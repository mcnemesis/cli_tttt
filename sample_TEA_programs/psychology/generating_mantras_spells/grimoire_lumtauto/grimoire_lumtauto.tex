%---[ TEMPLATE: BOOK | grimoire | mini-treatise ]
%---|original-author:JWL|date:NOV2025|email:jwl@nuchwezi.com
%---|TITLE: NOVUS MODERNUS GRIMOIRE LUMTAUTO MAGIA
%---|EDITOR:Prof. Joseph Willrich Lutalo, Oxford
%---[MANUSCRIPT-ORIGINAL-VERSION:15NOV2025]
%---[MANUSCRIPT-LATEST-VERSION:27NOV2025]
%----------------------------------------------------------------|
%\documentclass[a4paper, 18pt]{article} % A4 paper, readable font size
%\documentclass[a4paper, 12pt]{book} % A4 book-layout, readable font size
\documentclass[a4paper]{book} % A4 book-layout, readable font size
%\documentclass[12pt,a4paper]{article}
%\usepackage[a4paper,margin=1.4cm]{geometry}
%\usepackage[a4paper, left=2cm, right=1.5cm, top=1.5cm, bottom=1.5cm]{geometry} % Adjust these values as needed
\usepackage{hyperref}
\usepackage{parskip}


%%------ for poetry

\usepackage{setspace}
\usepackage{lmodern} % for typewriter font consistency
\usepackage{geometry}

% Define custom environment for padded poem block
\newenvironment{poembox}{%
  \begin{list}{}{%
    \setlength{\leftmargin}{0.05\textwidth}%
    \setlength{\rightmargin}{0.05\textwidth}%
  }%
  \item[]}{\end{list}}

% Define styles for each language
\newcommand{\Cwezi}[1]{\par\noindent\textbf{[Cwezi]}\\{\large #1}\par}
\newcommand{\Ganda}[1]{\par\noindent\textbf{[Ganda]}\\\textit{#1}\par}
\newcommand{\English}[1]{\par\noindent\textbf{[English]}\\\texttt{#1}\par}

%%----- end for poetry

% for multiline comments...
\newcommand{\comment}[1]{}

%for striking-through text
\usepackage{ulem}

% allow table of contents to also list subsections
\setcounter{tocdepth}{3}

% for better appendices
\usepackage[title,titletoc]{appendix}

% for controlling page numbers
\usepackage{fancyhdr}
\pagestyle{fancy}
\fancyhf{}
\fancyhead[R]{\thepage}

% throw in page-top header
\fancyhead[L]{IONA Inner Journal, Vol. 2}

% for line-breaks in table cells
\usepackage{makecell}

% for graphics
\usepackage{graphicx}
\usepackage{caption}
\usepackage{float}

%for multi-figure figures?
\usepackage{subcaption}

% for highlighting text
\usepackage{xcolor, soul}
% then define colors we shall use:
\definecolor{myteal}{RGB}{0, 128, 128}
\definecolor{lightgray}{HTML}{CCCCCC}
\definecolor{myorange}{HTML}{FFD7B3}

% for table with alternating row bg colors 
\usepackage[table]{xcolor}
\definecolor{lightgray}{gray}{0.9}  % or use HTML/RGB if preferred


%\definecolor{highcolor}{rgb}{0,255,255} % our default hl color for background, friendly on black text foreground
\definecolor{highcolor}{rgb}{0,255,255} %a accent background color, must be friendly on black text foreground
\sethlcolor{highcolor}

%for regular expression and TEA code presentation in console-like text-boxes
\usepackage{tcolorbox}
\tcbuselibrary{listings,skins,breakable}
\usepackage{listings}

% style for general terminal-like listings
\tcbset{
  myterminalstyle/.style={
    colback=black,       % background color
    coltext=white,       % text color
    fontupper=\ttfamily, % typewriter font
    boxrule=0pt,         % no border
    arc=0pt,             % square corners
    outer arc=0pt,
    left=2mm, right=2mm, top=1mm, bottom=1mm,
    enhanced,
    sharp corners,
  }
}

% define listings config for TEA language
\lstdefinelanguage{TEA}{
  morecomment=[l]{\#},
  sensitive=true,
  alsoletter={:*!},
  %morekeywords=[1]{i:, u!:, g:, l:, f:, x:, j:, q!:},
  morekeywords=[1]{%
a:, a.:, a*:, a!:, a.*:, a.!:, a*!:, b:, b.:, b*:, b!:, b.*:, b.!:, b*!:, c:, c.:, c*:, c!:, c.*:, c.!:, c*!:, d:, d.:, d*:, d!:, d.*:, d.!:, d*!:, e:, e.:, e*:, e!:, e.*:, e.!:, e*!:, f:, f.:, f*:, f!:, f.*:, f.!:, f*!:, g:, g.:, g*:, g!:, g.*:, g.!:, g*!:, h:, h.:, h*:, h!:, h.*:, h.!:, h*!:, i:, i.:, i*:, i!:, i.*:, i.!:, i*!:, j:, j.:, j*:, j!:, j.*:, j.!:, j*!:, k:, k.:, k*:, k!:, k.*:, k.!:, k*!:, l:, l.:, l*:, l!:, l.*:, l.!:, l*!:, m:, m.:, m*:, m!:, m.*:, m.!:, m*!:, n:, n.:, n*:, n!:, n.*:, n.!:, n*!:, o:, o.:, o*:, o!:, o.*:, o.!:, o*!:, p:, p.:, p*:, p!:, p.*:, p.!:, p*!:, q:, q.:, q*:, q!:, q.*:, q.!:, q*!:, r:, r.:, r*:, r!:, r.*:, r.!:, r*!:, s:, s.:, s*:, s!:, s.*:, s.!:, s*!:, t:, t.:, t*:, t!:, t.*:, t.!:, t*!:, u:, u.:, u*:, u!:, u.*:, u.!:, u*!:, v:, v.:, v*:, v!:, v.*:, v.!:, v*!:, w:, w.:, w*:, w!:, w.*:, w.!:, w*!:, x:, x.:, x*:, x!:, x.*:, x.!:, x*!:, y:, y.:, y*:, y!:, y.*:, y.!:, y*!:, z:, z.:, z*:, z!:, z.*:, z.!:, z*!:,%
A:, A.:, A*:, A!:, A.*:, A.!:, A*!:, B:, B.:, B*:, B!:, B.*:, B.!:, B*!:, C:, C.:, C*:, C!:, C.*:, C.!:, C*!:, D:, D.:, D*:, D!:, D.*:, D.!:, D*!:, E:, E.:, E*:, E!:, E.*:, E.!:, E*!:, F:, F.:, F*:, F!:, F.*:, F.!:, F*!:, G:, G.:, G*:, G!:, G.*:, G.!:, G*!:, H:, H.:, H*:, H!:, H.*:, H.!:, H*!:, I:, I.:, I*:, I!:, I.*:, I.!:, I*!:, J:, J.:, J*:, J!:, J.*:, J.!:, J*!:, K:, K.:, K*:, K!:, K.*:, K.!:, K*!:, L:, L.:, L*:, L!:, L.*:, L.!:, L*!:, M:, M.:, M*:, M!:, M.*:, M.!:, M*!:, N:, N.:, N*:, N!:, N.*:, N.!:, N*!:, O:, O.:, O*:, O!:, O.*:, O.!:, O*!:, P:, P.:, P*:, P!:, P.*:, P.!:, P*!:, Q:, Q.:, Q*:, Q!:, Q.*:, Q.!:, Q*!:, R:, R.:, R*:, R!:, R.*:, R.!:, R*!:, S:, S.:, S*:, S!:, S.*:, S.!:, S*!:, T:, T.:, T*:, T!:, T.*:, T.!:, T*!:, U:, U.:, U*:, U!:, U.*:, U.!:, U*!:, V:, V.:, V*:, V!:, V.*:, V.!:, V*!:, W:, W.:, W*:, W!:, W.*:, W.!:, W*!:, X:, X.:, X*:, X!:, X.*:, X.!:, X*!:, Y:, Y.:, Y*:, Y!:, Y.*:, Y.!:, Y*!:, Z:, Z.:, Z*:, Z!:, Z.*:, Z.!:, Z*!:%
},
  keywordstyle=[1]\color{green},
  commentstyle=\color{lightgray},
  morestring=[b]",
stringstyle=\color{myorange},
moredelim=[s][\color{myorange}]{\{}{\}},
}


% define custom terminal for TEA language snippets

\tcbset{
  teaterminalstyle/.style={
    enhanced,
    colback=myteal,
    coltext=white,
    fontupper=\ttfamily,
    boxrule=0pt,
    arc=0pt,
    outer arc=0pt,
    left=2mm, right=2mm, top=1mm, bottom=1mm,
    sharp corners,
    listing only,
    listing options={
      language=TEA,
     basicstyle=\ttfamily,
%keywordstyle=\color{cyan}\bfseries,
%commentstyle=\color{green}\itshape,
%stringstyle=\color{yellow}
    }
  }
}



% from google-gemini for verbatim listings:
\tcbuselibrary{listings,breakable}

% Define a language with no syntax highlighting
\lstdefinelanguage{none}{}

% Define a new tcblisting environment for verbatim content
\newtcblisting{tcbverbatim}[1][]{
  % Pass any user options to the new environment
  #1,
  breakable, % Allow long lines to wrap
  listing only, % The box contains only a listing
  listing options={
    language=none, % The 'none' language disables highlighting
    basicstyle=\ttfamily, % Use the typewriter font
    columns=flexible, % Allow flexible column widths for wrapping
    breaklines=true, % Enable line breaking
  },
}



%%---------------CONFIG JAVASCRIPT

%\usepackage{listings}
%\usepackage{xcolor}

% Define custom colors
\definecolor{jskeyword}{RGB}{0,0,180}
\definecolor{jsstring}{RGB}{163,21,21}
\definecolor{jscomment}{RGB}{0,128,0}
\definecolor{jsnumber}{RGB}{128,0,128}
\definecolor{jsbackground}{RGB}{245,245,245}

% JavaScript style for listings
\lstdefinelanguage{JavaScript}{
  keywords={break, case, catch, continue, debugger, default, delete, do, else, 
    finally, for, function, if, in, instanceof, new, return, switch, this, throw, 
    try, typeof, var, void, while, with, const, let, class, extends, super, import, 
    export, yield, async, await},
  keywordstyle=\color{jskeyword}\bfseries,
  ndkeywords={boolean, number, string, null, undefined, true, false, Array, Date, 
    eval, function, Math, Object, RegExp},
  ndkeywordstyle=\color{jsnumber}\bfseries,
  identifierstyle=\color{black},
  sensitive=true,
  comment=[l]{//},
  morecomment=[s]{/*}{*/},
  commentstyle=\color{jscomment}\ttfamily,
  stringstyle=\color{jsstring}\ttfamily,
  morestring=[b]',
  morestring=[b]"
}


% General listings setup
%\lstset{
%  language=JavaScript,
%  backgroundcolor=\color{jsbackground},
%  basicstyle=\ttfamily\small,
%  numbers=left,
%  numberstyle=\tiny\color{gray},
%  stepnumber=1,
%  numbersep=8pt,
%  showstringspaces=false,
%  tabsize=2,
%  breaklines=true,
%  frame=single,
%  rulecolor=\color{gray},
%  captionpos=b
%}



%%---------------END CONFIG JAVASCRIPT



% for maths
\usepackage{amsmath}
% for number sets symbols
\usepackage{amssymb}
%\usepackage{ntheorem}
\usepackage{amsthm}


% for writing our theorems and defs...
\newtheorem{comp}{Computation}
\newtheorem{theo}{Theorem}
\newtheorem{defn}{Definition}
\newtheorem{lem}{Lemma}
\newtheorem{prop}{Proposition}
\newtheorem{axiom}{Axiom}
\newtheorem{post}{Postulate}
\newtheorem{trans}{Transformation}
\newtheorem{transf}{Transformer}
\newtheorem{law}{Law}
\newtheorem{prob}{Problem}
\newtheorem{soln}{Solution}
\newtheorem{alg}{Algorithm}

\title{\textbf{NOVUS MODERNUS GRIMOIRE LUMTAUTO MAGIA} --- A Modern Grimoire of 5 Eternal Magickal Languages\thanks{Proceed with Caution. This is \textbf{A Call to Practice} Occult Mysteries. There is no guarantee that tears wont be involved\cite{crowley1948magick}.}}

\usepackage{amsmath, amssymb, graphicx}

% to include pdf pages
\usepackage{pdfpages}

% Define \invpi to flip the pi symbol and use it as a function
\newcommand{\invpi}[1]{\mathop{\rotatebox[origin=c]{180}{$\pi$}}#1}
\newcommand{\invdel}[1]{\mathop{\rotatebox[origin=c]{180}{$\Delta$}}#1}

\author{\textbf{M*A*P} Adept Psymaz\thanks{\textbf{Most Ancient Priest}, also known as Fut. Prof. J. Willrich Lutalo C.M.R.W; Curator, PI and President at Nuchwezi Research, GARUGA, Uganda. \textbf{ORCID:} \url{https://orcid.org/0000-0002-0002-4657}}\\Nuchwezi Research\\\href{mailto:joewillrich@gmail.com}{joewillrich@gmail.com}, \href{mailto:jwl@nuchwezi.com}{jwl@nuchwezi.com}}

%\date \today
\date {FINAL-EDITION: \textbf{26}$^{th}$ \textbf{NOV}, \texttt{2025}}


\begin{document}


%---[ START BOOK CONTENT/FRONT ]
\frontmatter

% insert [front] cover --- could just be a PNG or PDF
\includepdf[pages=1]{resources/front_cover.pdf}


% insert I*POW frontmatter
\includepdf[pages=-]{resources/FrontisMatter.pdf}


%\maketitle



\chapter*{Dedication}

{
\LARGE
\begin{center}
\textit{For the Creative Co-Creator, for the Illuminati.}
\end{center}
}



\tableofcontents


%---[ BEGIN BOOK CONTENT/CHAPTERS ]

\mainmatter

\Large

%\begin{abstract}
\chapter*{PREFACE}


\noindent
\begin{minipage}{1\textwidth}
\vspace{1em}
\begin{quotation}
{\ttfamily

If AHA does not build a house
in vain do its builders toil.
If AHA does not guard a nation
in vain do its armies keep watch.


In vain you get up earlier,
and put off going to bed,
sweating to make a living,
since it is he who provides for his beloved as they sleep.

Sons are a birthright from AHA,
children a reward from him.

}
\end{quotation}
\vspace{1em}
\end{minipage}


That opening verse is an adaptation of \textbf{Psalms 127:1-3} as presented in \cite{newjerusalem1985}. It is meant to remind the reader --- most likely coming to this book because of their interest in magic or magick and their applications solving one or another of their problems or just to learn something otherwise normally out of reach --- that, above all, things work how they do, or perhaps in our favour, not because we have invested sufficient effort or skill to make them so, but that instead, because a higher power --- essentially known to be God or the Mysterious First Cause and Architect of All Our Existence, let it be so. That adaptation appeals to \textbf{AHA} --- Runyakitara for ``Provider" --- because indeed, with or without magick, all our well-being stems from some Mysterious Divine Providence.

\vspace{2em}

Take for example the reality that every single one of you reading this book shares with me --- at some point at or during the start of your physical/material and perhaps also your spiritual existence --- when you were but just a basic fertilized ova dwelling in your mother's womb, and all the time while you dwelt in your mother's womb (even though some might have later, perhaps prematurely been born and had to dwell inside of some incubator or such...), whether or not you had some control over your reality, survival and well-being is not something we can objectively pin-down or agree upon, however, we can all objectively agree that, at least during that period of your life, you could not only not survive without some external help or support, but that also, and philosophically and spiritually so, that you were being sustained by some power or principle otherwise mysterious to you. Normally, we can refer to that kind of benevolent, life-giving, life-sustaining power as \textbf{Divine Providence} --- within IoN, we prefer to refer to such as \textbf{AHA} the \texttt{Mysterious Divine Hand of Providence}.

\vspace{2em}

And surely, the sacred name ``AHA", which also might resonate with the concept of ``Aha Moments", ``Eureka Moments"\cite{wordweb_assistant} should clearly help remind us of the fact that sometimes, \textit{out of the blue}, we come across a solution, an idea, a place, a dream-come-true, a miracle, etc. that somewhat defies our expectations or which we otherwise had not planned or didn't directly have a hand in making manifest --- and yet, as with most such moments or experiences, it leaves us satisfied, happy or thrilled by whatever it is we have accomplished. We are better off with than without such AHA moments!

\vspace{2em}

The topic of magic --- in much of this book, preferably referred to as ``magick" to differentiate it from mere trickery and illusory experiences, is something that underlies much of sane spiritual, cultured and civil human life since when we became aware of the need to align ourselves with or tap into some unseen, unheard, and mostly unknown life-giving, life-sustaining, life-preserving power that we might as well refer to as the Mysterious Lord Our God. Let's return to the above example of our selves while we were but infants... 

\vspace{2em}

As infants, or even as little children not just crying whenever we needed something or when we felt threatened or insecure, it was surely by divine design --- or perhaps \textbf{natural law} --- that, even when we had not yet learned to read or write, or even before we had learned to mumble babyish gibberish, that, when we needed --- especially \textit{when we badly needed something} --- that we resorted to some special form of communication and clever use of language so as to draw the attention of our immediate AHA to come to our rescue; the child might cry, might wear a smile or some other strong emotion\footnote{Talking of emotions; in the days while working on the final proof-reading of this manuscript, especially the mathematical formulations and the magical sigils, neither of which were easy to polish; exhausted and tired, I went to sleep one night, only to be awakened mysteriously as I dwelt in a dream in which my [now-deceased.. 20 years now] father's younger-self, together with my mother's younger-self, appeared to a grown-up me, and that we were at some exhibition of cultural artifacts and esoteric artworks and crafts, and that they wanted some fine piece of cultural wear on exhibit at some tent, but that they'd already spent most of their money and had run out of cash! I then inquired about their need, talked with the vendor, and covered the balance after some negotiations. That was all in a dream. Moreover, after I woke up --- that same night, something troubled me, and drew me towards reviewing this book once more, only to find that, among the critically important mathematical formulations being presented, we had a problem --- there seemed to be two competing implementations of the \textbf{modal sequence statistic} --- a vital statistic measure we have heavily used not only in this but also earlier work, and which underlies the computer programming TEA primitives \texttt{u:} and \texttt{u!:}. Interestingly, this came to my attention, when especially reviewing the sigils of \textbf{Leviathan} --- an angel sometimes associated with mysteries in the oceans and great lakes, but also, and mystically also associated with emotions.}, they might clench something tightly, pull or push things about, they might giggle or hysterically laugh and such, when it is the case that they need to draw someones or anyone's attention to them --- or specifically, to something they badly need or want. That's a child's way of applying magical language to obtain their needs. And many times, if not always, when a child performs such feats --- whether consciously or not, they soon get the attention of someone --- a mother most likely, but perhaps their father, a big sister or brother, a concerned neighbor, the nanny, etc. Basically, by expressing themselves well, a baby, a child, an infant, eventually gets what they want. In fact, for the case of kids and babies, it might not even be that they need or should express themselves well, but that they even just slightly need to do so --- some parents will quickly react, or come to the attendance of a child at the slightest indication that the child is either hungry, in trouble or afraid of something. 

\vspace{2em}

In case a child were to express themselves thus, and yet, no one present or within their reach, hearing or seeing them, came in to help or at least call for someone in-charge to attend to the child, it would not only be an act of evil, immorality or perhaps an inhumane act, but it would definitely leave whoever was in the know about the child's struggle and cries with a karmic debt. Most likely, God wouldn't let such an unconcerned person go unpunished, and this, genuinely, rightly so.

\vspace{2em}

Moreover, the use of ``special languages" and ``special communications" is an aspect of all humans --- whether as little children still looking towards our immediate mother and father for almost everything we need, or whether we are grown-ups --- that have learned to talk with neighbors and friends, that picked up formal languages at school, learned to read, draw and sing, etc. So that, by the time one is perhaps a teenager, they are a kind of well-equipped communicator; apart from psychologically tortured children and/or those that underwent some sort of physiological alterations or who have natural incapacitations, most humans, starting from around their early teen years and into adulthood and perhaps even into their ``evening years", very well know how to not only reach out to the human nearest to them for assistance or support, but can also very well put their intentions into either writing or clear utterances, that they can then send off/dispatch, to either a specific [remote] person or otherwise the universe --- this later case, for those who have also obtained some spiritual or religious training, and which is perhaps aimed towards their God, their angel(s), some ``higher power" or something they trust or believe is out there listening, and can react or come to their help; could be a ``nature spirit", ``my beloved but deceased grandfather", ``little old Timmy from the Peace Corps that used to welcome me back from school whenever I walked past their shop, and that he'd usually see me hungry, call to me and give me a slice of cake or a cup of milk", ``the Saint Kizito", etc. Such are the kinds of ``powers", seen or unseen, that most humans, most magicians appeal to, when they `cry out' using their basic or arcane rituals, fervent prayers\footnote{Concerning prayer, it should come to your attention that, indeed, and as mystical teachers of old always taught, many times, \textbf{the best kind of magick is just prayer!} I personally would likewise recommend this to most magicians --- whether beginners or advanced; to \textbf{first of all learn how to pray} --- many times, we resort to magick, not because we can't or do not know how to pray, but just because, we are tired or bored by ``usual kinds of prayers" --- especially the kinds which involve repetitive formulas, plastic prayers or prayers that are just ``empty" --- words without actions, actions without meaning or actions without emotions, etc. Especially, and in line with the philosophy being advocated for here, it is very worthwhile, effective, and satisfying, to pour emotions into prayer, but also, to pray[ and play] like children --- sometimes, with a little or sufficient play/fun/drama thrown-in, which, as Jesus once taught, is a sure way to get the attention of ``Our Heavenly Father" --- ``In truth I tell you, unless you change and become like little children you will never enter the kingdom of Heaven" (\textbf{Mathew 18:3}, but also \textbf{Mark 10:15}, \textbf{Luke 18:17}\cite{newjerusalem1985}, the overall message being a ``call to spiritual simplicity, trust and humility before God"\cite{copilot_assistant}}, chants or even and sometimes, just like a troubled infant that is not getting sufficient attention from those present; when they become destructive and perhaps send our shrieks, break or burn things, cut or torture themselves, kills pets or beasts etc. All, so they can draw the attention of some ``Provider"\footnote{Especially their \textbf{Sole Provider} --- perhaps also the provider of their ``soul"!}, some ``Care-taker", some ``Supervisor", some ``Prefect", some ``Mother" or fatherly power, some ``Friend" or ``Familiar", etc. to come to their rescue and help address what it is they are (hopefully honestly, earnestly) in need of.


\vspace{2em}


Thus, in this non-trivial manual on magical languages, we are going to take that above philosophy and line of thought, and develop it rigourously and into more than 1 system of magical communication and means of expression, with which, not just an adult, but perhaps even children, might be able to take their inherent needs, problems, desires, wants, etc. and carefully express them well, using either visual or verbal languages and expression forms, that they can then send out into the world --- inner/subjective world or outer/objective world --- and so that, with conscious [and for the case of magick, also ``unconscious"] effort, they can successfully draw the attention of something --- most likely God, AHA, their HGA/Holy Guardian Angel, their Patron Saint, their Ancestral Spirit, Mother Nature, a human or non-human neighbor, etc. to come to their aid, and attend to or help address what it is they are honestly looking for, in need of or hoping for. 


\vspace{2em}


Morever, and especially with the knowledge and somewhat deep familiarity of how the ``still mysterious" phenomena of magick operates, we shall also explore several methods of enchantment, gnosis, conjurations, association and such, with which, the operator --- a magician, a student of magick, an initiate, etc. --- might apply, so as to make such communication more permanent and naturally ``universal", when, instead of communicating directly [and mostly mundanely] using their physical apparatus alone --- mouth, hands, body and such --- they instead communicate using their ``inner-selves" as well; their spirit, soul, astral self, higher-self, their unconscious mind, etc. And so that, the kind of communication they then make using such methods, is generally guaranteed to cut-through most bullshit and effectively reach the intended target --- especially where such a target isn't another physical human who they might otherwise approach physically and directly, and talk to/with. Such is the reason we intend to dive deep into the study and exploration of \textbf{magical languages} and \textbf{magical communication systems}.

\vspace{2em}

Talking of languages we shall cover, note that, even though the focus of this grimoire is on languages we have developed or discovered ourselves (here at Nuchwezi Esoteric School --- the education/academic research arm of our ``Illuminates of Nuchwezi" collective), and yet, these \textbf{5 languages --- lumtauto, myrrh, ozin, medina and miti}, are not entirely alien and nor are they superficial. The careful student, or even the seasoned magician and researcher shall notice that our languages have some semblance to other ``magical languages" and ``magical alphabets" that others have developed, discovered or been taught throughout the ages and across cultures and traditions. One shall for example find, when studying the works of esotericism scholars such as \textbf{Janet Farrar and Gavin Bone}\cite{farrar1990spells}, that our \textit{visual languages} --- ozin, medina and miti, have some distant relatives too; ozin somewhat looks or feels like Germanic ``Runes" or the Hebrew-like ``Celestial Script" or perhaps the ``Malachim Script"; medina and miti, both spatially-defined ``magical alphabets"\cite{farrar1990spells}, somewhat relate to \textit{line-oriented} magical languages such as the ``Ogham Script"\cite{farrar1990spells}, etc. And so that, learning one or more of the languages we present in this book, is guaranteed to open ones eyes to a usually (and \textit{intentionally}) ``hidden"/occult world of both explicit and subliminal messages, both in nature, but also in artificial/man-made environments and contexts.

\vspace{2em}

Thus, the serious student of this work is guaranteed to leave this book, when they are very well equipped, transformed and empowered. Apart from picking up new mathematical and psychological ideas, they are surely bound to be inspired by the art, literature and poetry, cultures and traditions, history and overall, a kind of pragmatic magick that we love to classify as ``Scientific Illuminism".

\vspace{2em}

It is our hope (as IonN, but also IoNA), that, from among the readers and practitioners of the work presented in this grimoire, shall emerge serious communicators; commanders of physical and spiritual powers or forces of various kinds; ``special" seers/scryers and diviners --- who can read God's and spiritual messages in many ways and languages in nature; creatives and artisans --- especially not the kind who focus on creating harmful or ``evil" crafts, but otherwise who can create protective talismans, healing or companion dolls, dream-inducing pillows, etc; writers --- not just of mundane literature but also sacred and magical literature that can open minds, transform hearts, end conflicts, etc.; architects of wondrous and mystifying worlds or artifacts; occult philosophers and [post-]modern engineers with an eye for not just the literal mundane world, but also the deeper, hidden, and yet important, significant aspects of both the individual, but also collective mind\cite{Lutalo2025transpsy},etc. Essentially, the \textbf{True, Gifted and Beneficial Illuminati}.\\\\
{\LARGE | \textbf{Joseph Willrich Lutalo} \textit{Cwa Mukama Rwemera Weira}\\
Founder and First Initiator for the Illuminates of Nuchwezi
}



%\begin{abstract}
\chapter*{Abstract}

%\begin{abstract}
%\Large
In this manuscript, \textbf{a grimoire mini-treatise} on \textbf{5 magickal languages}, we are to present for the first time, a proper distillation of research and applications in the use of esoteric languages for the purpose of performing occult operations as explored by initiates at Nuchwezi Esoteric School (NES) for the past 1 decade. This is an original contribution to the universal esoteric tradition and is meant to help curate, promulgate and advance a sane, thoughtful appreciation of the mysteries --- a line of work that goes back through the ages, to the first psy-ops and occult workings attempted by primitive man, through generations of diverse and varying explorations by mystics and initiates from all kinds of schools, cultures and traditions, all the way to modern approaches best known to the true initiates of illuminism. In particular though, this manuscript shall focus on 4 different ORIGINAL RESULTS of hard-work, study and yes, applying occult philosophy at Nuchwezi;
\begin{enumerate}
\item \textbf{LUMTAUTO} --- an application of computational mysticism in the form of an algorithmic cipher that can transform ordinary English or any language into a form suitable for occult operations and conjurations.
\item \textbf{The Grand Myrrh Transform} --- a related, but otherwise later and shorter, more occult TEA\cite{cli_tttt} algorithm for turning ordinary phrases into sacred words of power reminiscent of sacred languages such as Hebrew and Aramaic. 
\item  The \textbf{Ozin Cipher} --- a visual code first presented in an earlier work\cite{lutalo_2025_trans_genetics}, useful in expressing secret or special messages in an occult and psychologically charged hand that was first developed by the Illuminates of Nuchwezi Angelic (IoNA) via esoteric workings reminiscent of the methods of medieval angel-working wizards such as John Dee and Edward Kelley.
\item \textbf{Crypt of Medina} --- a special occult cipher also first developed at NES, and which, unlike most ciphers ancient or modern, allows for the visual encoding of occult messages in such a way that \textbf{one must explicitly use the method of reading between the lines} in order to understand its messages.
\end{enumerate} 
We shall look into the underlying philosophies and supporting literature; history, best-practices and applications. We shall treat of how to practically apply the presented ideas; for example, for almost all the example application scenarios we consider for each language, we likewise develop and present not only associated mathematical formalisms, but also computational formalisms and especially present most solutions in the form of generic algorithms --- even where a problem is to be solved by an operator other than a machine or automaton. Otherwise, and in the spirit of Computational Mysticism, we shall many times encounter cases of mixing modern computer technology in applying these otherwise ancient mystical and magical ideas, and shall share lots of visuals, links to supporting media, community and online tools to help practitioners further and deepen their appreciation and application of these modern mysteries. Mostly though, this is a work for, and by the Illuminati.
 \newline\newline
     \textbf{Keywords}: Foundations, Scientific Illuminism, Psy-Ops, Modern Magick, Computational Mysticism, Grimoire
%\end{abstract}

\begin{figure}[H]
  \begin{center}
   \includegraphics[scale=1]{resources/emblem_ion.pdf}\\
   \caption{Illuminates of Nuchwezi}
  \label{FIG1}
  \end{center}
\end{figure}



\begin{figure}[H]
  \begin{center}
   \includegraphics[scale=0.2, angle=90, height=0.8\textheight]{resources/the_message.pdf}\\
   \caption{Welcome Seekers, Soldiers and Commanders, to Immersive Future Illuminism}
  \label{FIGMESSAGECUNEIFORM}
  \end{center}
\end{figure}


\newpage
\chapter{An Introduction to Magickal Languages}
\label{SECINTRO}



In the preface of his treatise on applying the new mathematics of Transformatics\cite{Lutalo2025_transformatics_thesis} to Genetics\cite{lutalo_2025_trans_genetics}, the author appeals to the sacred scriptures, particularly to the first book of the Torah (also known as the ``Pentateuch") --- which, not just for Kabbalists and Jewish mystics might be considered the foundational book of essential laws and instruction, but which also serves as a foundational text for many judaic-associated faiths and traditions --- the likes of Christians, Moslems, Rastafarians but also, and not very surprising, Theistic Satanists\cite{wikipedia_theistic_satanism}\footnote{Some critics might argue that actually, modern satanism only borrows the concept of Satan (as \textit{ha-satan}) from the Hebrew bible but doesn't appeal to Judaic roots or laws and that it instead is founded on twisting and elevating the Christian idea of Satan as a fallen angel\cite{copilot_assistant}\cite{wikipedia_theistic_satanism}, however, and logically so, it does make sense to pin them down, and assert their undeniable roots in ancient Judaism and its traditions for that reason alone. Also, note that we talk of \textbf{theistic satanism} here and not \textit{atheistic satanism} (also \textbf{LaVeyan Satanism}); Satan as a symbol not as a deity, and also not \textit{acosmic satanism} (more popular with edge-lords and ecclectic modern LHP philosophers); satan as the principle that opposes order --- perhaps \textit{chaos}?)}!. Lutalo calls us to consider the strong directive given man by God, when he is called to \textbf{``be fruitful, multiply, fill the earth and subdue it"} (Genesis 1:28).

In another work by Lutalo --- \textbf{3 Core Ideas in Computational Mysticism}\cite{Lutalo2024_3c}, a September 2024 paper that laid down the foundations of computational mysticism, he clearly lays down the fundamental significance of \textbf{language} not only in its use for bringing about transformations and the manifestation of will via computers, but also for general human life and affairs, when he says this in the abstract of that mini-paper:


\noindent
\begin{minipage}{1\textwidth}
\vspace{1em}
\begin{quotation}
{\ttfamily

Programming languages create a medium via which one can define and execute orders with certain effects at will, and certainly so. Basing on how language underlies the ability for humans to formulate and share thoughts with each other, we also see how the use of certain special languages underlies man's ability to command and control reality since ancient times.

}
\end{quotation}
\vspace{1em}
\end{minipage}


As we shall see and come to appreciate in this grimoire, a careful and willful use of ``special" languages --- especially, and from the perspective of the theme of this work, languages both inspired by or based on natural, but also artificial or \textit{synthetic} alphabets, can readily help yield results that we shall soon come to appreciate to be what ``magic"\cite{butler1952magic} or rather ``magick"\cite{crowley1929magick} is all about. Essentially, Lutalo tells us, in \cite{Lutalo2024_3c} that:


\noindent
\begin{minipage}{1\textwidth}
\vspace{1em}
\begin{quotation}
{\ttfamily

Basically, we see how it is indeed language, or rather, its use via communication --- a willful application
of language, that makes possible the creation and transmission of [any] thought.

}
\end{quotation}
\vspace{1em}
\end{minipage}


Thus, we argue that, \textbf{it is essentially via the willful application of language that man can come to subdue reality.} Moreover, and as we shall soon see when we consider the acceptable definitions of magick, any such willful acts, whether they leverage language in the form of expressions in the mind (``thoughtforms"), expressions in sound (``incantations", ``spells", ``mantras", ``commands", etc.), expressions in body/body-language (``gestures", ``mudras", ``assanas", ``signs", etc.) or as visual expressions (``sigils", ``mandalas", ``glyphs", ``signs" or ``symbols", etc.) are what make magick possible.


And as for the traditional concept of \textit{language}, still, \cite{Lutalo2024_3c}  offers a compelling and reliable working definition. But what of the idea of a \textbf{magickal language}? First, we shall return to the [modern] classics, \textbf{Aleister Crowley}\footnote{Apart from being a renown initiate into the ancient mysteries --- initiated circa 1898 into the \textbf{Hermetic Order of the Golden Dawn}\cite{cassiel1990encyclopedia}, he's also a famous and prolific writer and researcher on all matters occult and esoteric during the early 20$^{th}$ century, and was also well-known to have not only founded the tradition of THELEMA\cite{crowley1929magick} that has inspired many modern magical traditions such as the \textit{Wiccans} and several ``new-age" spirituality paths, but that he also considered himself not just a ``magus", but as also the ``BEAST"!} being one very undeniably reliable authority on the subject, and shall start by considering the ideas he presents in \textbf{Magick in Theory and Practice}\cite{crowley1929magick}:




\noindent
\begin{minipage}{1\textwidth}
\vspace{1em}
\begin{quotation}
{\ttfamily

Magick is the Science and Art of causing Change to occur in conformity with Will.

}
\end{quotation}
\vspace{1em}
\end{minipage}



To the best of our knowledge, it is not until when Crowley penned this definition, that magic as a stream of ``esotericism" or ``occult philosophy" and not as ``stage magic" or the practice of ``illusionism" passed from the realm of mere mysticism into a kind of formal science --- arguably, a great contribution to the galvanizing of so-called \textit{Scientific Illuminism}. Earlier authorities such as \textbf{Cornelius Agrippa}, despite having penned incredible tomes and compendiums on western esotericism and occult philosophy\cite{agrippa2014occult}, never actually, or rather, explicitly offered nor extended any such formal definitions that we know of.

That said, note that Crowley's definition isn't the only authoritative, nor usable \textbf{working definition} of magick, and as for that matter, though we shall not attempt to enumerate all of them here, there is a rare compilation of such definitions that was prepared by the author of this grimoire as far back as 2014\cite{lutalo_2025_definitions}, and that it not only lists Crowley's definition among a whooping total of \textbf{34 distinct definitions of magic[k]}, but that it also clearly offers their associated sources (authors and books/articles/websites/traditions, etc.). Among these, let us just recall but only 3:



\noindent
\begin{minipage}{1\textwidth}
\vspace{1em}
\begin{quotation}
\noindent {\ttfamily

The science and art of causing change (in consciousness) to occur in conformity with will, using means not currently understood by traditional Western science

}
\hspace*{\fill} --- \textbf{Modern Magick}, \textit{2010}, Donald Michael Kraig\cite{kraig2010modern}
\end{quotation}
\vspace{1em}
\end{minipage}


That one, especially resurfaced here, because DMK has really helped modern magick practitioners working outside of traditional initiatory systems and who might not be able to access proper initiators into hermeticism and practical western esotericism to actually get busy and attain results. His book\cite{kraig2010modern} on magick is very resourceful --- especially for solo practitioners, and we shall come back to it several times in later parts of this grimoire.



\noindent
\begin{minipage}{1\textwidth}
\vspace{1em}
\begin{quotation}
\noindent {\ttfamily

The Highest, most Absolute, and most Divine Knowledge of Natural Philosophy, advanced in its works and wonderful operations by a right understanding of the inward and occult virtue of things; so that true Agents being applied to proper Patients, strange and admirable effects will thereby be produced.

}
\hspace*{\fill} --- \textbf{The Goetia of the Lemegeton of King Solomon}, \textit{1904}, S. L. MacGregor Mathers and Aleister Crowley\cite{mathers1904goetia}
\end{quotation}
\vspace{1em}
\end{minipage}


That second one, not only because it is one of few that appeals to ``ancients" such as \textbf{Cornelius Agrippa}\cite{agrippa2014occult} or \textbf{Eliphas Levi}\footnote{This is the pen name of Alphonse Louis Constant (1810–1875), a highly influential French occult author and ceremonial magician whose writings significantly contributed to the revival of magic and esoteric thought in the 19th century\cite{wordweb_assistant}.} that especially championed concepts such as ``High Magic" or rather highly-eclectic Ceremonial [and priestly] Magick, but also because it might align well with classical kinds of magick such as the \textit{Alchemy} that scientists such as \textbf{Sir Isaac Newton} dabbled in occasionally, the \textit{Hermetic Medicine} that medieval esotericists such as \textbf{Paracelsus}\footnote{Also known as \textbf{Philippus Aureolus Theophrastus Bombastus von Hohenheim}\cite{FasanoSequeira2017}} --- also ``Father of Toxicology", championed, as well as mystics practicing arts such as \textit{Theurgy} --- the likes of \textbf{Emanuel Swedenborg}\footnote{Swedish scientist, theologian and mystic (1688-1772)\cite{wordweb_assistant}} enjoyed. Moreover, and also a contemporary of Crowley, \textbf{Mathers} does deserve a special place in the hearts of modern occultists for his many contributions to formalizing and promulgating ancient mysteries in the 20$^{th}$ century.


Finally, we shall also re-surface this definition:


\noindent
\begin{minipage}{1\textwidth}
\vspace{1em}
\begin{quotation}
\noindent {\ttfamily

The enhancement of the probabilities/likelihood of a desired outcome/result.

}
\hspace*{\fill} --- \textbf{PsyberMagick: Advanced Ideas in Chaos Magick}, \textit{1995}, Peter Carroll\cite{Carroll1995}
\end{quotation}
\vspace{1em}
\end{minipage}


That final one, especially because the founder of the \textit{Chaos Magick} meta-paradigm modern magick tradition has also greatly helped inspire and liberate many practitioners from ideological slavery to ancient and medieval dogmas, but also that, as one might find when studying many of \textbf{Peter Carroll}'s works, he, like the associated magical communities he inspired, such as the \textbf{Illuminates of Thanateros}\footnote{See IoT German Section: \url{https://iot-d.de/} or IoT BIS: \url{https://iotbritishisles.com/}}, but also our own \textbf{Illuminates of Nuchwezi}\footnote{Refer to IoNA home page: \url{https://iona.nuchwezi.com/}} and individual modern chaos magicians such as \textbf{Joshua Madara}\footnote{Originally \url{http://hyperritual.com/}, now \url{https://eldri.tech/}} find much utility in associating modern Tech, Maths and Science sensibilities with Esotericism and Magick\footnote{See for example, our take on Probabilistic Metaphysics\cite{Lutalo2023_metaphysics}, which strongly reflects or aligns with Carroll's definition and ideas of Magick.}.


\section{A Formal and Mathematical Definition of MAGICK and a Magickal Language}
\label{SECDEFMAGICK}


And so, having looked at all the past authorities, we shall consider just one more definition --- essentially, the one we consider to be our authoritative working definition, as laid out below:




\fbox{\begin{minipage}{\textwidth}
\large

Given the working definition of a \textbf{Certain Manifestor}\cite{Lutalo2025transpsy}\cite{Lutalo2023_metaphysics}:\\


\begin{transf}[The \textbf{Certain Manifestor}]
\label{TRANSFCM}
In a reality space $\Psi:N \times \psi_{k}$, \\
a \textbf{certain manifestor}, $\mathbb{k} : \Psi(\mathbb{k}): \psi_{\tau} : \tau \implies \mathbb{k} \quad \land \quad \invpi(\psi_\tau \in \Psi) = 1 \quad \\
\forall \mathbb{k} \in [1,|\Psi|]$ is the following operator:

\begin{trans}
 $\langle \tau \rangle \xrightarrow{O_{\lambda}(\cdot)}  \psi_\tau $\\
 \end{trans}
\end{transf}

we know that the operator $O_{\lambda}(\cdot)$, also defined as the \textbf{Certain Manifestor}, is a \textbf{Magician}, if we define magick as such:\\

\begin{defn}[\textbf{MAGICK}]
\label{DEFMAGICK}

Given some potential distinct event $e$ from the space of all possible events $\langle e* \rangle$, it can be manifested when requested for, willed or wanted --- by the application of strong faith, belief or [un]conscious effort on the part of the operator relative to that event, $\lambda (\langle e* \rangle)(e)$, in an act that is essentially one of applying psychology, but otherwise which we shall call \textbf{MAGICK}. Essentially, then, \textbf{Magick is any act that can make the following transformation happen:}\\

{\Large

\begin{trans}
\label{TRANSMAGICK}
 $\langle e \rangle \xrightarrow{O_{\lambda}(\psi_\infty)(e)} \psi(\overset{>}{e}) \approx  \psi_e $
 \end{trans}
 
 }


\end{defn} 

\end{minipage}}
\\


\begin{table}[htp]
  \begin{tabular}{|p{0.95\textwidth}} % Left border only
    \hline
    \begin{figure}[H]
      \centering
      \includegraphics[width=0.9\textwidth]{resources/nu_chwezi_people__mwaru_s_day_by_nemesisfixx_dbq604u.jpg}\\
  \caption{\textbf{The 3 Magi?} Then, \textit{Neophytes}, Students of Magick at Nuchwezi Esoteric School --- \texttt{L-R}: \textbf{Abraxas}, \textbf{Psymaz} and \textbf{Nishtar}; after performing an Ancient African Transmutation Ritual circa 2016, on \textbf{Halloween-Samhain} (``MWARUS Day")}
      \label{FIGNES}
    \end{figure} \\
    \cline{1-1} % Bottom border only
  \end{tabular}
\end{table}





The critical term $ \psi(\overset{>}{e}) $ in \textbf{\hyperref[TRANSMAGICK]{Transformation \ref{TRANSMAGICK}}} is meant to symbolize the ``desired result" --- a \textit{willed outcome}, that essentially is encoded in the language of transformatics\cite{Lutalo2025_transformatics_thesis} here as a kind of [identifying particular modal] sequence that captures the gist of that event or phenomena\footnote{Note that, as we saw in the important fundamental law --- \textbf{IGS}: \textbf{Identity Genome Sequence Law} first laid down in \cite{lutalo_2025_trans_genetics}, any potential event or phenomena might be somehow reducible to or expressible as some distinct sequence statistic, and as per the sensibilities of transformatics, any such sequence might be related to or reducible to some symbol set.}, whereas $\langle e \rangle$ could be just a symbol, a word, thought or label depicting that which is desired or wanted.


A \textbf{magickal language} then, is a concept we might formally define as such:\\

\begin{defn}[A \textbf{Magickal Language}: $\mathbb{L}: \mathbb{N} \times \psi_\infty$]
\label{DEFMAGICKLANG}

Any system of encoding will, and particularly, such as would allow one to encode some desire or need such as in the input of \textbf{\hyperref[TRANSMAGICK]{Transformation \ref{TRANSMAGICK}}}, so that, by presenting it to a certain manifestor --- a special kind of operator --- essentially a magician, the corresponding desired event or result can then readily be manifested, is a \textbf{Magical Language}, $\mathbb{L}: \mathbb{N} \times \psi_\infty$, that \textit{might} span the infinite set of symbols or events $\psi_\infty$.

\end{defn} 



At this juncture, and given what we have just clarified, it might help to set some things clearer concerning how modern magick needs be approached, but also how it might be appreciated. First, note that, magick, despite being an art, is also a science\cite{lutalo_2025_definitions}. In fact, even long before modern authorities on magick such as the Computer Scientist Peter Carroll gave us a somewhat mathematical treatment\cite{Carroll1995}\cite{carroll2010octavo} of an originally mostly speculative and mystical craft, and yet, in their work on the theory and practice of magick, Aleister Crowley did also greatly help cast magick as an enterprise one might come to properly appreciate or evaluate through the lens of a mathematician, when he laid down the following theorem and remarks:


\noindent
\begin{minipage}{1\textwidth}
\vspace{1em}
\begin{quotation}
\noindent {\ttfamily

THEOREMS.
1) Every intentional act is a Magickal act.

By ``intentional" I mean ``willed". But even unintentional acts so seeming are not truly so. Thus, breathing is an act of the Will to Live.

}
\hspace*{\fill} --- \textbf{Magick in Theory and Practice}, \textit{1929}, Aleister Crowley\cite{crowley1929magick}
\end{quotation}
\vspace{1em}
\end{minipage}

And so that, for our case, we can rest assured, that the following theorem likewise shall drive the point home concerning why developing, knowing and applying a magical language might be more useful than not:\\

\begin{theo}[\textbf{Effectiveness of a Magickal Language}]
\label{THEOMAGLANG}

Given some desire, $\tau$, and a magician that can operate on it, $\lambda(\cdot)$, we can assert that: presenting the suitable encoding of $\tau$ in some magical language that the operator can process effectively, shall yield better results/increase the likelihood of manifesting the desired outcome than not.

\begin{proof}
The proofs are several, but we shall call out just two:
\begin{enumerate}
\item Follows from \textbf{\hyperref[DEFMAGICK]{Definition \ref{DEFMAGICK}}} and \textbf{\hyperref[DEFMAGICKLANG]{Definition \ref{DEFMAGICKLANG}}}.
\item Experience and generations of successful practitioners confirm this; the most effective magical operations require or assume the need to express desire using some encoding method such as sigilization, mantras, visualization, specific asanas, specific programs in specific languages, etc. so as to readily or quickly and most efficiently bring the [desired] result into manifestation.
\end{enumerate}
$\qed$
\end{proof}

\end{theo}


By now, \textbf{\hyperref[THEOMAGLANG]{Theorem \ref{THEOMAGLANG}}} should be obvious to anyone reading this grimoire --- like, for example, considering the otherwise commonplace case of \textit{wanting to eat food presented on a plate}: no matter what one might do, say or think, unless they actually go ahead and employ the ``right" language of ``eating" --- which entails presenting something to the mouth, chewing and/or swallowing it so as to actually ``eat" it; any other actions the operator might apply --- such as smiling at the food, looking at a picture of the food, or perhaps merely smelling it, might only leave the food turning cold, and perhaps only satiate the mind (that ``I have food"), but otherwise leave the person not nourished and possibly still hungry (the food hasn't been ingested, and neither has it been digested). A related, and totally absurd case might be that of \textit{wanting to bare a [normal human] child} without having sex nor having ones sperm presented to someone's ready ova. It just wouldn't work!\footnote{Well, one might will a child into existence [without sex], say, by having some donor offer them a baby, or by adopting an abandoned kid, or adopting an already impregnated girl etc. However, the original intent of bearing one's own literal child shall not be met --- preternatural cases such as the immaculate conception of the Blessed Virgin Mary [dogma], and subsequently her \textbf{virgin birth} of Jesus Christ[Matthew (1:18–25), Luke (1:26–38)] might of course deviate from this, but also then, the birth of Jesus without Joseph having to copulate Mary perhaps didn't originate from an intent of Joseph's desire to bear \textit{his own child} --- as careful analysis of scripture\cite{newjerusalem1985} does indeed confirm.} 


In the rest of this manuscript, we are thus going to spend time exploring, in theory, but also in practice, several original and multipurpose (and arguably ``general", but not necessarily infallible) magickal languages that have been developed or discovered by the author while exploring and applying Magick at NES\footnote{Despite being a somewhat obscure institution by normal standards, \textbf{Nuchwezi Esoteric School} has come a long way as one might tell from one of the few photographs of past students and activities as depicted in \textbf{\hyperref[FIGNES]{Figure \ref{FIGNES}}} that shows 3 of the founding team, and who started a tradition at NES, of convening magical meetups around the international pagan holiday, ``Samhain", a day which, for members of NES especially, was renamed ``MWARUS" day --- after the quirky, but meaningful application of LUMTAUTO to the word ``NDAHURA" --- name of last king of the Cwezi Empire and Dynasty, also a deified entity in most pantheons in modern Ganda and Cwezi traditions. As with most things occult, much of what actually goes on at NES --- see \textbf{\hyperref[FIGNES2]{Figure \ref{FIGNES2}}} --- stays hidden or secret, but yes, a tradition started, and genuine reasons, is worth perpetuating and advancing.} , and yes, in sane, and for sane reasons\footnote{There is nothing as insane as assuming that a limited operator [\textit{an automaton and not a psymaton, a psymaton and not a man, a man and not an angel, an angel and not a god, a god and not God}] might be able to process an infinite sequence of [magickal] languages when presented with them or an infinite sequence of desires encoded using them, however, we shall still proceed to present these [limited] languages, which, even though they might seem finite and constrained for one or some operator, could otherwise find [infinite] purpose in the hands of yet another [well-prepared] operator, and thus, it might then not be insane to consider that all of magick might be somehow reducible to just the mastery of [one of] these \sout{four} five languages we are about to unravel.}.


\begin{table}[htp]
  \begin{tabular}{|p{0.95\textwidth}} % Left border only
    \hline
    \begin{figure}[H]
      \centering
      \includegraphics[height=0.8\textheight]{resources/nuchwezi_early_days.jpg}\\
  \caption{\textbf{A School at Home}: a collage showing the way Psymaz decided to dedicate several special spaces at his home for the purpose of private study and explorations of the mystical. So-called NES in the real-world.}
      \label{FIGNES2}
    \end{figure} \\
    \cline{1-1} % Bottom border only
  \end{tabular}
\end{table}





\fbox{\begin{minipage}{\textwidth}
\large

\textbf{Concerning Relevance of [Practicing] Magick}:\\

Especially for the uninitiated, much of what is being talked about in this volume might seem or come off as strange gibberish, total mysticism or scientific absurdity! However, like the scripture warns the wise to \textit{not throw pearls at swine};


\begin{quotation}
\noindent {\ttfamily

Do not give dogs what is holy; and do not throw your pearls in front of pigs, or they may trample them and then turn on you and tear you to pieces.

}
\hspace*{\fill} --- \textbf{Sermon on the Mount}, \textit{Mathew 7:6}, Jesus Christ\cite{newjerusalem1985}
\end{quotation}
\vspace{1em}
 

We [the initiated, genuine students and explorers of the sacred and mystical], ought be, first, unbothered that the mundane don't see any value in dedicated study and careful practice of Occult Science. But also that, we ought not concern ourselves with trying to appeal to plebeians and the unworthy, for the true rewards of this science and art are innumerable, ancient and eternal --- as one authority, \textbf{Walter Ernest Butler}, shared in his foundational treatise on the subject:

\begin{quotation}
\noindent {\ttfamily

...the old Hermetic axiom \textit{Solve et coagula}, which may be rendered as ``Dissolve and re-form," and so he uses the rites of the High Magic to effect both that dissolution and that reformation. But what is dissolved, and what is reformed? ...it is the personal self which he has for so long regarded as his only real self, this personality which he has so tenaciously clung to and defended, has pampered and indulged --- it this \textit{persona} this mask of the real man which must be dissolved and reformed. But how shall that which is itself imperfect produce perfection? ``Nature unaided, fails," said the old alchemists, and in the Scriptures we read ``Except the Lord build the House, the workman worketh in vain." So the magician in all humility seeks the Knowledge and Conversation of his Holy Guardian Angel --- that True Self of which his earthly personality is but the mask. This is the supreme aim of the magician. All else, spells and charms, rituals and circles, swords, wands and fumigations, all are but means by which he may accomplish that end. Then, being united with that True Self --- if only for a brief time --- he is instructed by that Inner Ruler in that Higher Magic which will one day bring up his manhood into his Godhood and will achieve that which the True Mysteries have ever declared to be the true end of man --- Deification.


}
\hspace*{\fill} --- \textbf{Magic: Its Ritual, Power and Purpose}, \textit{1952}, W.E. BUTLER\cite{butler1952magic}
\end{quotation}

\end{minipage}}
\\





\begin{figure}[H]
  \begin{center}
   \includegraphics[scale=1]{resources/iona.pdf}\\
   \caption{Illuminates of Nuchwezi Angelic}
  \label{FIG2}
  \end{center}
\end{figure}


\begin{figure}[H]
  \begin{center}
   \includegraphics[width=\textwidth]{resources/lumtauto_slogan.pdf}\\
  \end{center}
\end{figure}

\chapter{LANGuage $\rightarrow$ LUMTauto}
\label{SECLUMTAUTO}

\begin{transf}[The \textbf{Magical\footnote{\textbf{NOTE on Nomenclature:} Please realize that, although the author wishes to appeal to the special concept of ``Magick" attributed to Crowley\cite{crowley1929magick}, and yet, for purposes of keeping with the majority of scholarly writers such as Butler\cite{butler1952magic}, we shall mostly use the spelling ``Magic" or ``Magical" instead of ``magickal" unless where potential confusion warrants use of the Crowleyean form.} Language \texttt{Lumtauto}}]
\label{TRANSFLUMTAUTO}
If $\Theta^n$ is a sequence of $n > 0$ symbols (the original message) spanning the \textbf{Latin Alphabet} or the symbol set $\psi_{az}$, such that:

\begin{multline}
\label{EQLATINALPHABET}
\psi_{az} = \langle a, b, c, d, e, f, g, h, i, j, k, l, m, n, o, p, q, r, s, t, u, v, w, x, y, z \rangle: \invpi(\psi_{az}) = 26 \\ \quad \land \quad \Theta^n:\mathbb{N} \times \psi_{az}
\end{multline}

then the following transformation:

\begin{trans}
\label{TRANSLUMTAUTO}
$\Theta^n \xrightarrow{O_{lauto(\cdot)}} \Theta^* = \Omega^n;$\\
$\invpi(\Theta^n) = \invpi(\Theta^*) = \invpi(\Omega^n) = n$\\
$\land \quad \forall \theta_{i \in [1,n]} \in \Theta^n \quad \exists \omega_{j \in [1,n]} \in \Omega^n \quad \land \quad \invpi(\theta_i \in \psi(\Theta^n)) = \invpi(\omega_i \in \psi(\Omega^n)) = 1$\\
$\land \quad \forall \alpha \in \Theta^n: \invpi(\alpha \in \Theta^n) = f_\alpha \implies \alpha \in \Omega^n: \invpi(\alpha \in \Omega^n) = f_\alpha$\\
$\land \quad \overset{>}{\Theta^n} = \overset{>}{\Omega^n} \lor \overset{>}{\Theta^n} \neq \overset{>}{\Omega^n}$\\
$\land \quad \tilde{A}(\Theta^n \rightarrow \Omega^n) > 1 \qed$
\end{trans}

is guaranteed to always produce/generate a derivative message --- $\Theta^*$ that has the following properties:

{
\normalsize

\begin{multline}
\label{EQLUMTAUTO}
\forall \alpha \in \Theta^n \implies \beta \in \Omega^n \implies \begin{cases}
a \rightarrow u, & \\ \text{what happened: }\\ a \in \Theta^n \text{ became } u \in \Omega^n\\ \land \quad I(a,\Theta^n) = I(u,\Omega^n) = i \quad iff \quad \theta_{i=I(u,\Omega^n)} = a\\
b \rightarrow y, & \\ \text{what happened: }\\b \in \Theta^n \text{ became } y \in \Omega^n\\ \land \quad I(b,\Theta^n) = I(u,\Omega^n) = i \quad iff \quad \theta_{i=I(y,\Omega^n)} = b\\
c \rightarrow x,& \\
d \rightarrow w,& \\
e \rightarrow o,& \\
f \rightarrow f,& \\
g \rightarrow t,& \\
h \rightarrow s,& \\
i \rightarrow i,& \\
j \rightarrow q,& \\
k \rightarrow p,& \\
l \rightarrow l,& \\
m \rightarrow n,& \\
n \rightarrow m,& \\
o \rightarrow e,& \\
p \rightarrow k,& \\
q \rightarrow j,& \\
r \rightarrow r,& \\
s \rightarrow h,& \\
t \rightarrow g,& \\
u \rightarrow a,& \\
v \rightarrow v,& \\
w \rightarrow d,& \\
x \rightarrow c,& \\
y \rightarrow b,& \\
z \rightarrow z& \\ \text{what happened: }\\z \in \Theta^n \text{ became } z \in \Omega^n\\ \land \quad I(z,\Theta^n) = I(z,\Omega^n) = i \quad iff \quad \theta_{i=I(z,\Omega^n)} = z\\
\end{cases}
\end{multline}
$\qed$
}

And so that, the resultant [transformed] message, $\Theta^* = \Omega^n$, despite being the same exact length as the original message, is not exactly equivalent to it, and is an instance of text in the language \textbf{LUMTAUTO}.

\end{transf}


\textbf{\hyperref[EQLUMTAUTO]{Equation \ref{EQLUMTAUTO}}} explicitly specifies the magickal language \textbf{LUMTAUTO}, in which, if one had a starting message such as ``language", it is then transformed into an equivalently correct \textbf{magical message} ``lumtauto" via the transformation that a transformer such as \textbf{\hyperref[TRANSFLUMTAUTO]{Transformer \ref{TRANSFLUMTAUTO}}}, also encoded as the operator $O_{lauto}(\cdot) = O_{lauto}(m:\mathbb{N} \times \psi_{az})$ in \textbf{\hyperref[TRANSLUMTAUTO]{Transformation \ref{TRANSLUMTAUTO}}}. This transformation is certain, and always guaranteed to occur, if one applies that lumtauto transformer to any message.

\section{Relevance and MAGICKAL IMPACT of the magickal language LUMTAUTO}
\label{SECRELLUMTAUTO}

Of course, merely knowing that we have a mechanism by which any letter in the latin alphabet can be mapped to another letter of the same set in a certain way and not any other, might not immediately strike some people as either \textbf{odd, relevant or potent} --- especially for those who have never attempted to actually practice magick literally or rather practically (and not just in theory or just wishfully --- as most plebeians and the uninitiated would or might). But, for a good starter at how powerful this magickal language is, consider the impact the transformations we soon shall illustrate might have. However, first, let us first familiarize ourselves with the invariant properties of this language and its subtleties...


\section{The Four Subtleties of LUMTAUTO}
\label{SECPROPERTIESLUMTAUTO}

\begin{enumerate}
\item First, consider the impact of this language on the actor or agent --- the magician, attempting to process Lumtauto; perform or utter, read or evoke, chant or vibrate, charge or apply or even merely cast or visualize these messages --- they shall never be processing the exact original message!
\item Then consider that there is an impact, a transformation, on the immediate environment of whomever processes the resultant (transformed) messages.
\item The changes in semantics or meaning of the resultant message vis-a-vis the original and its intent is not a trivial one.
\item There is something peculiar about the structure of the resultant message vis-a-vis that of the original --- which, and provably so, has the \textbf{interesting properties} such as; all consonants in $\Theta^n$ are mapped to \textit{different}\footnote{\textit{Different} for consonants, because, we note that the \textbf{special quartet} $\{f, l, r, z \}$ is conserved under \textbf{\hyperref[TRANSLUMTAUTO]{Transformation \ref{TRANSLUMTAUTO}}}, while all other consonants in $\psi_{az}$ are not --- they are guaranteed to change under the lumtauto transformation.} consonants in $\Omega^n$, while all vowels in $\Theta^n$ are mapped to  \textbf{different} vowels in $\Omega^n$ \textbf{except} $i$.
\end{enumerate}




\begin{table}[htp]
  \begin{tabular}{|p{0.95\textwidth}} % Left border only
    \hline
\begin{figure}[H]
  \begin{center} % Rotate a single page of a PDF by 90 degrees
   \includegraphics[width=1\textwidth]{resources/scanned_example_magickal_art_sound_magick}\\
   \vspace{2em}
  \end{center}
\end{figure}\\
    \cline{1-1} % Bottom border only
  \end{tabular}
\end{table}


\section{FOUR Examples of Applying LUMTAUTO}
\label{SECEXAMPLESLUMTAUTO}

The four subtleties depicted in \textbf{\hyperref[SECPROPERTIESLUMTAUTO]{Section \ref{SECPROPERTIESLUMTAUTO}}}, shall be illustrated via the following four illustrative and relevant examples:


\begin{enumerate}
\item{\textbf{The Word ``language"}.
It becomes ``lumtauto" as we have already seen.
}

\item{\textbf{The \textbf{Magical Cogito Ergo Sum} Motto  ``I think, therefore I am"}. First attributed to classical philosopher \textbf{Rene Descartes} and discussed in \cite{Lutalo2025transpsy}, becomes the weird magical motto\footnote{For direct usage, prefer the formulation: {\LARGE \texttt{IGA SIMPE, GASOROFERO IUNA!}}} {\Large ``I gsimp, gsorofero I un"} under the lumtauto transformer (\textbf{\hyperref[TRANSFLUMTAUTO]{Transformer \ref{TRANSFLUMTAUTO}}}). 
}

\item{The \textbf{Magical Lord's Prayer} --- the \textit{Paternoster}, first attributed to god-man \textbf{Jesus Christ} of Nazareth and presented in \textbf{Mathew 6:9-13}\cite{newjerusalem1985}, here quoted verbatim as in the bible version \cite{newjerusalem1985}:\\


{\ttfamily

Our Father in heaven,
may your name be held holy,
your kingdom come,
your will be done,
on earth as in heaven.
Give us today our daily bread.
And forgive us our debts,
as we have forgiven those who are in debt to us.
And do not put us to the test,
but save us from the Evil One.

}

\vspace{2em}

becomes the [verbatim] and equivalent \textit{mystical} magical prayer\\

{\ttfamily

Ear Fugsor im souvom,
nub bear muno yo solw selb,
bear pimtwen xeno,
bear dill yo wemo,
em ourgs uh im souvom.
Tivo ah gewub ear wuilb yrouw.
Umw fertivo ah ear woygh,
uh do suvo fertivom gseho dse uro im woyg ge ah.
Umw we meg kag ah ge gso gohg,
yag huvo ah fren gso Ovil Emo.

}

\vspace{2em}

\item{The \textbf{Opening Sentence of the Bible}\footnote{For those familiar with the TEA programming language, note that this sentence has earlier on also shown up among the standard TEA program examples (check the TEA WEB IDE), when we notice the interesting result of applying the \textbf{modal sequence statistic} command: \texttt{u:} to it --- the result `` enthadgirIbGocvs." potentially telling that in transforming simple messages, one might many times uncover some very peculiar things!} --- also the first sentence in the \textbf{Pentateuch} and presented in \textbf{Genesis 1:1}\cite{newjerusalem1985}, here quoted verbatim as in the bible version \cite{newjerusalem1985}:\\


{\ttfamily

In the beginning God created heaven and earth.

}

\vspace{2em}

becomes the [verbatim] and corresponding magical utterance\\

{\ttfamily

Im gso yotimmimt Tew xrougow souvom umw ourgs.

}

\vspace{2em}

This particular example warrants some little more discussion concerning working with transformations of text for magical purposes. For example, \textbf{one of the major utilities of a magickal language like lumtauto would be to generate or construct arcane, charged, enchanting and perhaps obscure magical formulas, mantras and incantations} --- the stuff of serious, barbarous, ancient and seasoned magicians, sorcerers, wizards, witches, high-priests, spiritualists\footnote{Often needing strange utterances to compel or subdue spirits [using other spirits] or to command and operate spiritually using the strange \textit{tongues} of the Holy Spirit} and perhaps exorcists to name but a few. 

Concerning this then, it would be more useful --- especially for the case of rendering the outputs of such a transformer as \textbf{\hyperref[TRANSFLUMTAUTO]{Transformer \ref{TRANSFLUMTAUTO}}}, to be not only different from the original text, but that they are also rendered \sout{legible} readable and readily utterable so as to make them more useful in especially the sharing or reuse of standard and or, eternal spells, prayers and commands --- in a manner as how, using a normal natural language such as English, one can have the guarantee that the words and phrases thus encoded in a prayer, scripture or grimoire, shall stay readable and usable as originally intended across many generations of readers, users, students and practitioners. And thus, despite many phrases or words in English never requiring any modifications to their structure so as to make them readily and correctly pronounceable, and yet, for the outputs of the LUMTAUTO transformation thus presented, we find (especially out of experience, taste and necessity), that despite the succinctness and elegance of the transformer as presented either mathematically, or as a computer algorithm --- refer to \textbf{\hyperref[ALGLUMTAUTO]{Algorithm \ref{ALGLUMTAUTO}}}, we might want to sometimes \textit{further massage} the lumtauto transformer output so as to make them better \textbf{for verbal use}\footnote{Other than trying to make the resultant utterable, it might help to preserve the transformed message verbatim (perhaps in a form before any further changes other than just applying Lumtauto to the original input) --- this, for example, for other uses other than aural processing}. 

Thus, for example, the above transformed opening sentence of the scriptures\footnote{And it should not come as a surprise to those who are learned or who know why certain books have survived and been passed on from generation to generation for millennia.. such as the Bible and other Sacred books --- well, not just for their explicit wisdom and stories, but the fact that they contain many [sometimes] hidden nuggets of magical wisdom and wizardry if one knows what to look out for, or how to \textit{creatively use} the written word.} might become:

{\ttfamily
\LARGE

Imi geso yotimemimet Tewa xarougow souvom umwa ouregus.

}

%\vspace{2em}

And so that, people practicing magick --- especially \textbf{ritual or ceremonial magick}, where operators must speak, act, dance, shout, vibrate and do many expressive things during the operations or workings (so as to make them worthwhile and successful) --- so-called \textit{psycho-dramas}\cite{LaVey1969}\footnote{A term very popular with modern satanists such as \textbf{Anton Szandor LaVey}, the founder of the \textbf{Church of Satan} and author of \textbf{The Satanic Bible}\cite{LaVey1969}.}, can do so effectively, systematically and repeatably in a standardized way. 

%\vspace{2em}

And so, with sufficient practice and experimentation transforming words and phrases or texts into magical versions using LUMTAUTO, you shall learn how to be creative, and formulate very potent and interesting spells for almost any purpose!

}



}
\end{enumerate}


\section{The LUMTAUTO Algorithm}

\begin{table}[H]
  \begin{tabular}{|p{0.95\textwidth}} % Left border only
    \hline
    \begin{figure}[H]
      \centering
      \includegraphics[width=0.9\textwidth]{resources/lumtauto.jpg}\\
      \vspace{2em}
    \end{figure} \\
    \cline{1-1} % Bottom border only
  \end{tabular}
\end{table}

Using a formalism familiar to computer scientists, we shall here formally specify the LUMTAUTO algorithm using a method that could readily be translated into a compartible computer program so that an interest magician or researcher can then use the magickal language with any computer or programming language available to or familiar to them.

\vspace{2em}


\begin{alg}[The \textbf{LUMTAUTO Algorithm}: \texttt{lauto(msg\_in)}]
\label{ALGLUMTAUTO}
$ $\\
\begin{enumerate}
\item \textbf{GIVEN} source sequence (a plain-text message), \texttt{msg\_in} of length $n$.
\item \textbf{GIVEN} latin alphabet (as a list of letter symbols), \texttt{list\_a\_z} of 26 elements from $\psi_{az}$.
\item \textbf{GIVEN} list of vowels, \texttt{list\_vowels} = $\langle a,e,i,o,u \rangle$.

\item \textbf{COMPUTE} list of mirrored vowels, \texttt{list\_vowels\_mirror} = \texttt{reversed(list\_vowels)}.

\item{ \textbf{COMPUTE} first version of target alphabet, \texttt{list\_a\_z\_intermediate} thus: 

\begin{enumerate}
\item \textbf{INITIALIZE} \texttt{list\_a\_z\_intermediate} by cloning \texttt{list\_a\_z}.
\item \textbf{UPDATE} \texttt{list\_a\_z\_intermediate} by replacing each of its vowels, $v_i$ at index $i$ in that list, with another vowel character, $v_j$ from \texttt{list\_vowels\_mirror}, if the element $v_i$ is located at position $j$ in \texttt{list\_vowels}.
\end{enumerate}

}

\item{ \textbf{COMPUTE} final version of target alphabet, \texttt{list\_a\_z\_lumtauto} thus: 

\begin{enumerate}
\item \textbf{INITIALIZE} \texttt{list\_a\_z\_lumtauto} by cloning\\ \texttt{list\_a\_z\_intermediate}.
\item{ \textbf{UPDATE} \texttt{list\_a\_z\_lumtauto} thus:\\

\textbf{FOR} each element $v_i$ in \texttt{list\_a\_z\_lumtauto} between positions 1 to\\
 $\frac{1}{2} \times$ \texttt{length(list\_a\_z\_lumtauto)}:
\begin{enumerate}
\item \textbf{SET} element at $i$ as $v_i = $ \texttt{list\_a\_z\_lumtauto[i]}.
\item \textbf{SET} element $vm_i = $ \texttt{list\_a\_z\_lumtauto[n - i]}.
\item{ \textbf{IF} $v_i$ \textbf{NOT IN} \texttt{list\_vowels} \textbf{AND} $vm_i$ \textbf{NOT IN} \texttt{list\_vowels}: 

\begin{enumerate}
\item \textbf{SWAP} $v_i$ at position $i$ in \texttt{list\_a\_z\_lumtauto} with $vm_i$.
\item \textbf{SWAP} $vm_i$ at position $n - i$ in \texttt{list\_a\_z\_lumtauto} with $v_i$.
\end{enumerate}
}

\end{enumerate}

}
\end{enumerate}

}

\item{ \textbf{COMPUTE} resultant, transformed/translated message, \texttt{msg\_lumtauto} thus:

\begin{enumerate}
\item \textbf{INITIALIZE} \texttt{msg\_lumtauto} = \texttt{msg\_in}.
\item{ \textbf{FOR} each letter $c_i$ in \texttt{msg\_lumtauto} at position $i$:

\begin{enumerate}
\item \textbf{COMPUTE} the corresponding mirror letter position\\
$k = $ \texttt{index($c_i$ in list\_a\_z\_lumtauto)}.
\item{ \textbf{REPLACE} $c_i$ in \texttt{msg\_lumtauto} via the \textbf{OVERWRITE} operation:\\ \texttt{msg\_lumtauto[i] = list\_a\_z\_lumtauto[k]}
}
\end{enumerate}
}
\end{enumerate}

}

\item \textbf{RETURN} resultant sequence, \texttt{msg\_lumtauto}.
\end{enumerate}
$\qed$
\end{alg}


And as for how to go about testing or implementing this, note that an example implementation in the popular \textbf{PYTHON} computer programming language is shown in \textbf{\hyperref[SECLUMTAUTO_PY]{Section \ref{SECLUMTAUTO_PY}}}, and is based on \textbf{\hyperref[ALGLUMTAUTO]{Algorithm \ref{ALGLUMTAUTO}}}, itself based on \textbf{\hyperref[TRANSFLUMTAUTO]{Transformer \ref{TRANSFLUMTAUTO}}}, while a two-way, encode-decode version in the language this algorithm was first implemented (JavaScript) is shown in \textbf{\hyperref[SECLUMTAUTO_JS]{Section \ref{SECLUMTAUTO_JS}}}.



\section{The LUMTAUTO Algorithm in TEA}
\label{SECTEALUMTAUTO}

For purposes of helping you to immediately be able to study, apply and share this LUMTAUTO language and the associated text transformer, note that the following TEA\cite{cli_tttt}\cite{Lutalo2024TEATAZ} program is a first, and reliable, robust implementation of the transformer specified in \textbf{\hyperref[TRANSFLUMTAUTO]{Transformer \ref{TRANSFLUMTAUTO}}}.


 %\small
  \begin{tcolorbox}[teaterminalstyle, title=TEA Program: The LUMTAUTO Transformer, breakable]
  %\begin{lstlisting}[language=TEA, caption={TP C7}, label={LSTC7}, numbers=left]
  \begin{lstlisting}[language=TEA,breaklines=true]
i:{language} # given some message
v:vMESSAGE #store the original message

#COMPLETE LANGuage -> LUMTauto TRANSFORM
#lumtauto-TRANSFORM [lower-case]
#start transforming via the lumtauto cipher algorithm
r!:a:_%_ #U
r!:b:_%%_ #Y
r!:c:_%%%_ #X
r!:d:_%%%%_ #W
r!:e:_%%%%%_ #O
r!:f:_%%%%%%_ #F
r!:g:_%%%%%%%_ #T
r!:h:_%%%%%%%%_ #S
r!:i:_%%%%%%%%%_ #I
r!:j:_%%%%%%%%%%_ #Q
r!:k:_%%%%%%%%%%%_ #P
r!:l:_%%%%%%%%%%%%_ #L
r!:m:_%%%%%%%%%%%%%_ #N
r!:n:m
r!:o:e
r!:p:k
r!:q:j
r!:r:r
r!:s:h
r!:t:g
r!:u:a
r!:v:v
r!:w:d
r!:x:c
r!:y:b
r!:z:z
#complete the transform
r!:_%_:u
r!:_%%_:y
r!:_%%%_:x
r!:_%%%%_:w
r!:_%%%%%_:o
r!:_%%%%%%_:f
r!:_%%%%%%%_:t
r!:_%%%%%%%%_:s
r!:_%%%%%%%%%_:i
r!:_%%%%%%%%%%_:q
r!:_%%%%%%%%%%%_:p
r!:_%%%%%%%%%%%%_:l
r!:_%%%%%%%%%%%%%_:n
#FINISHED: for lower-case

#j:lFINISHED

#COMPLETE LUMTAUTO TRANSFORM [for uppercase]
#LUMTAUTO-TRANSFORM [upper-case]
#start transforming via the LUMTAUTO cipher algorithm
r!:A:_%_ #U
r!:B:_%%_ #Y
r!:C:_%%%_ #X
r!:D:_%%%%_ #W
r!:E:_%%%%%_ #O
r!:F:_%%%%%%_ #F
r!:G:_%%%%%%%_ #T
r!:H:_%%%%%%%%_ #S
r!:I:_%%%%%%%%%_ #I
r!:J:_%%%%%%%%%%_ #Q
r!:K:_%%%%%%%%%%%_ #P
r!:L:_%%%%%%%%%%%%_ #L
r!:M:_%%%%%%%%%%%%%_ #N
r!:N:M
r!:O:E
r!:P:K
r!:Q:J
r!:R:R
r!:S:H
r!:T:G
r!:U:A
r!:V:V
r!:W:D
r!:X:C
r!:Y:B
r!:Z:Z
#complete the transform
r!:_%_:U
r!:_%%_:Y
r!:_%%%_:X
r!:_%%%%_:W
r!:_%%%%%_:O
r!:_%%%%%%_:F
r!:_%%%%%%%_:T
r!:_%%%%%%%%_:S
r!:_%%%%%%%%%_:I
r!:_%%%%%%%%%%_:Q
r!:_%%%%%%%%%%%_:P
r!:_%%%%%%%%%%%%_:L
r!:_%%%%%%%%%%%%%_:N
#FINISHED: for upper-case

#Complete Original Message NOW Transformed

#then store transformed message :)
v:vTRANSFORMED_MESSAGE #such as "lumtauto"
   \end{lstlisting}
  \end{tcolorbox}
    \captionof{figure}{TEA Program: The LANGuage to LUMTauto transformer}
  \label{FIGLUMTAUTOTEACODE}

\vspace{2em}

\textbf{\hyperref[FIGLUMTAUTOTEACODE]{Figure \ref{FIGLUMTAUTOTEACODE}}} is the \textbf{source-code} of the non-interactive TEA program implementing this algorithm, and which, when actually cleaned of comments and MINIFIED and rendered interactive, would be the program depicted in  \textbf{\hyperref[FIGLUMTAUTOTEACODE_CLEAN]{Figure \ref{FIGLUMTAUTOTEACODE_CLEAN}}}. However, and in case one wishes to look at the code, modify or run \textbf{an interactive version} of it like on the Linux, Unix, Windows or MAC OS command-line or on the WEB, the most recent version should be what you might find or run directly and live via:
  
  
\vspace{2em}

 \url{https://tea.nuchwezi.com/?i=put+your+message+here&fc=https://gist.githubusercontent.com/mcnemesis/ae9d6226d49f5a8601a84241a08f07c8/raw/lumtauto_language_transformer.tea}

\vspace{1em}


\textbf{ALTERNATIVELY} just use the short-link: \url{https://bit.ly/lumtauto}

\vspace{2em}


 %\small
  \begin{tcolorbox}[teaterminalstyle, title=CLEAN TEA Program: The LUMTAUTO Transformer, breakable]
  %\begin{lstlisting}[language=TEA, caption={TP C7}, label={LSTC7}, numbers=left]
  \begin{lstlisting}[language=TEA,breaklines=true]
f!:^$:lDONTPROMPT|i!:{Enter Message to be Encoded:}|i*:|i:{language}|l:lDONTPROMPT|v:vMESSAGE|r!:a:_%_|r!:b:_%%_|r!:c:_%%%_|r!:d:_%%%%_|r!:e:_%%%%%_|r!:f:_%%%%%%_|r!:g:_%%%%%%%_|r!:h:_%%%%%%%%_|r!:i:_%%%%%%%%%_|r!:j:_%%%%%%%%%%_|r!:k:_%%%%%%%%%%%_|r!:l:_%%%%%%%%%%%%_|r!:m:_%%%%%%%%%%%%%_|r!:n:m|r!:o:e|r!:p:k|r!:q:j|r!:r:r|r!:s:h|r!:t:g|r!:u:a|r!:v:v|r!:w:d|r!:x:c|r!:y:b|r!:z:z|r!:_%_:u|r!:_%%_:y|r!:_%%%_:x|r!:_%%%%_:w|r!:_%%%%%_:o|r!:_%%%%%%_:f|r!:_%%%%%%%_:t|r!:_%%%%%%%%_:s|r!:_%%%%%%%%%_:i|r!:_%%%%%%%%%%_:q|r!:_%%%%%%%%%%%_:p|r!:_%%%%%%%%%%%%_:l|r!:_%%%%%%%%%%%%%_:n|r!:A:_%_|r!:B:_%%_|r!:C:_%%%_|r!:D:_%%%%_|r!:E:_%%%%%_|r!:F:_%%%%%%_|r!:G:_%%%%%%%_|r!:H:_%%%%%%%%_|r!:I:_%%%%%%%%%_|r!:J:_%%%%%%%%%%_|r!:K:_%%%%%%%%%%%_|r!:L:_%%%%%%%%%%%%_|r!:M:_%%%%%%%%%%%%%_|r!:N:M|r!:O:E|r!:P:K|r!:Q:J|r!:R:R|r!:S:H|r!:T:G|r!:U:A|r!:V:V|r!:W:D|r!:X:C|r!:Y:B|r!:Z:Z|r!:_%_:U|r!:_%%_:Y|r!:_%%%_:X|r!:_%%%%_:W|r!:_%%%%%_:O|r!:_%%%%%%_:F|r!:_%%%%%%%_:T|r!:_%%%%%%%%_:S|r!:_%%%%%%%%%_:I|r!:_%%%%%%%%%%_:Q|r!:_%%%%%%%%%%%_:P|r!:_%%%%%%%%%%%%_:L|r!:_%%%%%%%%%%%%%_:N|l:lFINISHED|v:vTRANSFORMED_MESSAGE|i!:{In LUMTAUTO
}|x*!: vMESSAGE|x!: {
-- becomes --
}|x*!: vTRANSFORMED_MESSAGE|i*:|y:vTRANSFORMED_MESSAGE
   \end{lstlisting}
  \end{tcolorbox}
    \captionof{figure}{MINIFIED TEA Program: The basic LUMTAUTO transformer program source-code}
  \label{FIGLUMTAUTOTEACODE_CLEAN}
  
  
 \section{The Background of LUMTAUTO}


This language and the associated algorithm was first developed at Nuchwezi Esoteric School around February 2015 (almost \textbf{15 years ago!}) as per the official repository of the associated original project --- \textbf{Font Crypto}\cite{nuchweziCrypto} --- an online/web app project with the source-code plus several other occult ciphers, text-transformation programs and magickal languages (including text to \textbf{Hieroglyphics, text to Greek, Cuneiform and Masonic cipher} among others) we explored back then\footnote{Refer to \url{https://github.com/NuChwezi/font-crypto}}\cite{nuchweziCrypto}.

\vspace{1em}

Back then, Nuchwezi as a technology startup and research community was just about a year old (since its founding in July 2014). And so, given what we have seen of the latest way that this language is approached, utilized and formalized, such as in \textbf{\hyperref[TRANSFLUMTAUTO]{Transformer \ref{TRANSFLUMTAUTO}}} and actual modern code for the associated transformer program as in \textbf{\hyperref[SECTEALUMTAUTO]{Section \ref{SECTEALUMTAUTO}}}, surely, this language has now reached some appreciable maturity and can be properly applied in actual workings and research.

\vspace{1em}

Talking of its applications though, note that the first formal mention and heavy use of LUMTAUTO was in the literary fiction work --- the novel, \textbf{Shrines of The Free Men}\cite{shrinesjwl}, first made public around 2018. In that book, a stealth link to the above mentioned online tool is shared, as part of a snippet of a text-chat session between various students and netizens in a school alumni social media community as depicted on page 24 of \cite{shrinesjwl}. It was shared back then as the link:


\vspace{1em}

 \url{http://tiny.cc/cry_dept#scrt}

\vspace{1em}

However, that link does not seem to be working anymore, and instead, one would better access the \textbf{original JavaScript implementation} of LUMTAUTO, together with the other ciphers and magical languages developed back then, via the up-to-date link:



\vspace{1em}

 \url{http://crypto.nuchwezi.com}

\vspace{1em}

Also, and more concerning how it was used in that novel, note that there are several parts in that book, where the author deliberately chose to present certain text --- especially specially formatted spells and incantations, via projections from the original plain versions into corresponding versions in LUMTAUTO. We shall call out just a few examples, showing the lumtauto that was depicted in the book Vs the plain text version one might decipher from them using a two-way encoder/transformer of LUMTAUTO such as the original JavaScript transformer program could.






\begin{table}[H]
  \begin{tabular}{|p{0.95\textwidth}} % Left border only
    \hline
    \begin{figure}[H]
      \centering
      \includegraphics[width=0.9\textwidth]{resources/illustration_trudy_olga_spirit_fight_2.jpg}\\
  \caption{A Spiritual Battle with a Malevolent Spirit}
      \label{FIGOLGA}
    \end{figure} \\
    \cline{1-1} % Bottom border only
  \end{tabular}
\end{table}



The first example use case we are to look into occurs on on page 166 of \cite{shrinesjwl} --- it is the case of a ``powerful mantra" --- in the context of the story, being employed by Trudy inside a lucid dream, to banish a feminine spirit that had been impersonating her friend Olga, and which had suddenly turned malevolent:\\



%\begin{figure}[H]
  \begin{tcbverbatim}[title=A Mantra from the novel ``Shrines of The Free Men"]
I yimw bea im gso muno ef Rasumtu umw ull gso hkirigh ef mugaro umw
nb umxohgerh soro krohomg.
I yimw bea roturwlohh ef dsogsor bea uro soro im hkirig, nimw er yewb.
I yimw bea umw xag fren bea ull hearxoh ef kedor gsug uro Uminahgimt
bea ritsg med, im gso muno ef gso Nehg Kedorfal Rasumtu!
I sarl bea imge um ogormul krihem, I xag bea ge hsrowh, I hannem
gsamwor umw firo gsug ough ug gso yewb, nimw umw hkirig ge hgripo
umw wokrivo bea ef ull bear hgromtgs umw soulgs.
I hannem gso samtriohg ef soll'h hkirigh ge fouhg em bea.
I yimw umw wohgreb bea im Tew'h nehg selb, ampmedm munoh!
Ritsg Soro, Ritsg Med!
Yo Temo, Yo Temo fren gsih kluxo umw gino Ritsg Med!
Im gso muno ef gso Imfimigo, Uyhelago umw Nbhgorieah Tew!
  \end{tcbverbatim}
%\end{figure}

\vspace{2em}

Which, if we run it backwards --- basically, reverse/decipher from LUMTAUTO into plain text (without attempting to make any manual adjustments and assuming the presented text is proper LUMTAUTO text) --- something we might do well via the online \textbf{Crypto} tool shared above, we would then get:

\vspace{2em}


{\ttfamily

I bind you in the name of Ruhanga and all the spirits of nature and
my ancestors here present.
I bind you regardless of whether you are here in spirit, mind or body.
I bind you and cut from you all sources of power that are Animusting
you right now, in the name of the Most Powerful Ruhanga!
I hurl you into an eternal prison, I cut you to shreds, I summon
thunder and fire that eats at the body, mind and spirit to strike
and deprive you of all your strength and health.
I summon the hungriest of hell's spirits to feast on you.
I bind and destroy you in God's most holy, unknown names!
Right Here, Right Now!
Be Gone, Be Gone from this place and time Right Now!
In the name of the Infinite, Absolute and Mysterious God!

}

\vspace{2em}

Which perhaps needs no explanations, apart from the word ``Animusting" potentially having been modified in the original LUMTAUTO version, so that it might possibly have been ``Animating", which would have been \textit{Uminugimt} in Lumtauto\footnote{Further support for this analysis is by the fact that in that book, we later come across creative use of two spirit names --- ``Uminu" and ``Uminah", corresponding to the \textit{anima} and the other the \textit{animus}\cite{jung1964symbols} of the protagonist \textbf{Ignatius Irumba} --- also refer to page 353 of \cite{shrinesjwl}.}.


\vspace{2em}


And so, with what we know now, a better, more practical rendition of this mantra\footnote{\textbf{mantra}(n): (Sanskrit) literally a `sacred utterance' in Vedism; one of a collection of orally transmitted poetic hymns\cite{wordweb_assistant}} and \textbf{exorcising formula} would be the following universally useful formulation:\\


  \begin{tcbverbatim}[title=A Universal EXORCISM Mantra against Malevolence]
Ayimwe bea imi geso muno, efa, RU USU OSOIOS BOSOHASAU!

Umwa ulli, geso hakirigaha, efe mugaro umwa!

Nuba umixo, higerah soro karohomiga.

Iyimwe bea, roturwa-loheha efa disogesora
 
Bea uro soro ima hakiriga, nimwe era yewabu.

Iyimwe bea umwa xagi, frena bea ulla hearaxohu,
 
Efa kedora, Efa kedora!

Gisuga uro uminugimate, bea ritasagi medu!
 
Ima geso, muno efe gaso!

Nehagi Kedora-falu, RU USU OSOIOS BOSOHASAU!

Isarale bea, image uma ogoremule kirihema,
Xagi bea; ge hasoro-wahe, hannemi
Gasamwora Gasamwora!

Umwe firo gasuga oua gahe!
 
Uga, Uga, Uga! Geso yewabu,
Nimwe umwa hikiriga.
 
Ge hagripo, Ge umwa wokarivo bea efa ulla beara

Hagiro mitugis umwa! Soulu gase ula ro uwebu!
 
Ihanneme geso samite riohaga efe!
Solulah Solulah Soluleh!
 
Hakiri gehe, Hakiri geha!
Gege fou Gege hagi, Gege eme bea.

Iyimwa umwa, woha gureba
Belebele bea ima Tewha!
 
Neha neha gise laaba! 
Amapenduma munoh munoh!

Rita sagu, Soro, Ritasage Medu!
Yo Tema, Yo Temu, fareni gisiha kalu exa!
 
Umwe gano-gine ARARITA! Sagu MEDUSA!

Ime gaso muno efe gaso Imi-fifi-migo..
 
Uya, hela, HERA! Go! Umwa Nabahigi!
Rie AHA Lerwa EFE Ulla Lerwah!
He Yo Igu! Unoma Unoma RU!
  \end{tcbverbatim}

\vspace{2em}

\begin{table}[H]
  \begin{tabular}{|p{0.95\textwidth}} % Left border only
    \hline
    \begin{figure}[H]
      \centering
      \includegraphics[width=0.9\textwidth]{resources/illustration_trudy_olga_spirit_fight.jpg}\\
  \caption{Warding Off Undesirable Spirits}
      \label{FIGFIGHT2}
    \end{figure} \\
    \cline{1-1} % Bottom border only
  \end{tabular}
\end{table}


\subsection{The ORIGINAL LUMTAUTO Algorithm in JAVASCRIPT}
\label{SECLUMTAUTO_JS}

Note that, for especially background reasons, and for those who wish to replicate the original implementation, note that the following JavaScript program would suffice to reflect the active and two-way (encode/decode) implementation of  \textbf{\hyperref[ALGLUMTAUTO]{Algorithm \ref{ALGLUMTAUTO}}} as currently found in \cite{nuchweziCrypto}:\\


 %\small
  \begin{tcolorbox}[ title=JavaScript Program: DECODE-ENCODE LUMTAUTO transformer function, breakable]
  \begin{lstlisting}[language=JavaScript,breaklines=true]
function transform_chaos(src,target,reverse){
    var v_alphabet = 'aeiou'.split('');
    var v_al_mirror = 'uoiea'.split('');
    var vu_alphabet = 'AEIOU'.split('');
    var vu_al_mirror = 'UOIEA'.split('');
    var l_alphabet = []; var l_al_mirror = [];
    var u_alphabet = []; var u_al_mirror = [];

    for(var l='a'.charCodeAt(0); l <= 'z'.charCodeAt(0); l++){
        var s = String.fromCharCode(l);
        var S = s.toUpperCase();
        if(v_alphabet.indexOf(s) >= 0) {// a vowel
            l_alphabet.push(s);
            u_alphabet.push(S);
            var _s = v_al_mirror[v_alphabet.indexOf(s)];
            var _S = _s.toUpperCase();
            l_al_mirror.push(_s);
            u_al_mirror.push(_S);
        }else {
            l_alphabet.push(s);
            l_al_mirror.push(s);
            u_alphabet.push(S);
            u_al_mirror.push(S);
        }
    }

    for(var i=0; i < l_al_mirror.length * 0.5; i++){
        var t = l_al_mirror[i];
        var _t = l_al_mirror[l_al_mirror.length-1-i];
        if((v_alphabet.indexOf(t) < 0 ) && (v_alphabet.indexOf(_t) < 0 )) {// skip vowel
            l_al_mirror[i] = _t;
            l_al_mirror[l_al_mirror.length-1-i] = t;

            var T = u_al_mirror[i];
            u_al_mirror[i] = u_al_mirror[u_al_mirror.length-1-i];
            u_al_mirror[u_al_mirror.length-1-i] = T;
        }
    }

    var _in = $(src).val();
    var _out = _in.split("").map(function(c){ 
        if(reverse){
            var _c = u_al_mirror.indexOf(c) >= 0 ? u_al_mirror[u_alphabet.indexOf(c)] : (l_al_mirror.indexOf(c) >= 0 ? l_al_mirror[l_alphabet.indexOf(c)] : c); 
            return _c;
        }else{
            var _c = l_al_mirror.indexOf(c) >= 0 ? l_al_mirror[l_alphabet.indexOf(c)] : (u_al_mirror.indexOf(c) >= 0 ? u_al_mirror[u_alphabet.indexOf(c)] : c); 
            return _c;
        }
    }).join("");
    
    $(target).val(_out);
}
   \end{lstlisting}
  \end{tcolorbox}
    \captionof{figure}{JavaScript Program: DECODE-ENCODE LUMTAUTO transformer function}
  \label{FIGLUMTAUTOJSCODE}

\vspace{2em}


\subsection{The LUMTAUTO Algorithm in PYTHON}
\label{SECLUMTAUTO_PY}


The up-to-date (one-way/encode) implementation of LUMTAUTO in Python is as follows:

 %\small
  \begin{tcolorbox}[ title=PYTHON Program: ENCODE LUMTAUTO transformer program, breakable]
  \begin{lstlisting}[language=Python,breaklines=true]
#!/usr/bin/env python3
def lauto(msg_in):
    a_z = ['a','b','c','d','e','f','g','h','i','j','k','l','m','n','o','p','q','r','s','t','u','v','w','x','y','z']
    vowels = ['a','e','i','o','u']

    #first, compute mirror of vowels
    vowels_mirror = list(reversed(vowels))

    #compute a new alphabet with all vowels replaced by their mirrors
    a_z_vm = [c if not c in vowels else vowels_mirror[vowels.index(c)] for c in a_z]

    #next compute alphabet mirror, by replacing all non-vowels in a_z_vm with their mirror element from the intermediate alphabet
    a_z_mirror = a_z_vm #first, make a copy

    #thus, we update the intermediate alphabet thus:
    for i in range(int(len(a_z_mirror) * 0.5)):
        t = a_z_mirror[i]
        _t = a_z_mirror[len(a_z_mirror) - 1 - i]
        # ensure neither t nor _t are vowels:
        if not(t in vowels) and not(_t in vowels):
            #swap the two opposite letters
            a_z_mirror[i] = _t
            a_z_mirror[len(a_z_mirror) - 1 - i] = t


    msg_lumtauto = msg_in.lower() # only work on lowercase messages for now
    msg_lumtauto = "".join([a_z_mirror[a_z.index(l)] for l in msg_lumtauto])
    return msg_lumtauto

print(lauto("LANGuage")) #prints 'lumtauto'
   \end{lstlisting}
  \end{tcolorbox}
    \captionof{figure}{PYTHON Program: ENCODE LUMTAUTO transformer program}
  \label{FIGLUMTAUTOPYCODE}




\section{More LUMTAUTO Spells and Mantras from ``Shrines of The Free Men"\cite{shrinesjwl}}

Before we continue, note that, considering the kind of spells, mantras, conjurations and such that we are dealing with here and in most of this grimoire, the fact that we somewhat design or re-format our formulas to make them palatable to not just the tongue, but also the mind (and especially the sub-conscious), can be best understood when we reflect on the advise given by \textbf{Janet and Stewart Farrar} in their essential book on spells:

\noindent
\begin{minipage}{1\textwidth}
\vspace{1em}
\begin{quotation}
\noindent {\ttfamily

A traditional spell is often rather like poetry, in that the power can manifest even though the reasons for its working remain a mystery. It is best to keep a cautiously open mind. 

}
\hspace*{\fill} --- \textbf{Spells and How They Work}, \textit{1990}, Janet and Stewart Farrar\cite{farrar1990spells}
\end{quotation}
\vspace{1em}
\end{minipage}

In fact, that book does remind and compel us to not readily discard or take for granted, spells we find well preserved in spell books, grimoires, prayer-books and the kind often passed down from generation to generation via oral tradition alone. This, especially because we might not always understand why the spell works, and yet it actually does work! See that they tell us:


\noindent
\begin{minipage}{1\textwidth}
\vspace{1em}
\begin{quotation}
\noindent {\ttfamily

The formulae, actions and objects or substances they involve may have many different purposes. Sometimes they may be, frankly, just to impress the uninitiated. Sometimes they may be a distorted memory of something which originally had a physically, psychologically or psychically practical reason. And even oftener they may be calling upon a symbolism, whether conscious or unconscious, which aids the inter-level communication... What one should not do is to dismiss them out of hand just because the reason for their form is not immediately obvious to rational analysis. You may find that they work though you don't know why --- that the power of their symbology is buried too deeply in your personal Unconscious (or even in the Collective Unconscious) for your awareness to find it.

}
\hspace*{\fill} --- \textbf{Spells and How They Work}, \textit{1990}, Janet and Stewart Farrar\cite{farrar1990spells}
\end{quotation}
\vspace{1em}
\end{minipage}



The next spell we are going to consider, is that regarding using magick to get visions concerning events or people not immediately within physical reach --- kind of like \textbf{remote viewing} or a kind of \textbf{clairvoyance} --- powers normally exercised by spirit mediums, seers, scryers and people with their \textbf{third eye} open and working actively.

\begin{table}[H]
  \begin{tabular}{|p{0.95\textwidth}} % Left border only
    \hline
    \begin{figure}[H]
      \centering
      \includegraphics[width=0.9\textwidth]{resources/conjurer.jpg}\\
  \caption{Conjuring Visions}
      \label{FIGCONJ1}
    \end{figure} \\
    \cline{1-1} % Bottom border only
  \end{tabular}
\end{table}


It is demonstrated on page 171 of \cite{shrinesjwl} and the equivalent literal wording would be:


%\begin{figure}[H]
  \begin{tcbverbatim}[title=A Conjuration for Visions from the novel ``Shrines of The Free Men"]
Mwimuke inywe abalibata omumuro
Mwimuke inywe abarora omukizima
Mwimuke inywe abarora omumbeho
Mwimuke inywe abarora emizimu
Mwimuke inywe abarubata omumwanya
Mwijje munyoleke omwana wange Nyamwezi.
  \end{tcbverbatim}
%\end{figure}

\vspace{2em}

However, that particular spell was for conjuring visions and images to help one old woman --- a witch and grandmother of a girl child that had gone missing in the deep of night while she slept. So, for purposes of rendering it generally usable, and applicable to any matter concerning wanting to solicit for visions from the spiritual realm, we instead have the following conjuration spell:


{\LARGE

 \begin{tcbverbatim}[title=A Universal Vision CONJURATION Spell Leveraging Familiar Spirits]
Nana dinape imabado uyuliyugu enanare
Nana dinapo imabado uyureru enapizinu
Nana dinape imabado uyureru enanyose
Nana dinapo imabado uyureru onizina
Nana dinape dinapo inapo imabado uyura-yugu, enanu-dumba
Nana dinapo imabado uyure-lara suru!
Nedoqqo, namebele ape pima padomawa padogoti iraze!

________ napi mabelo papa bobo sugi suma, enyahi, huma munura apu!
  \end{tcbverbatim}
  }
 

\vspace{2em}

\begin{table}[H]
  \begin{tabular}{|p{0.95\textwidth}} % Left border only
    \hline
    \begin{figure}[H]
      \centering
      \includegraphics[width=0.9\textwidth]{resources/conjurer_4.jpg}\\
  \caption{Conjuring Visions With an Assistant Medium or Seer}
      \label{FIGCONJ1}
    \end{figure} \\
    \cline{1-1} % Bottom border only
  \end{tabular}
\end{table}


In that spell, it shall be useful to utter some details about the actual vision being sought, in the space left blank (with blank line) in the conjuration formula. Also, in keeping with the aesthetics and method of the original source of that spell\cite{shrinesjwl}, one might want to have the following as part of their operation and operating space:

\begin{enumerate}
\item An altered state of consciousness --- performing the spell itself could cause a change in consciousness on the part of the magician, but it might be better sharpened and enhanced if other methods of inducing trance or gnosis were employed as earlier steps --- for example, meditation, vigorous or shamanic dance, intoxication with alcohol or a psycho-active substance, etc.
\item A source of creative power --- depending on taste and tradition, but some good alternatives might be proximity to a water source or focus on some reflective substance --- a mirror, water in a dark bowl, gazing at a lake, clouds, TV static/white-noise, etc.
\item A transmutation power source --- especially via the element of fire; could be a candle, a fireplace, a red cloth or blood.
\item An anchorage point for the spiritual entities inducing the visions --- could be a skull of a dead but familiar or benevolent person such as an ancestor or partner, could be just the bones of a dead animal (especially those of ones familiars or totem animal), could be belongings of such an entity or person or a symbolic expression of any such spiritual powers --- sigilized names of angels or some god, a sacred symbol of a Godform, an identifying sigil for a legion, etc.
\item A means to copy, capture or record any such visions once they are obtained --- might be as simple as having a blank piece of paper and a pen or pencil, might be a camera if the visions are to be projected or sourced external to the operator, might be a blank canvas, paints and a brush if the visions are to be obtained as via an invocation or artistic mediumship, etc.
\end{enumerate}



\subsubsection{Inducing Trance While Invoking Solar Deities}


Often times, there are moments during a ritual, when the operator needs the support or energy of his working to be boosted further, leveraging the powers of a legion or an entire army! A well coordinated chant, especially by the onlookers or subordinate attendants at a group ritual or communal working are what can readily make this proceed fruitfully --- essentially, while the main operator does their thing (perhaps performing a spell, casting or channeling some powers), the others would not merely stand or sit watching, but would join in, and sing or vibrate a particular power-evoking mantra, and if they do it in sync and with sufficient vigor, the entire space, room or the magic circle, would suddenly be filled with sufficient energy to help the magician effectively execute their operation.

\vspace{2em}

We see something of the sort happening, on page 365 of \cite{shrinesjwl}, when four spiritual beings working inside a pristine occult temple deep in the astral, chant continuously, the following LUMTAUTO mantra for raising power and invoking a solar deity...

{\LARGE

 \begin{tcbverbatim}[title=RA Invocation Mantra]
Epatiziydo Piguru! Piguru!
Epatiziydo Piguru! Piguru!
Epatiziydo Piguru! Piguru!
Epatiziydo Piguru! Piguru!
Epatiziydo Piguru! Piguru!
Epatiziydo Piguru! Piguru!
  \end{tcbverbatim}
  
}
  
  
So, while they repetitively vibrate that formula... essentially recanting the phrase ``Epatizido Piguru! Piguru!" again and again... things start happening... and the operator, but also the members participating, shall experience something of a solar deity... most likely \textbf{AMUN RA} or his attributes\footnote{It shall help some readers to realize that the preferred name ``Amun Ra" might vary [significantly] from what some authorities and especially Egyptologists might prefer --- for example, \textbf{Paul Johnson} prefers the phrasing ``Amun-Re"\cite{johnson2000civilization}. However, in practice, such as during invocations and such, one might default to which version resonates best with them.}, manifesting in the air all about them. It can be a great companion mantra to casting out evil spirits, performing healing, dispatching intents or charging tools.

\vspace{2em}

It is left as an exercise to the reader, to find out what exactly those words means. The necessary tools have already been shared.




\begin{figure}[H]
  \begin{center}
   \includegraphics[scale=0.9]{resources/maiu_sigil.pdf}\\
   \caption{The Mysteries: \textbf{MAIU}: Mysterious Absolute Infinite Unkowable --- invokable: \textbf{IEAU}}
  \label{FIG2}
  \end{center}
\end{figure}


\chapter{The Grand Myrrh: \textbf{MYRRH LANGUAGE}}
\label{SECMYRRH}


\begin{figure}[H]
  \begin{center}
   \includegraphics[width=\textwidth]{resources/myrrh_slogan.pdf}\\
  \end{center}
\end{figure}


Without wasting time, first, consider the following TEA program:\\

{
\normalsize

 %\small
  \begin{tcolorbox}[teaterminalstyle, title=TEA Program: The Grand Myrrh Transform, breakable]
  %\begin{lstlisting}[language=TEA, caption={TP C7}, label={LSTC7}, numbers=left]
  \begin{lstlisting}[language=TEA,breaklines=true]
m!:
r!:y:yua
r!:ht:th
r!:dn:dun
r!:tn:tan
r!:rp:rupa
r!:sy:s y
r!:tc:tauch
r!: :a 
r!:[gG]:su
r!:[dD]:v
z:
   \end{lstlisting}
  \end{tcolorbox}
    \captionof{figure}{The Grand Myrrh Transform}
  \label{FIGTEAMYRRH}

}

\vspace{2em}


That simple TEA program\footnote{Much more succinct and yet no less powerful than what we saw with LUMTAUTO in its simplest form --- refer to \textbf{\hyperref[FIGLUMTAUTOTEACODE_CLEAN]{Figure \ref{FIGLUMTAUTOTEACODE_CLEAN}}}} has \textbf{a very very powerful application} in Occult Science and Esotericism. Before we move on, note that it is all we need to specify \textbf{the magickal language MYRRH}\footnote{Pronounced ``mira", and also sometimes to be referred to as ``Grand Myrrh", because of the original TEA program it was based on --- early in the days (circa 2021 or so) when TEA was still just a small and niche android app (TTTT) with a very limited instruction set\cite{lutalo2024tea}.}. 

\vspace{2em}

The \textbf{MYRRH Language} can take any ordinary word, phrase or text, and apply some subtle transforms to it that automagically turn it into something of \textbf{the sacred words, words of power or magical spells!}


\vspace{2em}


Take for example, a basic, very common phrase --- actually, perhaps one of the most \textit{religious test-cases} in Computer Science and Software Engineering ever\footnote{If for no other reason, because all students and initiates of computer programming must at some point have to test their teeth at a new or unfamiliar language, and most likely, their teacher or initiator shall task them to write the so-called ``Hello World Program". But also, it is one of the best test-cases for evaluating or comparing [computer] languages\cite{lutalo2020dnap}.} --- such as: 

\vspace{2em}


{\ttfamily

Hello World

}


\vspace{2em}


Transformed using the MYRRH Transform depicted in \textbf{\hyperref[FIGTEAMYRRH]{Figure \ref{FIGTEAMYRRH}}}, it becomes as an occult conjuration...

\vspace{2em}

{\ttfamily
\LARGE

avalaraoawa aoalalaeaha

}

\vspace{2em}



Clearly then, the Grand Myrrh language has immediate use as an effective means of preparing payloads for magical incantations in say a ritual, as part of occult mantras or spell casting. In fact, two more observations might help drive the point home...\\

\begin{enumerate}
\item Of all the nifty little spell-preparation programs we have ever attempted to develop at NES, none has ever been so confounding as the Myrrh Transform! First, because, the way it was discovered... somewhat trial and error, but also creative and perhaps inspired hacks while exploring and studying several traditional and classic spells and grimoires. The original version merely had the name ``myrrh transform", mostly because of the opening TEA instruction required to make the entire transform work --- the \texttt{m:} TEA primitive command also formally known as \textbf{MIRROR} command\cite{Lutalo2024TEATAZ}. However, like the mind-altering, sacred and rare magical incense, myrrh\footnote{Gum myrrh was one of the gifts brought by the Magi in the biblical nativity story\cite{wordweb_assistant}.}, that was offered Jesus and his family at his birth in Bethlehem (\textbf{Mathew 2:12}\cite{newjerusalem1985}), how the discovery of this precious transformer program occurred is but a miracle! We might as well consider it a somewhat divine gift to the illuminates of Nuchwezi... perhaps by angels we don't clearly know yet!

\item{Especially thanks to TEA, it is short, concise and just works! Note that, as we observed and recommended in \textbf{\hyperref[SECEXAMPLESLUMTAUTO]{Section \ref{SECEXAMPLESLUMTAUTO}}} concerning payloads developed using the LUMTAUTO transformer, one might often find that they need to somewhat massage or further edit the resultant payload before it can be adequate for \textit{smooth use} in say a magickal operation. Take for example, the case of ``Hello World" as we have just encountered; if we apply \textbf{\hyperref[TRANSFLUMTAUTO]{Transformer \ref{TRANSFLUMTAUTO}}} to it, it shall result in the phrase:\\

{\centering
\ttfamily

Solle Derlw

}

\vspace{2em}
Which, and clearly, is no match to what we can accomplish with the Grand Myrrh Transformer:

{\centering
\ttfamily

avalaraoawa aoalalaeaha

}

\vspace{2em}

 Of course, experimentation and experience shall prove that each language has its place and purpose, depending on context or problem at hand, but in terms of usability, and with the examples we are going to consider, one shall come to surely love the MYRRH language!

}
\end{enumerate}

\vspace{2em}

Students and practitioners of magick and occult philosophy shall truly come to love such nifty, sacred transformer programs such as this one. Especially by realizing that.. it is ancient wisdom, that to readily bring about a transformation in real life, often, it starts with bringing about a corresponding transformation in the mind... (particularly, via proper manipulation of the subconscious mind) and the careful use of language (especially via magical transformations or projections of will.. such as this TEA program can make possible) underlie many such fundamental practices in (especially western) magick; 

\vspace{2em}

One no longer needs to cram or memorize arcane spells or mysterious incantations from ancient or closely guarded grimoires (\textbf{Ars Goetia, Emerald Tablets of Thoth, Bhagavad Gita,}...etc), and neither might they need to learn ``magical phrases or words of power" from some occult language (such as when a Roman Catholic resorts to use of \textbf{Latin, Aramaic or Hebrew} for the very occult parts of their [mass] rituals.. or when modern Occultists resort to incantations in supposedly ancient ``magical languages" such as \textbf{Egyptian, Sumerian, Aztec, Yiddish, Hindu, Arabic,}... etc for their more arcane/exacting rituals).

\vspace{2em} 

Instead, \textbf{one merely needs have a robust and reliable text-transformation formula or algorithm fit for magical purposes}; such as a [text-processing] programming utility like TEA (the Transforming Executable Alphabet), and knowledge of how to implement such magical transformations for themselves as say via a tool like the TEA WEB IDE\footnote{Checkout \url{https://tea.nuchwezi.com}} or that failing, perhaps having access to a specially crafted, and ready-to-use phrase transformer algorithm such as we see in the above example, and then, voila! They can take any desire, prayer or spell in ordinary natural language (doesn't have to be English), and the transformer program shall automatically generate for them a corresponding, magically (and psychologically potent) version ready to adapt or apply as is.

\vspace{2em}

So, with that introduction out of the way, let us first clearly understand what the underlying algorithm is --- mathematically or rather, formally so..

\vspace{2em}


\begin{transf}[The \textbf{Magical Language \texttt{MYRRH}}]
\label{TRANSFMYRRH}
If $\Theta^n$ is a sequence of $n > 0$ symbols (the original message) spanning the \textbf{Latin Alphabet} or the symbol set $\psi_{az}$ --- such as in \textbf{\hyperref[EQLATINALPHABET]{Equation \ref{EQLATINALPHABET}}}, then the following transformation:\\

\begin{trans}
\label{TRANSMYRRH}
$\Theta^n \xrightarrow{O_{gmyrrh(\cdot)}} \Theta^* = \Omega^k;$\\
$\invpi(\Theta^n) < \invpi(\Theta^*) = \invpi(\Omega^k) : n < k: k \geq n + 2 \quad \forall n,k \in \mathbb{N}$\\
$\land \quad \Omega^k \supset \Theta^n \quad \land \quad \Omega^k \approx \lnot \Theta^n $\\
$\land \quad \xi(\Theta^n \rightarrow \Omega^k) > 1 \quad \land \quad \Psi(\Theta^n \rightarrow \Omega^k) \geq 0$\\
$\qed$
\end{trans}

is guaranteed to always produce/generate a derivative message --- $\Theta^*$ that has the following \textbf{extra} properties:\\

{
\normalsize

\begin{multline}
\label{EQMYRRH}
\forall \alpha \in \Theta^n \implies \beta \in \Omega^k \implies \begin{cases}
y \rightarrow yua, & \\
ht \rightarrow th,& \\
dn \rightarrow dun,& \\
tn \rightarrow tan,& \\
rp \rightarrow rupa,& \\
sy \rightarrow s \quad y,& \\
tc \rightarrow tauch,& \\
g|G \rightarrow su,& \\
d|D \rightarrow v,& \\
. \rightarrow a.a,& \\ \text{what happened: every original character is padded by `a'}\\
. \in \psi_{AZ} \rightarrow . \in \psi_{az}& \\ \text{what happened: resultant is entirely all lowercase}\\
\end{cases}
\end{multline}
$\qed$
}

And so that, the resultant [transformed] message, $\Theta^* = \Omega^k$, despite being \textbf{somewhat} composed of the same exact symbols as in the original message, is not exactly equivalent to it, is also more like an \textbf{a-protracted} lateral mirror inversion\cite{transformatics} of the original message, and is an instance of text in the language \textbf{MYRRH}.

\end{transf}



\vspace{2em}

So, now that we have formally defined this language, it is time to explore some of its interesting applications and ``magical properties"!


\section{The Four Special Properties of MYRRH}


\begin{enumerate}
\item First of all, note that all phrases in this language are naturally pronounceable or utterable --- makes it a very useful language for practical use in all kinds of magick operations for practitioners of all levels of skill and experience.
\item The ordinary, but also magical phrases transformed into this MYRRH language are somewhat longer than their original versions, but not for the wrong reasons --- naturally, spells meant to be applied via sound/voice are either cast using exacting commands or enchanting conjurations and mantras. So, for the later two cases, longer phrases are naturally fine --- one would prefer to vibrate ``AARAAARIIITAAA" [as it is literally expressed here in a MYRRH-like phrasing] than trying to guess at how to \textit{correctly} vibrate ``ARARITA"\cite{kraig2010modern}.
\item It is arguably ideal for creating \textbf{reversal spells} --- the kind that are fit for effecting healing, undoing harm or hexes, returning lost or stolen things, bringing the dead back to life, resurfacing lost or forgotten memories, etc. Essentially, because it is founded on a mirror operation, and because we know\cite{farrar1990spells}, that mirror operations are a great basic technique in most practical magick, for reversing ill, deflecting curses and bad intent as well as helping reverse black magick.
\item It is ideal, if not one of the best methods of creating \textbf{little spells} or ``basis words of power" --- this, best reflected by the fact that, any single letter, when transformed using this method, shall become a word! Take the example ``R" $\rightarrow$ ``ARA", ``B" $\rightarrow$ ``ABA"\footnote{Perhaps reminiscent of the main incantation or words that Jesus spoke in Gethsemane in his most trying hour (\textbf{Mark 14:36})! The fact that it relates to ``Abba", aramaic for ``Father", and that, its use is closely associated with invoking divine assistance or that of the \textbf{Holy Spirit} (\textbf{Romans 8:15}, \textbf{Galatians 4:6}) would surely make this little facility of the MYRRH transform very powerful!}. But also the protraction of basic phrases such as ``AE" $\rightarrow$ ``AEAAA"\footnote{``Who sweeps" in Runyakitara!}, ``RA" $\rightarrow$ ``AAARA"\footnote{Funnily, but perhaps relevant --- ``Aaara" would translate as ``Who cries" in RUNYORO-RUNYAKITARA, the ancient language of the CWEZI, but also, this little find is interesting magically... it somewhat then synchronizes with ``Who Scries" --- essentially, who has magical sight or intuition,... and this, all just from ``RA", an ancient [solar] God-name also known to be associated with magick!}, etc.  
\end{enumerate}



\section{THREE Examples of Applying MYRRH}
\label{SECEXAMPLESMYRRH}


 %\small
  \begin{tcolorbox}[teaterminalstyle, title=TEA Program: The FINAL Grand Myrrh Transform, breakable]
  %\begin{lstlisting}[language=TEA, caption={TP C7}, label={LSTC7}, numbers=left]
  \begin{lstlisting}[language=TEA,breaklines=true]
m!:|r!:y:yua|r!:ht:th|r!:dn:dun|r!:tn:tan|r!:rp:rupa|r!:sy:s y|r!:tc:tauch|r!: :a|r!:[gG]:su|r!:[dD]:v|z:|z*:
   \end{lstlisting}
  \end{tcolorbox}
    \captionof{figure}{The Final Grand Myrrh Transform}
  \label{FIGTEAMYRRHFIN}


\vspace{2em}


\textbf{\hyperref[FIGTEAMYRRH]{Figure \ref{FIGTEAMYRRH}}} is the \textbf{source-code} of the non-interactive TEA program implementing this algorithm, and which, when enhanced with just one more step --- transforming the output of the original MYRRH transformer so that, instead of all lowercase output, and since the use-case is to generate spells or words of power, applies the useful Title-Case transform to the final output, and so that, the final, work-friendly version of that program (also minified) is as depicted in \textbf{\hyperref[FIGTEAMYRRHFIN]{Figure \ref{FIGTEAMYRRHFIN}}}. However, and in case one wishes to look at the code, modify/improve or run \textbf{a live version} of it like on the Linux, Unix, Windows or MAC OS command-line or the WEB, the most recent version should be what you might find or run directly and live via:
  
  
\vspace{1em}

 \url{https://tea.nuchwezi.com/?i=hello+world&fc=https://gist.githubusercontent.com/mcnemesis/89a8030ba026977574930f87273fedd0/raw/grand_myrrh.tea}

\vspace{1em}


\textbf{ALTERNATIVELY} just use the short-link: \url{https://bit.ly/gmyrrh}


\subsection{MIRROR Reversal SPELLS}


In exploring how one might go about applying the \textbf{GRAND MYRRH} magickal language and their associated text-transformer (\textbf{\hyperref[TRANSFMYRRH]{Transformer \ref{TRANSFMYRRH}}}), we shall start by revisiting one interesting (and important) case of how mirrors are vital in \textbf{Psychic Self-Defence} --- a very important addition to any magicians arsenal of weapons and utilities\footnote{Because, like soldiers or any army, it is important to keep in mind that where there are means for people or entities to cause harm, induce attacks or even merely make things happen, there is also need to be able to do the reverse; defend, prevent or oppose some actions from happening or perhaps to mirror-back any such undesirable actions [by others?]}. The example is first going to be picked verbatim from Farrar's book on spells\cite{farrar1990spells}:




\noindent
\begin{minipage}{1\textwidth}
\vspace{1em}
\begin{quotation}
\noindent {\ttfamily

Susa Morgan Black sends us another way of ensuring the Boomerang Effect and deflecting bad energy back to the sender...

`Take a special hand-held mirror,' she says, `and simply turn completely around with the mirror reflecting outward and state an affirmation like -

\vspace{1em}

{\centering

`Circle of Reflection,
Circle of Protection,
May the sender of all harm
Feel the power of this charm.'

}

\vspace{1em}

We have used mirrors for this purpose ourselves when we have known who was working against us, and in what direction he or she lived. We would put a mirror in a suitable window facing outwards in that direction, willing it to send the malevolence back to its source.

}
\hspace*{\fill} --- \textbf{Spells and How They Work}, \textit{1990}, Janet and Stewart Farrar\cite{farrar1990spells}
\end{quotation}
\vspace{1em}
\end{minipage}


Thus, for a compelling first example application of our MYRRH language, let us merely enhance the evil-reversal spell cited above, [hopefully] without undoing its original power!


{\LARGE

%\begin{figure}[H]
  \begin{tcbverbatim}[title=A Reversal Spell Against Black Magick]
Aamaraaahaca Asaiataha Afaoa Araeawaoapa Aeataha Alaeaeafa
Amaraaaha Alalaaa Afaoa Araeavauanaeasa Aeataha Ayauaaaaama
Aanaoaiataaauacahaeataoarapa Afaoa Aealacaraiaca
Aanaoaiataaauacahaealafaeara Afaoa Aealacaraiaca.
  \end{tcbverbatim}
%\end{figure}
}

\vspace{2em}


\subsection{NAMES of POWER and DIVINE NAMES}


Away from spells, let us first return to the stuff that many ceremonial magicians enjoy --- the stuff of \textbf{Assuming Godforms}\cite{kraig2010modern}! Essentially, there are certain parts of arcane and HIGHER MAGICK, that call for the operator to be something \textbf{greater than} their normal (basal) selves; something along the lines of \textbf{becoming ones Higher Self} --- this, so that then, the magician thus transformed and empowered, can execute tasks and manifest things that a normal human wouldn't be able to, or which they otherwise would only be able to do wishfully but not actually.

\vspace{2em}

One of the well-known methods for assuming a god-form is depicted in the modern magicians training and workbook manuscript attributed to \textbf{Donald Michael Kraig}, and which book, among many things, is a great guide to self-initiation into hermetics and especially the modern \textbf{Golden Dawn} tradition. And so, we see, on page 37 of DMK's magical treatise, the following step that is an essential part of practicing and performing the \textbf{Lesser Banishing Ritual of The Pentagram} (LBRP):


 

\noindent
\begin{minipage}{1\textwidth}
\vspace{1em}
\begin{quotation}
\noindent {\ttfamily

STEP THREE. Step forward with the left foot. At the same time thrust your hands forward so that they point at the exact middle of the glowing blue pentagram in front of you (this position is known as a ``GodForm" and is the God Form known as ``The Enterer"). As you do this you should exhale and feel the energy come back up your body, out your arms and hands, through the pentagram and to the ends of the universe. You should use the entire exhalation to vibrate the God Name: \textit{Yud-Heh-Vahv-Heh}.


}
\hspace*{\fill} --- \textbf{Modern Magick: Twelve Lessons in the High Magickal Arts}, \textit{2010}, Donald Michael Kraig\cite{kraig2010modern}
\end{quotation}
\vspace{2em}
\end{minipage}


And talking of god-forms, it shall help the reader to realize that the above procedure entails knowing a whole lot of theory and practical knowledge in order to pull it off well or effectively. Basically, it is not enough to just know or vibrate the \textbf{sacred names} --- such as the \textbf{TETRAGRAMMATON} --- YHWH --- sometimes expressed as DMK shows in the above instruction, but other times [and though it is generally known that it is not only a \textit{fearsome} name to utter, but is also without explicit pronunciation] as ``Yahweh", ``Jehovah" and other variations. One needs to known and employ their \textbf{entire self} when performing these acts of assuming a godform --- essentially, the meat of what invocations are all about; thus, details such as:

\begin{enumerate}
\item  What is expressed \textbf{physically}; such as the bits about which posture/asana to assume or express, what clothing/vestments/robes to wear, etc.
\item  \textbf{Psychologically}; usually, while in a relaxed but commanding and detached state of mind.
\item \textbf{Psychically}; such as when surrounded by symbols and implements of power --- pentagrams or hexagrams such as in the above example, but also [and based on school or tradition] what decorum the operator employs, etc.
\item \textbf{Astrally}; for example, in the DMK recipe for the LBRP, one comes to this step after having already performed the Kabbalistic Cross ritual, by which time, the magician's astral body is most likely way more pronounced and more protracted than their physical body, etc.
\end{enumerate}


\vspace{2em}



\begin{table}[H]
  \begin{tabular}{|p{0.95\textwidth}} % Left border only
    \hline
    \begin{figure}[H]
      \centering
      \includegraphics[width=0.9\textwidth]{resources/a_modern_chaos_magick_operating_circle_by_nemesisfixx_digtctm.jpg}\\
  \caption{A Magic Circle prepared for an operation at Nuchwezi Esoteric School}
      \label{FIGOLGA}
    \end{figure} \\
    \cline{1-1} % Bottom border only
  \end{tabular}
\end{table}


However, and back to our core purpose here, one needs to perform such ritual magick when they are well employed with the RIGHT WORDS or POWER FORMULAE. Sometimes it helps to know why a certain ritual is supposed to be performed how it is taught --- for example, why it is that DMK instructs the neophyte to vibrate that particular \textbf{word of power} and not any other --- even though it is the fact that there are other sacred names\cite{kraig2010modern} by which a Godform such as the one appealed to in such kaballistic rites might respond to:

\vspace{2em}
\begin{itemize}
\item YUD-HEH-VAHV-HE --- YHVH
\item AH-DOH-NYE --- ADONAI
\item EH-LOH-HEEM --- ELOHIM
\item EL
\item ME-AH-RAHB --- MEARAB
\item EH-HEH-YEH --- EHIEH
\item AH-GLAH --- AGLA
\end{itemize}
\vspace{2em}

But, and it is important to note this; \textbf{it is not ethical to knowingly corrupt or modify the sacred names} --- especially when they are meant to be used in sacred rituals or as part of a communal tradition or rite --- for example, it is well known that the name ``JESUS" isn't exactly what the original literal name of the Son of God was, based on the original language of the original scriptures:

\vspace{2em}
\begin{quotation}
{\ttfamily

The original name of Jesus of Nazareth was not ``Jesus." In his own time, he would have been called Yeshua in Hebrew/Aramaic, a shortened form of Yehoshua, meaning ``Yahweh is salvation." The name ``Jesus" is the English form that developed through Greek (Iēsous) and Latin (Iesus) translations.

}\cite{copilot_assistant}. 
\end{quotation}
\vspace{2em}


BUT, it is also important to realize that, as practicing magicians, sometimes one finds themselves in a tricky situation --- they know or have some form of a sacred name or word of power, such as \texttt{YHVH}, but they are not sure how best to go about pronouncing or uttering it while performing their ritual. And so, \textbf{it is acceptable that one modify or adapt a sacred word or name to a version easier or readily utterable.} And ADDITIONALLY and also out of experience, one finds that \textbf{it might also help to not work within the context of a ritual while using literal names of deities or entities --- principals and powers}. This might be for several reasons;

\begin{enumerate}
\item To conceal a secret from the uninitiated --- like when one must perform a ritual in the presence or within hearing distance of non-members or possibly adversaries.
\item To enhance the potency of a ritual --- just like one might prefer to perform a conjuration or cast a spell in some alien or barbarous language\footnote{Talking of barbarous languages used by actual contemporary magicians, one might for example appreciate the use of [artificial?] magical languages such as \textbf{Ouranian Barbaric}\cite{madara2019robotheosis} --- see \url{http://www.chaosmatrix.org/library/ob.php} --- employed by \textbf{The Illuminati} of IoT, but also several affiliate chaos magicians, and known to have been invented by \textbf{Peter J. Carroll}\cite{copilot_assistant}.} such as our own case of LUMTAUTO already well introduced in \textbf{\hyperref[SECLUMTAUTO]{Section \ref{SECLUMTAUTO}}}.
\item To help readily \textbf{short-circuit} programming of the unconscious mind --- as we have already encountered in the note from the Farrars\cite{farrar1990spells}, but also as eminent scholars like Jung taught concerning use of clever symbolic programming for effecting psycho-social change\cite{jung1964symbols} --- a topic so well covered by the present author in \cite{lutalo2025concerning_trans}.
\end{enumerate}

But also just for creativity's sake. Thus, we might for example prefer to recite mass, while referring to JESUS as ``YEHESHUA" --- a name that one shall find in our virtual shrines at \url{https://iona.nuchwezi.com}, but also, one might instead decide to take ordinary names and words of power, and use their transformed versions --- such as we show in the following table, applying the MYRRH transformer to traditional \textbf{Divine Names}:



\begin{table}[H]
  %\centering
{

\footnotesize  
  
	\begin{tabular}[t]{|l|l|l|}
\rowcolor{lightgray}\bfseries	\textbf{NAME} & \textbf{MYRRH'ed VERSION} & \textbf{RELEVANCE\cite{GoogleAI2025}\cite{copilot_assistant}}\\
	\hline
 \hline
 	\textbf{RA} & AAARA & Ancient Solar Deity in Khem/Egypt\\
	\hline
	\textbf{AMUN RA} & ARA ANAUAMAAA & Another form of RA\\
	\hline
	\textbf{KA} & AAAKA & The human soul/spirit (Egyptian)\\
	\hline
	\textbf{TETU} & AUATAEATA & \makecell[l]{Ancient Deity (Egyptian) also ``THOTH", ``TAHUTI",\\ ``HERMES" (Greek); of Magick and Writing}\\
	\hline
	\textbf{HECATE} & AEATACAEAHA & \makecell[l]{Deity (Greek) also ``HEKATE";\\ of Magick, the Moon, \textit{Liminal Spaces}, and VooDoo!}\\
	
	\hline
	\hline
	\textbf{YHVH} & AHAVAHAYA &\makecell[l]{The Tetragrammaton/God of Israel\\Creator of The Universe}\\
	\hline
	\textbf{ADONAI} & AIAAANAOAVAAA &{Deity Honor (Kabalah), ``My Lords"}\\
	\hline
	\textbf{EHIEH} & AHAEAIAHAEA &\makecell[l]{Deity Title (Kabalah) of Mysterious God of Moses}\\
	\hline
	\textbf{AGLA} & AAALASUAAA &{Deity Honor for ``ADONAI" (Kabalah)}\\
	\hline
	\textbf{EL} & ALAEA &\makecell[l]{Deity (Kabalah, Canaanite tradition)}\\
	\hline
	\textbf{IAO} & AOAAAIA &\makecell[l]{Greek form of ``YHVH", Invokable God}\\
	\hline
	\textbf{MAMMON} & ANAOAMAMAAAMA & \makecell[l]{An Arch-Daemon of Wealth and Money\\also ``NAMON" (Illuminati)}\\
	
	\hline
	\hline
	\textbf{YAHWEH} & AHAEAWAHAAAYA & \makecell[l]{A popular Christianized Deity\\A non-material Omnipotent Spirit}\\
	\hline
	\textbf{JESUS} & ASAUASAEAJA &\makecell[l]{God Incarnate (Christian Mysticism)\\a Benevolent, Merciful Lord; the Saviour}\\
	\hline
	\textbf{YEHESHUA} & AUAHASAEAHAEAYA & \makecell[l]{Also ``JESUS", who understands Humans and Mystics}\\
	\hline
	\textbf{ALLAH} & AHAAALALAAA &\makecell[l]{Deity (Arabic), also ``Al-Ilāh"; the ``God" in Judaism}\\
	
	\hline
	\hline
	\textbf{SATAN} & ANAAATAAASA & \makecell[l]{An Arch-Angel, the ``Devil", ``Adversary",\\ also ``HUGUM" (Illuminati); a Pagan Godform}\\
	\hline
	\textbf{BAPHOMET} & ATAEAMAOAHAPABA & \makecell[l]{Fictitious Deity (Knights Templars)\\often confused with ``MAHOMET" (Islam)}\\
	\hline
	\textbf{ASTAROTH} & ATAHAOARAAATASA & \makecell[l]{Demonised-Deity, Legion Commander\\in sync with Astarte, Lucifer and Beelzebub}\\
	
	\hline
	\hline
	\textbf{ISHTAR} & ARAAATAHASAIA & Deity (Sumerian) also ``INANNA"; of war, love, sex.\\
	\hline
	\textbf{ANU} & AUANAAA & Ancient Deity (Sumerian), ``Father of Gods"\\
	\hline
	\textbf{MARDUK} & AKAUAVARAAAMA & Deity (Babylon) of Creation\\
	
	\hline
	\hline
	\textbf{RAMA} & AMAAARA & \makecell[l]{Deity (Hindu) also ``RAMACHANDRA", ``VISHNU"; of Ideals}\\
	\hline
	\textbf{SHIVA} & AVAIAHASA & \makecell[l]{Deity (Hindu) also ``NATARAJA";\\ of Transformation and Drums}\\
	\hline
	\textbf{KAMA} & AMAAAKA & \makecell[l]{Deity (Hindu) also ``KAMADEVA"; of Love, Eros,\\ Desires, and Wishes}\\
	\hline
	\textbf{MAHE} & AEAHAAAMA & Deity (Hindu) also ``SHIVA", ``MAHENYU" (Cwezi); of Armies\\

	\hline	
	\hline
	\textbf{BANGA} & ASUANAAABA &\makecell[l]{Sovereign Deity (Ganda), also ``RUHANGA" (Cwezi);\\of Aliens, Invisibility and Space.\\In the form ``TONDA", as Creator}\\
	\hline
	\textbf{KIBUKA} & AKAUABAIAKA &\makecell[l]{Deity (Ganda), also ``OMUMBALE", sovereign over ``Air",\\Aerial Travel, Birds and Aerial Spirits}\\
	\hline
	\textbf{KINTU} & AUATAAANAIAKA &\makecell[l]{Deity Attribute (Ganda) of Matter, sovereign over ``Earth";\\ a kind of ``Adam"}\\
	\hline
	\textbf{MANYA} & AYANAAAMA &\makecell[l]{Deity Honor (Ganda) sovereign over Names and Knowledge}\\

	\hline	
	\hline
	\textbf{AHA} & AAAHAAA & \makecell[l]{Deity Honor (Cwezi) of Divine Providence, Wealth and Money}\\
	\hline
	\textbf{WEIRA} & ARAIAEAWA & \makecell[l]{Deity Honor (Cwezi) Mysterious ``First Cause"}\\
	\hline
	\textbf{IRAKA} & AKAAARAIA & \makecell[l]{Deity Attribute (Cwezi) of Communication and Languages}\\
	\hline
	\textbf{MANI} & AIANAAAMA & \makecell[l]{Deity Attribute (Cwezi) Mysterious ``Powers", an ``Almighty"}\\
	
	\hline	
	\hline
	\textbf{HAZORAHIN} & ANAIAHARAOAZAHA & \makecell[l]{Deity Honor (Illuminati) Mysterious 1\\of Psychic Powers}\\
	\hline
	\textbf{HANUZAMEK} & AKAEAMAZAUANAHA & \makecell[l]{Deity Honor (Illuminati) Mysterious 1\\also ``ZHANAMUKE", ``AMENUZAHK", ``AHEZANKUM",\\``KAHENAMUZ", and ``NKUZAMAHE"; of Fun and Drama!}\\
	\hline
	              
\end{tabular}
}
\caption{\texttt{The PANTHEONICS:} \textbf{Divine Names} of POWER: in MYRRH language, Explained}
  \label{TABMYRRHNAMES}
\end{table}


\textbf{NOTE:} Because of space constraints, some of the MYRRH-form of the names in \textbf{\hyperref[TABMYRRHNAMES]{Table \ref{TABMYRRHNAMES}}} have been simplified by reducing extraneous `A's where necessary. However, those interested in securing the proper MYRRH-form of any of the listed divine names, can readily do so using the programs and algorithms already shared.



\subsection{WORDS of POWER, FORMULAE for BLACK and WHITE MAGICK}




\begin{table}[H]
  \begin{tabular}{|p{0.95\textwidth}} % Left border only
    \hline
\begin{figure}[H]
  \begin{center}
   \includegraphics[width=\textwidth]{resources/om_mani.jpg}\\
  \end{center}
  \caption{The \texttt{OM MANI PADME HUM} mantra}
  \label{FIGOMMANI}
\end{figure}\\
    \cline{1-1} % Bottom border only
  \end{tabular}
\end{table}


Having considered \textbf{Names of POWER} in \textbf{\hyperref[TABMYRRHNAMES]{Table \ref{TABMYRRHNAMES}}}, and which names might be well applied during either \textbf{Invocations} (for God-forms) or in \textbf{Evocations} (for daemons, thoughtforms or angels), or which might be incorporated into other formulations and arbitrary rituals as one sees fit, let us now turn our attention to \textbf{Words of POWER} --- in fact, \textbf{MANTRAS of POWER} and some \textbf{Magical Formulae}\footnote{\textbf{NOTE}: It might help some readers unfamiliar with how things typically work, to realize that unlike ``Words or Names of Power" that require \textit{other stuff} in order to properly use or apply them, \texttt{MANTRAS} and the kind of magical formulae being discussed in this section can usually be used as-is --- i.e. one can just pick up the mantra and start uttering or vibrating it however way they see fit, and it shall just work! But also, these mantras and formulae might be incorporated into other, longer or more sophisticated spells or ritual formulations by experienced students and magicians who know what they want and know what they are doing.}



\begin{table}[H]
{

\footnotesize  

\begin{tabular}[t]{|l|l|l|}
 \hline
 
	\rowcolor{lightgray}\bfseries \textbf{PHRASE/FORMULA} & \makecell[l]{\textbf{MYRRH'ed UTTERANCE}\\or \textbf{MANTRA}} & \textbf{RELEVANCE/NOTES}\cite{GoogleAI2025}\cite{copilot_assistant}\\
	\hline
	\hline
	
	
	\makecell[l]{\textbf{OM AH HUM}\\\textbf{VAJRA GURU}\\ \textbf{PADMA SIDDHI HUM}} & \makecell[l]{AMAUAHA\\AIAHAVAVAIASA\\AMAVAAAPA\\AUARAUASUA\\ARAJAAAVA\\AMAUAHA\\AHAAA\\AMAOA} &\makecell[l]{\textbf{For Defense and Healing:}\\\\Also as: \texttt{OM AH HUNG BENZA GURU}\\ \texttt{PEMA SIDDHI HUNG} --- for Tibetan Buddhists,\\Attributed to \textit{Padmasambhava} --- founder of\\Tibetan Budhism\cite{Rinpoche1992}. It is said to be:\\\\``the mantra of all the budhas, masters, and\\ realized beings, and so uniquely powerful for\\ peace, for healing, for transformation and for\\protection in this violent, chaotic age."\cite{Rinpoche1992}\\}\\
	\hline
	\hline
	
	
	\makecell[l]{\textbf{OM MANI}\\\textbf{PADME HUM}} & \makecell[l]{AMAUAHA\\AEAMAVAAAPA\\AIANAAAMA\\AMAOA} &\makecell[l]{\textbf{For Compassion and Mercy:}\\\\Also as: \texttt{OM MANI PEME HUNG}, it is attributed\\to Tibetans, and this mantra \\(of \textit{Avalokiteshvara} --- the Budha\\ of Compassion), apart from invoking blessings,\\shall help a deceased family member\\ or friend peacefully pass on into the after-life\\ --- especially when performed while in the\\ presence of the dead person's body or grave\cite{Rinpoche1992}.\\}\\
	\hline
	\hline
	

		\makecell[l]{\textbf{LEAD ME}\\\textbf{OUT OF PRISON}\\\textbf{THAT I MAY PRAISE}\\\textbf{YOUR NAME O LORD.}} & \makecell[l]{AVARAOALA AOA\\AEAMAANA ARAUAOAYA\\AEASAIAARAPA AYAAMA\\ AIA ATAAHATA\\ANAOASAIARAPA AFAOA\\ATAUAOA AEAMA\\AVAAEALA} & \makecell[l]{\textbf{A Powerful Liberation Psalm:}\\attributed to King David (\textbf{Psalms 142:7})\cite{newjerusalem1985}.\\As with most Psalms, its worth is not\\just in its poetic value, but also the spiritual \\value;\\\\ ``The spiritual riches of the Psalter need no\\ commendation... prayers of the Old Testament\\in which God inspired the feelings that his\\children ought to have towards him and\\the words they ought to use when speaking\\to him. They were recited by Jesus himself, \\by the Virgin Mary, the apostles and the\\early martyrs."\cite{newjerusalem1985}}\\
	\hline
	\hline
	
	\makecell[l]{\textbf{MAY WORDS}\\\textbf{SPOKEN BECOME}} & \makecell[l]{AEAMAOACAEABA\\ANAEAKAOAPASA\\ASAVARAOAWA\\AYAAMA} & \makecell[l]{\textbf{For Charging and Manifesting Intents:}\\attributed to JOHN The Apostle (\textbf{John 1:14})\cite{newjerusalem1985}, \\this magical formula is derived from the mystical\\statement:\\\\ \texttt{And the Word became flesh}\\\texttt{and dwelt among us.}\\\\Which, when applied to magick, especially \\ritual magick, is leveraged such that,\\the desire or will/intent(s) of \\the operator or gathered magicians \\(say at a ceremony such as mass or a seance),\\and which typically are expressed as \\either intentions spoken, written or expressed\\in some creative way, are thus charged\\and so willed into being via the mantra,\\and which is founded on this Occult Philosophy\\that John appeals to. Thus then,\\this formula is BEST APPLIED at the very end\\of the ceremony or [perhaps] before [and after]\\ the expression of intents\\ --- such as after the 5$^{th}$ bead of the\\Transformative Rosary Rite\cite{transformation_rosary_rite}.}\\
	\hline
	\hline

	              
\end{tabular}
}
\caption{\texttt{The BENEVOLENT MANTRAS}: \textbf{WORDS of Power} Transformed by MYRRH}
  \label{TABMYRRHWORDS}
\end{table}








\begin{table}[H]
{

\footnotesize  

\begin{tabular}[t]{|l|l|l|}
 \hline
 
	\rowcolor{lightgray}\bfseries \textbf{PHRASE/FORMULA} & \makecell[l]{\textbf{MYRRH'ed UTTERANCE}\\or \textbf{MANTRA}} & \textbf{RELEVANCE/NOTES}\cite{GoogleAI2025}\cite{copilot_assistant}\\
	\hline
	\hline
	
	\textbf{I AM THAT I AM} & \makecell[l]{AMAAA\\AIA\\ATAAAHATA\\AMAAA\\AIA} & \makecell[l]{\textbf{For Commanding Powers:}\\\\Of MASONIC relevance, also formula\\ of God's name given to MOSES at the\\burning bush (\textbf{Exodus 3:14})\cite{newjerusalem1985}\cite{butler1952magic}.\\Great for affirming power, divinity, and command.\\}\\
	\hline
	\hline
	
	
	\makecell[l]{\textbf{PROCUL PROCUL}\\\textbf{ESTE PROFANI}} & \makecell[l]{AIANAAFAOARAPA\\AEATASAEA\\ALAUACAOARAPA\\ALAUACAOARAPA} & \makecell[l]{\textbf{Banishing/Suspending Profane Reality:}\\\\Originally Latin for ``Far hence, far hence,\\ye profane!" and attributed to Book VI\\of \textbf{Virgil}'s epic poem, the \textit{Aeneid}\cite{virgil_aeneid_book6},\\this phrase is a somewhat \textit{underground}\\opening utterance for arcane\\rituals in several secret societies only for\\the initiated, and like how formulas\\such as ``The License to Depart" might\\chase away demons after an evocation, this \\formula is useful in ``chasing away" normal\\consciousness so the participants\\can operate while in a [sufficient]\\altered state of consciousness.}\\
	\hline
	\hline
	
	
	\makecell[l]{\textbf{MORTEM MIHI}\\\textbf{INDUITE NOVUM}\\\textbf{HOMINEM}} & \makecell[l]{AMAEANAIAMAOAHA\\AMAUAVAOANA\\AEATAIAUAVANAIA\\AIAHAIAMA\\AMAEATARAOAMA} & \makecell[l]{\textbf{Induce Metamorphosis or Transfiguration:}\\\\In English:``Darkness, clothe me with\\ a different body."\\\\It is attributed to Lutalo's novel;\\ \textit{Shrines of The Free Men}\cite{shrinesjwl}, and this formula\\is a kind of Death Magick practiced by Black \\Magicians exploring \textbf{Vampire Magick},\\ OBEs or Astral Travel. The basic idea is to enter\\a state of deep sleep-like trance or perhaps\\ a Near-Death Experience (NDE) after\\ performing the mantra, and that, upon waking\\or regaining awareness (possibly in the dream\\ or altered-state), the magician shall find that they\\are someone else or are inside someone else's body!}\\
	\hline
	\hline
	              
\end{tabular}
}
\caption{\texttt{The ARCANE MANTRAS}: \textbf{WORDS of Power} Transformed by MYRRH}
  \label{TABMYRRHWORDS2}
\end{table}


And thus we come to the conclusion of this treatment of the \textbf{MYRRH Language}. As a final note, realize that, even though we have split the above two sets of mantras or words of power collections into ``Benevolent" and ``Arcane", and yet, in reality, either can be used or applied in contexts relating to White or Black Magick --- for example, many people might regard the idea of performing magick that raises the dead or which causes a dead person to ``prematurely" return to life as a kind of necromancy or black magick, and yet, we see such acts having been performed well by the [well-intending and benevolent] Jesus Christ as depicted in the scriptures!


\newpage

\begin{figure}[H]
  \begin{center}
  \vspace{6em}
   \includegraphics[scale=0.9]{resources/ozin_code.pdf}\\
   \caption{CODE: \textbf{Church of Dance Eternal}: [e][in]vokable: \textbf{OEIAU}}
  \label{FIGCODE}
  \end{center}
\end{figure}


\newpage

\begin{table}[H]
  \begin{tabular}{|p{0.95\textwidth}} % Left border only
    \hline
\begin{figure}[H]
  \begin{center}
  \includegraphics[height=0.9\textheight]{resources/scanned_example_use_ozin_for_shrine_archictecture_notes_iona.pdf}
  \caption{At NES, we originally toyed with the idea of Constructing a Modern Esoteric Shrine}
  \label{FIGESOTERIC}
  \end{center}
\end{figure}\\
    \cline{1-1} % Bottom border only
  \end{tabular}
\end{table}


\newpage
\begin{figure}[H]
  \begin{center}
  \vspace{6em}
   \includegraphics[width=0.6\textwidth]{resources/code_mysteries_meta_hymn.pdf}\\
      \caption{CODE: \textbf{First META hymn}: fit to be sung, chanted, vibrated, reversed or looped as long as one wishes to operate in a Highest State of Mind.}
  \label{FIGCODEMETAHYMN}
  \end{center}
\end{figure}


\begin{figure}[H]
  \begin{center}
  %\vspace{6em}
   \includegraphics[width=0.8\textwidth]{resources/code_mysteries_meta_hymn_painting.pdf}\\
  \end{center}
\end{figure}



\newpage

\begin{table}[H]
  \begin{tabular}{|p{0.95\textwidth}} % Left border only
    \hline
\begin{figure}[H]
  \begin{center}
  \includegraphics[height=0.9\textheight]{resources/scanned_example_use_ozin_for_shrine_archictecture_notes.pdf}
  \caption{The idea in this sketch is for an operating space where a modern magician or priest of the mysteries might operate from, perhaps with a very small audience. It is meant for private Illuminati experiments and study of the occult work in sufficient elegance and privacy.}
  \label{FIGESOTERICB}
  \end{center}
\end{figure}\\
    \cline{1-1} % Bottom border only
  \end{tabular}
\end{table}




\begin{figure}[H]
  \begin{center}
   \includegraphics[width=\textwidth]{resources/ozin_slogan.pdf}\\
  \end{center}
\end{figure}


\begin{figure}[H]
  \begin{center}
   \includegraphics[scale=0.8]{resources/nucwa_place_by_nemesisfixx_ddivx9v.jpg}\\
  \end{center}
\end{figure}




\chapter{The OZIN Cipher and Magickal Language}
\label{SECOZIN}

\begin{figure}[htp]
  \begin{center}
   \includegraphics[height=\textheight]{resources/COMPLETE_OZINLANGUAGE.pdf}\\
   \caption{The COMPLETE OZIN ALHABET Symbol Set}
  \label{FIGOZINALPHABET}
  \end{center}
\end{figure}

{
\Large

Next, we turn our attention to a language unlike any that we have covered yet in this grimoire; \textbf{a visual magickal language!} Yes, much as the two languages, LUMTAUTO (\textbf{\hyperref[TRANSFLUMTAUTO]{Transformer \ref{TRANSFLUMTAUTO}}}) and MYRRH (\textbf{\hyperref[TRANSFMYRRH]{Transformer \ref{TRANSFMYRRH}}}) are each powerful and have special applications as we have seen in their respective chapters, and yet, they can not do certain things that only a language such as \textbf{OZIN} can do. But what exactly is this OZIN language all about? Why is it special? 

}

%\vspace{2em}


\section{Some History on Use of Visual Languages in Magick}


First, before we dive into exploring our third magical language system --- the \textbf{OZIN Magickal Language}\cite{lutalo_2025_ozin} --- refer to \textbf{\hyperref[FIGOZINALPHABET]{Figure \ref{FIGOZINALPHABET}}} , let us first jump into a time machine, and travel back to the most ancient days... let us first trace our steps back to when writing and use of written language was still a thing exclusively only for high priests and the most dignified occultists in the land... back in [and before!] the days of \textbf{Hieroglyphics} and \textbf{the earliest forms of writing and written expressions}!

\vspace{2em}


In the [now-rare] encyclopedic compendium about the history and progress of human languages and communication --- \textbf{Communication and Language: Networks of Thought and Action} edited by \textbf{Sir Gerald Barry}, Dr. J. Bronowski, James Fisher and \textbf{Sir Julian Huxley}, we see, in the introduction, a telling of how the magical faculties of formal communication that humans exhibit first came to be...


\begin{figure}[H]
  \begin{center}
   \includegraphics[width=1\textwidth]{resources/kids_in_ancient_khem_studying_rubics_games_LEARNING.jpg}\\
  \end{center}
\end{figure}


\noindent
\begin{minipage}{1\textwidth}
\vspace{1em}
\begin{quotation}
\noindent {\ttfamily

Most animals communicate with their kind in one degree or another, some by methods unknown to or imperfectly understood by men. But man alone has acquired the faculty of communication by \textit{speech}. This unique achievement has been the biggest single factor in the success of \textit{Homo sapiens} in developing complex societies. Only now are we beginning to realize how deeply relevant communication is to the story of human progress. Communication theory, as it is called, is now one of the basic areas of research into human intercourse and understanding.

Speech was the first great leap forward in the development of human communication. The second was the invention of \textit{writing}. By this means, what men thought to themselves or said to one another could be recorded, read by others, and stored for the benefit of future generations. There now existed a communal ``memory."

}
\hspace*{\fill} --- \textbf{Communication and Language: Networks of Thought and Action}, \textit{1965}, Gerald Barry et al.\cite{Barry1965}
\end{quotation}
\vspace{1em}
\end{minipage}



Thus, we start to realize that, apart from [formal] languages underlying our distinctiveness as humans when compared to beasts, they also serve the purpose of setting us apart --- particularly, knowledge of and possession of, ability to read, speak or wield a certain formal language can be all \textbf{the difference between those with a power and privilege and those without!} We clearly see this, when Lutalo, in his treatment of special symbols, words and languages\cite{lutalo2025concerning_trans}, says, as part of his analysis of \textbf{Professor Pierre Bourdieu}'s classic text ``LANGUAGE and SYMBOLIC POWER":




\noindent
\begin{minipage}{1\textwidth}
\vspace{1em}
\begin{quotation}
\noindent {\ttfamily

Also, we see that in many instances of control of access to power, authority and in the exercising of order in society --- especially where many classes and diversities are concerned, that certain ``special" knowledge such as that of ``special" languages, endows or lifts certain members of the community/society to a status and privilege that isn't or can't be readily shared by everyone.


}
\hspace*{\fill} --- \textbf{Concerning a Transformative Power in Certain Symbols, Letters and Words}, \textit{2025}, Joseph Willrich Lutalo\cite{lutalo2025concerning_trans}
\end{quotation}
\vspace{1em}
\end{minipage}


And he goes on to cite Pierre Bourdieu on this one, and which is also relevant to our present explorations, when he informs us that:\\


 \begin{quotation}
%\small
{\ttfamily

The members of these local bourgeoisie of priests, doctors or teachers, who owed their position to their mastery of the instruments of expression, had everything to gain from the Revolutionary policy of linguistic unification... defacto monopoly of politics, and more generally of communication with the central government and its representatives

}
\end{quotation}


\vspace{2em}


Not to deviate significantly from our main discourse, note that in that last quote, the context was about use of and access to ``special languages" --- or particularly, \textit{special dialects} --- by a few privileged people and social-classes during the French Revolution. However, these scenarios and cases aren't few across human history, and neither are they absent in contemporary life and reality. Moreover, and with regards to \textit{the kind} of communication that is most leveraged in magick --- essentially, and usually, languages and expression forms that mostly appeal to the unconscious mind\footnote{For a great and up-to-date model of the mind, treating of the conscious, subconscious and overall unconscious and their attributes, also refer \textbf{a model of the mind} depicted in Lutalo's treatment of modern Psychology\cite{Lutalo2025transpsy}.} --- tools and methods with which not only individuals, but entire communities, generations and all of the human race might be conditioned and programmed, the special use of symbolic forms of expression is one of the cornerstones of many power-wielding elites in all circles of human life since antiquity. 


Concerning use of and particularly, the significance of using symbols in communication, note what Barry et al. (1965) say:



\noindent
\begin{minipage}{1\textwidth}
\vspace{1em}
\begin{quotation}
\noindent {\ttfamily

So far we have looked mainly at symbols that merely convey information. But there is another class of symbols that both carry information and also stand for a body of thought or belief that is not easily expressed in any other way. Just as a simple word can in the course of time gather around it a mass of associations and a variety of meanings that are not easily explained by using other words, so, too, a symbol can stand for much more than its basic meaning in the original code. In other words, symbols can become ``charged" with meaning and with the power to evoke emotions.

}
\hspace*{\fill} --- \textbf{Communication and Language: Networks of Thought and Action}, \textit{1965}, Gerald Barry et al.\cite{Barry1965}
\end{quotation}
\vspace{1em}
\end{minipage}



In that quote's opening sentence, most likely, the reference is to \textit{signs} --- a matter perhaps best treated of in philosophies and studies such as \textbf{Semiotics}. However, the core of that quote and its relevance to us, concerns use of special symbols and writing systems whose value or impact has less to do with the explicit messages or information they carry, and more to do with what they evoke when one encounters them or attempts to process them. In that book\cite{Barry1965}, Barry et al. go on to show us some examples such as the use of the ``fish" symbol to depict or relate ideas to Jesus and/or Christianity especially among the early Christians, but also shows us some other common symbols such as the ``Tau" cross and others as depicted in \textbf{\hyperref[FIGCHRISTSYMBOLS]{Figure \ref{FIGCHRISTSYMBOLS}}}\footnote{Note that this figure attempts to express the symbols as close as possible to what Barry et al. (1965) show in their book, but doesn't exactly use their drawing style verbatim.}




\begin{figure}[H]
  \begin{center}
   \includegraphics[width=\textwidth]{resources/earlychristiansymbols.pdf}\\
   \caption{\textbf{Symbolism from Early Christianity:} The Fish Sign (``Pisces" also 12$^{th}$ sign of the Zodiac) (left). Then first two letters of Christos superimposed --- the \textbf{Christogram} (center) and then the ``Tau" Cross with Greek letters from the beginning and end of the alphabet --- ``I am the Alpha and the Omega, the beginning and the end."}
         \label{FIGCHRISTSYMBOLS}
  \end{center}
\end{figure}



Please note that sometimes, correct or proper rendition of ancient symbols can be tricky or troublesome. For example, concerning the last symbol in \textbf{\hyperref[FIGCHRISTSYMBOLS]{Figure \ref{FIGCHRISTSYMBOLS}}} --- the special \textbf{Tau Cross}, Barry et al. use the sometimes confusing rendition that mixes an upper-case ``alpha" with a lower-case ``omega" --- as shown in figures 



\begin{figure}[H]
  \begin{center}
   \includegraphics[scale=0.8]{resources/tau_cross_earlychristiansymbols.pdf}\\
   \caption{The decorated Tau Cross as depicted in Gerald Barry et al. (1965)}
         \label{FIGCHRISTSYMBOLSTAU}
  \end{center}
\end{figure}


\begin{figure}[H]
  \begin{center}
   \includegraphics[scale=0.3]{resources/1091px-Chrisme_Colosseum_Rome_Italy.jpg}\\
   \caption{The decorated Christogram sourced from Wikipedia}
         \label{FIGCHRISTSYMBOLCHRISTOGRAM}
  \end{center}
\end{figure}


Moreover, that same confusion might arise with also the decorated Christogram\footnote{Source: \url{https://en.wikipedia.org/wiki/File:Chrisme_Colosseum_Rome_Italy.jpg}} as depicted in \textbf{\hyperref[FIGCHRISTSYMBOLCHRISTOGRAM]{Figure \ref{FIGCHRISTSYMBOLCHRISTOGRAM}}}, and so that, both precaution and care need be taken when trying to reproduce symbols or symbolic messages from antiquity. However, given that sometimes the necessary fonts or typefaces are missing, and yet one might wish to write or draw such symbols electronically, it then also helps to invest some effort into understanding where they originated from, how they are meant to be used and/or what the symbols might mean --- so as to properly reproduce or perhaps adapt them for their original purposes or in new cases and contexts.


And so, back to our tracing of the history of \textbf{visual magical languages}. Again, we find in Barry et al. an enlightening discussion under the title ``Pictures and Magic", that shall help offer proper context to our use of and development of the language \textbf{OZIN}\cite{lutalo_2025_trans_genetics}. We learn that\cite{Barry1965}:




\noindent
\begin{minipage}{1\textwidth}
\vspace{1em}
\begin{quotation}
\noindent {\ttfamily

Long before man had learned to write, and perhaps even before he had developed a proper spoken language, he was painting pictures.

}
\end{quotation}
\vspace{1em}
\end{minipage}


Moreover, we further learn that, not only because of the style in which these painting were made, but also where they were made, that it is very likely that these primitive people had a truly special purpose for making these drawings\cite{Barry1965}:


\noindent
\begin{minipage}{1\textwidth}
\vspace{1em}
\begin{quotation}
\noindent {\ttfamily

...we have quite recently discovered animal paintings that are at least 20,000 years old. They have survived only because they are far from the cave entrances and therefore unaffected by weather conditions. Their remoteness also makes it certain that these parts of the caves were not lived in, but [were] only visited for special purposes.

}
\end{quotation}
\vspace{1em}
\end{minipage}


Also, it is worth noting that, it is not just about ``how" or ``where" these drawings were made, but also about the ``why", that researchers and students of these early forms of visual communication might be concerned. \cite{Barry1965} puts it clearly enough for our purpose:



\noindent
\begin{minipage}{1\textwidth}
\vspace{1em}
\begin{quotation}
\noindent {\ttfamily

...they were painted under extremely difficult conditions. What was their purpose? The question is important because these paintings are the earliest known attempts by man to represent the world in which he lived. They are the beginnings of art --- a major form of visual communication. In questioning their purpose, we are asking: ``With whom was the artist communicating?"

...The skills that have gradually made us masters of the earth...

Primitive man is a weakling in the world of animals, and yet he has to gain mastery over them or die. One way in which he can build up his courage against animals bigger and stronger than himself is to make pictures of them.

}
\end{quotation}
\vspace{1em}
\end{minipage}


In that final statement, we see the traces of a special logic that underlies much of what actually goes on in the modern practice of magick --- actually, not just in magick, but also, and based on our experience and knowledge, also heavily utilized in hard-science fields such as in Architecture (the drawing of plans ahead of actual construction of structures), in Computer Science (the drawing of system design and flow-charts, state-diagrams, etc. ahead of actual writing or implementation of computer programs), etc. --- whereby, the skilled and learned magician, takes their intent, desire or objective --- typically referred to as a \textbf{Statement of Intent} --- and then, after either writing or thinking about it as clearly as possible, goes ahead to transform it into a visual form; by modern standards, as a \textbf{SIGIL} --- a special, magically relevant kind of picture, and which they then proceed to operate on or with, so as to resolve or \textit{subdue} their main problem. Thus, to the uninitiated, such procedures and rigor might seem useless or ``over-complicating things", and yet, as Barry et al. argue, this is pretty basic and ancient magical wisdom\cite{Barry1965}:


\noindent
\begin{minipage}{1\textwidth}
\vspace{1em}
\begin{quotation}
\noindent {\ttfamily

By making these pictures, Neolithic man was talking to himself, thinking about his coming combats. In effect he was saying: ``This is the animal that I have to kill. He is part of me and the world I live in; by painting this picture I am that much nearer achieving my object." He believed that the act of picturing the animal magically ensured that it would be killed.

}
\end{quotation}
\vspace{1em}
\end{minipage} 



\begin{figure}[H]
  \begin{center}
   \includegraphics[height=0.9\textheight]{resources/seal_of_victory.pdf}\\
   \caption{The SEAL of VICTORY --- Insignia of Inner Soldier}
  \label{FIGOZINVICTORY}
  \end{center}
\end{figure}


\subsection{The Development of Writing Systems and The Priesthood of Thoth}


\begin{table}[H]
  \begin{tabular}{|p{0.95\textwidth}} % Left border only
    \hline
\begin{figure}[H]
  \begin{center}
   \includegraphics[width=1\textwidth]{resources/hieroglyphics-ancient-egypt-time-machine-images-examples---ANCIENT-scribes-home.jpg}\\
   \vspace{2em}
  \end{center}
\end{figure}\\
    \cline{1-1} % Bottom border only
  \end{tabular}
\end{table}


We shall not attempt to exhaust our explorations of, analysis of or interpretation of ancient art nor the use of visual or rather, pictographic writing and communication in early and primitive man societies, however, and of interest to us as the Illuminates of Nuchwezi, is the ancient visual communication systems that were employed by the elite and especially the various priesthoods of ancient Africa. We especially wish to bring attention to the fact that the \textbf{Priesthood of Tehuti} not only deified Thoth as a god of writing and magick, but also, and jealously so, practiced and preserved the use of writing and especially visual expressions (in the form of \textit{hieroglyphics}) in both scientific and religious, but also for political ends. We for example learn from \textbf{Paul Johnson}'s reference book on many things concerning ancient Egypt:


\noindent
\begin{minipage}{1\textwidth}
\vspace{1em}
\begin{quotation}
\noindent {\ttfamily

The cult of the supergod and of theological imperialism, in which gods left their localities and crossed frontiers without loosing their power, necessarily presupposed the possibility of monotheism, of a universal, all-powerful and solitary deity... A papyrus in the Cairo Museum contains a hymn to Amun-Re with the following passage:\\

Thou art the sole one, who made all there is.\\
The solitary one, who made what exists,\\
From whose eyes mankind came forth,\\
And upon whose mouth the gods came into being...\\
Hail to thee who did this!\\
Solitary sole one, with many hands.\\\\

...Sometimes, the idea was reformulated in a characteristic Egyptian way to embrace the three leading supergods in a trinitarian concept of three natures in the same god. Thus a papyrus now in Leyden dating from the Nineteenth Dynasty gives this formulation: ``All gods are three, Amun, Re and Ptah, and there is no second to them. `Hidden' is his name Amun. He is Re in face and his body is Ptah. Their cities are on earth for ever: Thebes, Heliopolis and Mephis to eternity."


}
\end{quotation}
\vspace{1em}
\end{minipage}

\vspace{2em}

Further that...




\noindent
\begin{minipage}{1\textwidth}
\vspace{1em}
\begin{quotation}
\noindent {\ttfamily



If there was one supreme god, not confined to a particular place, what was his relationship to non-Egyptians? With the growth of empire, and the parallel tendency for foreigners to settle in Egypt, there is evidence that the latter were granted the protection of the Egyptian pantheon, though not exactly of equal status. The tomb document known as the \textit{Book of the Gates} credits Horus with creating all the main races of mankind; Horus or Sekhmet `protects the souls' of Asians, Negroes and Libyans. The illustrations in the \textit{Book of the Dead} certainly imply that foreigners and interpreters will be present in eternity. Thoth, the god of wisdom and scribes, was credited with inventing all languages, thus giving a kind of legitimacy to foreign customs and culture.

}
\hspace*{\fill} --- \textbf{The Civilization of Ancient Egypt}, \textit{2000}, Paul Johnson\cite{johnson2000civilization}
\end{quotation}
\vspace{1em}
\end{minipage}


We see here, the indications that despite not having originally been a part of the core pantheon of deities in Egypt, and yet, the need to write, or rather, the need to work formal magick, possibly introduced to Egypt by foreigners or perhaps attributed to an originally `alien deity', eventually took its rightful place among the most revered gods, and unto this day, we might not be able to easily dismiss the role that deities such as Thoth and his priesthood played in making modern writing and much of civilization possible. Perhaps Johnson puts it best, when he informs us that:



\noindent
\begin{minipage}{1\textwidth}
\vspace{1em}
\begin{quotation}
\noindent {\ttfamily

And, to judge by tomb biographies, most men who made their way to the top of the State served at some time in roles which demanded a degree of literacy. Literacy was clearly an enormous advantage in the highly centralized, and therefore highly bureaucratic, Egyptian theocracy.\\

But if most important men in Egypt were partly literate, the complexity of the written language, and the multiplicity of the scripts, created a need for professional writers or scribes. Each important religious establishment had a `House of Life', which was a scriptorium and library. Not all priests were professional scribes: scribal priests and lector-priests formed specialized branches of the clergy. Each department of government had its own special scribes: army scribes, navy scribes, treasury scribes, and so forth, who tended to develop specialized scripts of their own. There were business scribes and accountant scribes too. Scribes had their own god, Thoth, the baboon, sometimes portrayed with an ibis-head.

}
\hspace*{\fill} --- \textbf{The Civilization of Ancient Egypt}, \textit{2000}, Paul Johnson\cite{johnson2000civilization}
\end{quotation}
\vspace{1em}
\end{minipage}


And so we start to appreciate the central role of writing and language systems in all of government, religion and even domestic culture! But that's not all.. Johnson does add some few details here, that might further illuminate the practice and style with which these ancient scribes approached their craft:




\noindent
\begin{minipage}{1\textwidth}
\vspace{1em}
\begin{quotation}
\noindent {\ttfamily

They wrote sitting with their legs crossed before them using palettes of wood, with recesses for black and red inks (sometimes other colors); they kept their brushes --- pens, after the Greeks introduced them in the third century BC --- in recesses in the middle, sometimes with a sliding wooden cover, so that these boxes, which they carried in satchels, were the same in all essentials as the pencil boxes in use in Western schools until a few years ago.

}
\end{quotation}
\vspace{1em}
\end{minipage}




\begin{table}[H]
  \begin{tabular}{|p{0.95\textwidth}} % Left border only
    \hline
\begin{figure}[H]
  \begin{center} % Rotate a single page of a PDF by 90 degrees
   \includegraphics[angle=0, width=1\textwidth]{resources/ivory_towers}\\
  \end{center}
\end{figure}\\
    \cline{1-1} % Bottom border only
  \end{tabular}
\end{table}



 Thus, we can surely appreciate that to be a well-endowed modern magician, one surely ought also have a good taste for not only writing, but also for practicing their craft with the care and attention that our ancient predecessors gave to their craft and profession. Moreover, concerning the importance of picking up, polishing and creatively using a [writing] system or language, he goes on to inform us thus:
 
 
 
 \noindent
\begin{minipage}{1\textwidth}
\vspace{1em}
\begin{quotation}
\noindent {\ttfamily

As among the Christian clergy later, scribes were recruited from even the lowest classes --- especially orphans --- and it was one of the ways in which a poor man could set his children's feet on the bottom ladder of social advancement.


}
\end{quotation}
\vspace{1em}
\end{minipage}



But it wasn't so easy, and not everyone could pick up a language and master it...



 \noindent
\begin{minipage}{1\textwidth}
\vspace{1em}
\begin{quotation}
\noindent {\ttfamily

Scribal exercises, on papyrus, board and other materials, form one of the largest categories of surviving writings from ancient Egypt, and it is plain from them that acquiring professional status as a scribe was a very long and arduous business --- devoted chiefly to the copying of classical texts and didactic exercises --- whose rigours the scribes and their teachers justified to themselves by dwelling on the dignity and security of their profession. 


}
\end{quotation}
\vspace{1em}
\end{minipage}


Which reminds us that, to become a scribe, much as it was not easy, and that it was jealously guarded, brought with it not just the comfort of being in a class different from the ordinary, but, and as we see in the next excerpt, also had its certain rewards:



 \noindent
\begin{minipage}{1\textwidth}
\vspace{1em}
\begin{quotation}
\noindent {\ttfamily

Scribes, they boasted, were exempt from the \textit{corv\'ee}, from active military service, and from land taxes. Thus an Egyptian learning proverb states: ``What you gain in one day at school is for eternity. The work done there lasts as long as mountains." Or again: ``Do you not carry a palette? That is what constitutes the difference between you and a man who has to pull an oar." Or yet again: ``Put writing in they heart [i.e. learn to write], so that thou mayest protect thine own person from any kind of manual labour, and be a respected official."


}
\end{quotation}
\vspace{1em}
\end{minipage}


Not that the author is advocating against people whose well-being stems from doing mundane or manual work... People working in factories, in quarries, fishermen, blacksmiths, builders or masons and such... But that, by at least investing some effort into picking up some form of [sacred] literacy, especially the kind being taught in this book, one surely shall elevate and transform their previously lowly craft or profession, into something of finer and a higher [godly] kind. History in Europe confirms this for people like the original stone masons!

\vspace{2em}

Finally, and back to the point we raised earlier, that the development of writing systems in Africa underlies much of later human progress and civilization, Johson has this to say:


  \noindent
\begin{minipage}{1\textwidth}
\vspace{1em}
\begin{quotation}
\noindent {\ttfamily

The Neoplatonists, like Plato himself, were actually looking for metaphysical short-cuts to the tedious business of acquiring knowledge by exact observation and logical proof --- what we call science. They thought they could fathom the secrets of the universe in their own minds provided they had the right key and methodology. Plotinus, the most influential of the Neoplatonists, believed and taught that the Egyptians had discovered (or perhaps had been given by `Theuth' or Hermes) this methodology, and put it into hieroglyphs. Thus they wrote with separate pictures of individual objects --- not only letters expressing sounds and syllables --- and these pictures did not directly portray the objects they apparently represented but penetrated to the very essence of things and encapsulated knowledge possessed only by the gods. The Egyptians had thus discovered `pure' philosophy, uncomplicated and unclouded by the barrier of an alphabetic and phonetic language. This material was embodied in a collection called \textit{Corpus Hermeticum}, reputedly by `Hermes Trismegistus' compiled in the late antiquity but supposed, even then, to go back to prehistoric times.\\


}
\end{quotation}
\vspace{1em}
\end{minipage}


and did this knowledge and ancient wisdom continue to allure and draw hordes of modern elites and especially magicians or perhaps the Illuminati...

 \noindent
\begin{minipage}{1\textwidth}
\vspace{1em}
\begin{quotation}
\noindent {\ttfamily

The medieval Christians in the West knew nothing of these texts at first hand... In 1414, fragments of Ammianus, the last great historian of antiquity, turned up in a German monastery. He referred to the hieroglyphs on the obelisks brought to imperial Rome and summarized the view that throughout ancient times scholars had drawn on the ancient, god-given knowledge of the Egyptian seers and priests.\\\


...Egyptian knowledge, or pseudo-knowledge, operated at a number of levels simultaneously. Until the seventeenth century, when an unabridged fissure opened between empirical science, on the one hand, and metaphysics, astrology and esoteric knowledge on the other, men saw natural philosophy as a \textit{continuum}. Even as late a figure as Isaac Newton thought the fissure could be crossed. Hence in the sixteenth century, a great mathematician and physicist like John Dee studied the Hermetic texts with fervent hope... imperial Prague was both the leading scientific center in Europe and the place where Egyptian Hieroglyphics and the secrets they were believed to contain were most eagerly scrutinized.


}
\end{quotation}
\vspace{1em}
\end{minipage}



\begin{table}[htp]
  \begin{tabular}{|p{0.95\textwidth}} % Left border only
    \hline
\begin{figure}[H]
  \begin{center} % Rotate a single page of a PDF by 90 degrees
   \includegraphics[width=1\textwidth]{resources/scanned_example_magickal_art_magickal_languages_in_comics_part2}\\
      \vspace{2em}
  \end{center}
\end{figure}\\
    \cline{1-1} % Bottom border only
  \end{tabular}
\end{table}



So, both mainstream occultists, but also empirical scientists were passionate about and engrossed substantially in the study of these ancient pictograms...


 \noindent
\begin{minipage}{1\textwidth}
\vspace{1em}
\begin{quotation}
\noindent {\ttfamily

At the level of allegoric art, compendia of hieroglyphic signs, real and invented, such as Colonna's \textit{Hypnerotomachia Polifil} (2499) and Valeriano's \textit{Hieroglyphica}, were used by court artists throughout the sixteenth and seventeenth centuries. Masons and other secret societies such as the Rosicrucians used these signs and symbols also, especially in relation to the Roman obelisks; they were among the first to relate the esoteric knowledge of ancient Egypt to its physical memorials in stone, and to the mythology of Egyptian religion


}
\end{quotation}
\vspace{1em}
\end{minipage}





\begin{figure}[H]
  \begin{center}
   \includegraphics[width=1\textwidth]{resources/kids_in_ancient_khem_studying_rubics_games_LEARNING_B.jpg}\\
  \end{center}
\end{figure}
   


{%\LARGE
%\hl{TODO: ADD SOME HIEROGLYPHICS PHOTOGRAPHS from WIKIPEDIA HERE}

}


\subsection{Modern Use of Visual Languages in Magick}


Away from primitive man, note that visual languages are heavily utilized in modern magical systems and practices --- both privately and in public, and for purposes and levels of applications spanning all the way from basic and near mundane (such as the use of a ``heart" symbol in a modern social-media account profile name, tag-line or inline a text-message so as to attract or influence peers and onlookers towards ``loving" the one thus communicating) to sophisticated and complex political, commercial and/or religious mind-programming of the citizenry (partisan symbolism and national identity for example), consumers (subliminal messaging in advertising and branding) and/or believers or followers (faith-specific symbols of God, use of color to depict or distinguish spiritual powers, etc.) respectively.



\begin{figure}[H]
  \begin{center}
   \includegraphics[width=1\textwidth]{resources/hieroglyphics-ancient-egypt-time-machine-images-examples---ANCIENT-SEX-MAGiCK-temple.jpg}\\
  \end{center}
\end{figure}



\vspace{2em}


For the practicing magician, one interesting case of where use of visual writing or visual communication --- especially, and \textbf{specifically pictorial communication} and not sounds or mere verbal writing as with ordinary alphabets such as the commonplace Latin Alphabet, can be seen in a ritual recipe that 20$^{th}$ century magician and author, \textbf{Donald Michael Kraig} gives in his treatment of use of \textbf{Sigilization in Sex Magick}. We shall not attempt to reproduce the entire ritual formula here, nor attempt to provide all the essential theory and context necessary to make these methods sensible, but shall focus on just the essential bits given our present discussion:



\noindent
\begin{minipage}{1\textwidth}
\vspace{1em}
\begin{quotation}
\noindent {\ttfamily

Let both participants be aware that a magickal act, \textbf{a spiritual act} is about to be performed, not just a common act of sexual intercourse.

STEP ONE. Let both participants know the purpose of this act. A divination should be done, with both persons present, to insure the ``karmic correctness" of the magickal act.

STEP TWO. Let a suitable sigil, representing the purpose of the magickal act, be designed. Although a sigil taken from a grimoire will do, designing an original sigil is a good idea. The system of A. O. Spare is quite good for this.

STEP THREE. Let large versions of the above sigil be made and placed around the room. This must include the ceiling so that no matter which way you look you will see the magickal sigil.

}
\hspace*{\fill} --- \textbf{Modern Magick}, \textit{2010}, Donald Michael Kraig\cite{kraig2010modern}
\end{quotation}
\vspace{1em}
\end{minipage}


Thus, we see that, it is not a by-the-way, or even a last-step procedure, but is actually at the CORE of how modern magick might be approached or performed. DMK does help us here, by bringing up the name of another 20$^{th}$ century magician and authority in both the creative and magickal fields --- \textbf{Austin Osman Spare}!


\fbox{\begin{minipage}{\textwidth}
\LARGE

\textbf{BRIEF BIO: Austin Osman Spare, ``Grandfather" of Chaos Magick:}\\

\textbf{Austin Osman Spare} (1886--1956) was an English artist and occultist, born into a working-class family in London. A gifted draughtsman from an early age, he won a scholarship to the Royal College of Art and exhibited at the Royal Academy while still a teenager. Influenced by Symbolism and Art Nouveau, Spare’s artwork was noted for its precise line work and often unsettling imagery, depicting fantastical, monstrous, and erotic themes. His early success drew comparisons to Aubrey Beardsley, but Spare soon turned toward esoteric interests, pioneering techniques such as automatic drawing, automatic writing, and the creation of sigils, which were rooted in his theories about the interplay between the conscious and unconscious mind. Though briefly associated with Aleister Crowley and the Hermetic Order of the Golden Dawn (A.*.A.*.), he rejected organized occultism in favor of his own visionary approach. His writings, particularly \emph{The Book of Pleasure} (1913), became foundational for later currents of chaos magic, ensuring his reputation as one of the most original occult thinkers of the twentieth century \cite{baker2014spare}\cite{hine1995condensed}.


\end{minipage}}
\\



\begin{table}[H]
  \begin{tabular}{|p{0.95\textwidth}} % Left border only
    \hline
\begin{figure}[H]
  \begin{center} % Rotate a single page of a PDF by 90 degrees
   \includegraphics[height=0.9\textheight]{resources/scanned_example_magickal_art_exploring_light}\\
   \vspace{2em}
  \end{center}
\end{figure}\\
    \cline{1-1} % Bottom border only
  \end{tabular}
\end{table}



\begin{table}[H]
  \begin{tabular}{|p{0.95\textwidth}} % Left border only
    \hline
\begin{figure}[H]
  \begin{center} % Rotate a single page of a PDF by 90 degrees
   \includegraphics[width=1\textwidth]{resources/scanned_example_use_ozin_for_shrine_archictecture_notes_iona_egyptian_magick}\\
   \vspace{2em}
  \end{center}
\end{figure}\\
    \cline{1-1} % Bottom border only
  \end{tabular}
\end{table}


\section{The OZIN Language System}
\label{SECOZINSYSTEM}

\begin{defn}[The \textbf{Extended Latin Alphabet}, $\psi_{09az}$]
\label{DEFXLATINALPHABET}
The 26 symbols of the standard Latin Alphabet, $\psi_{az}$ as expressed in \textbf{\hyperref[EQLATINALPHABET]{Equation \ref{EQLATINALPHABET}}}, prefixed by the 10 symbols of the standard base-10 symbol set, $\psi_{10} = \psi_{09}$, fully specify the symbol set we are calling \textbf{The Extended Latin Alphabet}. Equivalently:

Given

\begin{equation}
\label{EQDECIMALSYMBOLSET}
\psi_{09} = \langle 0, 1, 2, 3, 4, 5, 6, 7, 8, 9 \rangle
\end{equation}

Then


\begin{multline}
\label{EALPHANUMSYMBOLSET}
\psi_{09az} = \psi_{09} \cdot \psi_{az} =\\
 \langle 0, 1, 2, 3, 4, 5, 6, 7, 8, 9, a, b, c, d, e, f, g, h, i, j, k, l, m, n, o, p, q, r, s, t, u, v, w, x, y, z \rangle
\end{multline}


The special symbol set $\psi_{09az}$, which, based on \textbf{Theorem 2} in \cite{base36paper} we might also refer to as the \textbf{Base-36} symbol set, $\psi_{36}$, is also equivalently, the \textbf{Extended Latin Alphabet} for our purposes in this manuscript and future purposes.
\end{defn}

\vspace{2em}

And with that definition out of the way, we can then formally define the ozin language as such:

\vspace{2em}


\begin{transf}[The \textbf{Magical Language \texttt{OZIN}}]
\label{TRANSFOZIN}
If $\Theta^n$ is a sequence of $n > 0$ symbols (the original message) spanning the \textbf{Extended Latin Alphabet} or the symbol set $\psi_{36}$, 

then the following transformation:\\

\begin{trans}
\label{TRANSOZIN}
$\Theta^n \xrightarrow{O_{lozin(\cdot)}} \Theta^* = \Omega^n : \mathbb{N} \times (\psi_{36} \rightarrow \psi_{ozin}) : \mathbb{N} \times \psi_{ozin};$\\
$\invpi(\Theta^n) = \invpi(\Theta^*) = \invpi(\Omega^n) = n$\\
$\land \quad \forall \theta_{i \in [1,n]} \in \Theta^n \quad \exists \omega_{j \in [1,n]} \in \Omega^n: \theta_i \longleftrightarrow \omega_j \quad \land \quad \invpi(\theta_i \in \psi(\Theta^n)) = \invpi(\omega_i \in \psi(\Omega^n)) = 1$\\
$\land \quad \forall \alpha \in \psi(\Theta^n): \invpi(\alpha \in \Theta^n) = f_\alpha \implies \beta \in \psi(\Omega^n): \invpi(\beta \in \Omega^n) = f_\alpha  \quad$\\
$ \text{ iff } \quad \alpha \in \psi_{36} \longleftrightarrow \beta \in \psi_{ozin}$\\
$\land \quad \overset{>}{\Theta^n} \implies \overset{>}{\Omega^n}$\\
$\land \quad \tilde{A}(\Theta^n \rightarrow \Omega^n)_{\psi_{36}} = 0 \qed$
\end{trans}

\vspace{2em}

Which is one way of saying, the transformation of a message from the Extended Latin Alphabet into the OZIN language, is guaranteed to always produce/generate a derivative message --- $\Theta^*$ that has the following properties:\\


\begin{multline}
\label{EQOZIN}
\forall \alpha \in \Theta^n \implies \beta \in \Omega^n: \mathbb{N} \times \psi_{ozin}
\end{multline}

Where $\psi_{ozin}$ is an extension of the symbol set $\psi_{oz}$ first introduced in \textbf{Figure 10.2} of \cite{lutalo_2025_trans_genetics}\footnote{For those that had not yet become familiar with on-going research at Nuchwezi concerning development and formalization of the new mathematics of Transformatics\cite{transformatics} \cite{Lutalo2025_transformatics_thesis}, the OZIN language we are treating of here, was actually first well introduced in a mini-treatise on applying Transformatics in GENETICS\cite{lutalo_2025_trans_genetics}, and in which work, focus was mostly placed on using just the number-subset of $\psi_{36}$, so that then, in a hypothetical genome expression system --- the \textbf{Lu Genome Expression System}(LGES), it would be possible to take a DNA or na-Sequence describing some species or organism or entity, and based on some sound logic relating to how living things manifest based on their genetic code, render or approximate/predict a visual-spatial expression that best describes the simplest form of that resulting or associated entity. The key assumption then, was that, unlike the typical DNA code that spans the symbols A-T-G-C, that these codes would need to be mapped to a sequence of just digits. There might be special occult science applications for that model too, however, in this grimoire, we shall not treat of that application of OZIN in GENETICS, and shall instead focus on its use as a normal magickal language.} , and which is now expressed in full as depicted in \cite{lutalo_2025_ozin} and \textbf{\hyperref[FIGOZINALPHABET]{Figure \ref{FIGOZINALPHABET}}} as a \textbf{one-to-one} mapping from \textbf{Base-36 symbol set}, $\psi_{36} \rightarrow \psi_{ozin}$.
$\qed$

\vspace{2em}

And so that, the resultant [transformed] message, $\Theta^* = \Omega^n \approx \Theta^n$, is not only semantically equivalent to the original message, but also its length \textbf{is exactly equivalent to that of the original message}, with the only basic different being that its message is expressed using the alternative symbols or visual glyphs different from the original symbol set, but otherwise corresponding to it item-wise. Any such instance of text thus transformed is then an expression in the language \textbf{OZIN}.

\end{transf}


\vspace{2em} For clearer appreciation and study of the $\psi_{ozin}$ alphabet, let us consider its elements when split up into two figures --- \textbf{\hyperref[FIGOZINALPHABETNUMBERS]{Figure \ref{FIGOZINALPHABETNUMBERS}}} for just the numbers: mappings from $\psi_{10} = \psi_{09} \rightarrow \psi_{ozin}$, while \textbf{\hyperref[FIGOZINALPHABETLETTERS]{Figure \ref{FIGOZINALPHABETLETTERS}}} for the letters only: mappings from $\psi_{az} \rightarrow \psi_{ozin}$.



\begin{figure}[H]
  \begin{center}
   \includegraphics[width=\textwidth]{resources/NUMBERS_OZINLANGUAGE.pdf}\\
   \caption{The NUMBER OZIN ALHABET: $\psi_{10} = \psi_{09} \rightarrow \psi_{ozin}$}
  \label{FIGOZINALPHABETNUMBERS}
  \end{center}
\end{figure}



\begin{figure}[H]
  \begin{center}
   \includegraphics[height=1\textheight]{resources/LETTERS_OZINLANGUAGE.pdf}\\
   \caption{The LETTERS OZIN ALHABET: $\psi_{az} \rightarrow \psi_{ozin}$}
  \label{FIGOZINALPHABETLETTERS}
  \end{center}
\end{figure}


Next, let us consider some of the interesting magical applications of the OZIN language.


\section{5 Examples of Applying OZIN}
\label{SECEXAMPLEOZIN}


In the case of how to apply OZIN, we shall first take a step-back, and return to the helpful and time-proven techniques of \textbf{SIGIL Magick} that would best underlie most of the esoteric work involving correctly applying this (and other) visual magickal languages. We shall mostly focus on the core of what makes sigil magick work --- especially as taught in schools and in traditions that focus on or champion modern chaos magick\cite{hine1995condensed}.


\subsection{SIGIL MAGICK: The [core] Method of Chaos Magick in Brief}
\label{SECSIGILMAGICKCORE}




In the appendix (\textbf{\hyperref[APPENDCHAOSMAGICK]{Section \ref{APPENDCHAOSMAGICK}}}) we offer a wider comparative summary of what Chaos Magick is all about; not because it is the system of magick we would most recommend for those working with this grimoire, but because, of all modern and traditional magick systems, it is one of few that resonates most with the methods and ethos of creative approach to ritual magick that this grimoire champions. Also, it shall help to note that, sigil magick, or rather, the method of pursuing results in magick via use of sigils, is a core, perhaps mainstay aspect of Chaos Magick, and thus, and especially while exploring magickal languages such as OZIN, one shall want to familiarize themselves with what it is all about first. Sufficient sources and reference works for the interested reader and student have been provided in the bibliography.

\vspace{2em}


That said, note that the gist of the chaos magick system dwells in ensuring that one can attain results with the most minimal of operating sets; at miminimum, all that is crucial is to be able to distill your intent into a visual glyph or mantra, charge it, dispatch it into the universe, and then let go --- for results to naturally manifest (sooner or later).

\vspace{2em}


Originally attributed to 20th century occultist A.O. Spare, chaos magick, and particularly its sub-genre of sigil magick has come a long way, and has seen siginificant contributions being made by prominent occultists such as Peter Carrol (especially in formulating a system around it, and promulgating the core ideas), but also, and arguably so, one can trace the development of sigil magick to earlier traditions and magick systems such as the creation of and use of magick seals (planetary seals, angelic seals, demon or spirit seals, etc) ideas which have been surfaced in the past by occultists such as Cornelius Agrippa, John Dee and in the far ancient past as we have already discussed in the introduction of this chapter, all the way back to primitive man and their use of pictograms and hieroglyphics as a form of magick.

\vspace{2em}

For the modern student though, we can distill the essential method of applying Sigil Magick as the following core process --- which, for consistency, we shall summarize as using a computer program or algorithm:\\


\begin{alg}[The \textbf{SIGIL MAGICK ALGORITHM}]
\label{ALGOSIGILMAGICK}
\begin{enumerate}
$ $\\

\item \textbf{INTENT:} Have a clear Intention --- a desire, a wish, an objective, a goal, an aim. 

\item \textbf{STATEMENT:} Formulate it well and unambigously, and especially write it down into a single clear \textbf{sentence in the present tense}, for example using the Extended Latin Alphabet ($\psi_{36}$). At this point, you might want to be writing using \textbf{a PENCIL}, not a pen.

\item \textbf{REVISE:} Read it out aloud to yourself to see if it makes sense, revise or edit it till it is exactly what you want. At this moment too, take note of ``caveats": is the intent attainable? shall it not create problems for you or someone? is it complete? is it ethical or would it not create a karmic debt? etc. Sometimes it might help to also visualize having already obtained the thing you have written down as your intent, and then see if it is exactly what you wanted. If not, then return to it, and revise it till it is tick. Most importantly, ensure that your intent is expressed not as a [future] wish or a [past] memory, but as an affirmation of something already and actually happening in the present as you read/write it.

\item \textbf{FINALIZE:} Having clearly thought through your intent (also called ``Statement of Intent"; SOI), then write it down clearly, in big enough letters, possibly using \textbf{a PEN}.

\item \textbf{REDUCE:} That SOI that you have needs to be transformed into a form that you can then readily apply magick to. This shall depend on the system of magick you wish to use, and your experience or tastes, however, for sigil magick, which is a visual magick system, you want to be able to express your SOI as some picture and/or sound at minimum. The simplest method, attributed to A.O. Spare, is to take the SOI, and reduce it to a phrase that is somewhat like \textbf{a modal sequence statistic}\cite{Lutalo2025_transformatics_thesis} or \textbf{lexical basis}\cite{Lutalo2024TEATAZ} or another method\footnote{We have noticed that most contemporary magick systems and schools, including so-called best-of-the-league-illuminati, didn't actually have a formal system nor mathematics for correctly constructing sigils until the IoN invested tirelessly into making this present formalism happen.}; \textbf{basically, the SOI reduced}; for example, to only its unique characters or letters, initially sorted in the order of first-occurrence. This becomes your ``SOURCE".

\item \textbf{TRANSFORM:} Now take that Source, and turn it into \textbf{a pictogram} (Sigil) and/or \textbf{a sonogram} (Mantra). This needs be done carefully, but also creatively; for example, start by simply joining the remaining letters in the SOURCE into a single connected picture, each of the letters connected to the others in some creative way. Simple and iconographic drawings that look like logos or outline pictures are typically preferable than complex near-realistic drawings. At this point, you want to be using \textbf{a MARKER or PAINT BRUSH}. For Mantras, you want it to be pronounceable, repeatitive and true to the ``Source" structure or composition. That becomes your ``Payload".

\item \textbf{CHARGE:} Depending on context, your experience or knowledge of magick and taste, take the Payload you have, and while in an ecstatic state (also known as ``Gnosis" --- an intense, altered state of consciousness), give sufficient attention to the Payload and let it speak directly to your unconscous mind; for cases where you also have a Mantra Payload, effectiveness in charging might be enhanced by visually focusing on the Sigil Payload, while simultaneously vibrating/chanting the associated Mantra Payload. Done well, these two acts on their own can induce the gnosis state required, while also effecting the necessary charging of the intent, and this step might be conducted by more than one person (peers or assistants) even though the Payload being operated on is just 1.

\item \textbf{DISPATCH:} For Payloads in visual form such as SIGILs, after sufficient charging, then cast off the payload into the universe... You might literally burn the sigil if it is on say a piece of paper, but others might burry it into the earth, cast it off into the ocean or conceal it inside some \textbf{accumulator device} --- for example, a small magick pot having fresh blood in it, and then into which you dip the sigil (either burnt to ashes[for destructive magick] or left as is on paper [for creative magick]) and cover or bury the pot somewhere to kick-start the chain-reaction. Importantly, it is preferable to let go, and forget --- do not try to keep either the original written intent nor the transformed ``Source" in your mind at this moment. This, so as to avoid problems such as ``Lust for Results", ``Conscious Interference", etc. For aural payloads such as SOIs cast as Mantras, perhaps merely finish chanting or vibrating the mantra and forget (unless part of your intention was to create a reusable payload such as a mantra for use every morning or before starting a certain act).

\item \textbf{RECORD:} [Recommended] Immediately, or perhaps later on, record in-brief, into your journal or magickal diary, the operation that was conducted, and what you results if any, you have attained or observed. This, so that you can later evaluate the effectiveness of your operation, and so that in the future you known where or how to make necessary enhancements or if you might need to repeat the ritual operation [differently]. Also helps to document and keep-track of your esoteric experiments and progress for future analysis or reference.

\end{enumerate}
\end{alg}


\vspace{2em}

For all practical purposes, \textbf{\hyperref[ALGOSIGILMAGICK]{Algorithm \ref{ALGOSIGILMAGICK}}} presents and summarizes all the essential steps that both a neophyte still learning or just starting to explore practical ritual magick might want to leverage, but also shall serve as a great foundation for any serious initiate or adept of the esoteric science that wishes to leverage the modern methods of chaos and sigil magick as part of their magical repertoire and system. For those interested in diving deeper or who wish to explore alternative approaches to this system, another recommended resource is \textbf{Phil Hine}'s \textit{Condensed Chaos} book\cite{hineCondensedChaos}. But also, and for those that have followed our earlier formulations and magick systems --- see \textbf{Church of Dance Eternal}(CODE)\cite{lutalo_2025_ogf}: ceremonial magick, fraternal workings, etc.; \textbf{TRP}\cite{transformation_rosary_rite}: prayers for magicians, kaballah, fraternal workings,.. or \textbf{IP Govern Rite}\cite{Lutalo2025_govern_rite}: masonic rituals, public ceremonies, political magick, conferences and group workings, etc. ---- you shall come to appreciate that the method we lay down in \textbf{\hyperref[ALGOSIGILMAGICK]{Algorithm \ref{ALGOSIGILMAGICK}}}, despite being friendly to many ad-hoc method practioners as many contemporary chaottes and urban magicians might like, and yet, it can properly be placed into the context of grander [\textit{more traditional}] ritual programs and ceremonies such as we have come championing in the paradigm of \textbf{Ceremonial Chaos Magick} (CCM). Overall, it is another noteworthy addition to your repetoire of \textbf{Computational Mysticism}\footnote{Essentially, a method of approaching most magick or rituals as though one were a processor or operator executing some program that must always result in some tangible result.}



\begin{table}[H]
  \begin{tabular}{|p{0.95\textwidth}} % Left border only
    \hline
\begin{figure}[H]
  \begin{center} % Rotate a single page of a PDF by 90 degrees
   \includegraphics[height=0.7\textheight]{resources/scanned_example_magickal_art_ozin
}\\
  \end{center}
\end{figure}\\
    \cline{1-1} % Bottom border only
  \end{tabular}
\end{table}



\subsection{EXAMPLE 1: APPLYING SIGIL MAGICK using The OZIN Language}


Assuming you need something, and that after evaluating all other ways to obtain it, you find that it is certain, you wont get it. So then, you are essentially faced with a problem such as this:

\vspace{2em}

\begin{prob}[The \textbf{Manifesting Something Otherwise Impossible} Problem\footnote{Essentially, the kind of problems which anything but magick can not hope to solve.}]
\label{PROBMANIFEST}
\textbf{An Operator} $\lambda(\cdot)(\psi_\beta)$ is a magician that can process anything in case it is expressible in any magical language based on the alphabet $\psi_\beta$ --- for example, the alphabet for the \textbf{extended latin alphabet}:(ELA), $\psi_{36}$, in which case, $\beta = 36$.

\vspace{2em}


\begin{defn}[The \textbf{OIE}: Otherwise Impossible Event]
\label{DEFOIE}
 Any \textbf{Otherwise Impossible Event}, is some descriptor sequence --- potentially a \textbf{Genome Sequence} for any observable entity or thing --- which, when processed via the operation:

\vspace{2em}

{\Large


\begin{trans}
\label{TRANSOIE}
$e_i:\lambda(e_i)(\psi_\beta) \rightarrow \emptyset$
\end{trans}
 
 }
\end{defn}




\vspace{2em}


Results in nothing! The \textbf{Manifesting Something Otherwise Impossible} Problem (MSOIP), challenges the magician thus concerned, to come up with some Alternative Processing Program (APP), in which, when presented with the OIE $e_i$, it then becomes possible to produce, generate or otherwise manifest some result $e^*_i$, that is a non-empty expression of the \textbf{desired} event $e_i$.

\vspace{2em}


\begin{soln}[\textbf{Manifestation via SIGILIZATION}]
\label{SOLNMANSIGIL}
To remedy the problem identified when \textbf{\hyperref[TRANSOIE]{Transformation \ref{TRANSOIE}}} always results in an empty set --- meaning, the wanted result isn't being obtained, a \textbf{magical solution} is a modification of the original problem into/as such:

\vspace{2em}


{\Large

\begin{trans}
\label{TRANSOIESOL}
$ $\\
 $\langle e_i \rangle \rightarrow e_i:\Lambda(\lambda(e_i)(\psi_\beta), \psi_\infty) \xrightarrow{O_{\lambda}(e_*)(\psi_\infty)} \psi_{\overset{>}{e_i}} \approx  \psi_{e_i}  \quad \neq \emptyset$
\end{trans}
 
 }
 
 \vspace{2em}
 
 And thus, via \textbf{\hyperref[TRANSOIESOL]{Transformation \ref{TRANSOIESOL}}} we solve the MSOIP by a process that involves modifying or augmenting the original constrained operator by applying a higher-order operator $\Lambda(\lambda(\langle e \rangle)(\cdot))(\psi_\infty)$ to the problem, and that can then enable that otherwise impossible event to be manifested successfully, as is depicted in \textbf{\hyperref[TRANSOIESOL]{Transformation \ref{TRANSOIESOL}}}. 
 
 \vspace{2em}
 
 In more practical terms, such an operation might proceed as such:
 
 \vspace{2em}
 
 \begin{enumerate}
 \item The OIE $e_i$ is expressed in its original language or base as $\langle e_i \rangle:\mathbb{N} \times \psi_\beta$ --- this might for example be an expression in intelligible natural human languages such as English, Runyoro or Chinese, \textbf{BUT} might also be in some alphabet or language other than natural or immediately intelligible. 
 
\item  We then take that initial expression of the wanted, and re-express it in some intelligible and or straight-foward parsable language such as plain English. For example, writing or computing a sentence or phrase --- essentially a \textbf{Statement of Intent} (SOI) --- in an expression form spanning the alphabet \textbf{ELA}: $\psi_{36}$. We shall denote such a clear and specific SOI with $\langle e_i \rangle_{\psi_{36}}$.
 \item So that, then, instead of processing the SOI in its OIE form, $\langle e_i \rangle_{\psi_\beta}$, we instead compute an Alternative Event Expression (AEE) corresponding to the OIE thus concerned, by casting the event in a different, otherwise better input payload form that the original operator can then process or which can be processed on their behalf by the higher-order operator, and so that it is then guaranteed that they shall produce the required non-empty result --- essentially, \textbf{making the impossible possible} --- via magick. Such an alternative approach to the solution is as depicted in:
 
 \vspace{2em}

\begin{trans}
\label{TRANSOIESOLOZIN}
 $\langle e_i \rangle_{\psi_\beta} \rightarrow \langle e_i \rangle_{\psi_{36}} \rightarrow e_i:\Lambda(\lambda(e_i)(\psi_{36}), \psi_{ozin}) \xrightarrow{O_{\lambda}(e_*)(\psi_\infty)} \psi_{\overset{>}{e_i}} \approx  \psi_{e_i} $
\end{trans} 

\vspace{2em}

\textbf{NOTE:}
It essentially is like helping a kid that only knows to read base 10 numbers, to correctly call a friend whose name is the sound of calling a certain number in base 10, and yet, the only information the kid had before the solution, was their friend's t-shirt back-tag, that reports the correct number, only that is it written in base-2. By using for example an app that converts from binary to decimal --- or perhaps modifying the kid by teaching them how to convert from binary to decimal for themselves, the originally constrained or limited operator (the kid) becomes endowed and empowered via the solution [app or transformation method], so that they can then naturally call their friend and she comes, or direct them to do anything they wish or at least to cause them to react, because they now possess a method of evoking a willed reaction from them by having their correct name or handle --- or a guaranteed means of affectant communication. By expressing SOIs via \textbf{the OZIN Language} ($\psi_{ozin}$) then, we are proposing a solution to MSOIP kinds of problems; involving failing to manifest events simply because they cant be processed [correctly] by the magician in their original [or other] forms. Of course, for the case of manifesting things via magick, it is being assumed that the operation is approached as per \textbf{\hyperref[ALGOSIGILMAGICK]{Algorithm \ref{ALGOSIGILMAGICK}}}.
 \item $\qed$
 \end{enumerate}
\end{soln}
\end{prob}



\subsubsection{OZIN MAGICK}


For purposes of helping those unfamiliar with the system and methods of our general solution in \textbf{\hyperref[SOLNMANSIGIL]{Solution \ref{SOLNMANSIGIL}}} learn how to apply or leverage the theory thus presented, we shall concretize the idea by working on a realistic example case of \textbf{MSOIP} whose SOI is expressed explicitly as:


``By The Time The Sun Sets Today, I Want To Have An Extra 100 Dollars On My Account" (good, but not affirmative enough). A better version, more manifestation-friendly being:

{\LARGE

%\begin{figure}[H]
  \begin{tcbverbatim}[title=A Statement of Intent for a hypothetical MSOIP]

BY THE TIME THE SUN SETS TODAY I HAVE AN EXTRA 100 DOLLARS ON MY ACCOUNT

  \end{tcbverbatim}
%\end{figure}
}

\vspace{2em}


For purpose of helping you appreciate the formal process we are about to undertake, consult the worksheet depicted in \textbf{\hyperref[FIGEXASOISIGILA]{Figure \ref{FIGEXASOISIGILA}}}, and which is all of what is practically crucial for sigil magick and for our case, applying the OZIN magickal language to manifesting [the impossible] via sigilization [and chaos magick].

\vspace{2em}



\begin{table}[H]
  \begin{tabular}{|p{0.95\textwidth}} % Left border only
    \hline
\begin{figure}[H]
  \begin{center}
   \includegraphics[width=\textwidth]{resources/scanned_soi_sigilization_process_via_ozin_language_example.pdf}\\
   \caption{Example CASE: Sigilizing SOI for Earning 100 Dollars before Day End, Using OZIN Magickal Language}
  \label{FIGEXASOISIGILA}
  \end{center}
\end{figure}\\
    \cline{1-1} % Bottom border only
  \end{tabular}
\end{table}


\vspace{2em}


And so, we see that, the process of generating the final sigil as depicted via the process shown in  \textbf{\hyperref[FIGEXASOISIGILA]{Figure \ref{FIGEXASOISIGILA}}}, involves some simple, but non-trivial mathematical processing and analysis as we shall try to help illustrate and break-down in the following formalism:\\


\textbf{STEP 1: Having the SOI expressed clearly in plain intelligible sentence spanning} $\psi_{36}$

\vspace{2em}

\begin{multline}
\label{EQSOI1}
\Theta_{soi} = \langle \text{BY THE TIME THE SUN SETS TODAY} \\\text{I HAVE AN EXTRA 100 DOLLARS ON MY ACCOUNT} \rangle
\end{multline}



\vspace{2em}

{
\LARGE
\hl{TODO: REPEAT THE MANUAL COMPUTATION ON PAPER FOR THE GIVEN SOI, USING THE NEW/UPDATED METHOD OF AN MSS BASED ON A RANK/WEIGHT SORT, AND COMPARE WITH THE OUTPUT FOR THE CLI TEA. iF THEY MATCH, THEN UPDATE THE CLI TEA TO USE THE RANK METHOD, BOTH FOR WORD AND CHARACTER MSS, THEN ALSO DO THE SAME FOR WEB TEA. THEN UPDATE THE TABLE IN FIGURE, UPDATE THE SIGIL ASSOCIATED, ALSO UPDATE ALL REFERENCES TO MSS SOI IN THE TEXT, AND WE SHALL LATER FIX ALL MSS-ASSOCIATED SOURCES FOR SIGILS AND MANTRAS ETC..

NOTE: The current method depicts an approach we might call ``WEIGHED FREQUENCY" or ``POSITION-WEIGHED FREQUENCY" --- which, for items that have the same frequency, the earliest occurrence ranks highest, but otherwise, an item that occurs later, but has higher frequency, occurs earlier. So, this somewhat feels like the original MSS, but is not. We might instead refer to this as ``POSITIONAL-MODAL SEQUENCE STATISTIC" (PMSS)

We could retain this in TEA, via the two commands:

u.: PMSS for words
u.!: PMSS for letters
u.*: PMSS for vault word
u.!*: PMSS for vault letters

then keep original MSS via:

u: MSS for words
u!: MSS for letters
u*: MSS for vault word
u*!: MSS for vault letters

}
}

\textbf{STEP 2: Reduce the SOI to the ``SOURCE" still spanning}\label{STEP2} $\psi_{36}$. In our illustration here, we are reducing using the method of computing a Modal Sequence Statistic\cite{Lutalo2025_transformatics_thesis} as shown in the computations happening in the table depicted in \textbf{\hyperref[FIGEXASOISIGILA]{Figure \ref{FIGEXASOISIGILA}}}, and which we also reproduce here as in \textbf{\hyperref[TABLESOIMSS]{Table \ref{TABLESOIMSS}}}.

\vspace{2em}


\begin{equation}
\label{EQSOISOURCE}
\Theta_{soi} \xrightarrow{O_{mss}(\cdot)} \langle TEYHSANOIMUBDRVX01LC\rangle
\end{equation}

\vspace{2em}


The value `` TEYHSANOIMUBDRVX01LC" being what we obtain, when we use the TEA programming language program code --- just \texttt{u!:} --- to compute the corresponding MSS value for our example SOI; also refer to program source-code in \textbf{\hyperref[FIGTEAEXASOISRC]{Figure \ref{FIGTEAEXASOISRC}}}.

\vspace{2em}


 %\small
  \begin{tcolorbox}[teaterminalstyle, title=TEA Program: computing MSS for SOI using TEA, breakable]
  %\begin{lstlisting}[language=TEA, caption={TP C7}, label={LSTC7}, numbers=left]
  \begin{lstlisting}[language=TEA,breaklines=true]
i!:{BY THE TIME THE SUN SETS TODAY I HAVE AN EXTRA 100 DOLLARS ON MY ACCOUNT}
u!:

#RESULTANT MEMORY STATE: (= TEYHSANOIMUBDRVX01LC, VAULTS:{})
   \end{lstlisting}
  \end{tcolorbox}
    \captionof{figure}{EXAMPLE: computing MSS (``Source") for a SOI (in ELA) using TEA}
  \label{FIGTEAEXASOISRC}


\vspace{2em}

And clearly, right here, we see why it is preferable to work on these tasks with care, and if possible, with some automation such as with a computer or computer program like the one in \textbf{\hyperref[FIGTEAEXASOISRC]{Figure \ref{FIGTEAEXASOISRC}}}, because, as you might see when you compare our manual computations in \textbf{\hyperref[FIGEXASOISIGILA]{Figure \ref{FIGEXASOISIGILA}}}, we might not have computed ``the correct" SOURCE for our given SOI. However, and luckily, this is Magick; there is typically much room for making errors or being creative, and it is not like an exact science such as Physics or Computer Science, \textbf{BUT}, and as we would prefer to argue here --- \textbf{in Computational Mysticism} especially --- it is preferable that we develop, practice and apply our computational and mathematical methods for magick, well and good enough so that even with just a pen and paper, one can correctly execute or conduct complex metaphysical operations such as we are trying to do here.

That said, note that the discrepancies between the TEA program result we obtain in the above reduction step Vs what we obtained with manual calculation on paper, might be explained by the observation that, in the manual process, we reduced \textit{strictly}; apart from not counting or evaluating for white-space in the SOI, we also eliminated all punctuation marks or symbols at the reduction step. Thus, a more tolerant SOURCE value from the computer for this SOI, might be ``TEYHOSINAMUBDVRXLC", which is what we obtain with the modified TEA program depicted in \textbf{\hyperref[FIGTEAEXASOISRC2]{Figure \ref{FIGTEAEXASOISRC2}}}. In the manual computation too, note that, some simplifications such as treating ``1" a number, the same as ``I" the letter , or ``0" the number as the letter ``O", might also cause difference... etc.

\vspace{2em}

 %\small
  \begin{tcolorbox}[teaterminalstyle, title=TEA Program: computing MSS for SOI using TEA, breakable]
  %\begin{lstlisting}[language=TEA, caption={TP C7}, label={LSTC7}, numbers=left]
  \begin{lstlisting}[language=TEA,breaklines=true]
i!:{BY THE TIME THE SUN SETS TODAY I HAVE AN EXTRA 100 DOLLARS ON MY ACCOUNT}
d!:[0-9a-z-A-z]
r!:0:O | r!:1:I #replace some number with letters
u!:

#RESULTANT MEMORY STATE: (=TEYHOSINAMUBDVRXLC, VAULTS:{})
   \end{lstlisting}
  \end{tcolorbox}
    \captionof{figure}{UPDATED EXAMPLE: computing MSS (``Source") for a SOI (in ELA) using TEA}
  \label{FIGTEAEXASOISRC2}

\vspace{2em}



\begin{table}[H]
  \begin{tabular}{|p{0.95\textwidth}} % Left border only
    \hline
\begin{figure}[H]
  \begin{center}
   \includegraphics[width=\textwidth]{resources/computing_source_from_soi_via_MSS_corrected.pdf}\\
   \caption{[UPDATED]Example CASE: Computing SOURCE for SIGIL from SOI via PMSS Algorithm}
  \label{FIGEXASOIMSS}
  \end{center}
\end{figure}\\
    \cline{1-1} % Bottom border only
  \end{tabular}
\end{table}

\vspace{2em}


And so, by reproducing that table --- \textbf{with corrections made} --- a process shown worked out manually as in \textbf{\hyperref[FIGEXASOIMSS]{Figure \ref{FIGEXASOIMSS}}} --- we see in \textbf{\hyperref[TABLESOIMSS]{Table \ref{TABLESOIMSS}}}, the correct and recommended mathematically sound, computationally-friendly method of arriving at a correct SOURCE for use in implementing the SIGIL of anything --- using, \textbf{not the original MSS method}, but rather, a new and alternative \textbf{PMSS method} --- refer to \textbf{\hyperref[SECPMSS]{Section \ref{SECPMSS}}} for details on this alternative approach to the MSS.



\vspace{2em}

\begin{table}[H]
  \centering
	\begin{tabular}[t]{|c|c|c|c|c|}
	\hline
	\textbf{\makecell{INDEX:\\$i\in[1,n]$}}& \textbf{\makecell{INPUT SYMBOL:\\$\Theta_i$}} & \textbf{\makecell{FREQ:\\$f_i = \invpi(\Theta_i, \Theta)$}} & \textbf{\makecell{WEIGHT:\\$\omega_i = (\invpi(\psi(\Theta)) - i) \times f_i$}} & \textbf{\makecell{RANK:\\$ \geq \omega_i $}} \\
	\hline

	\hline	
	1 & B & 1 & $(18 - 1) \times 1 = 17 \times 1 = 17$ & 12\\

	\hline	
	2 & Y & 3 & $(18 - 2) \times 3 = 16 \times 3 = 48$ & 3\\	
	
		\hline	
	3 & T & 7 & $(18 - 3) \times 7 = 15 \times 7 = 105$ & 1\\	
	
		\hline	
	4 & H & 3 & $(18 - 4) \times 3 = 14 \times 3 = 42$ & 4\\	

		\hline	
	5 & E & 6 & $(18 - 5) \times 6 = 13 \times 6 = 78$ & 2\\	
		
		\hline	
	6 & I & 3 & $(18 - 6) \times 3 = 12 \times 3 = 36$ & 7\\	

		\hline	
	7 & M & 2 & $(18 - 7) \times 2 = 11 \times 2 = 22$ & 10\\	
								
		\hline	
	8 & S & 4 & $(18 - 8) \times 4 = 10 \times 4 = 40$ & 6\\	

		\hline	
	9 & U & 2 & $(18 - 9) \times 2 = 9 \times 2 = 18$ & 11\\	
				
			\hline	
	10 & N & 4 & $(18 - 10) \times 4 = 8 \times 4 = 32$ & 8\\	
			
						\hline	
	11 & O & 6 & $(18 - 11) \times 6 = 7 \times 6 = 42$ & 5\\	
		
							\hline	
	12 & D & 2 & $(18 - 12) \times 2 = 6 \times 2 = 12$ & 13\\	
		
	
								\hline	
	13 & A & 6 & $(18 - 13) \times 6 = 5 \times 6 = 30$ & 9\\	
		
								\hline	
	14 & V & 1 & $(18 - 14) \times 1 = 4 \times 1 = 4$ & 14\\	
			
									\hline	
	15 & X & 1 & $(18 - 15) \times 1 = 3 \times 1 = 3$ & 16\\	
		
											\hline	
	16 & R & 2 & $(18 - 16) \times 2 = 2 \times 2 = 4$ & 15\\	
		
													\hline	
	17 & L & 2 & $(18 - 17) \times 2 = 1 \times 2 = 2$ & 17\\	
		
															\hline	
	18 & C & 2 & $(18 - 18) \times 2 = 0 \times 2 = 0$ & 18\\	
		
	
	\hline	              
\end{tabular}
  \caption{Tabular Analysis to compute $\overset{>}{\Theta}$ --- the MSS of SOI ($\Theta = \Theta_{soi}$)}
    \label{TABLESOIMSS}
\end{table}


\vspace{1em}

And so, from \textbf{\hyperref[TABLESOIMSS]{Table \ref{TABLESOIMSS}}}, we have that:

\vspace{1em}

\begin{trans}
\label{TRANSSOI}
$ $\\
$\Theta_{soi} = \langle \text{BY THE TIME THE SUN SETS TODAY,} \\\text{I HAVE AN EXTRA IOO DOLLARS ON MY ACCOUNT} \rangle \xrightarrow{O_{mss(\cdot)}} \overset{>}{\Theta_{soi}} $\\\\
$ = \overset{>}{\Theta} = \langle \text{\textit{TEYHOSINAMUBDVRXLC}} \rangle  $
\end{trans}

\vspace{2em}

Note that the method used, is well documented and illustrated in \textbf{Section 4.1} of \cite{transformatics}, and also summarized in \textbf{Proposal 5} of \cite{Lutalo2025_transformatics_thesis}, and would give us the result \texttt{TEYHSANOIMUBDRVX01LC} --- essentially, the input symbols (from the SOI) sorted in ascending order based on their computed rank (most frequent or earliest to occur for ties) as shown in \textbf{\hyperref[TABLESOIMSS]{Table \ref{TABLESOIMSS}}}. However, let us also realize that, if, instead of reducing to a modal sequence statistic, we had chosen to reduce to just the set of unique letters, with their order sorted alphabetically, then, we would essentially have the SOURCE as ``ABCDEHILMNORSTUVXY" --- a \textbf{Lexical Basis} (after reducing some numbers to letters, removing punctuation, etc). as shown using just the TEA program instruction \texttt{b!:} as shown in 

\vspace{1em}

 %\small
  \begin{tcolorbox}[teaterminalstyle, title=TEA Program: computing Lexical Basis for SOI using TEA, breakable]
  %\begin{lstlisting}[language=TEA, caption={TP C7}, label={LSTC7}, numbers=left]
  \begin{lstlisting}[language=TEA,breaklines=true]
i!:{BY THE TIME THE SUN SETS TODAY, I HAVE AN EXTRA 100 DOLLARS ON MY ACCOUNT}
d!:[0-9a-z-A-z]
r!:0:O | r!:1:I #replace some numbers with letters in the SOI

#u!: #use modal sequence statistic (for chars) as SOURCE
#returns: TEYHOSINAMUBDVRXLC

b!: #use lexical basis as FINAL SOURCE

#RESULTANT MEMORY STATE: (=ABCDEHILMNORSTUVXY, VAULTS:{})
   \end{lstlisting}
  \end{tcolorbox}
    \captionof{figure}{UPDATED EXAMPLE: computing Lexical Basis (``Source") for a SOI (in ELA) using TEA}
  \label{FIGTEAEXASOISRC2}

\vspace{2em}

And then, we can thus proceed to the next major step in the sigilization process:


\vspace{2em}

\textbf{STEP 3: TRANSFORM the ``SOURCE" into OZIN language}: that is, from $\psi_{36}$ into $\psi_{ozin}$. In our illustration here, we see depicted in the hieroglyphics-like writing at the bottom of the scanned manual computation in \textbf{\hyperref[FIGEXASOISIGILA]{Figure \ref{FIGEXASOISIGILA}}}. Equivalently, and if we for example operate on the SOI SOURCE in the form of its lexical basis, we would merely have something as depicted in   \textbf{\hyperref[FIGEXASOISIGILAPROCESS]{Figure \ref{FIGEXASOISIGILAPROCESS}}}.

\vspace{2em}



\begin{figure}[H]
  \begin{center}
   \includegraphics[width=\textwidth]{resources/sample_soi_source_in_ozin.pdf}\\
   \caption{Example CASE: PROCESS of Sigilizing SOI Using OZIN Magickal Language}
  \label{FIGEXASOISIGILAPROCESS}
  \end{center}
\end{figure}

\vspace{2em}


In fact, we see in \textbf{\hyperref[FIGEXASOISIGILAPROCESS]{Figure \ref{FIGEXASOISIGILAPROCESS}}}, the entire process from when we have a plain english SOI, to when we finally generate or draw a complete, final sigil for the thing we wish to manifest using magick. For approaching this problem manually, we see one potential final sigil, drawn using a MARKER, as depicted in \textbf{\hyperref[FIGEXASOISIGILB]{Figure \ref{FIGEXASOISIGILB}}}.

\vspace{2em}


\begin{table}[H]
  \begin{tabular}{|p{0.95\textwidth}} % Left border only
    \hline
\begin{figure}[H]
  \begin{center}
   \includegraphics[width=0.6\textwidth]{resources/scanned_soi_sigil.pdf}\\
   \caption{Example Hand-Drawn and Finished SOI SIGIL}
  \label{FIGEXASOISIGILB}
  \end{center}
\end{figure}\\
    \cline{1-1} % Bottom border only
  \end{tabular}
\end{table}

\vspace{2em}

While, the equivalent, not only simplified, but also \textbf{DECORATED FINAL SIGIL} is as shown in  \textbf{\hyperref[FIGEXASOISIGILC]{Figure \ref{FIGEXASOISIGILC}}}

\vspace{2em}

\begin{figure}[H]
  \begin{center}
   \includegraphics[width=0.8\textwidth]{resources/100dollar_soi_sigil.pdf}\\
   \caption{Example Computer-Drawn and Finished+DECORATED SOI SIGIL (and associated MANTRA via MYRRH language) --- fit for ATTRACTING MONEY}
  \label{FIGEXASOISIGILC}
  \end{center}
\end{figure}

\vspace{2em}

As for the mantra --- which might also serve the dual purpose of giving us a name with which to refer to the final sigil (or the entity it encodes), note that we used the methods introduced in \textbf{\hyperref[SECEXAMPLESMYRRH]{Section \ref{SECEXAMPLESMYRRH}}}, by which we can take the final simplified, reduced SOI SOURCE, and apply the MYRRH language transformer to it as shown in \textbf{\hyperref[FIGTEAEXASOIMANTRA]{Figure \ref{FIGTEAEXASOIMANTRA}}}.

\vspace{2em}

 %\small
  \begin{tcolorbox}[teaterminalstyle, title=TEA Program: computing Lexical Basis for SOI using TEA, breakable]
  %\begin{lstlisting}[language=TEA, caption={TP C7}, label={LSTC7}, numbers=left]
  \begin{lstlisting}[language=TEA,breaklines=true]
i!:{TEYHOSINAMUBDVRXLC} #the SOI SOURCE as input

#then apply the MYRRH transformer to generate a MANTRA
m!:|r!:y:yua|r!:ht:th
|r!:dn:dun|r!:tn:tan|
r!:rp:rupa|r!:sy:s y
|r!:tc:tauch|r!: :a|r!:[gG]:su|
r!:[dD]:v|z:|z:


#then make it all CAPITAL for proper use
z!:
#might also randomly break long mantra into chunks
s: | s:

#so that we end up with something like...
#ACALAXARAVAVABA UAMAAANAIASAOAHA YAEATA
   \end{lstlisting}
  \end{tcolorbox}
    \captionof{figure}{GENERATING MANTRA from SIGIL SOURCE: applies MYRRH language transformer}
  \label{FIGTEAEXASOIMANTRA}

\vspace{2em}

\subsection{EXAMPLE 2: Using The OZIN Language to Create TALISMANS, CHARMS or PENDANTS}
\label{SECTALISMANS}

\begin{figure}[H]
  \begin{center}
   \includegraphics[width=1\textwidth]{resources/OZIN_MAGICK.pdf}\\
   \caption{The message ``Ozin Magick", expressed using the OZIN language}
  \label{FIGOZINEX1}
  \end{center}
\end{figure}

\vspace{2em}


So, first, consider the fact that using symbols as a means of psychic self-defense, is a practice and arcana that could be traced back to the discussions on applying sigil magic to protection and augmentation of the practicing magician since ancient eras... --- in more recent times, \textbf{Dion Fortune} (1890–1946)\cite{copilot_assistant}, is most popularly known for her contributions to the development and formalization of \textbf{Self-Defense Magick}\cite{butler1952magic}\cite{cassiel1990encyclopedia} --- for example, her writtings about karma, astral shields and applications of mirrors. Dion also comes up when some writers talk about \textbf{The Mystical Qabala}\cite{farrar1990spells}. However, in the present section, our focus is to fix on using sigils and computational mysticism to design and apply \textbf{TALISMANIC MAGICK} --- essentially, creating physical implements, shields or weapons, that are then charged with powers, identities and tasks such as self-defense, protection, guidance, enlightenment, etc. especially in psycho-social, large-scale and realistic contexts.

\vspace{2em}

\begin{table}[H]
  \begin{tabular}{|p{0.95\textwidth}} % Left border only
    \hline
\begin{figure}[H]
  \begin{center}
   \includegraphics[width=\textwidth]{resources/crafting_charged_pillows.jpg}\\
   \caption{Example of an Artisan Magician crafting Pillows with Charms in them}
  \label{FIGEXACRAFTS}
  \end{center}
\end{figure}\\
    \cline{1-1} % Bottom border only
  \end{tabular}
\end{table}



\vspace{2em}
Here is a problem:

\vspace{2em}

\begin{prob}[\textbf{Augmenting and Automating Psychic Self-Defense} --- \textit{Psymatons, Servitors, Guardian Angels and Doubles}]
\label{PROBPSYMATON} 
Assuming the person --- a normally healthy, psychologically stable and otherwise sane human being, suddenly starts to experience outre, preternatural, and or mysterious phenomena in their individual or collective awareness --- in the later, such as when psychologically bewildering events manifest at a large scale socially, or that they are likewise witnessed by, experienced by more than a single mind, we might talk of cases of \textbf{Mass Hysteria}, \textbf{Psy-Ops}, defending or analyzing the \textbf{Collective Unconscious}. In such cases, as is the tradition or practice in many schools and paths of magic and esotericism, there might be a justification to craft, wield and advance ancient and modern esoteric mathematics in terms of solving such problems which, because of their nature, no ordinary, physiologically founded medicine might be established scientifically, but which, by creative use of pathological and paranormal psychology, a magician can implement solutions that can resolve the mystery and pathology --- angels that can end a war, guardian angels that can detect and fight off attackers to some particular individual, anti-demons, health-monitors and financial angels, etc.

\vspace{2em}

The problem is, given any one such case or phenomena description as text spanning the ELA language $\psi_{36}$, how might you bring to life, a physical manifestation of the necessary and associated defense actor --- an \textit{astral}, \textit{spiritual} but most importantly, \textit{psychological defense agency} as required? 

\vspace{2em}

\begin{soln}[\textbf{PHYSICAL MANIFESTATION of a Named Godform, Angel, Spirit, Power, Element, Principle or Attribute of the Self or Collective Mind at Will}]
\label{SOLNMANIFEST}

It is pretty standard, if not dogmatic belief among Christians for example, that the \textbf{Angel Michael} is a generally approachable and satisfactory guardian angel for most faithfuls. But also among people of other faiths and cultures, such as those that practice Islam, Judaism or many faiths founded on the Qabalah. 

\vspace{2em}

We for example see that angel, Michael, alongside \textbf{Gabriel} --- the spirit of Truth, being talked about in a reference manuscript by \textbf{Professor James R. Lewis} and Evelyn D. Oliver, in their lexicon-like compilation of specifics about various kinds of angels and angelic beings from all cultures around the world\cite{lewis1996angels}. They tell us:

\vspace{1em}

\begin{quotation}
\noindent {\ttfamily

The name Gabriel, which means ``God is my strength," is of Chaldean origin and was unknown to the Jews prior to Babylonian captivity... The Sumerian root of the word \textit{gabri} is \textit{gbr} or \textit{gubernator}, meaning ``steersman" or ``governor". Gabriel, who is described as possessing 140 pairs of wings, is the governor of Eden and ruler of the \textbf{cherubim}. He is the angel of Annunciation and Resurrection, as well as an angel of mercy, vengeance, death and revelation.

Gabriel is a unique archangel ... the only female in the higher echelons... said to sit on the left hand side of God.


According to tradition, Mikhail [Michael] lives in the seventh heaven [together with Gabriel], has wings of an emerald color, and has hairs of saffron. Each hair has a million faces with mouths that speak in a million dialects, all imploring the pardon of God. A steadfast friend of humankind, Mikhail has not laughed once since hell was created.

Mikhail and Djibril [Gabriel] were said to be the first angels to obey God's order to worship \textbf{Adam}, a command to which \textbf{Iblis}, the Muslim \textbf{Satan}, objected.

}
\hspace*{\fill} --- \textbf{Angels A to Z}, \textit{1996}, Lewis et. al\cite{lewis1996angels}
\end{quotation}

 \vspace{2em}

 Thus, by knowing which angel or entity to manifest so as to save or defend the people, an individual or for purposes of applying this sacred magick to such psychological pathologies as we have talked about, we might proceed as follows, to manifest one, two or a set of special angels or powers, having manifested who, it is guaranteed, the angels --- beings that are neither limited by physical laws or constraints of human imagination and psychology, shall operate --- autonomously so, but also in ways that the involved humans might not or never understand, so as to effect truth, natural law, divinely sanctioned Order, justice, end or start wars, dismiss epidemics or natural disasters, etc. Things normally beyond the reach, knowledge and or powers of an individual human being.
 
 \vspace{2em}
 
 The method to be followed, as long as we know the name of any such entity, is as follows:

 \vspace{2em}
 
\begin{alg}[The \textbf{Entity PHYSICAL Manifestation Algorithm}] 
\label{ALGENTYMANIFEST}
$ $\\
 \begin{enumerate}
 \item \textbf{GIVEN} the name \texttt{NAME} = $\Theta$.
 \item \textbf{COMPUTE} its modal sequence statistic, $\overset{>}{\Theta}$ = $\Theta_{mss}$
 \item \textbf{ENCODE} $\Theta_{mss}$ in the \textbf{OZIN} alphabet as $\Theta^*_{mss}: \mathbb{N} \times \psi_{ozin}$
 \item \textbf{DRAW} a pictogram or sigil based on \textbf{REDUCING} $\Theta^*_{mss}$, such as $\boxed{\Theta^*_{mss}}$.
 \item \textbf{SCULPT}\footnote{Essentially, \textbf{PRINT}} a living or physical object that best expresses the sigil $\boxed{\Theta^*_{mss}}$ --- that is, we then have a sculpture $\boxed{\Theta^*_{mss}}_{\psi_{\infty}}$.
 \item{ \textbf{NAME}\footnote{Essentially, \textbf{BAPTIZE}} the physically manifested entity using \texttt{NAME} and \textbf{BIND} it to the physical entity $\boxed{\Theta^*_{mss}}_{\psi_{\infty}}$; essentially the transformation:
 
 \begin{trans}
 $\boxed{\Theta^*_{mss}}_{\psi_{\infty}} \rightarrow \boxed{\Theta^*_{mss}}_{\psi_{\infty}}:\text{\textit{NAME}}$
 \end{trans}
 
 }
 \item \textbf{APPLY} the associated entity --- an angel, a godform, spirit, etc. --- by merely approaching it physically, astrally or psychologically via its name, \texttt{NAME}, and manifested form, $\boxed{\Theta^*_{mss}}_{\psi_{\infty}}:\text{\textit{NAME}}$, so as to assign it tasks, commands or requests, so it can react and effect a desired solution.
 \end{enumerate}
 \end{alg}
 
\end{soln}
\end{prob}


\begin{table}[H]
  \begin{tabular}{|p{0.95\textwidth}} % Left border only
    \hline
\begin{figure}[H]
  \begin{center}
   \includegraphics[width=\textwidth]{resources/talismanic_rosary.jpg}\\
   \caption{Example of a Modern Talismanic Rosary}
  \label{FIGEXATALIS}
  \end{center}
\end{figure}\\
    \cline{1-1} % Bottom border only
  \end{tabular}
\end{table}


\vspace{2em}

One shall realize that, even though the solution \textbf{\hyperref[SOLNMANIFEST]{Solution \ref{SOLNMANIFEST}}} is metaphysically sufficient, and yet, without someone knowing how exactly to conduct any of the steps in the process or algorithm (such as \textbf{\hyperref[ALGENTYMANIFEST]{Algorithm \ref{ALGENTYMANIFEST}}}) underlying it, the solution might otherwise remain impractical or out-of-reach for most practitioners and empirical magicians. For this purpose then, we shall go ahead to ease the task, by demonstrating, using the example names of various angels, godforms and spirits, how it is, the necessary affectant sigils required to physically manifest the named entity might be approached or executed --- all, based on applying \textbf{sigil magick}, the \textbf{ozin language} and an algorithm for sigilization such  as in \textbf{\hyperref[ALGOSIGILMAGICK]{Algorithm \ref{ALGOSIGILMAGICK}}} and which is also illustrated well in \textbf{\hyperref[FIGEXASOISIGILA]{Figure \ref{FIGEXASOISIGILA}}}, and introduced in \textbf{STEP 3} of \textbf{OZIN Magick}, so that then, the interested user, student or practitioner, merely needs to take a seal or sigil of the entity or power or principle they are interested in manifesting or applying, and then print or sculpt it into physical form. Adepts might not even need to physically print, but only use these seals to visualize and manifest the associated entities in the astral --- the most efficient, most advanced method perhaps.


\newpage
{
\centering

\textbf{GABRIEL:}


\normalsize
To manifest the arch-angel GABRIEL, note that the necessary sigil would be based on the SOURCE: \texttt{GABRIEL}, and so that, the necessary expression via OZIN is as:



\begin{figure}[H]
  \begin{center}
   \includegraphics[width=0.8\textwidth]{resources/angel_seal_gabriel.pdf}\\
   \caption{Angelic SEAL for \textbf{GABRIEL}}
  \label{FIGOZINSEAL1}
  \end{center}
\end{figure}

\vspace{2em}


\newpage
\textbf{MICHAEL}

\normalsize
To manifest the arch-angel MICHAEL, note that the necessary sigil would be based on the SOURCE: \texttt{MICHAEL}, and so that, the necessary expression via OZIN is as:



\begin{figure}[H]
  \begin{center}
   \includegraphics[width=0.9\textwidth]{resources/angel_seal_michael.pdf}\\
   \caption{Angelic SEAL for \textbf{MICHAEL}}
  \label{FIGOZINSEAL2}
  \end{center}
\end{figure}

\vspace{2em}



\begin{table}[H]
  \begin{tabular}{|p{0.95\textwidth}} % Left border only
    \hline
\begin{figure}[H]
  \begin{center}
   \includegraphics[width=\textwidth]{resources/the_art_of_making_candles_by_nemesisfixx_djcjzej.jpg}\\
   \caption{Example of Applying SIGIL MAGICK to CANDLE MAGICK: details in \textbf{BONUS Chapter CCM}   }
  \label{FIGEXATALISCANDLE}
  \end{center}
\end{figure}\\
    \cline{1-1} % Bottom border only
  \end{tabular}
\end{table}






\newpage
\textbf{SATAN}

\normalsize
To manifest the arch-angel SATAN, note that the necessary sigil would be based on the SOURCE: \texttt{ASTN}, and so that, the necessary expression via OZIN is as:



\begin{figure}[H]
  \begin{center}
   \includegraphics[width=0.7\textwidth]{resources/angel_seal_satan.pdf}\\
   \caption{Angelic SEAL for \textbf{SATAN}}
  \label{FIGOZINSEAL3}
  \end{center}
\end{figure}

\vspace{2em}

\newpage
\textbf{RAPHAEL}

\normalsize
To manifest the arch-angel RAPHAEL, note that the necessary sigil would be based on the SOURCE: \texttt{ARPHEL}, and so that, the necessary expression via OZIN is as:



\begin{figure}[H]
  \begin{center}
   \includegraphics[width=0.9\textwidth]{resources/angel_seal_raphael.pdf}\\
   \caption{Angelic SEAL for \textbf{RAPHAEL}}
  \label{FIGOZINSEAL4}
  \end{center}
\end{figure}

\vspace{2em}


\newpage
\textbf{LEVIATHAN}

{
\LARGE

\hl{TODO: Revisit the MSS definition and case because seems like the SOURCE for LEVIATHAN was wrong and this might have cascaded to all other MSS-based cases! So, check mail, fix the MSS tests for TEA cli and WEB, then re-generate the necessary SOURCES, and also cross-check the manual computations reported in associated tables and statistical formulations. THIS COULD BE A SHOW-STOPPER!}

}

\normalsize
To manifest the arch-angel LEVIATHAN, note that the necessary sigil would be based on the SOURCE: \hl{\texttt{alevithn}}, and so that, the necessary expression via OZIN is as:



\begin{figure}[H]
  \begin{center}
   \includegraphics[width=0.8\textwidth]{resources/angel_seal_leviathan.pdf}\\
   \caption{Angelic SEAL for \textbf{LEVIATHAN}}
  \label{FIGOZINSEALLEV}
  \end{center}
\end{figure}

\vspace{2em}


\newpage
\textbf{AMUN RA}

\normalsize
Of course, it is not only angels or angelic forms that we can manifest using this method. We can also manifest God-forms, and as an example, consider AMUN RA. Note that the necessary sigil would be based on the SOURCE: \texttt{AMUNR}, and so that, the necessary expression via OZIN is as:



\begin{figure}[H]
  \begin{center}
   \includegraphics[height=0.8\textheight]{resources/godform_seal_amunra.pdf}\\
   \caption{SIGIL/SEAL for \textbf{AMUN RA}}
  \label{FIGOZINSEAL5}
  \end{center}
\end{figure}

\vspace{2em}


\newpage
\textbf{PEACE}

\normalsize
Finally, note that we might also apply this method to thought-forms or principles. As an example, we shall only consider the consider two opposite principles; PEACE and WAR. Note that the necessary sigil would be based on the SOURCE: \texttt{EPAC}, and so that, the necessary expression via OZIN is as:



\begin{figure}[H]
  \begin{center}
   \includegraphics[width=0.6\textwidth]{resources/thoughtform_seal_peace.pdf}\\
   \caption{SIGIL/SEAL for \textbf{PEACE}}
  \label{FIGOZINSEAL6}
  \end{center}
\end{figure}

\vspace{2em}



\newpage
\textbf{WAR}

\normalsize
Note that the necessary sigil would be based on the SOURCE: \texttt{WAR}, and so that, the necessary expression via OZIN is as:



\begin{figure}[H]
  \begin{center}
   \includegraphics[height=0.8\textheight]{resources/thoughtform_seal_war.pdf}\\
   \caption{SIGIL/SEAL for \textbf{WAR}}
  \label{FIGOZINSEALWAR}
  \end{center}
\end{figure}

\vspace{2em}




\newpage
\textbf{JESUS}

As a bonus, and for especially readers from paths such as \textbf{Christian Mysticism}, \textbf{Rosicrucians} and some \textbf{Christian Cabalists}, we shall also add some two more seals --- both associated with the \textbf{God-man} egregore, \textbf{Jesus Christ}. It might come as a surprise to some, that one might want to manifest Jesus or contact him via a magical seal! But well... it is very possible! Note that the necessary sigil would be based on the SOURCE: \texttt{JESU}, and so that, the necessary expression via OZIN is as:



\begin{figure}[H]
  \begin{center}
   \includegraphics[height=0.8\textheight]{resources/seal_jesus.pdf}\\
   \caption{SIGIL/SEAL for \textbf{JESUS}}
  \label{FIGOZINSEAL7}
  \end{center}
\end{figure}

\vspace{2em}





\newpage
\textbf{YEHESHUA}

As discussed earlier on, some folks prefer to refer to \textbf{Jesus Christ} by the more authoritative name ``YEHESHUA". In that case, the necessary sigil would be based on the SOURCE: \texttt{EHYSUA}, and so that, the necessary expression via OZIN is as:



\begin{figure}[H]
  \begin{center}
   \includegraphics[height=0.8\textheight]{resources/seal_yeheshua.pdf}\\
   \caption{SIGIL/SEAL for \textbf{YEHESHUA}}
  \label{FIGOZINSEAL8}
  \end{center}
\end{figure}

\vspace{2em}


}


And thus, for any need related to manifesting some power or entity for either health, security, happiness, etc. The essential method would be to obtain its name, create its sigil, and make some physical object that one can carry around with them, or which they can anchor at a location of significance given the problem or task being worked on, and thus, one might then have the semblance of a talisman, charm or channel via which the associated entity or power can manifest and have influence in the physical, and not astral or psychological/subjective realm.

\vspace{2em}


\begin{table}[H]
  \begin{tabular}{|p{0.95\textwidth}} % Left border only
    \hline
\begin{figure}[H]
  \begin{center}
   \includegraphics[width=\textwidth]{resources/magician_tools.jpg}\\
   \caption{Example of a Travelling Magician's operating tools and vestments}
  \label{FIGEXAROBES}
  \end{center}
\end{figure}\\
    \cline{1-1} % Bottom border only
  \end{tabular}
\end{table}




\vspace{2em}

The following examples all essentially build upon this idea and the methodology we have already well established in this section.


\subsection{EXAMPLE 3: Using The OZIN Language to Create SERVITORS or SERVANT SPIRITS}


The case of creating servitors, also ``Servant Spirits" has already been catered for in the the previous example. However, note that, unlike cases where one wishes to apply sigil magick to manifesting an angel or godform --- sometimes which are actually aspects of the collective mind or collective unconscious, one might be interested in manifesting an entity that is known to, and solely works for just themselves --- it is their personal thing or an extension of their own individual mind. This might for example be the kind of entity you create, for the sole purpose of taking you to interesting places during your dreams --- in which, the entity might be considered a kind of \textbf{tulpa} or a \textbf{succuba}, and which only actually exists in your mind alone. But it might also be for example, a spirit or entity that watches over your house, or some particular property, so that, whenever anything is threatening or significantly interacting with that property, the being you have created and assigned the task of overseeing it shall react or cause you to become aware of whatever it is that is happening or needs be given attention to.

\vspace{2em}

However, once the spirit or entity thus created begins to engage with or concern other people, such as when your dream-mate spirit also starts to manifest in in the dreams of your neighbors, your partner or for cases of helper spirits created by say a family or community head, and which then go on to start becoming real to other members of the family or community, then, the proper classification of the entity ceases to be merely a servitor or servant, and might instead better be treated as an \textbf{egregore} --- also covered in \textbf{\hyperref[SECEGREGORE]{Section \ref{SECEGREGORE}}}. 



\subsection{EXAMPLE 4: Using The OZIN Language to EXORCISE The MIND}


There are times when a person might be troubled by something that seems to be of a spiritual or psychological kind, but which, after careful evaluation and observation, might seem to be having some specific identity or ideological basis that an expert (such as a psychiatrist, a demonologist, an exorcist or a healer) might be able to isolate from both the collective mind and the mind of the individual[s] thus concerned. One particularly powerful approach to solving such problems --- typically, where \textbf{a person is possessed} by this undesirable or even malevolent entity or a psychological complex, the best starting step to solving the problem is by profiling the problem or entity thus involved.

\vspace{2em}


In general though, we might propose the following method for properly identifying, naming, framing and then utterly destroying or banishing the associated entity or power:

\vspace{2em}


\begin{alg}[An \textbf{EXORCISM PROGRAM}]
\label{ALGOEXORCISM}
$ $\\
\begin{enumerate}
\item \textbf{OBSERVE} or \textbf{STUDY} the subject (person or place) thus disturbed; for example, take note of their waking state habits relative to other people, but also, and if possible, study their sleeping state habits --- especially if the problem is associated with just an individual. Build sufficient notes about, but also obtain photographs, audio-recordings and past memories of the person and the places where they stay, live or work from.
\item \textbf{PROFILE} that subject; understand their character and personality from both the external perspective, but also from their subjective/inner reality based on what has been observed or elicited from them.
\item \textbf{CLASSIFY} the subject; based on the profile thus accumulated, draft one or more classes or categories that the subject best fits in or which they best associate with, and so that it is possible to understand or isolate what aspects of the subject are general, and thus, well-understood Vs those that are particular or unique, and thus possibly less-known or not understood.
\item \textbf{NAME} the subject; at this point, it might not matter what the actual or normal name is that is associated with or assigned the subject, but instead, based on the profile and classification developed concerning the subject, distill and/or assign a suitable, meaningful name. At this point too, especially for cases of trying to exorcise or treat or heal a pathology associated with an individual, it might help to try to solicit for an explicit name from the subject themselves --- this other name that is then obtained using such a method, sometimes even more useful or telling than might be gleaned merely from external analysis and observation. However, the better route would be to have as many compelling or authoritative names for the involved subject or entity as possible, and then, via careful analysis and distillation or reduction, arriving at a final, most descriptive name or handle for the entity possessing the subject; perhaps a compound-name.
\item \textbf{SIGILIZE} the named entity using the methods we have already explored and developed in earlier sections. The aim here, is to arrive at some useful pictorial or symbolic representation of the entity or power possessing the subject. Note that, just like with how we might evaluate the effectiveness and correctness of a generated or derived name --- say, by testing how effective it is in controlling or evoking the associated entity or problem when issued before the subject under investigation, also, the sigil thus developed for the entity might be tested or evaluated, by assessing how effective its presence might be when certain things happening to the sigil reflect effects on the actual subject, and vice-versa; the idea of \textbf{Voodoo Dolls} comes to mind!
\item \textbf{BIND} the sigil thus developed or identified as suitable, with the name identified for the entity using the best methods the magician or operator can harness or that they know shall work. By doing this, the sigil begins to actually reflect or correspond with the entity thus named. Also, and for advanced operators and operations, the sigil might have already been turned into some advanced form --- such as a figurine, doll or sculpture of some sort, so that it actually has a real physical presence and anchorage point via which the entity might be trapped, or from which it can be interacted with even independently of the actual subject thus concerned.



\begin{table}[H]
  \begin{tabular}{|p{0.95\textwidth}} % Left border only
    \hline
\begin{figure}[H]
  \begin{center}
   \includegraphics[width=\textwidth]{resources/exorcism_via_figurine.jpg}\\
   \caption{Case of Exorcism via Operating on a Double/Alter Entity}
  \label{FIGEXORCISM}
  \end{center}
\end{figure}\\
    \cline{1-1} % Bottom border only
  \end{tabular}
\end{table}


\item \textbf{BANISH} the entity; using methods and skills that we might not have to divulge here, but which essentially boil down to either astrally, psychologically and or physically destroying or attacking the entity --- especially by performing the necessary operations on the \textbf{entity's double or projection} in the sigilized form accessible to the magician, so that, by banishing and destroying the \textit{alternate} form of the entity causing trouble, the actual entity shall likewise be tormented, destroyed and or dispelled. This approach is many times preferable to actually operating on the involved subject directly, since it helps preserve the health and well-being of the subject --- such as if the involved subject is an innocent such as a child or a sick-person, or an elderly person, or if the subject is a special artifact such as a church, home or public utility such as a school or hospital, so that instead of desecrating or damaging the actual structure or location, it is exorcised indirectly and ``smartly" so. However, and this is to be taken with \textbf{precaution}; if all fails, and that the subject involved is in danger of loosing their lives (such as when it is a human subject), then, the knowledgeable operator might have no other alternatives but to directly operate on the subject themselves, or the location where they are anchored. And about this last point, note that, unlike what some people might not known, location is quite critical in many kinds of pathologies --- sometimes, by merely taking the associated subject to a different, clean or neutral location (for example, taking a psychologically tortured child from the location of their usual home to an entirely different home or community), it might be found that their original pathology also wanes or even entirely disappears --- in which case then, the exorcism better be conducted upon the location itself and not necessarily the subject thus troubled.
\end{enumerate}
\end{alg}



\subsection{EXAMPLE 5: Using The OZIN Language to EXPRESS EGGREGORES}
\label{SECEGREGORE}



\begin{table}[H]
  \begin{tabular}{|p{0.95\textwidth}} % Left border only
    \hline
\begin{figure}[H]
  \begin{center}
   \includegraphics[width=\textwidth]{resources/magic_egregore_at_hotel.jpg}\\
   \caption{Example of an Egregore's SIGILIC Sculpture Working at an Urban Hotel}
  \label{FIGEXASOISIGILEGGREG}
  \end{center}
\end{figure}\\
    \cline{1-1} % Bottom border only
  \end{tabular}
\end{table}


The case of expressing or manifesting egregores might appeal to especially modern magicians with a bias towards commercial or corporate magick. Why? Because, in the modern world, there seems to be much heavy use of and need for specialized thoughtforms and entities operating almost daily in the workplace, in markets, at public schools, in supermarkets, offices, courtrooms, hospitals, etc. Such entities are usually associated with or affiliated to the brand, identity, ethos or mission of an organization --- a church for example, but also a company or political party; it might be a social club, a family or a product line. Essentially, the concept of the eggregore consists in having some well defined idea, that is then assigned a particular name, motto or slogan, and from which identifier the creative magician --- sometimes a marketeer or branding guru, or designer, can draft, develop and eventually manifest some symbol, a monument, emblem, or ``sign", by which the energies, identity and powers of the associated idea or rather, collective-thoughtform\footnote{We say ``collective-thoughtform" because, most egregores are powered by, or founded on concepts or ideas that operate based on the powers and existence of more than one person --- essentially, and typically, egregores, just like the organisations or social-entities they represent, can outlive their founders or originators, can exist in places and locations where their originators have never been or can't reach, and might even be better known/well-known than their creators are/were.} (thus ``egregore"), might then be able to operate or have effect/influence things and events in the real world.

\vspace{2em}

Often, these ideas are active in the public with or without conscious knowledge of the people concerned. For example, such egregores might manifest in forms such as the brand symbol of a particular car brand --- such as the \textbf{Mercedes Benz} or perhaps \textbf{Mazerati} brand symbol. It might be the symbol of a special clothing line such as \textbf{Woolworth}, and so that, their entity is made present in all places people wear their clothes, since it has their egregore's symbol drawn say in the tag of a T-shirt, pair of jean pants or upon the sole or padding of a shoe, etc. 

\vspace{2em}

Such egregores, once designed and given a form, like via the methods of sigilization that we have seen and developed well in earlier parts of this grimoire, might then be turned into pendants that members of a particular club or society wear at their formal ceremonies, it might be turned into a wall-mountable emblem or even be placed in electronic media works such as the symbols one might encounter either at the start or end of a convention holywood film or some TV shows. Etc.

\vspace{2em}

A good example of egregores that might defeat all attempts to explain or understand them are like say \textbf{the crucifix} and its association with ``holiness" or ``sacredness" Vs \textbf{the pentagram} and its association with ``evil" or ``occultism".



\begin{figure}[H]
  \begin{center}
   \includegraphics[height=0.9\textheight]{resources/the_pentagram_commander_seal.pdf}\\
  \end{center}
\end{figure}




\begin{table}[H]
  \begin{tabular}{|p{0.95\textwidth}} % Left border only
    \hline
\begin{figure}[H]
  \begin{center} % Rotate a single page of a PDF by 90 degrees
   \includegraphics[height=.9\textheight]{resources/scanned_example_use_ozin_for_magickal_diary_divination_notes}\\
   \vspace{2em}
  \end{center}
\end{figure}\\
    \cline{1-1} % Bottom border only
  \end{tabular}
\end{table}






\newpage
%\vfill

\begin{figure}[H]
\vspace{10em}
  \centering
   %\includegraphics[trim=LEFT BOTTOM RIGHT TOP, clip, width=0.9\textwidth,]{resources/pdfs/EXAPLATONICFORM-PFA.pdf}\\
  \includegraphics[trim=0cm 0cm 0cm 0cm, clip]{resources/philosopher.pdf}
  \caption{The Philosopher: [e][in]vokable: \textbf{OIE AEU}}
  \label{FIGPHILOS}
\end{figure}
%\vfill


\begin{table}[H]
  \begin{tabular}{|p{0.95\textwidth}} % Left border only
    \hline
\begin{figure}[H]
  \begin{center} % Rotate a single page of a PDF by 90 degrees
   \includegraphics[height=0.8\textheight]{resources/scanned_example_use_ozin_and_miti_for_meditation_magickal_diary}\\
   \caption{One shall readily find that, as in this example of a page from one of our initiate's magickal diary, that it is often better, and more satisfying, for a student of magick, to keep their notes and procedures well written using not just creativity, but also secrecy --- use of ``occult" languages is surely one of the things that makes \textbf{Occult Science} what it is.}
  \end{center}
\end{figure}\\
    \cline{1-1} % Bottom border only
  \end{tabular}
\end{table}


\begin{figure}[H]
  \begin{center}
   \includegraphics[width=\textwidth]{resources/medina_slogan.pdf}\\
  \end{center}
\end{figure}



\chapter{The MEDINA Cipher and Magickal Language}
\label{SECMEDINA}


\begin{table}[H]
  \begin{tabular}{|p{0.95\textwidth}} % Left border only
    \hline
\begin{figure}[H]
  \begin{center} % Rotate a single page of a PDF by 90 degrees
   \includegraphics[height=0.6\textheight]{resources/crypt_of_medina}\\
   \caption{\textit{Kiro kimu, tulio okwo mumwanya tugurukire... munyonyi... nkarora orulimi ruhandikirwe, nirwisana ruti.} Basically that is it. It came to me in a dream or a vision. Almost like child's play... the stuff of pure magic. This here is the only surviving original copy of the entire key --- it specifies the alphabet $\psi_{medina}:\mathbb{N} \times \psi_{az}\cdot\psi_{space}$. \textbf{That is the magical language MEDINA}.}
  \end{center}
\end{figure}\\
    \cline{1-1} % Bottom border only
  \end{tabular}
\end{table}



The language depicted in the most original record of the associated cipher --- refer to \cite{lutalo_2025_medina} and \textbf{\hyperref[FIGMEDINAALPHABET]{Figure \ref{FIGMEDINAALPHABET}}}, was originally known as \textbf{Crypt of Medina}, and is one of those very special, albeit, rarely used, and less documented of the occult languages being studied at Nuchwezi Esoteric School. It however might be among the most versatile, alien-like and undeniably difficult to find visual magickal languages ever. It has one very special and useful property though: its entire symbol-set depicts or calls for literally reading between [vertical] lines! Originally, it was based on the \textbf{Latin Alphabet} --- extended with the single WHITE-SPACE character, and no special treatment yet, of the base-10 digits or any particular numbers. Thus, when compared to the other visual magickal language we have explored in this grimoire, such as OZIN (\textbf{\hyperref[SECOZIN]{Section \ref{SECOZIN}}}), and the other in the appendices --- \textbf{The MITI language (\textbf{\hyperref[SECMITILANGUAGE]{Section \ref{SECMITILANGUAGE}}})}, one shall find that those other visual languages span the entire \textbf{ELA}, but MEDINA originally did not. However, that doesn't make it any less useful or powerful\footnote{And, there always was the possibility, that in the future, or out of interest, we might extend it further, so as to also fully span $\psi_{36}$.}. 

\vspace{2em}

After studying that original symbol set, it was decided to actually go ahead and extend this language's symbol set, so that, just like the other two visual languages we introduce in this grimoire, it is complete for all coding and mathematical purposes. Thus, with the extra originally missing symbols --- spanning the digits of the most typically useful number set, $\psi_{10}$, specified and to be expressed as in the proposal in \textbf{\hyperref[FIGMEDEXT]{Figure \ref{FIGMEDEXT}}}.



\begin{table}[H]
  \begin{tabular}{|p{0.95\textwidth}} % Left border only
    \hline
\begin{figure}[H]
  \begin{center} % Rotate a single page of a PDF by 90 degrees
   \includegraphics[height=0.8\textheight]{resources/medina_digits_set}\\
   \caption{In a spirit similar to how we obtained the original, creativity and higher inspiration gave us this extension that then makes MEDINA span $\psi_{36}$}
   \label{FIGMEDEXT}
  \end{center}
\end{figure}\\
    \cline{1-1} % Bottom border only
  \end{tabular}
\end{table}


\begin{figure}[htp]
  \begin{center}
   \includegraphics[height=0.9\textheight]{resources/language_cipher_medina.pdf}\\
   \caption{The COMPLETE MEDINA ALHABET Symbol Set}
  \label{FIGMEDINAALPHABET}
  \end{center}
\end{figure}


\section{3 Examples of Applying MEDINA}
\label{SECEXAMPLEMEDINA}


In the case of how to apply this language, first, note that it is the only other language we shall cover, that involves both vertical writing\footnote{Of course, and as one might have noticed while reading some of the texts in the shared magickal diary notes and paintings leveraging languages such as OZIN, that, we can also write a language normally written horizontally, and more importantly, a spatially-free alphabet language --- OZIN's symbol-set being a good example, but also the ordinary Latin Alphabet and the entire $\psi_{36}$ symbol set expressed using standard glyphs.}, as well as the need for expressing symbols in a spatially-dependent style --- for the case of MEDINA; writing symbols between two parallel lines, while, for the case of MITI; writing symbols perpendicular to or laterally balancing on a [typically vertical] line.



\subsection{EXAMPLE 1: Using The MEDINA Language to Create \textbf{TALISMANIC or MAGICKAL BELTS}}
\label{SECTALISMANBELTS}


\begin{figure}[H]
  \begin{center}
   \includegraphics[width=0.9\textwidth]{resources/collage_belts}\\
   \caption{Examples of special belts --- not just for their appearance, materials or shape, but some belts are appreciated as power decorum, trophies or grades, powers and such.}
  \label{FIGBELTS}
  \end{center}
\end{figure}



First, note that the ideas we are about to explore in this section deal with the aspect of \textbf{modifying The Self}. This is a fundamental part of most traditional ritual magick systems, especially ones in which the obligations for the operator, magician, priest or commander of some special office require special vestments, regalia and decorum in order to properly exercise their roles and obligations --- often, in a different capacity than would be the case in their normal, basal self. However, we shall not explore all aspects of cosplay and ritual robes but just \textbf{the belt of the magician}.

  
\vspace{2em}  
  

Assuming we start with a \textbf{plain sacred belt as shown below}


\begin{figure}[H]
  \begin{center}
   \includegraphics[width=0.9\textwidth]{resources/talismanic_belt_plain.pdf}\\
  % \caption{The COMPLETE MEDINA ALHABET Symbol Set}
  %\label{FIGMEDINAALPHABET}
  \end{center}
\end{figure}

That picture depicts a stylistically general and abstractly simple core element in the formal vestments and clothing of most professionals --- men and women, boys and girls, students at school, nurses or doctors, lawyers and gentlemen, soldiers and priests; their belt. Moreover, for the magician, and especially modern magician, it should not come as a surprise that this simple aspect of fashion perhaps has very ancient roots --- the \textit{greeks}? \textit{romans}? \textit{egyptians} or the first semblance of mankind's civilization and civility? --- and that, even in cases such as for the modern solo magician that might operate not as part of a tradition, caste or officious class of operators that require their members to don some kind of uniform or formal wear while operating, and yet, the careful basic magician with taste and culture, shall realize that they can take normal clothes and after creatively setting them apart from their other wardrobe items, end up with their own special magical uniform and instruments of power. 

\vspace{2em}

That aside, first take note of how important this aspect of a magician's character and signature might be, considering that in \textbf{Walter Ernest Butler}'s book on Magic and the Magician, particularly with a strong bias towards nurturing effective initiates aligned with or better than members of the \textbf{Golden Dawn}, he has a chapter on just \textbf{Vestments}, and we can glimpse of some important cues from what he has to say:

\vspace{2em}


\noindent
\begin{minipage}{1\textwidth}
\vspace{1em}
\begin{quotation}
\noindent {\ttfamily

One of the weak points of the Order of the Golden Dawn was its excessive eclecticism. It tried to include far too much, and some very doubtful attributions crept into use. Though through use these attributions do act as channels of power, a good deal would be gained if they were, by process of neglect and the cultivation of the true correspondence, allowed to slip back into disuse.\\

...the use of colours is fundamental. They are to be found throughout the whole magical scheme, and the use of the Flashing Colours is foundational work.\\

As the colours key us to certain forces, it follows that the use of vestments of the appropriate colour will help us to link up with those forces. That is the simple rationale of the vestments. Since the Western Tradition has been strongly influenced by the Greek, Hebrew and Egyptian traditions on the one hand and the medieval Catholic Church on the other, it will be found that the robes worn in the magical fraternities reflect one or other of these sources. Many of them are magnificent pieces of work, but it is necessary for the student to remember that their value does not depend merely upon their magnificence. Plain robes of the appropriate colour are every bit as effective as the most exotic designs!


}
\hspace*{\fill} --- \textbf{Magic: Its Ritual, Power and Purpose}, \textit{1952}, W.E. BUTLER\cite{butler1952magic}
\end{quotation}
\vspace{1em}
\end{minipage}


\vspace{2em}

In that regard, Butler is mostly putting in our heads the basic idea that clothing matters\footnote{Of course, some clever, and more liberal magicians throughout the ages have sometimes preferred to approach \textit{some aspects} of their magical operations while working entirely clothed in nothing but \textbf{their natural skin}\cite{cassiel1990encyclopedia}; we know this trend might likewise be found in some modern and contemporary fraternities, even for cases where the rituals aren't say, of a sexual or even LHP kind. In African traditions too, several occult groups including those in Uganda, might require (sometimes not the operator, but the participants), to attend a ritual while naked!}. That excerpt though, put more emphasis on the colours of the vestments. However, that is not their only role as we also learn that...




\noindent
\begin{minipage}{1\textwidth}
\vspace{1em}
\begin{quotation}
\noindent {\ttfamily

Apart from their value as ``colour-suggestions," they serve a very useful purpose; they screen off the personality of the operator, and so make for impersonality... of great importance, especially when magical work is being done by a group comprising of both sexes. In some lodges, cowls or hoods are [also] used...\\
The robes have another interesting effect. They act as a very strong auto-suggestion, which has the power of keying the mind to the operation in hand.


}
\end{quotation}
\vspace{1em}
\end{minipage}



And thus we come to the matter of special belts that would help or automatically set the wearer's mind into some frame ideal or conducive for certain work or to imbue them with particular qualities --- \textbf{\hyperref[FIGBELTS]{Figure \ref{FIGBELTS}}} shows us some examples, including special \textbf{macho belts} that might remind one of classical heroes and god-men such as Hercules, Theseus, gladiators, emperors and certain kinds of occult priests\footnote{The idea of belts as accolades in power games such as boxing and wrestling might also have links to ancient occult traditions --- why are they awarded belts and not cups or plaques?}. We particularly wish to realize that, a normal plain belt, just like any other magical vestment --- a cape or helmet, a shoe or gown, ring or wand --- might be carefully enhanced by writing, inscribing or etching upon it, some symbols, sacred names or sigils of some particular power, principle or even the magicians' own name or inner-genius seals, etc.

\vspace{2em}





\begin{figure}[htp]
  \begin{center}
   \includegraphics[height=0.8\textheight]{resources/chakra_system_body}\\
   \caption{8 Energy Centers (\textbf{Chakras}) of the ideal human, with our primary focus right now, about the second-last (or central chakra) around the genital area, also conventionally called ``Svadhistana" in Oriental Occultism}
  \label{FIGCHAKRAS}
  \end{center}
\end{figure}


For purposes of not wasting much time going into the proper underlying philosophies and esotericism of both men and ladies' belts, note that the following few aspects might help anchor our soon to be proposed system of creating talismanic or magically charged belts:\\


\begin{enumerate}
\item The belt is typically essential to secure one's manhood against abuse --- this, both for ladies and men.
\item{ The belt is one of the few vestments --- apart from knickers and underpants, that operates nearest to the middle-chakra (``sacral center"\cite{farrar1999healing}).
}
\item The belt can serve sufficiently as a status, grade or class symbol in most formal attire\footnote{In african cultures and traditions such as the \textbf{Ganda Traditional Formal Wear} of Uganda, a man might forego the belt given that they formally appear dressed in the \textit{kanzu} --- a kind of dress-like attire for men, while for women especially, they must complete their formal wear in \textit{busuuti}, with a special, strangely large, decorated and extraneous belt known as ``Ekitambala".}.
\end{enumerate}

\vspace{2em}

That said, note that, concerning point \#2 in the list above, we are here especially focused on matters that might relate to or influence the wearer's mind or self --- their physical body perhaps, but especially their astral and spiritual body, while they are dressed in a particular way. Concerning this region of the human body, the \textbf{Farrars} and \textbf{Bone Gavin} have this to say:\\


\noindent
\begin{minipage}{1\textwidth}
\vspace{1em}
\begin{quotation}
\noindent {\ttfamily

This is the sacral center, and is positioned in the genital area. Early Western practitioners of Laya Yoga avoided this center altogether... The reason for the avoidance... the Theosophist's worry that concentration on this center would result in the awakening of sexual feelings. Such an act was considered inherently black in nature...\\
It appears as orange in color and relates to the element of Water. It is traditionally divided into six petals. It has obvious associations with sexual love, controlling energy to the reproductive organs. Psychologically it is associated with the emotions. It is actively involved in all forms of intimacy and interpersonal feelings and strongly affects the sense of emotional well-being.

}
\hspace*{\fill} --- \textbf{The Healing Craft}, \textit{1990}, Janet and Stewart Farrar\cite{farrar1990spells}
\end{quotation}
\vspace{1em}
\end{minipage}


And with that introduction out of the way, note that, our proposed and practically explored system of BELT MAGICK is as outlined below in brief:

\vspace{2em}


\textbf{BASIC BELT MAGICK:}

\begin{enumerate}
\item Include a special belt [of power] in your formal wardrobe or traveling vestments as a magician.
\item Pick a durable and strong leather belt if possible, but taste and creativity might also guide you towards a special kind of magicians belt perhaps made of other materials --- animal skin or not (such as those that venture towards reptilian belts; pythons and crocodiles especially), metallic and chain-like or not, etc.
\item \textbf{The Buckle} is a critical part of most belts, and so, the careful magician shall ensure to pick or make a buckle with the same care and special attention that a seasoned magician gives to their pendant (as part of their ``magical necklace").
\item Though mundane and ordinary people might not care about, nor ever focus or treat of the inside of a belt apart from whether it is made of durable material or not --- especially since this side of the belt is normally hidden from everyone while the belt is being worn --- and yet, for the creative magician, this is the region of the belt that one might use to adorn, augment, consecrate, signify or associate their belt with some particular powers, principles, forms and special magical attributes. Also, because this is among the hidden aspects of a magician's but also anyone's normal clothing, it means, it is also one of those aspects of fashion that might be used to condition or program the ``hidden mind" --- essentially, the unconscious or subconscious mind.
\item A belt can then be programmed or associated with something, by adorning it with the symbolism or name or seal of some thing. In our system, we shall use the special \textbf{magickal language of MEDINA} --- particularly because, it can allow us to make simple codes on the inside of a belt using just a MARKER (its ink can be permanent but also prominent on most belt-inside materials) or some special ink.
\item The magician can prepare or make several belts, one for each special or particular attribute or power they wish to imbue themselves with while wearing the belt or while operating with some other thing.
\end{enumerate}

\vspace{2em}

Our examples then, are going to exploit the above system, and we shall simply show some images of various formulations and designs of belt-insides, that depict or speak to some kinds of magician character or attributes. Of course, these are but examples... just like people go to a tattooist with the aim of either decorating or modifying themselves (albeit, ``literally" and which perhaps is unsafe or irreversible when compared to instead writing on one's clothes and which can later be changed easily), one can either approach an artist or craft person to make for them one of these belt designs, or they can just pick up a black marker of permanent black ink, and make the designs on their own belt.

\vspace{2em}


Definitely, as with other magical vestments, merely writing on it a special seal, sigil or name might immediately cause the piece of clothing to take on a new identity and purpose, but, and as we again learn from Butler below, special care should be given such special clothes which have been specifically modified for use in special ceremonies or occasions --- especially belts made specifically for use in priestly, magical or formal contexts. This, so that the person doesn't unintentionally alter themselves in places and situations they hadn't intended, but also so that the powers and new character of their clothing isn't abused or lost from using it for mundane purposes.



\noindent
\begin{minipage}{1\textwidth}
\vspace{1em}
\begin{quotation}
\noindent {\ttfamily

...the robes are of use. During the many magical operations undertaken through the years, the robes become ``charged" with a certain etheric energy or ``magnetism", and though the fairly frequent cleaning process, which are necessary, though not always carried out, will disperse some of this magnetism, they soon become charged up again. In this state they play a part in the interplay of etheric forces which occurs in the Magical Lodge.\\\\

A word of warning. When you have once used your magical robe, \textit{never under any conditions}, thereafter use it for any purely secular purpose


}
\hspace*{\fill} --- \textbf{Magic: Its Ritual, Power and Purpose}, \textit{1952}, W.E. BUTLER\cite{butler1952magic}
\end{quotation}
\vspace{1em}
\end{minipage}



Also note that, unlike the case of creating SIGILS that we explored in \textbf{\hyperref[SECSIGILMAGICKCORE]{Section \ref{SECSIGILMAGICKCORE}}}, for the case of BELT MAGICK, unless when you are also adding a sigil to the otherwise plain-MEDINA-language annotations, one might just directly take the name, word or phrase of power they wish to use, and encode it into MEDINA, on the belt. But obviously, and for reasons that might already be obvious to the careful student by now, it might always be preferable, to proceed as with SIGIL MAGICK, and instead of working with or writing the original message directly, instead compute the \textbf{modal sequence statistic} of the message, and instead use or write that as the \textbf{SOURCE}. Thus, in most of our examples here, we are to show scenarios using the MSS as the approach to the coded message.

\vspace{2em}

Further, and for inspiration to those wishing to explore with us, note that we shall apply some of the names, phrases and mantras that we already encountered, developed and explained well in \textbf{\hyperref[TABMYRRHWORDS2]{Table \ref{TABMYRRHWORDS2}}} (The Arcane Mantras), \textbf{\hyperref[TABMYRRHWORDS]{Table \ref{TABMYRRHWORDS}}} (The Benevolent Mantras), and \textbf{\hyperref[TABMYRRHNAMES]{Table \ref{TABMYRRHNAMES}}} (The Divine Names). 


\vspace{2em}
For computing the \textbf{modal sequence statistic} SOURCE we shall be using, the essential computer program we shall need is as follows:


\vspace{2em}

 %\small
  \begin{tcolorbox}[teaterminalstyle, title=TEA Program: computing MSS for any input, breakable]
  %\begin{lstlisting}[language=TEA, caption={TP C7}, label={LSTC7}, numbers=left]
  \begin{lstlisting}[language=TEA,breaklines=true]
i:{your message here} # can place it here, or directly pass it to the TEA runtime
d!:[0-9a-z-A-z]
r!:0:O | r!:1:I #replace some number with letters
u!:
   \end{lstlisting}
  \end{tcolorbox}
    \captionof{figure}{COMPUTING MSS for MAGICK-purposes using TEA}
  \label{FIGTEAEXAMSSMAGICK}

\vspace{2em}


For reference purposes, especially consult the annotated version of the MEDINA language as depicted in \textbf{\hyperref[FIGMEDINAALPHABETANOTATED]{Figure \ref{FIGMEDINAALPHABETANOTATED}}}.


\begin{figure}[H]
  \begin{center}
   \includegraphics[height=0.9\textheight]{resources/annoted_language_cipher_medina_extended.pdf}\\
   \caption{The COMPLETE and Annotated MEDINA ALHABET Symbol Set}
  \label{FIGMEDINAALPHABETANOTATED}
  \end{center}
\end{figure}




\begin{figure}[H]
  \begin{center}
   \includegraphics[width=0.9\textwidth]{resources/talismanic_belt_defender_healer.pdf}\\
   \caption{\textbf{Talismanic Belt for Defender and Healer} based on Budhist OM AH HUM mantra}
  \label{FIGMEDINABELT1}
  \end{center}
\end{figure}

 
 \vspace{2em}
 
 
 \begin{figure}[H]
  \begin{center}
   \includegraphics[width=0.9\textwidth]{resources/talismanic_belt_self_empowerment.pdf}\\
   \caption{\textbf{Talismanic Belt for Self-Empowerment and Command} based on ``I AM THAT I AM" formula}
  \label{FIGMEDINABELT2}
  \end{center}
\end{figure}


 \vspace{2em}
 
 
 \begin{figure}[H]
  \begin{center}
   \includegraphics[width=0.9\textwidth]{resources/talismanic_belt_magician.pdf}\\
   \caption{\textbf{Talismanic Belt for presencing THOTH and Magical Personality} based on ``TETU" sacred name}
  \label{FIGMEDINABELT3}
  \end{center}
\end{figure}


 \vspace{2em}
 
 
 \begin{figure}[H]
  \begin{center}
   \includegraphics[width=0.9\textwidth]{resources/talismanic_belt_marduk.pdf}\\
   \caption{\textbf{Talismanic Belt for Creative Powers and Stunning Persona} based on ``MARDUK" sacred name}
  \label{FIGMEDINABELT4}
  \end{center}
\end{figure}


\vspace{2em}
 
 
 \begin{figure}[H]
  \begin{center}
   \includegraphics[width=0.9\textwidth]{resources/talismanic_belt_illuminati.pdf}\\
   \caption{\textbf{The ILLUMINATI Belt} based on ``ILLUMINATI", the seal of mysteries, ``EL RA" and the Star of Ishtar}
  \label{FIGMEDINABELT5}
  \end{center}
\end{figure}



\subsection{EXAMPLE 2: Using The MEDINA Language to HIDE CODES and PASSWORDS in PUBLIC}
\label{SECMEDINAEXA2CODES}


Having properly treated of how to practically leverage the \textbf{MEDINA language} in the foregoing sections, we leave it as an exercise to the student to apply their knowledge to go out in the world out there, and establish if it might not be the case that codes and occult languages such as this one might be in use in the general public --- as part of structures in architecture, as signs embedded within public signage and commercials, as subtle messaging in clothes, on roads, on vehicles, etc.

\vspace{2em}


Of course, as with many things symbolic, one might sometimes encounter a message or symbol that somewhat resembles what we have encountered with the MEDINA language, and yet, it might just be a mere coincidence; the symbol looks like what you think it is, but it is meant for something totally else, or that you are trying to process it out of context. 

\vspace{2em}

That said, of course, now that this key knowledge is out there... who knows... it might not be just students and initiates of IoN nor IoNA that know of or decide to use this language and its symbolism in the future. And yes, we encourage you to go ahead and use it!



\subsection{EXAMPLE 3: Using The MEDINA Language to READ or DIVINE MESSAGEs from NATURE}
\label{SECMEDINAEXA3MESSAGES}



Finally concerning reading messages potentially encoded using this language. Note that it might not be by coincidence or surprise that you encounter MEDINA-like messages in nature! For example, do you notice that the divisions of a sugarcane can be read as instances of either ``A" or ``M" or perhaps the ``SPACE" character from the symbol set depicted in \textbf{\hyperref[FIGMEDINAALPHABETANOTATED]{Figure \ref{FIGMEDINAALPHABETANOTATED}}}? What of the markings along the bodies of various kinds of snakes? 


\vspace{2em}


As they say ``God works in mysterious ways", but also that, once you know something, it influences and affects your outlook on reality! And thus, once you learn and can read the MEDINA language, you shall find that, in cases where you take conscious effort to observe and study nature, or even when you practice \textbf{natural divination} methods, that, certain messages might start appearing or coming to you in real life, which messages you must decode using this language in order to properly or correctly understand them!


\begin{table}[H]
  \begin{tabular}{|p{0.95\textwidth}} % Left border only
    \hline
\begin{figure}[H]
  \begin{center}
   \includegraphics[width=1\textwidth]{resources/scanned_ijuka_hymn_scribe_sketch_in_medina}\\
   \vspace{2em}
  \end{center}
\end{figure}\\
    \cline{1-1} % Bottom border only
  \end{tabular}
\end{table}




\section{The MEDINA Language System}
\label{SECMEDINASYSTEM}

\vspace{2em}


\begin{transf}[The \textbf{Magical Language \texttt{MEDINA}}]
\label{TRANSFMEDINA}
If $\Theta^n$ is a sequence of $n > 0$ symbols (the original message) spanning the \textbf{Extended Latin Alphabet} or the symbol set $\psi_{36}$, 

then the following transformation:\\

\begin{trans}
\label{TRANSMEDINA}
$\Theta^n \xrightarrow{O_{lmedina(\cdot)}} \Theta^* = \Omega^n : \mathbb{N} \times (\psi_{36} \rightarrow \psi_{medina}) : \mathbb{N} \times \psi_{medina};$\\
$\invpi(\Theta^n) = \invpi(\Theta^*) = \invpi(\Omega^n) = n$\\
$\land \quad \forall \theta_{i \in [1,n]} \in \Theta^n \quad \exists \omega_{j \in [1,n]} \in \Omega^n: \theta_i \leftrightarrow \omega_j \quad \land \quad \invpi(\theta_i \in \psi(\Theta^n)) = \invpi(\omega_i \in \psi(\Omega^n)) = 1$\\
$\land \quad \forall \alpha \in \psi(\Theta^n): \invpi(\alpha \in \Theta^n) = f_\alpha \implies \beta \in \psi(\Omega^n): \invpi(\beta \in \Omega^n) = f_\alpha  \quad$\\
$ \text{ iff } \quad \alpha \in \psi_{36} \longleftrightarrow \beta \in \psi_{medina}$\\
$\land \quad \overset{>}{\Theta^n} \longleftrightarrow \overset{>}{\Omega^n}$\\
$\land \quad \tilde{A}(\Theta^n \rightarrow \Omega^n)_{\psi_{36}} = 0 \qed$
\end{trans}

Which is one way of saying, the transformation of a message from the Extended Latin Alphabet into the MEDINA language, is guaranteed to always produce/generate a derivative message --- $\Theta^*$ that has the following properties:\\


\begin{multline}
\label{EQMEDINA}
\forall \alpha \in \Theta^n \implies \beta \in \Omega^n: \mathbb{N} \times \psi_{medina}
\end{multline}

Where $\psi_{medina}$ is an extension\footnote{Unlike $\psi_{ozin}$ that exactly maps to $\psi_{36}$, we see that MEDINA also explicitly defines an extra symbol for a single WHITE-SPACE character --- which we could merely imply and not explicitly write when using the $\psi_{medina}$ symbol set, but which, especially if we wish to make the language able to be written in ``compact form" --- where white-space might be explicitly specified with a special \textit{non-white-space} symbol, but otherwise the text or message appearing as though there was no white-space --- and so, we might perhaps think of $\psi_{medina}$ spanning $\psi_{37}$? But otherwise, all the other formalisms that applied to OZIN might as well apply here.} of the symbol set $\psi_{36}$ as shown in  \textbf{\hyperref[FIGMEDINAALPHABETANOTATED]{Figure \ref{FIGMEDINAALPHABETANOTATED}}} as a \textbf{one-to-one} mapping from \textbf{Base-36 symbol set}, $\psi_{36} \rightarrow \psi_{medina}$.
$\qed$

\vspace{2em}

And so that, the resultant [transformed] message, $\Theta^* = \Omega^n \approx \Theta^n$, is not only semantically equivalent to the original message, but also its length \textbf{is exactly equivalent to that of the original message}, with the only basic different being that its message is expressed using the alternative symbols or visual glyphs different from the original symbol set, but otherwise corresponding to it item-wise. Any such instance of text thus transformed is then an expression in the language \textbf{MEDINA}.

\end{transf}















\chapter{CONCLUSION}
\label{SECFIN}


This is a no-bullshit distillation and preservation of the mysteries of computational mysticism --- ancient till modern. This grimoire has presented new information on several metaphysical problems and their general solutions under several new and existing magical systems; new technologies and languages treating of matters such as the creation of spells, mystical prayers, conjurations, hexes, sacred and power names or words, and all the transformations of text into pictorial-visual languages and then into angelic, entity and abstract glyphs such as seals and magical emblems. We have also covered non-trivial introductions to methods of manifestation into physical forms, of arbitrary named entities. We have introduced \textbf{Talismanic Magick} using godforms, eggregores, and angels. Also presented introductions to magick vestments and decorum treating of \textbf{Belt Magick} via an ancient-like power manifesting language --- \textit{crypt of medina}. Part of the \textbf{IONA Inner Journal series}\footnote{Refer to earlier and related publications via: \url{https://t.me/wwwrite/}}, this \textit{special} volume, also the second, features original art, notes, literature and book reviews, new mathematics applied to magick and computing in magick --- such as use of the new Transformatics mathematics in the formalisation of post-Spare sigilization (\textbf{\hyperref[ALGOSIGILMAGICK]{Algorithm \ref{ALGOSIGILMAGICK}}}) and more. \textbf{5 Magickal Languages} have been formally introduced; lumtauto (\textbf{\hyperref[SECLUMTAUTO_PY]{Section \ref{SECLUMTAUTO_PY}}}), [grand] myrrh (\textbf{\hyperref[SECMYRRH]{Section \ref{SECMYRRH}}}), ozin (\textbf{\hyperref[SECOZIN]{Section \ref{SECOZIN}}}), [crypt of] medina (\textbf{\hyperref[SECMEDINA]{Section \ref{SECMEDINA}}}), and miti (\textbf{\hyperref[SECMITILANGUAGE]{Section \ref{SECMITILANGUAGE}}}) languages. It is our will that, apart from helping bring more of the universally applicable ideas in pragmatic and general occult philosophy, science and art to surface via our publicly-shared correspondences and proceedings, that, \textbf{IoN}, and its education branch, \texttt{NES}, inspire, nurture and guide the current and future generations of especially African initiates into Western Esotericism on how to self-initiate, study and apply the sacred, new and ancient wisdoms in practically useful ways for our collective well-being and advancement. Look forward to a future edition of this grimoire, and other volumes of the IoN[A] Journal to follow in years to come.\\\\
{\LARGE | $^{*.*}$ Psymaz MAGINA}




\begin{appendices}


\newpage
\chapter{Applying TRANSFORMATICS in MAGICK}
\label{APPENDTRANSFORMMAGICK}



It shouldn't come as a surprise to the seasoned and learned modern magician, that mathematics underlies or perhaps, can offer us a solid foundation for much of modern magick --- there is so much in reality that without careful, special mathematical modeling or analysis, would otherwise remain entirely out of reach of the human operator. Also, despite aspects of basic math like numbers and basic geometry having been around us for eons... thousands and thousands of years, since the days of  esotericists like Pythagoras and Plato, and yet, even unto this day, much of basic, but also advanced concepts in occult philosophy, esoteric science, and magick still borrow from or keep returning to foundational, basic mathematics --- so that, perhaps, we should have actually titled this book the somewhat convenient and relevant \textbf{``Applying TRANSFORMATICS in MAGICK"}. However, and of course, magick and esoteric science is wider than, or calls for more mathematics than just transformatics...

\vspace{2em}

We still treat some basic mathematical realities as sacred or perhaps mystical; for example, the \textbf{Golden Ratio} --- $\frac{1}{2} \times (1 + \sqrt{5})$, that many find to be ideal for describing sacred and naturally aesthetic proportions; the ratio of the circumference of a circle to its diameter, \textbf{Pi} --- $\pi = \frac{C}{d} = 3.1415926353898$, that can help us demystify many intricate problems in nature; and the interesting fact --- refer to our NES tutorial on YouTube: \textbf{\url{https://youtu.be/BrRihQ4Orgg}} --- that, by dividing any circle's circumference using a compass subtended by the angle that spans $72^\circ$ from some starting point along the circumference, one eventually describes the 5 vertices of \textbf{a perfect pentagram}; the generation of, derivation of random or special messages in any language, based on manipulations of just the digits 0 and 1 (such as in the ancient Aaron Priesthood use of \textbf{Urim and Thumim} [\textbf{Exodus 28:30}\cite{newjerusalem1985}]) or from the 10 digits spanning $\psi_{09}$ --- example in \textbf{Chapter 10} of \cite{lutalo2025rock}, and other such interesting mathematical, but also magical curiosities.


\vspace{2em}


And thus we come to the matter of \textbf{Transformatics}\cite{Lutalo2025_transformatics_thesis}; the new field of mathematics that treats of [ordered] sequences and operations on them. In the foundational paper, a thesis, \textbf{Transformatics 101 Explained}, much of what would draw one to this field of mathematics is laid bare, and though the essential 1-pager summary of this field is available online at \cite{lutalo2025transformatics_1page}, we shall also reproduce it here for the interested reader's purpose...



% insert [front] cover --- could just be a PNG or PDF
\includepdf[pages=1]{resources/transformatics_101_1pager.pdf}


For those unfamiliar with the field, or why it is important, note that, the greatest allure of transformatics lies in the fact that many problems --- in mathematics itself, but also in mathematical sciences, in computing, in art, in philosophy et cetera, can be properly modeled and then solutions to them be arrived at, merely by structuring them around the concept of either ordered or unordered sequences. 


\vspace{2em}

For example, consider the matter of how to formally and mathematically define magick --- a problem that we encountered at the very beginning of this grimoire: \textbf{\hyperref[SECDEFMAGICK]{Section \ref{SECDEFMAGICK}}}. Note that, we can simplify much of the complexity around such problems, by reducing the problem to one of one or more special operations known as \textbf{transformers}, that essentially take one or more sequences as inputs, and by processing them, produce one or more sequences as output. This basic philosophy underlies much of what makes computers work as they do --- consider the case of the \textbf{UPLT} --- \textbf{Universal Programming Language Theory} --- first laid out in a critical paper on Software Language Engineering by the author\cite{lutalo2024software} --- in which we find that, even as simple and as basic a machine as a Turing Machine, conforms to this law, and that essentially, all kinds of computing, and especially when expressed using computer programs, boil down to just text-processing --- in fact, and more interesting to both contemplate and say, ``text processing text!". 


\vspace{2em}

The TEA language --- also known as \textbf{Transforming Executable Alphabet}\cite{Lutalo2024TEATAZ}\cite{cli_tttt}, is one of the best examples of a modern, and also contemporary computing machinery that is founded on, and inherently operates on this mathematics and philosophy of Transformatics. Good enough, we have encountered many examples how this language might be of use to the student and practitioner of magick, in many of the chapters and problems treated of in this grimoire, and so, its relevance and importance need not be repeated or stressed here.

\vspace{2em}

That said, note that, for our core purposes in this grimoire, especially treating of magical languages and their applications --- particularly, for the case of applying our visual magical languages to the matter of creating special magical payloads such as sigils (refer to \textbf{\hyperref[SECSIGILMAGICKCORE]{Section \ref{SECSIGILMAGICKCORE}}}), note that, among the important mathematical operations we need understand and exploit well, is that of \textbf{how to reduce some sequence to a most representative version} --- basically, and for the kind of applications we have dealt with here, especially \textbf{how to compute a statistically sound summary of an otherwise long and verbose message or sequence of symbols}. The \textbf{modal sequence statistic} (MSS) is a great example of one such approach to resolving this matter, and it was well introduced in \cite{transformatics}, but also in \cite{Lutalo2025_transformatics_thesis}. Moreover, we find that, especially for the case of approaching how to arrive at a [correct] working name or mantra associated with some sigil --- and thus, some Statement of Intent (SOI), we see that, other ways of reducing sequences --- such as using the \textbf{Lexical Basis} (\texttt{b!:} in TEA), might not be convincing\footnote{Much as the lexical basis of two distinct phrases such as ``ABRACA" and ``ARC B" both reduce to ``ABCR", and yet, we see in this kind of reduction, the problem that the transformation doesn't respect the relative order of the distinct symbols in the original sequence.} or useful enough. And so, given that magicians care so much about subtleties such as whether a deity is better invoked as ``AMUN RA" Vs ``AMUN RE", or whether a SOI is expressed in past, future or present tense when it is to be processed magically, or whether it is more useful to recite the Pater Noster in Latin or Hebrew and not in English, etc.

\vspace{2em}

We thus find that, having our magical mathematics right is also such a critically important matter, and so that, we shall devote this chapter to treating of the special matter of how to correctly derive the SOURCE of a given SOI\footnote{Refer back to \textbf{\hyperref[ALGOSIGILMAGICK]{Algorithm \ref{ALGOSIGILMAGICK}}} --- specifically the \textbf{REDUCE} step} --- a task that would underlie not just the process and tasks of creating magical payloads such as sigils, but also the creation and operationalization of talismans, hexes, voodoo dolls, conjurations, deity names, mantras, formulaic auto-spells, etc.


\section{The Basic Modal Sequence Statistic (BMSS)}
\label{SECBMSS}


First presented in \cite{transformatics} (\textbf{Section 4.1}), the concept of the \textbf{modal sequence statistic} (MSS) is founded on the problem of wanting to determine which of the unique symbols or elements in a given ordered sequence occur either the most frequently, or otherwise the earliest relative to the other distinct members of the same sequence. An interesting and enlightening approach to its appreciation is depicted in \textbf{Proposal 5} of \cite{Lutalo2025_transformatics_thesis} --- particularly, in a formal problem dealing with determining which of any two distinct symbols of a sequence has the highest rank based on their relative frequency:

\vspace{2em}


\begin{prob}
\label{PROB1}
Given\\ $\Theta^n : \mathbb{N} \times \psi_\beta : \theta_i, \theta_j \in \Theta^n : \theta_i \neq \theta_j  \quad \land \quad i,j \in [1,n] : n \in \mathbb{N}: 1 \leq \invpi(\theta_i \in \Theta^n) < n$,\\ is it true that $\invpi(\theta_i \in \Theta^n) \geq \invpi(\theta_j \in \Theta^n)$?\\
\end{prob}


\vspace{2em}

Reproduced here verbatim, as shown above, and relating to any two symbols or elements $\theta_i$ and $\theta_j$ from the sequence of $n$ elements, $\Theta^n$, we see that, by computing the MSS of that sequence --- and by MSS, meaning:


\vspace{2em}

\noindent
\begin{minipage}{1\textwidth}
\vspace{1em}
\begin{quotation}
\noindent {\ttfamily

a resultant sequence based on/derived from the source sequence $\Theta^n$, such that the elements/members of the former exactly span the symbol set of the later/source sequence --- i.e. $\psi(\Theta^n)$, and yet, there are not only no duplicates, but that the elements are sorted such that the member/symbol that occurs the most (with highest frequency) in $\Theta^n$, occurs earliest in  $\overset{>}{\Theta^n}$ than any other element, or at least (such as where any two elements have the same relative frequency in $\Theta^n$), they occur in their natural order of first-occurrence.

}
\hspace*{\fill} --- \textbf{Transformatics 101 --- explained}, \textit{2025}, Joseph Willrich Lutalo\cite{Lutalo2025_transformatics_thesis}
\end{quotation}
\vspace{1em}
\end{minipage}


And so, we find that, having obtained the MSS as some sequence --- conventionally denoted as $\overset{>}{\Theta^n}$ for a source sequence $\Theta^n$, that then, we can just look-up the position of both elements under scrutiny;  $\theta_i$ and $\theta_j$ in this case; and if we find that $\theta_j$ occurs earliest in the MSS than its partner, then it has the higher frequency in the original sequence, otherwise, both had the same frequency, but it occurred earliest, and vice-versa if $\theta_i$ occurs earliest in $\overset{>}{\Theta^n}$.


\vspace{2em}

And so, in the case of dealing with our problems in magick, especially in Chaos Magick, and especially SIGIL MAGICK; if we had some SOI as we encountered in \textbf{\hyperref[EQSOI1]{Equation \ref{EQSOI1}}} --- and which, as we did in \textbf{\hyperref[TRANSSOI]{Transformation \ref{TRANSSOI}}}, we then reduce to its modal sequence ``TEYHOSINAMUBDVRXLC", we shall have essentially computed and applied the \textbf{basic modal sequence statistic} (BMSS).

\vspace{2em}

In the example referenced above, some pre-processing was actually involved prior to the derivation of that final sequence --- essentially, given that there were some extraneous symbols in the original sequence; such as white-space, numbers and punctuation; it was somewhat easier, or better, that the final SOURCE be computed using a modified/cleaned version of the SOI. However, for purposes of illustrating to newcomers, how the basic process works, we shall repeat this computation of the MSS or BMSS, using a different, simple starter sequence as we do hereafter...


\vspace{2em}


So, assuming we start out with the following SOI: ``I LOVE YOU", and which, because we do not wish to process invisible characters, we first transform into ``ILOVEYOU", so that then, our starter SOI is:


\begin{equation}
\label{EQNSOI1}
\Theta = \langle \text{ILOVEYOU} \rangle
\end{equation}

\vspace{2em}

We can proceed to compute the BMSS for $\Theta$, by noting the following things about the sequence:

\vspace{2em}

\begin{enumerate}
\item{First, we need to know all the unique symbols/letters in the sequence --- this is what is known as the \textbf{symbol set} of the sequence: i.e $\psi(\Theta)$. And so, given the value of our input sequence in \textbf{\hyperref[EQNSOI1]{Equation \ref{EQNSOI1}}}, the corresponding value shall be as shown below:\\

\begin{trans}
\label{TRANSSOISS}
$ $\\
$\Theta = \langle \text{ILOVEYOU} \rangle \xrightarrow{O_{ss}} \langle \text{ILOVEYU} \rangle = \psi(\Theta)$ 
\end{trans}

\vspace{2em}

And so, for all our practical purposes, we now know that the SOI only contains the 7 unique letters ``ILOVEYU", and which, \textbf{if we wish} to have them sorted not in their order of natural occurrence, but rather, in their order of alphabetical occurrence, then we can also obtain what is known as the \textbf{Lexical Basis} of $\Theta$ from the above result, simply by sorting the result above in alphabetical order, so that we obtain ``EILOUVY". Thus, we can compute a SOI's symbol set by computing its Lexical Basis.
}

\item{ Next, for each of the symbols in the SOI's symbol set, we want to determine when it first occurred in the SOI. Mathematically, we shall be computing the value $I(\theta_i \in \psi(\Theta), \Theta)$ for each distinct element of $\Theta$. In a very simple tabulation, we can see that:

\vspace{2em}

\begin{tabular}[t]{|l|c|c|c|c|c|c|c|c|}
\hline

$\theta_i \in \Theta$ & I & L & O & V & E & Y & \hl{O} & U \\
\hline
$i = I(\theta_i \in \Theta,\Theta)$ & 1 & 2 & 3 & 4 & 5 & 6 & \hl{7} & 8 \\
\hline
$i^* = I(\theta_i \in \psi(\Theta), \Theta)$ & 1 & 2 & 3 & 4 & 5 & 6 & \hl{3} & 8 \\
\hline
\end{tabular}

\vspace{2em}

In which case we notice that all the letters in the original SOI have unique values of this metric other than the letter ``O", which in the SOI occurred first at position 3, and then later at position 7. The highlighted value in the bottom row is so that we realize that, despite this multiple occurrence of the symbol, for the purposes of computing the BMSS, we shall only be considered with the \textbf{first occurrence} of a symbol, and thus, we apply the first value of $i$ for all cases of $\theta_i$ that occur multiple times in $\Theta$, when computing the metric $I(\theta_i \in \psi(\Theta), \Theta)$, that we have also denoted $i^*$ to differentiate it from basic $i$.

}

\item {Next, we want to also determine, how many times each of the distinct elements in the SOI's symbol set occurred in the SOI. For this, we simply iterate over the items from \textbf{\hyperref[TRANSSOISS]{Transformation \ref{TRANSSOISS}}}, and merely count how many times each appeared in the SOI. A suitable tabulation might help drive the method/point home:

\vspace{2em}

\begin{tabular}[t]{|l|c|c|c|c|c|c|c|}
\hline

$\theta_i \in \psi(\Theta)$ & E & I & L & O & U & V & Y \\
\hline
$f_i = \invpi(\theta_i \in \psi(\Theta), \Theta)$ & 1 & 1 & 1 & \hl{2}  & 1 & 1 & 1 \\
\hline
\end{tabular}

\vspace{2em}

}

\item { Having computed those two values then, we can then proceed to compute the BMSS, simply by sorting the values in the SOI's symbol set using the compound criteria: 
\begin{enumerate}
\item First, sort items in descending order of their frequency, $f_i$, such that the most occurring item comes first.
\item In case any two items have similar values of frequency, then sort them in ascending order of their first-occurrence in the original sequence, SOI. Essentially, sort them in ascending order by $i^*$
\end{enumerate}

\vspace{2em}

We can see this happening for our example SOI, as shown in the table:

\vspace{2em}

\begin{tabular}[t]{|l|c|c|c|c|c|c|c|}
\hline

$\theta_i \in \psi(\Theta)$ & E & I & L & O & U & V & Y \\
\hline
$f_i = \invpi(\theta_i \in \psi(\Theta), \Theta)$ & 1 & 1 & 1 & 2  & 1 & 1 & 1 \\
\hline
$i^* = I(\theta_i \in \psi(\Theta), \Theta)$ & 5 & 1 & 2 & 3 & 8 & 4 & 6  \\
\hline
$Rank(\theta_i \in \psi(\Theta): > f_i, < i^*) = R_i$ & 5 & 2 & 3 & 1 & 7 & 4 & 6  \\
\hline
\end{tabular}
    \captionof{table}{Tabular Analysis: the process of computing BMSS Rank for each distinct item in the SOI symbol set $\psi(\Theta)$}
  \label{TABSOIBMSSRANK}

\vspace{2em}

or rather, after properly sorting them by the final row --- the rank, $R_i$, we have the equivalent tabulation:


\vspace{2em}

\begin{tabular}[t]{|l|l|c|c|c|c|c|c|c|}
\hline

&$Rank(\theta_i \in \psi(\Theta): > f_i, < i^*) = R_i$ & 1 & 2 & 3 & 4 & 5 & 6 & 7  \\
\hline
&$f_i = \invpi(\theta_i \in \psi(\Theta), \Theta)$ & 2 & 1 & 1 & 1  & 1 & 1 & 1 \\
\hline
&$i^* = I(\theta_i \in \psi(\Theta), \Theta)$ & 3 & 1 & 2 & 4 & 5 & 6 & 8  \\
\hline
$\overset{>}{\Theta}$ & $\theta_i \in \psi(\Theta)$ & O & I & L & V & E & Y & U \\
\hline

\end{tabular}
    \captionof{table}{Tabular Analysis: computing BMSS (``Source") for SOI $\Theta$ based on sorted mappings based on Rank}
  \label{TABSOIBMSS}

\vspace{2em}

And clearly then, in that last table, we arrive at the desired \textbf{Basic Modal Sequence Statistic} or just the \textbf{Modal Sequence Statistic} as originally specified in \cite{transformatics}, and which value we would also obtain using the TEA programming language when we run the command \texttt{u!:} against the given SOI as shown below:


\vspace{2em}


 %\small
  \begin{tcolorbox}[teaterminalstyle, title=TEA Program: computing BMSS for SOI using TEA, breakable]
  %\begin{lstlisting}[language=TEA, caption={TP C7}, label={LSTC7}, numbers=left]
  \begin{lstlisting}[language=TEA,breaklines=true]
i!:{ILOVEYOU}
u!:

#RESULTANT MEMORY STATE: (=OILVEYU, VAULTS:{})
   \end{lstlisting}
  \end{tcolorbox}
    \captionof{figure}{EXAMPLE: computing BMSS (``Source") for a SOI (in ELA) using TEA}
  \label{FIGTEAEXASOIBMSS}


\vspace{2em}



}
\end{enumerate}


And thus, using the standard methods, our BMSS for the given SOI in $\Theta$ is just ``OILVEYU".


\subsubsection{Computing the BMSS using a Rank Function}
\label{SECBMSSRANK}


For stricter mathematicians or those in seek of a more rigorous computation of the BMSS than what we have in \textbf{\hyperref[TABSOIBMSS]{Table \ref{TABSOIBMSS}}} --- particularly, noting that, in the first row of that table --- where we read off the ranks of each item, and so that we can then know which of the items in $\psi(\Theta)$ corresponds to which position in $\overset{>}{\Theta}$, we might leave some people with qualms or unsatisfied. 

\vspace{2em}

For example, one might wonder and correctly so... \textbf{How, when given two criteria by which to sort a list of items, might we mathematically and correctly so, arrive at the correct sorted list given what we know?}

\vspace{2em}

That problem is not ill-founded because, indeed, though our case in the shown example spanned just 7 unique symbols, there might be scenarios in which one must process a larger dataset, perhaps with hundreds of unique symbols --- for example, if processing symbols in languages with large alphabets such as perhaps Chinese? And so, for completeness's sake, we shall here augment the treatment of the BMSS that we got in the last section, with a simple formulation, still similar to what we have used, but which concretizes the rank function with an algebraically meaningful and computationally feasible function (and not the heuristic, abstract $Rank(\theta_i \in \psi(\Theta): > f_i, < i^*) = R_i$).

\vspace{2em}

For keeping things simple, and [re-]utilizing what we already know, we shall merely build upon that last example, and thus, just go straight to the final step of computing the BMSS. Essentially, we are going to replace the Rank metric with the following formula:


\begin{multline}
\label{EQTNBMSSRANK}
Rank(\theta_i \in \psi(\Theta): > f_i, < i^*) = R_i \implies R_i = -f_i + \epsilon \times i^*
\end{multline}



And where $\epsilon$ is a suitably very small \textbf{pure fractional number}\cite{Lutalo2024gtnc}\footnote{This step is one of the critical differences between the two alternative ways to arrive at an MSS: in this method, we use the $\epsilon$, so that, where any two items in $\psi(\Theta)$ have a tie on their relative frequency, $f_i$, we resolve that by their assuredly different relative position in the associated first-occurrence symbol set, $i^*$}, and so that, if we return to our example, and set the necessary constant $\epsilon$ to something like $1 \times 10^{-6} = 1e^{-6} = 0.000001$, then, using the same values for the per-item frequency, $f_i$, and their first-occurrence positional indices, $i^*$ from \textbf{\hyperref[TABSOIBMSS]{Table \ref{TABSOIBMSS}}}, we then shall compute a table that still correctly gets us the corresponding, and correct BMSS as such:


\vspace{2em}

\begin{tabular}[t]{|c|c|c|c|}
\hline
Sort($< R_i$) & & & $\overset{>}{\Theta}$\\
\hline
\makecell[c]{$Rank(\theta_i \in \psi(\Theta): > f_i, < i^*) =$\\$ -f_i + \epsilon \times i^*$}
& \makecell[c]{$f_i =$\\$ \invpi(\theta_i \in \psi(\Theta), \Theta)$}
& \makecell[c]{$i^* =$\\$ I(\theta_i \in \psi(\Theta), \Theta)$} 
& $\theta_i \in \psi(\Theta)$ \\
\hline
 -1.999998 & 2 & 3 & O \\
\hline
 -1.0 & 1 & 1 & I \\
\hline
 -0.999999 & 1 & 2 & L \\
\hline
 -0.999997 & 1 & 4 & V \\
\hline
 -0.999996 & 1 & 5 & E \\
\hline
 -0.999995 & 1 & 6 & Y \\
\hline
 -0.999993 & 1 & 8 & U \\
\hline
\end{tabular}
    \captionof{table}{Tabular Analysis: computing BMSS for SOI $\Theta$, using a Rank Heuristic}
  \label{TABSOIBMSS2}


\vspace{2em}

And thus, by this method, depicted in \textbf{\hyperref[TABSOIBMSS2]{Table \ref{TABSOIBMSS2}}}, we arrive at the correct BMSS, also to become the magical ``SOURCE" corresponding to our original/input SOI, after ranking the symbols from $\psi(\Theta)$ in ascending order based on the computed heuristic rank metric $R_i$. Among the allures and beauty of this method, is the observation that it is easy to implement in most programming languages whether or not they support sorting of collections or sequences based on composite or compound criteria or not --- because, unlike the original MSS definition\cite{transformatics}, this method then reduces the sorting criteria from two, to just one --- we sort in descending order of the BMSS Rank, i.e Sort($< R_i$).


\section{The Positional Modal Sequence Statistic (PMSS)}
\label{SECPMSS}


Away from the original method of computing a modal sequence statistic, a new, somewhat different, but mathematically and conceptually useful alternative came up in the course of researching and working on the material of this grimoire --- refer to  \textbf{\hyperref[FIGEXASOISIGILA]{Figure \ref{FIGEXASOISIGILA}}}, and later, also updated and corrected, as in \textbf{\hyperref[FIGEXASOIMSS]{Figure \ref{FIGEXASOIMSS}}}. That method, somewhat based on the same initial logic as what we saw of computing the MSS in \textbf{\hyperref[SECBMSS]{Section \ref{SECBMSS}}}, instead leads us to a metric --- a statistical measure nonetheless, and no less relevant than the original MSS --- that we shall refer to as \textbf{Positional Modal Sequence Statistic} (PMSS), or perhaps ``Position-Weighed Frequency Measure" (PWFM) --- mostly because of how its computation proceeds:

\vspace{2em}

\begin{alg}[The \textbf{Position-Weighed Frequency Measure} (PWFM)]
\label{ALGPWFM}
$ $\\
\begin{enumerate}
\item \textbf{GIVEN} the input ordered sequence of symbols as $\Theta$.
\item \textbf{COMPUTE} the corresponding symbol set: $\psi(\Theta)$.
\item \textbf{COMPUTE} a mapping of each of the unique symbols in $\psi(\Theta)$ to its \textbf{first-occurrence position} (FOP) in the original sequence's \textbf{unspecific symbol set}\cite{ossipaper} $\breve{i} : I(\theta_i \in \psi(\Theta), \hat{\psi}(\Theta)) = \breve{i}$
\item \textbf{COMPUTE} a mapping of each of the unique symbols in $\psi(\Theta)$ to its \textbf{relative frequency} (RF) in the original sequence: $f_i : f_i = \invpi(\theta_i \in \psi(\Theta), \Theta) = f_i$
\item{ \textbf{COMPUTE} a mapping of each of the unique symbols in $\psi(\Theta)$ to its PMSS RANK --- a measure, a rank --- the \textbf{Position-Weighed Frequency} (PWF), $R_i$, based on the size of the symbol set sequence, $\invpi(\psi(\Theta))$, the RF of the associated symbol, $f_i$ and its  FOP, $i^*$ as such:

\begin{multline}
\label{EQTNPMSSRANK}
Rank(\theta_i \in \psi(\Theta): > f_i, < \breve{i}) = R_i \implies R_i = (\invpi(\psi(\Theta)) - \breve{i}) \times f_i
\end{multline}

}
\item \textbf{SORT} the items in the symbol set, $\psi(\Theta)$, in descending order of their corresponding PWF.
\item \textbf{RETURN} the final sorted sequence as the \textbf{PMSS}.
\end{enumerate}
\end{alg}


\vspace{2em}

So, to see how this method would work given a realistic example or scenario, we shall leverage the same simple starting sequence as we did in the last section, and thus, we shall proceed to compute the PMSS of the SOI depicted in \textbf{\hyperref[EQNSOI1]{Equation \ref{EQNSOI1}}}, and which, as we also did in that section, and as we should also do as per the second step of  \textbf{\hyperref[ALGPWFM]{Algorithm \ref{ALGPWFM}}}, we shall be operating on the symbol set $\psi(\Theta) = ILOVEYU$ as we arrived at in \textbf{\hyperref[TRANSSOISS]{Transformation \ref{TRANSSOISS}}}, such that, $\invpi(\psi(\Theta)) = 7$. And so that, given the other steps in the algorithm are [almost] as what we already encountered in \textbf{\hyperref[SECBMSSRANK]{Section \ref{SECBMSSRANK}}} --- for example, we can re-use the same values of $\theta_i$, and $f_i$ as we already obtained --- see \textbf{\hyperref[TABSOIBMSSRANK]{Table \ref{TABSOIBMSSRANK}}}, and so that, we only need compute the FOP values, $\breve{i}$, and then using the new rank formulation depicted in \textbf{\hyperref[EQTNPMSSRANK]{Equation \ref{EQTNPMSSRANK}}}, we arrive at a tabulation of the necessary computation as depicted in the following table:



\vspace{2em}

\begin{tabular}[t]{|c|c|c|c|}
\hline
$\theta_i \in \psi(\Theta)$ 
& \makecell[c]{$f_i =$\\$\invpi(\theta_i, \Theta)$}
& \makecell[c]{$\breve{i} =$\\$ I(\theta_i, \hat{\psi}(\Theta))$}
& \makecell[c]{$R_i =$\\$(\invpi(\psi(\Theta)) - \breve{i}) \times f_i$} \\
\hline
E & 1 & 5 & $(7-5) \times 1 = 2$ \\
\hline
I & 1 & 1 & $(7-1) \times 1 = 6$ \\
\hline
L & 1 & 2 & $(7-2) \times 1 = 5$ \\
\hline
O & 2 & 3 & $(7-3) \times 2 = 8$ \\
\hline
U & 1 & 7 & $(7-7) \times 1 = 0$ \\
\hline
V & 1 & 4 & $(7-4) \times 1 = 3$ \\
\hline
Y & 1 & 6 & $(7-6) \times 1 = 1$ \\
\hline
\end{tabular}
    \captionof{table}{Tabular Analysis: the process of computing PMSS Rank for each distinct item in the SOI symbol set $\psi(\Theta)$}
  \label{TABSOIPMSSRANK}

\vspace{2em}

And so that, having worked through the first 5 steps of \textbf{\hyperref[ALGPWFM]{Algorithm \ref{ALGPWFM}}} as depicted in \textbf{\hyperref[TABSOIPMSSRANK]{Table \ref{TABSOIPMSSRANK}}}, we can proceed to the 6$^{th}$ step --- sorting the items in our mapping as per the PMSS Ranks --- in descending order of the PWF ($R_i$), and so that, we can just directly read-off the resulting PMSS as shown in the following tabulation:



\vspace{2em}

\begin{tabular}[t]{|c|c|c|c|}
\hline
Sort($< R_i$) & & & $\overset{>}{\Theta}$\\
\hline
\makecell[c]{$R_i =$\\$(\invpi(\psi(\Theta)) - \breve{i}) \times f_i$} 
& \makecell[c]{$f_i =$\\$\invpi(\theta_i, \Theta)$}
& \makecell[c]{$\breve{i} =$\\$I(\theta_i, \hat{\psi}(\Theta))$}
& $\theta_i \in \psi(\Theta)$ \\
\hline
8 & 2 & 3 & O \\
\hline
6 & 1 & 1 & I \\
\hline
5 & 1 & 2 & L \\
\hline
3 & 1 & 4 & V \\
\hline
2 & 1 & 5 & E \\
\hline
1 & 1 & 6 & Y \\
\hline
0 & 1 & 7 & U \\
\hline
\end{tabular}


\vspace{2em}

So, just like we obtained using the first method --- of the BMSS/Basic Modal Sequence Statistic --- we also can see that we still obtain the same [and correct] modal sequence statistic for our similar initial input sequence, ``ILOVEYOU", when we apply this otherwise different computation method of the Position-Weighed Frequency, that gives us what we are calling the \textbf{Positional Modal Sequence Statistic}, as [still] ``OILVEYU".


\section{Further Analysis and Comparison of BMSS to PMSS}
\label{SECMSSANALYSIS}


For those with a keen eye for beautiful mathematics, especially from the perspective of analysis, you shall realize that, the allure and core justification of the second method is not only in the fact that, like the BMSS method, it allows us to compute (whether manually or using a computer) the MSS of some sequence without having to worry about dealing with the composite sorting criteria --- in this case, reducing the criteria to just one --- the PWF, while, at the same time, and unlike BMSS, the PMSS can be justified by the fact that it is simple to see that it essentially gives higher-priority to the frequency of occurrence of a symbol in the source sequence ($f_i$), so that, the first sorting criteria in the original MSS definition is obeyed in the formulation of the PWF (see \textbf{\hyperref[EQTNPMSSRANK]{Equation \ref{EQTNPMSSRANK}}}), but also that, where any two terms have a tie on frequency, they definitely shall never have a tie on their \textbf{position of first-occurrence} (POF) in the original sequence's \textbf{unspecific symbol set} ($\hat{\psi}(\Theta)$) --- this last symbol set, not to be confused with any other symbol set of the sequence $\Theta$ (such as $\psi(\Theta)$, $\psi(\Theta)_\beta$, $\overset{>}{\psi(\Theta)}$, etc) because, it is the only one of its symbols sets, that specifically treats of the order of first-occurrence of symbols in $\Theta$ irrespective of any other conditions or criteria.

\vspace{2em}

Further, note that ``Weight" aspect in the PMSS method's WPF stems from the fact that we want to adjust an item's occurrence frequency ($f_i$) by the item's distance from the end of the unspecific symbol set of the originating sequence --- thus the term $(\invpi(\psi(\Theta)) - \breve{i})$ in the formulation of the PWF. That said, note that, in the BMSS method of obtaining a MSS, we encounter the term $\epsilon$ that didn't exist or originate anywhere in the original sequence, sequence specifications or the original problem. This might not always be an issue, however, it becomes a matter of major concern once you realize that the choice of the value to assign to $\epsilon$ has not much to do with the properties of the sequence under analysis apart from the fact that it is chosen to be \textbf{suitably small} so that when computing the rank ($R_i$) in the BMSS method, we ensure that similar or close values of $f_i$ never collide as long as positional indices of the concerned terms --- that is $i^*$ --- are different. This matter then, unless we reformulate (\textbf{\hyperref[EQTNBMSSRANK]{Equation \ref{EQTNBMSSRANK}}}) such that $\epsilon$ is a for example a function of $\invpi(\Theta)$ --- especially since this value puts an upper limit on any $f_i$ we shall ever encounter, it otherwise remains a matter that the user of that method must manually decide upon, and so that, some careless people might use that equation with a wrong value of $\epsilon$, and obtain ``wrong" rankings, and thus think or proceed to work with a MSS that isn't semantically correct.



\vspace{2em}

Finally, note that, the PMSS algorithm (\textbf{\hyperref[ALGPWFM]{Algorithm \ref{ALGPWFM}}}) also has the extra allure that, it somewhat would never necessitate nor involve processing of fractions or fractional numbers as one might have to do when using the BMSS method. And for that reason alone, it might be more preferable for those computing MSS values by-hand or manually. But also, because of sometimes tricky quirks in platform-specific handling of floating-point numbers and such, even for computer programs computing the MSS, \textbf{in case we don't use an algorithm that can compute the MSS based on sorting using the original composite criteria}, then the PMSS method is preferable --- mostly for its algebraic and numerical simplicity. However, as these are all new innovations we have identified and developed while conducting research in \textbf{applying Transformatics to Magick}, who knows... perhaps magicians prefer having more than one way to skin a cat, but, more research, applications and further analysis shall tell, which of these methods of computing an MSS eventually wins or if all survive unto posterity!


\vspace{2em}

Oh! And last, but not least... we hope that the reader of this chapter has developed deeper appreciation for and love for applying mathematics to matters usually relegated to faith, trust and mere belief. For example, do you notice anything peculiar about the example we have treated of in this chapter? The idea of analyzing the universally common phrase ``I love you" --- don't you see something interesting (if not weird) about its MSS being ``OILVEYU"? Like... perhaps, and especially from the creative perspective of a magician... like, people that utter such words to each other, eventually end up also sharing ``oily" things! But, and also as how certain types of magicians such as diviners, kaballists, ``readers", etc. arrive at some of the answers they use to solve problems for their clients, we see in this method of applying transformatics to ``demystifying the hidden side of what is otherwise always in before our eyes" a way to read the unseen; a call to indeed \textit{read between lines} for example... What else do you see in the phrase ``I LOVE YOU", when it is mathematically reduced to ``ILOVEYU" or better, ``OILVEYU"?\footnote{Not to spoil the fun, but, perhaps one sees Lubricants? Sex Magick? Aloe vera? Oliver Twists? It's definitely just crazy and exciting! What of then analyzing our heart's deepest desires and wants using these methods? See? That is Magick!}


\newpage
\chapter{Brief Detour: Modern CHAOS MAGICK}
\label{APPENDCHAOSMAGICK}

Modern chaos magick is a pragmatic, results-oriented approach that treats belief as a tool: practitioners adopt, discard, and remix symbols, deities, and techniques according to efficacy rather than lineage or dogma. Common practices include sigilization, gnosis induction (via trance, excitement, or stillness), servitor creation, and fluid paradigm shifting, often framed in psychological terms and adapted for solo practice or small working groups. Contemporary discussions emphasize versatility, personal experimentation, and minimal metaphysical commitments beyond ``what works,'' reflecting a culture that blends occult technique with modern psychology and media aesthetics \cite{carrollLiberNullPsychonaut,carrollLiberKaos,hine1995condensed}.

\section{Core Practices in Contemporary Settings}

\begin{itemize}
  \item \textbf{Paradigm shifting:} Practitioners intentionally ``wear'' belief systems (e.g., Kabbalistic, animist, scientific) for the duration of a working, then drop them to avoid dogma. This strategic flexibility is central to modern schools, forums, and informal lodges focused on method rather than metaphysics \cite{carrollLiberNullPsychonaut}.
  \item \textbf{Sigils and servitors:} Personal symbols (sigils) encode intent and are charged under gnosis; servitors are designed semi-autonomous constructs built to execute sustained tasks. Both suit individual practice and small group labs due to their portability and low ritual overhead \cite{carrollLiberNullPsychonaut,hine1995condensed}.
  \item \textbf{Gnosis methods:} Techniques range from breathwork and meditation to music, dance, and sensory overload or deprivation—selected for reliability and personal fit rather than tradition \cite{carrollLiberNullPsychonaut}.
  \item \textbf{Group labs:} Contemporary chaos circles often operate like experimental workshops: set a hypothesis, select a paradigm and method, run the ritual, log outcomes, iterate. Carroll outlines temple setup and group protocols; Hine favors modular, remixable rites \cite{carrollLiberKaos,hine1995condensed}.
\end{itemize}

\section{Contrasts with Traditional Systems}

\subsection{Kabbalah}
\begin{itemize}
  \item \textbf{Structure:} Highly mapped correspondences (Sephiroth, paths), graded initiation, angelic hierarchies.
  \item \textbf{Chaos contrast:} Uses Kabbalistic maps instrumentally when helpful; resists fixed cosmology, prioritizing technique and results over lineage.
\end{itemize}

\subsection{Voodoo/Vodou}
\begin{itemize}
  \item \textbf{Structure:} Community-rooted rites, stable pantheons (lwa), initiations, ancestral veneration.
  \item \textbf{Chaos contrast:} Borrows interfaces (spirit masks, possession-style gnosis) tactically but avoids cultural claims; emphasizes personal constructs over pantheons.
\end{itemize}

\subsection{Ceremonial Magick}
\begin{itemize}
  \item \textbf{Structure:} Formal invocations, graded systems, strict correspondences, elaborate ritual tech.
  \item \textbf{Chaos contrast:} Retains ritual forms as options but strips dogma; reframes tools psychologically, compresses complexity into result-driven protocols \cite{carrollLiberKaos,hine1995condensed}.
\end{itemize}

\subsection{Spiritualism}
\begin{itemize}
  \item \textbf{Structure:} Mediumship, evidential communication, moral frameworks.
  \item \textbf{Chaos contrast:} May adopt spirit-model paradigms ad hoc (e.g., servitors as ``constructed spirits''), but centers on intentional design and testing rather than evidential survival claims.
\end{itemize}

\subsection{Psionics}
\begin{itemize}
  \item \textbf{Structure:} Instrumental or mental ``energy'' models, devices, protocols.
  \item \textbf{Chaos contrast:} Overlaps in technique (focus, visualization, feedback loops), yet chaos treats models as provisional lenses; no commitment to a singular psi ontology \cite{carrollLiberNullPsychonaut}.
\end{itemize}

\subsection{Alchemy}
\begin{itemize}
  \item \textbf{Structure:} Transformative symbolism, matter-spirit analogies, long-cycle operations.
  \item \textbf{Chaos contrast:} Reinterprets alchemical symbolism psychologically (intent-transmutation), compressing timelines via direct sigil/gnosis work; favors measurable outcomes over allegorical fidelity \cite{hine1995condensed}.
\end{itemize}

\section{Contexts of Practice}
\begin{itemize}
  \item \textbf{Independent practitioners:} Most work solo, using sigils, servitors, and short rites designed for daily life; journals emphasize metrics and iteration \cite{hine1995condensed}.
  \item \textbf{Small circles and online schools:} Informal lodges and study groups share protocols, swap paradigms, and run joint workings \cite{carrollLiberKaos}.
  \item \textbf{Hybrid ceremonialists:} Some ceremonial magicians adopt chaos methods to de-dogmatize practice while keeping formal aesthetics \cite{carrollLiberKaos}.
\end{itemize}




\begin{table}[H]
  \begin{tabular}{|p{0.95\textwidth}} % Left border only
    \hline
\begin{figure}[H]
  \begin{center} % Rotate a single page of a PDF by 90 degrees
   \includegraphics[height=0.7\textheight]{resources/scanned_example_magickal_art_shrine_design}\\
   \vspace{2em}
  \end{center}
\end{figure}\\
    \cline{1-1} % Bottom border only
  \end{tabular}
\end{table}



\newpage
\chapter{MORE STANDARD SEALS (G*D): Grand Order Divine SET}


In this section we have a few more vital, reference-friendly, and readily applicable sacred and mystical seals of principles, words and names that a pragmatic occultist might source and or adapt for their core formulations --- sigilic programs for example --- refer to \textbf{\hyperref[SECOZINSYSTEM]{Section \ref{SECOZINSYSTEM}}}.






\begin{table}[htp]
  \begin{tabular}{|p{0.95\textwidth}} % Left border only
    \hline
\begin{figure}[H]
  \begin{center} % Rotate a single page of a PDF by 90 degrees
   \includegraphics[angle=90,width=1\textwidth]{resources/enchantment_path__2022__by_nemesisfixx_dfj38uz}\\
   \caption{Titled \textbf{Enchantment Path (2022)}, this occult painting by \textbf{J. Willrich Lutalo C.M.R.W}\cite{jwl2025art}, via a state machine, encodes a kind of ritual formula --- a \textbf{Ritual Program} that either automatons or psymatons; human magicians or angels may process and execute as a means of performing \textbf{Transcendental Magick} --- \texttt{G.*.D} Modern Rites}
   \vspace{2em}
  \end{center}
\end{figure}\\
    \cline{1-1} % Bottom border only
  \end{tabular}
\end{table}



{
\centering

\newpage
\textbf{GOD:}


\normalsize
To manifest the PURE ABSTRACT Godform - \textbf{GOD}, following in the system of manifesting and expression introduced in \textbf{\hyperref[SECOZINSYSTEM]{Section \ref{SECOZINSYSTEM}}},  the necessary sigil would be based on the \textit{strange} SOURCE: \texttt{GOD}, and so that, the necessary expression via OZIN is as:


\begin{figure}[H]
  \begin{center}
   \includegraphics[height=0.8\textheight]{resources/seal_of_god.pdf}\\
   \caption{OZIN System SEAL for \textbf{GOD}}
  \label{FIGOZINSEALGD1}
  \end{center}
\end{figure}

\vspace{2em}



\newpage
\textbf{AHA:}


\normalsize
To manifest the PURE ABSTRACT Godform - \textbf{AHA}, following in the system of manifesting and expression introduced in \textbf{\hyperref[SECOZINSYSTEM]{Section \ref{SECOZINSYSTEM}}}, the necessary sigil would be based on the SOURCE\footnote{Note that, for purposes of enlightening students, it might also be possible to proceed for certain special purposes such as with abstract forms, to merely  proceed by using the method of directly processing the word as shown for this case.} \texttt{AH}, and so that, the necessary expression via OZIN is as:


\begin{figure}[H]
  \begin{center}
   \includegraphics[height=0.8\textheight]{resources/seal_of_aha.pdf}\\
   \caption{OZIN System SEALS for \textbf{AHA}}
  \label{FIGOZINSEALGD2}
  \end{center}
\end{figure}

\vspace{2em}












}










%%%%%%%%%%%%------------[ END APPENDICES ]------------------%%%%%%%%%%%%

\end{appendices}

\bibliographystyle{unsrt}
\bibliography{references}


\newpage
\begin{figure}[H]
  \begin{center}
   \includegraphics[scale=0.8, angle=90,height=0.8\textheight]{resources/the_message_masonic.pdf}\\
   %\caption{Welcome Seeker, to Immersive Future Illuminism}
  %\label{FIGMESSAGECUNEIFORM}
  \end{center}
\end{figure}


\newpage
\chapter*{BONUS: The MITI Magickal Language}
\addcontentsline{toc}{chapter}{BONUS: The MITI Magickal Language}
\label{SECMITILANGUAGE}

\begin{table}[H]
  \begin{tabular}{|p{0.95\textwidth}} % Left border only
    \hline
\begin{figure}[H]
  \begin{center}
  \includegraphics[height=0.7\textheight]{resources/scanned_key_miti_magickal_language_definition.pdf}
  \caption{This is the only existing copy of the \textbf{ORIGINAL MITI Cipher} --- a magickal language that we developed at Nuchwezi Esoteric School, for use among the Illuminates of Nuchwezi. This is a scan of the only copy of this powerful language, from our archives. It is a mapping from $\psi_{36}$ to $\psi_{miti}$; the first three lines encode A-Z, the $\infty$ added to encode ``white-space", while the lower, rightmost line encodes digits 1-9, the last symbol $\boxed{}$ encoding ``0". Those glyphs are meant to appear and be written as they are --- the letters always balance along a vertical line on both sides, while the numbers/digits STRICTLY are placed on ONLY a specific side of the vertical line. Of course, MITI can be written horizontally or in any orientation, however, these rules must be adhered to, to maintain consistency, elegance and correctness of the mystical language.}
  \label{FIGMAGICKMITI}
  \end{center}
\end{figure}\\
    \cline{1-1} % Bottom border only
  \end{tabular}
\end{table}

This final language is a bonus, originally not meant to be included in this grimoire, however, and especially for posterity's sake, here included, so that there is no excuse for the student or practicing magician, to not pick up and apply a no-bullshit occult tool fit for the most advanced magical work, and apply it well.

\vspace{2em}

\textbf{``Miti"} is Runyoro-Runyakitara for ``Trees", and given this is the only language we know that allows one to not only properly write on-trees, but also write things that look like and are well anchored like trees, we decided to call it that --- the \textbf{MITI Magickal Language}. Its prominent and distinguishing feature is that it allows one to write by expressing glyphs along a vertical line --- actually, MITI could be written in horizontal mode once you get the idea, however, the standard practice within IoNA is to write it vertically, somewhat like some oriental languages such as Mandarin (Chinese), Japanese (Tategaki) and Hanja (Korean). Written thus, the text is then meant to be read vertically, column-by-column, typically left-to-right as we shall soon come to see.



\begin{table}[H]
  \begin{tabular}{|p{0.95\textwidth}} % Left border only
    \hline
\begin{figure}[H]
  \begin{center} % Rotate a single page of a PDF by 90 degrees
   \includegraphics[angle=90, height=.5\textheight]{resources/scanned_example_use_ozin_and_miti_for_meditation_magickal_diary_notes}\\
   \caption{In many a magician's mind, there is hope, and honestly so, that humans shall one day win over the hearts of an entirely \textit{other kind} --- aliens perhaps, or perhaps our distant relatives from other worlds.}
  \end{center}
\end{figure}\\
    \cline{1-1} % Bottom border only
  \end{tabular}
\end{table}



Note that, for all practical purposes, the mathematical definition of the MITI language is like that of the MEDINA language (check \textbf{\hyperref[SECMEDINASYSTEM]{Section \ref{SECMEDINASYSTEM}}}). We shall not attempt to reproduce it here. Moreover, the actual and complete specification of the language's symbol-set is as depicted in \cite{lutalo_2025_miti} and also here in \textbf{\hyperref[FIGMITIALPHABET]{Figure \ref{FIGMITIALPHABET}}}.




\begin{figure}[htp]
  \begin{center}
   \includegraphics[height=0.9\textheight]{resources/annoted_language_cipher_miti_extended.pdf}\\
   \caption{The COMPLETE MITI ALHABET Symbol Set}
  \label{FIGMITIALPHABET}
  \end{center}
\end{figure}


And as for how it is applied\footnote{Also, a video mini-lecture concerning how to professionally write and read the MITI language is offered online via YouTube: \url{https://youtu.be/G1RGW7OR_QE}}, a good example, and one we shall also come back to in the bonus chapter --- \textbf{\hyperref[APPENDCODEHYMNS]{Section \ref{APPENDCODEHYMNS}}}, is as depicted in \textbf{\hyperref[FIGMITIHYMNJUKILA]{Figure \ref{FIGMITIHYMNJUKILA}}}. Note that it is meant to be read top-to-bottom, one line/column at a time, left-to-right. However, once familiar, you can read cases where it is written in horizontal mode too --- such as in the title of the hymn shown in that figure, and which spells out the word ``JUKILA", but is written horizontally!




\begin{table}[H]
  \begin{tabular}{|p{0.95\textwidth}} % Left border only
    \hline
\begin{figure}[H]
  \begin{center}
   \includegraphics[height=0.7\textheight]{resources/scanned_ijuka_jukila_hymn_scribe_sketch_in_miti_filled}\\
   \caption{Page 1 of the Luganda Hymn ``JUKILA", written using the MITI magical language}
\label{FIGMITIHYMNJUKILA}
   \vspace{2em}
  \end{center}
\end{figure}\\
    \cline{1-1} % Bottom border only
  \end{tabular}
\end{table}



\newpage
\chapter*{BONUS: CCM: CANDLE CHAOS MAGICK}
\addcontentsline{toc}{chapter}{BONUS: CCM: CANDLE CHAOS MAGICK}
\label{APPENDCANDLECHAOSMAGICK}


In this section, we shall take a moment to make concrete, some of the ideas we explored in \textbf{\hyperref[SECTALISMANS]{Section \ref{SECTALISMANS}}}. Essentially, and generally for many kinds of purposes --- including merely having beautiful natural light and a source of illumination, but also, for a myriad applications and solutions via use of light and the element of fire (a transmuation, transformation power), we here explore how to create special kinds of candles we shall refer to as \textbf{Chaos Candles}. These candles might look or smell or form literal light just like any other candle, however, one [or more] critical steps go into their making --- for example, and for the purpose of this grimoire, we inform you that it is very possible to take a sigil that has been specifically crafted for the purpose of manifesting something, and then, after incinerating it like via your magical cauldron, you can then take the ashes thus formed, and mix them into the molten wax to be used to form the candle, so that, once the candle solidifies, the candle that you end up with, is not only charged, but shall literally bring your desire to life, and especially for the case of manifesting servitors or spirits, makes the entity come to life and work, as long as the candle is being lit!

\vspace{2em}

Traditionalists, old-generation witches and wizards in pragmatic sorcery shall realize that, traditional methods such as charging or consecrating a magical candle via use of methods such as dressing it up in a magical oil --- or better, making it out of wax mixed with special oil, charging it via a ritual, etching occult sigils on it, etc. might all still work, and yet, the simple method described above shall most times offer the most stealth, and yet effective, clean-result approach! It is also more precise.

\vspace{2em}

\textbf{NOTE:} that this recipe has long been out there\footnote{First written as a submission for inclusion in an IoT journal back in 2022, the original copy is also at: \url{https://www.scribd.com/document/636864989/Chaos-Candle-Recipe-GUIDE}, but is also the one [re-]shared here verbatim.}... but, for purposes of completeness, we shall re-share it here verbatim.

\vspace{2em}

Does it work? YES! Does it make great candles? YES! Check out some examples of what we have produced in the past, and which we also sale via our \textbf{MENUZA BYTES} shop\footnote{Originally at \url{https://menuzabytes.com}, but now at \url{https://menuzabytes.nuchwezi.com/}} --- refer to \textbf{\hyperref[FIGEXATALISCANDLE]{Figure \ref{FIGEXATALISCANDLE}}}.





% insert all the pages
\includepdf[pages=-]{resources/candle_chaos_magick.pdf}



\newpage
\chapter*{BONUS: POETRY and HYMNS for CODE}
\addcontentsline{toc}{chapter}{BONUS: POETRY and HYMNS for CODE}
\label{APPENDCODEHYMNS}


As a finale, realize that just like many other creatives, especially those who go around chasing after mysteries, the magicians, students and philosophers that found this grimoire exciting are also often the kind of folks that enjoy a good and enlightening spiritual experience. In that case, and even though this book wasn't about a [post-]modern church or path inclined towards approaching the divine and mysterious with the respect and sublimity they deserve --- such as Gnostics and early churches did, but also some modern faiths such as Catholics and especially the Eastern Orthodox Churches, we shall take a moment to look into a special hymn --- originally attributed to the \texttt{melodic metal band} \textbf{MAZERA}\footnote{Visit their online museum: \url{https://t.me/mazeraworks}} from and based in Uganda --- that is great stuff for treating of how we might apply occultism and especially occult languages in a church-hymn-book context or perhaps in formal writings meant to be clear and well understood by members and otherwise entirely opaque, nonsense and perhaps outlandish gibberish for non-members.

\vspace{2em}

This hymn, titled ``IJUKA", is picked from the few favorite songs to be found in practice within the \textbf{CODE}: \texttt{Church of Dance Eternal}\cite{lutalo_2025_ogf}. To build rapport, check out the specifics about this song:

\vspace{2em}


\begin{enumerate}
\item It is said that the original words in the underlying poem were received by Cwa Mukama in a dream.
\item The official video --- a lyric video, but mostly focusing on the audio, is accessible via \textbf{\url{https://youtu.be/WSJfE7hnuxw}}
\item The corresponding HD-audio file, plus context about the song can be found at the MAZERA portfolio channel: \textbf{\url{https://t.me/mazeraworks/636}}
\item A Talk-Show discussing this poem and the associated song, with essential questions and answers regarding its origins, meaning and variations, was recorded too and can be found online at YouTube: \textbf{\url{https://youtu.be/5adQVXpZr94}}
\item \textbf{IJUKA} has been immortalized as an official hymn for CODE, and the words of the song as they were first expressed in this song's lyrics in Runyoro-Runyakitara, have also been expressed in the ancient-like, hieroglyphic-like magical language; MEDINA, as found in \cite{lutalo_2025_medina_ijuka}.
\end{enumerate}


\vspace{2em}

However, first, we shall study the words and message of the poem in three languages...

\vspace{2em}


\begin{poembox}

\Cwezi{Ijuka bukyali mukizima, ensi ebyamire niturota}
\Ganda{Jukila nga bukyali munzikiza, ensi ngayebase ngatuloota}
\English{Recall while twas still dark, whole world still asleep and dreaming}

\vspace{2em}

\Cwezi{Binyira mumwanya nibiguruka, biccu, nenyunyuzi mwiguru}
\Ganda{Biwundo mubanga nga bibukaa, ebbire, nemunyenye mugulu}
\English{Vampire bats flying in the air, against a starlit sky}

\vspace{2em}

\Cwezi{Mu Biroto ebitakukengeka}
\Ganda{Mu Biloto ebitategelekeka}
\English{In mysterious dreams impossible to be understood}

\vspace{2em}

\Cwezi{Tuzoire amabale gekitinisa}
\Ganda{Tuzude amayinja g'ekitibwa}
\English{We've discovered precious rocks of honour}

\vspace{2em}

\Cwezi{Garoho ibaralye ogwo Aha}
\Ganda{Galiko amanya ge oyo Awa}
\English{They bear the names of AHA (God of Divine Providence)}

\vspace{2em}

\Cwezi{Aizire kutweta mukizima}
\Ganda{Nga aze kutuyita munzikiza}
\English{He'd come to summon us while twas still dark}

\vspace{2em}

\Cwezi{Kyererezi kijja kwijja mumanye}
\Ganda{Kitangaala kijja kujja mumanye}
\English{Illumination shall come, and you'll gain knowledge}

\vspace{2em}

\Cwezi{Mugume abensi ya Mukama}
\Ganda{Mugume abensi ya Mukama}
\English{Be strong, people of the Lord's nation}

\vspace{2em}

\Cwezi{Eebiro byona mumanye..}
\Ganda{Enaku zoona mumanye..}
\English{Forever, know...}

\vspace{2em}

\Cwezi{Ninywe abaana bbee..}
\Ganda{Yemwe abaana bbee..}
\English{You're his children..}

\vspace{2em}

\Cwezi{Yei yeaaa ooogwo Aaaaha}
\Ganda{Yei yeaaa ooooyo Aaaawa}
\English{Yeah yeah, that One, AHA, The Provider.}

\end{poembox}


\vspace{3em}

Next, let us focus just on the poem in each of the languages on its own.


\vspace{2em}

\newpage
%draw horizontal line
\noindent\rule{\textwidth}{0.4pt}

\vspace{2em}


\begin{poembox}

\section{IJUKA (in CWEZI/Runyoro-Runyakitara)}

\vspace{2em}

{\Large

\begin{verbatim}
Ijuka bukyali mukizima, ensi ebyamire niturota
Binyira mumwanya nibiguruka, biccu, nenyunyuzi mwiguru
Mu Biroto ebitakukengeka
Tuzoire amabale gekitinisa
Garoho ibaralye ogwo Aha
Aizire kutweta mukizima
Kyererezi kijja kwijja mumanye
Mugume abensi ya Mukama.

Eebiro byona mumanye..
Ninywe abaana bbee..
Yei yeaaa ooogwo Aaaaha.
\end{verbatim}
}

\end{poembox}


\vspace{2em}

%%%---LEAVE as IS HERE ----%

\begin{table}[H]
  \begin{tabular}{|p{0.95\textwidth}} % Left border only
    \hline
\begin{figure}[H]
  \begin{center}
   \includegraphics[angle=90, width=0.9\textwidth]{resources/scanned_ijuka_hymn_scribe_sketch_in_medina_filled}\\
   \vspace{2em}
  \end{center}
\end{figure}\\
    \cline{1-1} % Bottom border only
  \end{tabular}
\end{table}


%draw horizontal line
\noindent\rule{\textwidth}{0.4pt}

Realize that in this last figure --- a scan of the hymn written using the \textbf{MEDINA} language --- we have intentionally oriented the otherwise vertical text in a horizontal frame, to emphasize the fact that this writing style could as well be used to write horizontally. However, when used in that context, the scribe would need to instead draw horizontal guiding lines when writing the text by hand, and so the usually vertically-aligned letters would then be aligned horizontally, but still, using the method of keeping the letters spanning the space between two ``guiding lines". 

\vspace{2em}

The example text shown wasn't written with this in mind, and so, when one attempts to read it as is, one shall find that they are then having to read line-by-line, bottom-to-top, and for each column, reading left-to-right. Whereas, if it had been written with the intention of allowing horizontal reading, then, we would be able to read it top-to-bottom per line, while for each row, we would then read left-to-right still. Thus, it is currently somewhat-readable in horizontal mode, except, it tasks the reader to read in the peculiar way of bottom-to-top!

\vspace{2em}

The next figure though, still of the same exact text, and writing style, presents the hymn in the more natural reading orientation given how the text was written, and also has some scribal-decoration added to the calligraphy to bring back those aesthetics that perhaps mostly remain hidden in medieval books by Christian Monks and wizards!

\vspace{2em}

Also, for those interested, a video about how this version of the hymn/poem was written using the MEDINA language, and also covering some of the peculiarities of how to write any text using this writing style, is covered well in the associated video by the author: \textbf{\url{https://youtu.be/Unra0QgLeRE}}



\begin{table}[H]
  \begin{tabular}{|p{0.95\textwidth}} % Left border only
    \hline
\begin{figure}[H]
  \begin{center}
   \includegraphics[width=0.9\textwidth]{resources/scanned_ijuka_hymn_scribe_sketch_in_medina_filled_decorated}\\
   \vspace{2em}
  \end{center}
\end{figure}\\
    \cline{1-1} % Bottom border only
  \end{tabular}
\end{table}



\newpage
%draw horizontal line
\noindent\rule{\textwidth}{0.4pt}

\vspace{2em}


\begin{poembox}

\section{IJUKA/JUKILA (in GANDA/Luganda)}

\vspace{2em}

{\Large

\begin{verbatim}
Jukila nga bukyali munzikiza, ensi ngayebase ngatuloota
Biwundo mubanga nga bibukaa, ebbire, nemunyenye mugulu
Mu Biloto ebitategelekeka
Tuzude amayinja g'ekitibwa
Galiko amanya ge oyo Awa
Nga aze kutuyita munzikiza
Kitangaala kijja kujja mumanye
Mugume abensi ya Mukama.

Enaku zoona mumanye..
Yemwe abaana bbee..
Yei yeaaa oooyo Aaaawa.
\end{verbatim}
}

\end{poembox}


\vspace{2em}


\begin{table}[H]
  \begin{tabular}{|p{0.95\textwidth}} % Left border only
    \hline
\begin{figure}[H]
  \begin{center}
   \includegraphics[angle=90, width=0.9\textwidth]{resources/scanned_ijuka_jukila_hymn_scribe_sketch_in_miti_filled-decorated-p1}\\
   \vspace{2em}
  \end{center}
\end{figure}\\
    \cline{1-1} % Bottom border only
  \end{tabular}
\end{table}

%draw horizontal line
\noindent\rule{\textwidth}{0.4pt}

First, notice that, like for the case of ``IJUKA" written using the MEDINA language style that we have covered in the previous section, we also wish to indicate with the above figure, that the MITI language, despite naturally having been designed to work well with vertical reading and writing of text, and yet, the seasoned writer and reader might use it to render text for horizontal reading. Moreover, and similar to what we saw with the case of MEDINA, unless the scribe had written it out with horizontal reading mode in mind, as it is in that figure shown, we would also face the same quirks of having to read from bottom-to-top, though left-to-right as usual, in case a text originally written in MITI for vertical reading is instead presented oriented horizontally.

\vspace{2em}

That said, and for those interested in learning more about how to write using this MITI language system, a video presenting how this hymn version was written in the language is also available via: \textbf{\url{https://youtu.be/G1RGW7OR_QE}}
 
 
\vspace{2em}

Otherwise, we present the entire hymn/poem in Luganda, written as shown, and presenting in full, spanning two pages though, as shown in the figures hereafter.


\begin{table}[H]
  \begin{tabular}{|p{0.95\textwidth}} % Left border only
    \hline
\begin{figure}[H]
  \begin{center}
   \includegraphics[width=0.9\textwidth]{resources/scanned_ijuka_jukila_hymn_scribe_sketch_in_miti_filled-decorated-p1}\\
   \vspace{2em}
  \end{center}
\end{figure}\\
    \cline{1-1} % Bottom border only
  \end{tabular}
\end{table}


\begin{table}[H]
  \begin{tabular}{|p{0.95\textwidth}} % Left border only
    \hline
\begin{figure}[H]
  \begin{center}
   \includegraphics[width=0.9\textwidth]{resources/scanned_ijuka_jukila_hymn_scribe_sketch_in_miti_filled-decorated-p2}\\
   \vspace{2em}
  \end{center}
\end{figure}\\
    \cline{1-1} % Bottom border only
  \end{tabular}
\end{table}



\newpage
%draw horizontal line
\noindent\rule{\textwidth}{0.4pt}

\vspace{2em}


\begin{poembox}

\section{IJUKA/RECALL (in English)}

\vspace{2em}

{\Large

\begin{verbatim}
Recall while twas dark, the whole world still sleeping
Them vampire bats in a starlit night sky
In mysterious dreams hard to understand
We've found them precious rocks that are gems of glory
With scribbles upon them... many names of AHA
He'd come to summon us while twas still in darkness
He tells us, light shall come and that we shall all know
So be strong, ye people of his great nation.

Forever, know it surely...
You're his lovely childs..
Yeah yeah, that One, AHA.
\end{verbatim}
}

\end{poembox}


\vspace{2em}

%draw horizontal line
\noindent\rule{\textwidth}{0.4pt}

Concerning this English version of the hymn, first, you might notice that it doesn't immediately flow smoothly off the tongue as the original Runyoro version does nor like the Luganda version, however, for someone that knows either of those two, or the underlying melody, a little practice shall make it just tick! The translation was somewhat a last-minute effort ;)

\vspace{2em}

Moreover, we have gone ahead to demonstrate how this English version might be expressed creatively, and beautifully using one of the languages we have covered in this grimoire --- \textbf{OZIN} --- and the process of writing the lyrics in this language is covered well in the associated video at \textbf{\url{https://youtu.be/AdjO2bW1RCo}}.

\vspace{2em}

Talking of which, it shall be worth calling out the fact that, despite using this language to write a hymn might not seem magical enough to some people, it indeed might be, especially for those familiar with ancient European magical languages such as the Germanic-RUNES! For those familiar with such languages, try to compare using OZIN where one might have used RUNES and you'll start to love and appreciate that OZIN is also a great language at expressing both visual, but also verbal expressions --- especially for magical or sacred purposes.

\vspace{2em}

In the next two figures, we see how the song, now titled ``RECALL", has been written and decorated, using the OZIN language. Also you shall notice that we wrote horizontally to emphasize the fact that OZIN is a spatially-independent writing system just like using ordinary Latin Alphabet glyphs, and yet, if we had wanted, we could also have written it vertically like in the last two versions of the hymn that we have encountered. That said, and perhaps because we wanted to keep the lyrics legible enough, notice that the song couldn't fit on a single page, and so, spans two pages. This might make some people think or feel like, using the vertical writing style of \textbf{MEDINA} is possibly the most space-efficient of the writing styles we have covered, given similar text, but more research and explorations might cause others to think otherwise.


\begin{table}[H]
  \begin{tabular}{|p{0.95\textwidth}} % Left border only
    \hline
\begin{figure}[H]
  \begin{center}
   \includegraphics[height=\textheight]{resources/scanned_ijuka_hymn_scribe_sketch_in_ozin_filled_decorated_p1}\\
   \vspace{2em}
  \end{center}
\end{figure}\\
    \cline{1-1} % Bottom border only
  \end{tabular}
\end{table}



\begin{table}[H]
  \begin{tabular}{|p{0.95\textwidth}} % Left border only
    \hline
\begin{figure}[H]
  \begin{center}
   \includegraphics[height=\textheight]{resources/scanned_ijuka_hymn_scribe_sketch_in_ozin_filled_decorated_p2}\\
   \vspace{2em}
  \end{center}
\end{figure}\\
    \cline{1-1} % Bottom border only
  \end{tabular}
\end{table}



%%----------------------------------[ END ---------------------------%%



\begin{table}[H]
  \begin{tabular}{|p{0.95\textwidth}} % Left border only
    \hline
\begin{figure}[H]
  \begin{center} % Rotate a single page of a PDF by 90 degrees
   \includegraphics[height=0.6\textheight]{resources/scanned_example_magickal_art}\\
   \caption{As one might have noticed while reading this grimoire, a lot in magick has much to do with art --- visual art especially --- and so, for those interested, note that many of the original artworks exhibited in this book are also available and with notes or related works, via the visual arts portfolios of \textbf{Fut. Professor of MAGICK, COMPUTING and MATHEMATICS, Joseph Willrich Lutalo}, also once a student of architecture and a life-time student of fine-art. Checkout: \url{https://t.me/jwlart} OR \url{https://deviantart.com/nemesisfixx}}
   \vspace{2em}
  \end{center}
\end{figure}\\
    \cline{1-1} % Bottom border only
  \end{tabular}
\end{table}


\newpage
\chapter*{BONUS: HOW TO MAKE the MOST of this BOOK?}
\addcontentsline{toc}{chapter}{BONUS: HOW TO MAKE the MOST of this BOOK?}
\label{APPENDHOWTO}

Participate, and Apply \cite{lutalo_2025_grimlumtauto}!

\vspace{2em}

It is a \textbf{grimoire} --- \cite{wordweb_assistant} defines this as \texttt{A manual of black magic (for invoking spirits and demons)} --- however, ours isn't about calling or contacting demons or spirits as such, nor is such its gist --- we instead focus on introducing or deepening someones knowledge for how such might be accomplished among other magical feats, especially using specially crafted magical payloads and ritual programs. Overall though, it is a great book, {\LARGE \textbf{a manual}} on picking up special kinds of \textbf{artistic writing}; special knowledge of how to leverage computation and computer-like programs for special kinds of literature and communication --- with or without spirits being involved; special knowledge of mathematics that treats of concerns in psychology and most belief systems, but also the matter of formally specifying magical languages and ciphers; the history and applications of \textbf{scribe magick} and \textbf{chaos magick}, and lots of visual art inspirations and ideas!


%draw horizontal line
\noindent\rule{\textwidth}{0.4pt}

\texttt{HOW THIS BOOK\cite{lutalo_2025_grimlumtauto} WAS WRITTEN} - \textit{a story in contemporary scribe magick.}

1. \textbf{The Design} and \textbf{Planning}:

\texttt{Nuchwezi Esoteric School} | Work Behind {\LARGE Grimoire LUMTauto} (2025)

\vspace{1em}

\textbf{\url{https://youtu.be/rwxcTqQsBYM}}


\vspace{1em}

{\LARGE In that video} we look at the work that went into preparing the first ever grimoire to come from AFRICA! The Grimoire LUMTAUTO that features \textbf{5 magical languages} developed at NES by initiates of \textbf{Scientific Illuminism} and Occult \textbf{Philosophy} over the past 10 years of active research and occult workings; studying, writing, programming and sharing findings and experiences. We also get to look at some of the rare internationally published occult books that are featured in this grimoire, among which is the only publicly known work of literature to ever feature and heavily use the LUMTAUTO cipher and esoteric language (as far back as 2016 while the author was exploring esotericism in Ethiopia/Abyssinia). So, not only go download and read that novel, \textbf{Shrines of The Free Men} by Sir. Joseph Willrich Lutalo ( \textbf{\url{https://bit.ly/read1novel}} ) , but also turn to \textbf{the grimoire}, LUMTAUTO\cite{lutalo_2025_grimlumtauto} that furthers that work, and also helps most readers of that book understand how to decipher the many codes and magic spells in that novel, as well as how to use similar magical languages in their own works and practices --- especially for academic/study purposes, but also for research into \texttt{Computational Mysticism}\cite{Lutalo2024_3c} and Illuminism\footnote{LINK: \textbf{\url{https://iona.nuchwezi.com}}}.

\vspace{2em}


2. \textbf{More Background, Context and Notes:} 


\vspace{1em}
\texttt{[THIS] | }\url{https://bit.ly/grimoirelumtauto}
\vspace{1em}\\
\texttt{[COMMUNITY] | }\url{https://t.me/bclectures}
\vspace{1em}\\
\texttt{[PROF] | }\url{https://mak.academia.edu/JosephWillrichLutalo}
\vspace{1em}\\
\texttt{[I*POW] | }\url{https://t.me/ipowriters}
\vspace{1em}


3. \textbf{About \texttt{Typesetting}, Publication and Content:} 

\vspace{1em}

\textit{A mini-lecture: } \textbf{\url{https://youtu.be/Unra0QgLeRE}}

\vspace{1em}


In which you get a practical idea how to Write Professional Calligraphy in the Magical Language MEDINA. The exercise features working on the Lyrics of Hymn IJUKA.


%draw horizontal line
\noindent\rule{\textwidth}{0.4pt}

\textbf{NOTE:} Though no one funded the preparation nor publication of this manuscript --- in your hands right now, most likely at no cost to you, and yet, it hasn't been free on our part, to make readily available \textbf{the precious jewels this rare manuscript contains.} As indicated earlier, upon publication of the first-draft, FigShare.com disabled our author's account, and all references to much of Joseph Willrich Lutalo's work (about 80+ articles, papers, some books and figures) originally hosted on that platform are now INACCESSIBLE. Joseph wasn't being paid, and yet he has contributed to UGANDA's future possibly more than all his contemporaries combined. \textbf{Note that this book was written out of necessity and urgency.} Especially UGANDANs need a new liberator, but the future shall thank us all --- we all needed it! There have been hardships --- people need what we have to offer, but no one is willing to sacrifice or support those doing the hard work, and so, we must create our own paths towards the TOP, but also define and shape the FUTURE we all wish to see manifest. Thank you in advance, those few that have contributed in making this possible, one way or another. We are in this together. For God and My Country.


%draw horizontal line
\noindent\rule{\textwidth}{0.4pt}

Follow the \texttt{AUTHOR}: \url{https://orcid.org/0000-0002-0002-4657}





\newpage
\vspace{2cm}
\fbox{
\begin{minipage}{0.9\textwidth}
\textbf{TO CITE:}\\

Lutalo, J. W. (2025, November 26). \textbf{Novus Modernus Grimoire Lumtauto Magia} — \texttt{A modern grimoire of 5 eternal magickal languages}. I*POW. \url{https://www.academia.edu/resource/work/145086513}

\end{minipage}}
\\
%}





% insert [front] cover --- could just be a PNG or PDF
%\includegraphics[width=\textwidth]{resources/GrimoireLumtauto-InnerArt}
\includepdf[pages=1]{resources/GrimoireLumtauto-InnerArt.pdf}


% insert [front] cover --- could just be a PNG or PDF
\includepdf[pages=1]{resources/inner_back_cover.pdf}


% insert [front] cover --- could just be a PNG or PDF
\includepdf[pages=1]{resources/back_cover.pdf}

\end{document}

% try to explore how to fit the entire paper on 1 page. Especially using A4 size paper.
