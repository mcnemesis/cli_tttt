\documentclass[12pt,a4paper]{article}
\usepackage[a4paper,margin=1.4cm]{geometry}
%\usepackage[left=1cm, top=0cm, bottom=0.2cm]{geometry} % Adjust these values as needed
\usepackage{hyperref}
\usepackage{parskip}

%for multi-figure figures?
\usepackage{subcaption}

% for highlighting text
\usepackage{xcolor, soul}
% then define colors we shall use:
\definecolor{myteal}{RGB}{0, 128, 128}
\definecolor{lightgray}{HTML}{CCCCCC}
\definecolor{myorange}{HTML}{FFD7B3}

% for table with alternating row bg colors 
\usepackage[table]{xcolor}
\definecolor{lightgray}{gray}{0.9}  % or use HTML/RGB if preferred


%\definecolor{highcolor}{rgb}{0,255,255} % our default hl color for background, friendly on black text foreground
\definecolor{highcolor}{rgb}{0,255,255} %a accent background color, must be friendly on black text foreground
\sethlcolor{highcolor}

%for regular expression and TEA code presentation in console-like text-boxes
\usepackage{tcolorbox}
\tcbuselibrary{listings,skins,breakable}
\usepackage{listings}

% style for general terminal-like listings
\tcbset{
  myterminalstyle/.style={
    colback=black,       % background color
    coltext=white,       % text color
    fontupper=\ttfamily, % typewriter font
    boxrule=0pt,         % no border
    arc=0pt,             % square corners
    outer arc=0pt,
    left=2mm, right=2mm, top=1mm, bottom=1mm,
    enhanced,
    sharp corners,
  }
}

% define listings config for TEA language
\lstdefinelanguage{TEA}{
  morecomment=[l]{\#},
  sensitive=true,
  alsoletter={:*!},
  %morekeywords=[1]{i:, u!:, g:, l:, f:, x:, j:, q!:},
  morekeywords=[1]{%
a:, a.:, a*:, a!:, a.*:, a.!:, a*!:, b:, b.:, b*:, b!:, b.*:, b.!:, b*!:, c:, c.:, c*:, c!:, c.*:, c.!:, c*!:, d:, d.:, d*:, d!:, d.*:, d.!:, d*!:, e:, e.:, e*:, e!:, e.*:, e.!:, e*!:, f:, f.:, f*:, f!:, f.*:, f.!:, f*!:, g:, g.:, g*:, g!:, g.*:, g.!:, g*!:, h:, h.:, h*:, h!:, h.*:, h.!:, h*!:, i:, i.:, i*:, i!:, i.*:, i.!:, i*!:, j:, j.:, j*:, j!:, j.*:, j.!:, j*!:, k:, k.:, k*:, k!:, k.*:, k.!:, k*!:, l:, l.:, l*:, l!:, l.*:, l.!:, l*!:, m:, m.:, m*:, m!:, m.*:, m.!:, m*!:, n:, n.:, n*:, n!:, n.*:, n.!:, n*!:, o:, o.:, o*:, o!:, o.*:, o.!:, o*!:, p:, p.:, p*:, p!:, p.*:, p.!:, p*!:, q:, q.:, q*:, q!:, q.*:, q.!:, q*!:, r:, r.:, r*:, r!:, r.*:, r.!:, r*!:, s:, s.:, s*:, s!:, s.*:, s.!:, s*!:, t:, t.:, t*:, t!:, t.*:, t.!:, t*!:, u:, u.:, u*:, u!:, u.*:, u.!:, u*!:, v:, v.:, v*:, v!:, v.*:, v.!:, v*!:, w:, w.:, w*:, w!:, w.*:, w.!:, w*!:, x:, x.:, x*:, x!:, x.*:, x.!:, x*!:, y:, y.:, y*:, y!:, y.*:, y.!:, y*!:, z:, z.:, z*:, z!:, z.*:, z.!:, z*!:,%
A:, A.:, A*:, A!:, A.*:, A.!:, A*!:, B:, B.:, B*:, B!:, B.*:, B.!:, B*!:, C:, C.:, C*:, C!:, C.*:, C.!:, C*!:, D:, D.:, D*:, D!:, D.*:, D.!:, D*!:, E:, E.:, E*:, E!:, E.*:, E.!:, E*!:, F:, F.:, F*:, F!:, F.*:, F.!:, F*!:, G:, G.:, G*:, G!:, G.*:, G.!:, G*!:, H:, H.:, H*:, H!:, H.*:, H.!:, H*!:, I:, I.:, I*:, I!:, I.*:, I.!:, I*!:, J:, J.:, J*:, J!:, J.*:, J.!:, J*!:, K:, K.:, K*:, K!:, K.*:, K.!:, K*!:, L:, L.:, L*:, L!:, L.*:, L.!:, L*!:, M:, M.:, M*:, M!:, M.*:, M.!:, M*!:, N:, N.:, N*:, N!:, N.*:, N.!:, N*!:, O:, O.:, O*:, O!:, O.*:, O.!:, O*!:, P:, P.:, P*:, P!:, P.*:, P.!:, P*!:, Q:, Q.:, Q*:, Q!:, Q.*:, Q.!:, Q*!:, R:, R.:, R*:, R!:, R.*:, R.!:, R*!:, S:, S.:, S*:, S!:, S.*:, S.!:, S*!:, T:, T.:, T*:, T!:, T.*:, T.!:, T*!:, U:, U.:, U*:, U!:, U.*:, U.!:, U*!:, V:, V.:, V*:, V!:, V.*:, V.!:, V*!:, W:, W.:, W*:, W!:, W.*:, W.!:, W*!:, X:, X.:, X*:, X!:, X.*:, X.!:, X*!:, Y:, Y.:, Y*:, Y!:, Y.*:, Y.!:, Y*!:, Z:, Z.:, Z*:, Z!:, Z.*:, Z.!:, Z*!:%
},
  keywordstyle=[1]\color{green},
  commentstyle=\color{lightgray},
  morestring=[b]",
stringstyle=\color{myorange},
moredelim=[s][\color{myorange}]{\{}{\}},
}


% define custom terminal for TEA language snippets

\tcbset{
  teaterminalstyle/.style={
    enhanced,
    colback=myteal,
    coltext=white,
    fontupper=\ttfamily,
    boxrule=0pt,
    arc=0pt,
    outer arc=0pt,
    left=2mm, right=2mm, top=1mm, bottom=1mm,
    sharp corners,
    listing only,
    listing options={
      language=TEA,
     basicstyle=\ttfamily,
%keywordstyle=\color{cyan}\bfseries,
%commentstyle=\color{green}\itshape,
%stringstyle=\color{yellow}
    }
  }
}



% for maths
\usepackage{amsmath}
% for number sets symbols
\usepackage{amssymb}
%\usepackage{ntheorem}
\usepackage{amsthm}


% for writing our theorems and defs...
\newtheorem{comp}{Computation}
\newtheorem{theo}{Theorem}
\newtheorem{defn}{Definition}
\newtheorem{lem}{Lemma}
\newtheorem{prop}{Proposition}
\newtheorem{axiom}{Axiom}
\newtheorem{post}{Postulate}
\newtheorem{trans}{Transformation}
\newtheorem{transf}{Transformer}
\newtheorem{law}{Law}
\newtheorem{prob}{Problem}
\newtheorem{soln}{Solution}
\newtheorem{alg}{Algorithm}

\title{\textbf{NOVUS MODERNUS GRIMOIRE AETERNUS MAGIA LUMTAUTO} --- A Grimoire of Eternal Magickal Languages\thanks{Proceed with Caution. This is \textbf{A Call to Practice} Occult Mysteries. There is no guarantee that tears wont be involved\cite{crowley1948magick}.}}

\usepackage{amsmath, amssymb, graphicx}

% to include pdf pages
\usepackage{pdfpages}

% Define \invpi to flip the pi symbol and use it as a function
\newcommand{\invpi}[1]{\mathop{\rotatebox[origin=c]{180}{$\pi$}}#1}
\newcommand{\invdel}[1]{\mathop{\rotatebox[origin=c]{180}{$\Delta$}}#1}

\author{\textbf{M*A*P} Adept Psymaz\thanks{\textbf{Most Ancient Priest}, also known as Fut. Prof. J. Willrich Lutalo C.M.R.W; Curator, PI and President at Nuchwezi Research, GARUGA, Uganda. \textbf{ORCID:} \url{https://orcid.org/0000-0002-0002-4657}}\\Nuchwezi Research\\\href{mailto:joewillrich@gmail.com}{joewillrich@gmail.com}, \href{mailto:jwl@nuchwezi.com}{jwl@nuchwezi.com}}

%\date \today
\date {EDITION: \textbf{12}$^{th}$ \textbf{NOV}, \texttt{2025}}


\begin{document}

% insert [front] cover --- could just be a PNG or PDF
\includepdf[pages=1]{resources/front_cover.pdf}

\maketitle

\begin{abstract}
%\Large
In this manuscript, intentionally designed like a grimoire, we are to present for the first time, a proper distillation of research and applications in the use of esoteric languages for the purpose of performing occult operations as explored by initiates at Nuchwezi Esoteric School (NES) for the past 1 decade. This is an original contribution to the universal esoteric tradition and is meant to help curate, promulgate and advance a sane, thoughtful appreciation of the mysteries --- a line of work that goes back through the ages, to the first psy-ops and occult workings attempted by primitive man, through generations of diverse and varying explorations by mystics and initiates from all kinds of schools, cultures and traditions, all the way to modern approaches best known to the true initiates of illuminism. In particular though, this manuscript shall focus on 4 different ORIGINAL RESULTS of hard-work, study and yes, applying occult philosophy at Nuchwezi;
\begin{enumerate}
\item \textbf{LUMTAUTO} --- an application of computational mysticism in the form of an algorithmic cipher that can transform ordinary English or any language into a form suitable for occult operations and conjurations.
\item \textbf{The Grand Myrrh Transform} --- a related, but otherwise later and shorter, more occult TEA\cite{cli_tttt} algorithm for turning ordinary phrases into sacred words of power reminiscent of sacred languages such as Hebrew and Aramaic. 
\item  The \textbf{Ozin Cipher} --- a visual code first presented in an earlier work\cite{lutalo_2025_trans_genetics}, useful in expressing secret or special messages in an occult and psychologically charged hand that was first developed by the Illuminates of Nuchwezi Angelic (IoNA) via esoteric workings reminiscent of the methods of medieval angel-working wizards such as John Dee and Edward Kelley.
\item \textbf{Crypt of Medina} --- a special occult cipher also first developed at NES, and which, unlike most ciphers ancient or modern, allows for the visual encoding of occult messages in such a way that \textbf{one must explicitly use the method of reading between the lines} in order to understand its messages.
\end{enumerate} 
We shall [briefly] look at the underlying philosophies and supporting literature; shall look at many guidelines concerning how to practically apply the presented ideas; shall treat of the matter of mixing modern computer technology in applying these ideas, and shall share lots of visuals, links to supporting videos, community and online tools to help practitioners further and deepen their appreciation of these modern mysteries. This is a work for the Illuminati.
 \newline\newline
     \textbf{Keywords}: Foundations, Psy-Ops, Ceremonial Magick, Computational Mysticism, Grimoire
\end{abstract}

\section{An Introduction to Magickal Languages}
\label{SECINTRO}

In the preface of his treatise on applying the new mathematics of Transformatics\cite{Lutalo2025_transformatics_thesis} to Genetics\cite{lutalo_2025_trans_genetics}, the author appeals to the sacred scriptures, particularly to the first book of the Torah (also known as the ``Pentateuch") --- which, not just for Kabbalists and Jewish mystics might be considered the foundational book of essential laws and instruction, but which also serves as a foundational text for many judaic-associated faiths and traditions --- the likes of Christians, Moslems, Rastafarians but also, and not very surprising, Theistic Satanists\cite{wikipedia_theistic_satanism}\footnote{Some critics might argue that actually, modern satanism only borrows the concept of Satan (as \textit{ha-satan}) from the Hebrew bible but doesn't appeal to Judaic roots or laws and that it instead is founded on twisting and elevating the Christian idea of Satan as a fallen angel\cite{copilot_assistant}\cite{wikipedia_theistic_satanism}, however, and logically so, it does make sense to pin them down, and assert their undeniable roots in ancient Judaism and its traditions for that reason alone. Also, note that we talk of \textbf{theistic satanism} here and not \textit{atheistic satanism} (also \textbf{LaVeyan Satanism}); Satan as a symbol not as a deity, and also not \textit{acosmic satanism} (more popular with edge-lords and ecclectic modern LHP philosophers); satan as the principle that opposes order --- perhaps \textit{chaos}?)}!. Lutalo calls us to consider the strong directive given man by God, when he is called to \textbf{``be fruitful, multiply, fill the earth and subdue it"} (Genesis 1:28).

In another work by Lutalo --- \textbf{3 Core Ideas in Computational Mysticism}\cite{Lutalo2024_3c}, a September 2024 paper that laid down the foundations of computational mysticism, he clearly lays down the fundamental significance of \textbf{language} not only in its use for bringing about transformations and the manifestation of will via computers, but also for general human life and affairs, when he says this in the abstract of that mini-paper:


\noindent
\begin{minipage}{1\textwidth}
\vspace{1em}
\begin{quotation}
{\ttfamily

Programming languages create a medium via which one can define and execute orders with certain effects at will, and certainly so. Basing on how language underlies the ability for humans to formulate and share thoughts with each other, we also see how the use of certain special languages underlies man's ability to command and control reality since ancient times.

}
\end{quotation}
\vspace{1em}
\end{minipage}


As we shall see and come to appreciate in this grimoire, a careful and willful use of ``special" languages --- especially, and from the perspective of the theme of this work, languages both inspired by or based on natural, but also artificial or \textit{synthetic} languages, can readily help yield results that we shall soon come to appreciate to be what ``magic"\cite{butler1952magic} or rather ``magick"\cite{crowley1929magick} is all about. Essentially, Lutalo tells us, in \cite{Lutalo2024_3c} that:


\noindent
\begin{minipage}{1\textwidth}
\vspace{1em}
\begin{quotation}
{\ttfamily

Basically, we see how it is indeed language, or rather, its use via communication --- a willful application
of language, that makes possible the creation and transmission of [any] thought.

}
\end{quotation}
\vspace{1em}
\end{minipage}


Thus, we argue that, \textbf{it is essentially via the willful application of language that man can come to subdue reality.} Moreover, and as we shall soon see when we consider the acceptable definitions of magick, any such willful acts, whether they leverage language in the form of expressions in the mind (``thoughtforms"), expressions in sound (``incantations", ``spells", ``mantras", ``commands", etc.), expressions in body/body-language (``gestures", ``mudras", ``assanas", ``signs", etc.) or as visual expressions (``sigils", ``mandalas", ``glyphs", ``signs" or ``symbols", etc.) are what make magick possible.


And as for the traditional concept of \textit{language}, still, \cite{Lutalo2024_3c}  offers a compelling and reliable working definition. But what of the idea of a \textbf{magickal language}? First, we shall return to the [modern] classics, \textbf{Aleister Crowley}\footnote{Apart from being a renown initiate into the ancient mysteries --- initiated circa 1898 into the \textbf{Hermetic Order of the Golden Dawn}\cite{cassiel1990encyclopedia}, he's also a famous and prolific writer and researcher on all matters occult and esoteric during the early 20$^{th}$ century, and was also well-known to have not only founded the tradition of THELEMA\cite{crowley1929magick} that has inspired many modern magical traditions such as the \textit{Wiccans} and several ``new-age" spirituality paths, but that he also considered himself not just a ``magus", but as also the ``BEAST"!} being one very undeniably reliable authority on the subject, and shall start by considering the ideas he presents in \textbf{Magick in Theory and Practice}\cite{crowley1929magick}:




\noindent
\begin{minipage}{1\textwidth}
\vspace{1em}
\begin{quotation}
{\ttfamily

Magick is the Science and Art of causing Change to occur in conformity with Will.

}
\end{quotation}
\vspace{1em}
\end{minipage}



To the best of our knowledge, it is not until when Crowley penned this definition, that magic as a stream of ``esotericism" or ``occult philosophy" and not as ``stage magic" or the practice of ``illusionism" passed from the realm of mere mysticism into a kind of formal science --- arguably, a great contribution to the galvanizing of so-called \textit{Scientific Illuminism}. Earlier authorities such as \textbf{Cornelius Agrippa}, despite having penned incredible tomes and compendiums on western esotericism and occult philosophy\cite{agrippa2014occult}, never actually, or rather, explicitly offered nor extended any such formal definitions that we know of.

That said, note that Crowley's definition isn't the only authoritative, nor usable \textbf{working definition} of magick, and as for that matter, though we shall not attempt to enumerate all of them here, there is a rare compilation of such definitions that was prepared by the author of this grimoire as far back as 2014\cite{lutalo_2025_definitions}, and that it not only lists Crowley's definition among a whooping total of \textbf{34 distinct definitions of magic[k]}, but that is also clearly offers their associated sources (authors and books/articles/websites/traditions, etc.). Among these, let us just recall but only 3:



\noindent
\begin{minipage}{1\textwidth}
\vspace{1em}
\begin{quotation}
\noindent {\ttfamily

The science and art of causing change (in consciousness) to occur in conformity with will, using means not currently understood by traditional Western science

}
\hspace*{\fill} --- \textbf{Modern Magick}, \textit{2010}, Donald Michael Kraig\cite{kraig2010modern}
\end{quotation}
\vspace{1em}
\end{minipage}


That one, especially resurfaced here, because DMK has really helped modern magick practitioners working outside of traditional initiatory systems and who might not be able to access proper initiators into hermeticism and practical western esotericism to actually get busy and attain results. His book\cite{kraig2010modern} on magick is very resourceful --- especially for solo practitioners, and we shall come back to it several times in later parts of this grimoire.



\noindent
\begin{minipage}{1\textwidth}
\vspace{1em}
\begin{quotation}
\noindent {\ttfamily

The Highest, most Absolute, and most Divine Knowledge of Natural Philosophy, advanced in its works and wonderful operations by a right understanding of the inward and occult virtue of things; so that true Agents being applied to proper Patients, strange and admirable effects will thereby be produced.

}
\hspace*{\fill} --- \textbf{The Goetia of the Lemegeton of King Solomon}, \textit{1904}, S. L. MacGregor Mathers and Aleister Crowley\cite{mathers1904goetia}
\end{quotation}
\vspace{1em}
\end{minipage}


That second one, not only because it is one of few that appeals to ``ancients" such as \textbf{Cornelius Agrippa}\cite{agrippa2014occult} or \textbf{Eliphas Levi}\footnote{This is the pen name of Alphonse Louis Constant (1810–1875), a highly influential French occult author and ceremonial magician whose writings significantly contributed to the revival of magic and esoteric thought in the 19th century\cite{wordweb_assistant}.} that especially championed concepts such as ``High Magic" or rather highly-eclectic Ceremonial [and priestly] Magick, but also because it might align well with classical kinds of magick such as the \textit{Alchemy} that scientists such as \textbf{Sir Isaac Newton} dabbled in occasionally, the \textit{Hermetic Medicine} that medieval esotericists such as \textbf{Paracelsus}\footnote{Also known as \textbf{Philippus Aureolus Theophrastus Bombastus von Hohenheim}\cite{FasanoSequeira2017}} --- also ``Father of Toxicology", championed, as well as mystics practicing arts such as \textit{Theurgy} --- the likes of \textbf{Emanuel Swedenborg}\footnote{Swedish scientist, theologian and mystic (1688-1772)\cite{wordweb_assistant}}. Moreover, and also a contemporary of Crowley, \textbf{Mathers} does deserve a special place in the hearts of modern occultists for his many contributions to formalizing and promulgating ancient mysteries in the 20$^{th}$ century.


Finally, we shall also re-surface this definition:


\noindent
\begin{minipage}{1\textwidth}
\vspace{1em}
\begin{quotation}
\noindent {\ttfamily

The enhancement of the probabilities/likelihood of a desired outcome/result.

}
\hspace*{\fill} --- \textbf{PsyberMagick: Advanced Ideas in Chaos Magick}, \textit{1995}, Peter Carroll\cite{Carroll1995}
\end{quotation}
\vspace{1em}
\end{minipage}


That final one, especially because the founder of the \textit{Chaos Magick} meta-paradigm modern magick tradition has also greatly helped inspire and liberate many practitioners from ideological slavery to ancient and medieval dogmas, but also that, as one might find when studying many of \textbf{Peter Carroll}'s works, he, like the associated magical communities he inspired, such as the \textbf{Illuminates of Thanateros}\footnote{See IoT German Section: \url{https://iot-d.de/} or IoT BIS: \url{https://iotbritishisles.com/}}, but also our own \textbf{Illuminates of Nuchwezi}\footnote{Refer to IoNA home page: \url{https://iona.nuchwezi.com/}} and individual modern chaos magicians such as \textbf{Joshua Madara}\footnote{Originally \url{http://hyperritual.com/}, now \url{https://eldri.tech/}} find much utility in associating modern Tech, Maths and Science sensibilities with Esotericism and Magick\footnote{See for example, our take on Probabilistic Metaphysics\cite{Lutalo2023_metaphysics}, which strongly reflects or aligns with Carroll's definition and ideas of Magick.}.


And so, having looked at all the past authorities, we shall consider just one more definition --- essentially, the one we consider to be our authoritative working definition, as laid out below:




\fbox{\begin{minipage}{\textwidth}
\large

Given the working definition of a \textbf{Certain Manifestor}\cite{Lutalo2025transpsy}\cite{Lutalo2023_metaphysics}:\\


\begin{transf}[The \textbf{Certain Manifestor}]
\label{TRANSFCM}
In a reality space $\Psi:N \times \psi_{k}$, \\
a \textbf{certain manifestor}, $\mathbb{k} : \Psi(\mathbb{k}): \psi_{\tau} : \tau \implies \mathbb{k} \quad \land \quad \invpi(\psi_\tau \in \Psi) = 1 \quad \\
\forall \mathbb{k} \in [1,|\Psi|]$ is the following operator:

\begin{trans}
 $\langle \tau \rangle \xrightarrow{O_{\lambda}(\cdot)}  \psi_\tau $\\
 \end{trans}
\end{transf}

we know that the operator $O_{\lambda}(\cdot)$, also defined as the \textbf{Certain Manifestor}, is a \textbf{Magician}, if we define magick as such:\\

\begin{defn}[\textbf{MAGICK}]
\label{DEFMAGICK}

Given some potential distinct event $e$ from the space of all possible events $\langle e* \rangle$, it can be manifested when requested for, willed or wanted --- by the application of strong faith, belief or conscious effort on the part of the operator relative to that event, $\lambda (\langle e* \rangle)(e)$, in an act that is essentially one of applying psychology, but otherwise which we shall call \textbf{MAGICK}. Essentially, then, \textbf{Magick is any act that can make the following transformation happen:}\\

{\Large

\begin{trans}
\label{TRANSMAGICK}
 $\langle e \rangle \xrightarrow{O_{\lambda}(\psi_\infty)(e)} \overset{>}{\psi_e} \approx  \psi_e $
 \end{trans}
 
 }


\end{defn} 

\end{minipage}}
\\

The critical term $ \overset{>}{\psi_e} $ in \textbf{\hyperref[TRANSMAGICK]{Transformation \ref{TRANSMAGICK}}} is meant to symbolize the ``desired result" --- a \textit{willed outcome}, that essentially is encoded in the language of transformatics\cite{Lutalo2025_transformatics_thesis} here as a kind of [identifying particular modal] sequence that captures the gist of that event or phenomena\footnote{Note that, as we saw in the important fundamental law --- \textbf{IGS}: \textbf{Identity Genome Sequence Law} first laid down in \cite{lutalo_2025_trans_genetics}, any potential event or phenomena might be somehow reducible to or expressible as some distinct sequence, and as per the sensibilities of transformatics, any such sequence might be related to or reducible to some symbol set.}, whereas $\langle e \rangle$ could be just a symbol, a word, thought or label depicting that which is desired or wanted.


A \textbf{magickal language} then, is a concept we might formally define as such:\\

\begin{defn}[A \textbf{Magickal Language}: $\mathbb{L}: \mathbb{N} \times \psi_\infty$]
\label{DEFMAGICKLANG}

Any system of encoding will, and particularly, such as would allow one to encode some desire or need such as in the input of \textbf{\hyperref[TRANSMAGICK]{Transformation \ref{TRANSMAGICK}}}, so that, by presenting it to a certain manifestor --- a special kind of operator --- essentially a magician, the corresponding desired event or result can then readily be manifested, is a \textbf{Magical Language}, $\mathbb{L}: \mathbb{N} \times \psi_\infty$, that \textit{might} span the infinite set of symbols or events $\psi_\infty$.

\end{defn} 



At this juncture, and given what we have just clarified, it might help to set some things clearer concerning how modern magick needs be approached, but also how it might be appreciated. First, note that, magick, despite being an art, is also a science\cite{lutalo_2025_definitions}. In fact, even long before modern authorities on magick such as the Computer Scientist Peter Carroll gave us a somewhat mathematical treatment\cite{Carroll1995}\cite{carroll2010octavo} of an originally mostly speculative and mystical craft, and yet, in their work on the theory and practice of magick, Aleister Crowley did also greatly help cast magick as an enterprise one might come to properly appreciate or evaluate through the lens of a mathematician, when he laid down the following theorem and remarks:


\noindent
\begin{minipage}{1\textwidth}
\vspace{1em}
\begin{quotation}
\noindent {\ttfamily

THEOREMS.
1) Every intentional act is a Magickal act.

By ``intentional" I mean ``willed". But even unintentional acts so seeming are not truly so. Thus, breathing is an act of the Will to Live.

}
\hspace*{\fill} --- \textbf{Magick in Theory and Practice}, \textit{1929}, Aleister Crowley\cite{crowley1929magick}
\end{quotation}
\vspace{1em}
\end{minipage}

And so that, for our case, we can rest assured, that the following theorem likewise shall drive the point home concerning why developing, knowing and applying a magical language might be more useful than not:\\

\begin{theo}[\textbf{Effectiveness of a Magickal Language}]
\label{THEOMAGLANG}

Given some desire, $\tau$, and a magician that can operate on it, $\lambda(\cdot)$, we can assert that: presenting the suitable encoding of $\tau$ in some magical language that the operator can process effectively, shall yield better results/increase the likelihood of manifesting the desired outcome than not.

\begin{proof}
The proofs are several, but we shall call out just two:
\begin{enumerate}
\item Follows from \textbf{\hyperref[DEFMAGICK]{Definition \ref{DEFMAGICK}}} and \textbf{\hyperref[DEFMAGICKLANG]{Definition \ref{DEFMAGICKLANG}}}.
\item Experience and generations of successful practitioners confirm this; the most effective magical operations require or assume the need to express desire using some encoding method such as sigilization, mantras, visualization, specific asanas, specific programs in specific languages, etc. so as to readily or quickly and most efficiently bring the [desired] result into manifestation.
\end{enumerate}
$\qed$
\end{proof}

\end{theo}


By now, \textbf{\hyperref[THEOMAGLANG]{Theorem \ref{THEOMAGLANG}}} should be obvious to anyone reading this grimoire --- like, for example, considering the otherwise commonplace case of \textit{wanting to eat food presented on a plate}: no matter what one might do, say or think, unless they actually go ahead and employ the ``right" language of ``eating" --- which entails presenting something to the mouth, chewing and/or swallowing it so as to actually ``eat" it; any other actions the operator might apply --- such as smiling at the food, looking at a picture of the food, or perhaps merely smelling it, might only leave the food turning cold, and perhaps only satiate the mind (that ``I have food"), but otherwise leave the person not nourished and possibly still hungry (the food hasn't been ingested, and neither has it been digested). A related, and totally absurd case might be that of \textit{wanting to bare a [normal human] child} without having sex nor having ones sperm presented to someone's ready ova. It just wouldn't work!\footnote{Well, one might will a child into existence [without sex], say, by having some donor offer them a baby, or by adopting an abandoned kid, or adopting an already impregnated girl etc. However, the original intent of bearing one's own literal child shall not be met --- preternatural cases such as the immaculate conception of the Blessed Virgin Mary [dogma], and subsequently her \textbf{virgin birth} of Jesus Christ[Matthew (1:18–25), Luke (1:26–38)] might of course deviate from this, but also then, the birth of Jesus without Joseph having to copulate Mary perhaps didn't originate from an intent of Joseph's desire to bear \textit{his own child} --- as careful analysis of scripture\cite{newjerusalem1985} does indeed confirm.} 


In the rest of this manuscript, we are thus going to spend time exploring, in theory, but also in practice, several original and multipurpose (and arguably ``general", but not necessarily infallible) magickal languages that have been developed or discovered by the author while exploring and applying Magick at NES, and yes, in sane, and for sane reasons\footnote{There is nothing as insane as assuming that a limited operator [\textit{an automaton and not a psymaton, a psymaton and not a man, a man and not an angel, an angel and not a god, a god and not God}] might be able to process an infinite sequence of [magickal] languages when presented with them or an infinite sequence of desires encoded using them, however, we shall still proceed to present these [limited] languages, which, even though they might seem finite and constrained for one or some operator, could otherwise find [infinite] purpose in the hands of yet another [well-prepared] operator, and thus, it might then not be insane to consider that all of magick might be somehow reducible to just the mastery of [one of] these four languages we are about to unravel.}.


\begin{figure}[h]
  \begin{center}
   \includegraphics[scale=0.8]{resources/iona.png}\\
   \caption{A CHAOSPHERE}
  \label{FIG1}
  \end{center}
\end{figure}


\section{Language $\rightarrow$ LUMTAUTO}
\label{SECLUMTAUTO}

blah blah

\section{The Grand Myrrh Transform}
\label{SECMYRRH}

blah blah


\section{The OZIN Cipher}
\label{SECOZIN}


blah blah


\section{The Crypt of MEDINA}
\label{SECMEDINA}

blah blah


\section{Finale}
\label{SECFIN}

blah blah...

\bibliographystyle{unsrt}
\bibliography{references}


\vspace{5cm}
\fbox{
\begin{minipage}{0.9\textwidth}
\textbf{TO CITE:}\\

Lutalo, Joseph Willrich (2025). \textbf{TRANSFORMATICS 101 - explained.} figshare. Thesis. \url{https://doi.org/10.6084/m9.figshare.30305056}

\end{minipage}}
\\
%}


% insert [front] cover --- could just be a PNG or PDF
\includepdf[pages=1]{resources/back_cover.pdf}

\end{document}

% try to explore how to fit the entire paper on 1 page. Especially using A4 size paper.