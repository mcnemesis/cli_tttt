\documentclass[12pt,a4paper]{article}
%\usepackage[a4paper,margin=1.4cm]{geometry}
\usepackage[a4paper, left=2cm, right=1.5cm, top=1.5cm, bottom=1.5cm]{geometry} % Adjust these values as needed
\usepackage{hyperref}
\usepackage{parskip}

% for multiline comments...
\newcommand{\comment}[1]{}

%for striking-through text
\usepackage{ulem}

% allow table of contents to also list subsections
\setcounter{tocdepth}{2}

% for better appendices
\usepackage[title,titletoc]{appendix}

% for controlling page numbers
\usepackage{fancyhdr}
\pagestyle{fancy}
\fancyhf{}
\fancyhead[R]{\thepage}

% throw in page-top header
\fancyhead[L]{IONA Inner Journal, Vol. 2}

% for line-breaks in table cells
\usepackage{makecell}

% for graphics
\usepackage{graphicx}
\usepackage{caption}
\usepackage{float}

%for multi-figure figures?
\usepackage{subcaption}

% for highlighting text
\usepackage{xcolor, soul}
% then define colors we shall use:
\definecolor{myteal}{RGB}{0, 128, 128}
\definecolor{lightgray}{HTML}{CCCCCC}
\definecolor{myorange}{HTML}{FFD7B3}

% for table with alternating row bg colors 
\usepackage[table]{xcolor}
\definecolor{lightgray}{gray}{0.9}  % or use HTML/RGB if preferred


%\definecolor{highcolor}{rgb}{0,255,255} % our default hl color for background, friendly on black text foreground
\definecolor{highcolor}{rgb}{0,255,255} %a accent background color, must be friendly on black text foreground
\sethlcolor{highcolor}

%for regular expression and TEA code presentation in console-like text-boxes
\usepackage{tcolorbox}
\tcbuselibrary{listings,skins,breakable}
\usepackage{listings}

% style for general terminal-like listings
\tcbset{
  myterminalstyle/.style={
    colback=black,       % background color
    coltext=white,       % text color
    fontupper=\ttfamily, % typewriter font
    boxrule=0pt,         % no border
    arc=0pt,             % square corners
    outer arc=0pt,
    left=2mm, right=2mm, top=1mm, bottom=1mm,
    enhanced,
    sharp corners,
  }
}

% define listings config for TEA language
\lstdefinelanguage{TEA}{
  morecomment=[l]{\#},
  sensitive=true,
  alsoletter={:*!},
  %morekeywords=[1]{i:, u!:, g:, l:, f:, x:, j:, q!:},
  morekeywords=[1]{%
a:, a.:, a*:, a!:, a.*:, a.!:, a*!:, b:, b.:, b*:, b!:, b.*:, b.!:, b*!:, c:, c.:, c*:, c!:, c.*:, c.!:, c*!:, d:, d.:, d*:, d!:, d.*:, d.!:, d*!:, e:, e.:, e*:, e!:, e.*:, e.!:, e*!:, f:, f.:, f*:, f!:, f.*:, f.!:, f*!:, g:, g.:, g*:, g!:, g.*:, g.!:, g*!:, h:, h.:, h*:, h!:, h.*:, h.!:, h*!:, i:, i.:, i*:, i!:, i.*:, i.!:, i*!:, j:, j.:, j*:, j!:, j.*:, j.!:, j*!:, k:, k.:, k*:, k!:, k.*:, k.!:, k*!:, l:, l.:, l*:, l!:, l.*:, l.!:, l*!:, m:, m.:, m*:, m!:, m.*:, m.!:, m*!:, n:, n.:, n*:, n!:, n.*:, n.!:, n*!:, o:, o.:, o*:, o!:, o.*:, o.!:, o*!:, p:, p.:, p*:, p!:, p.*:, p.!:, p*!:, q:, q.:, q*:, q!:, q.*:, q.!:, q*!:, r:, r.:, r*:, r!:, r.*:, r.!:, r*!:, s:, s.:, s*:, s!:, s.*:, s.!:, s*!:, t:, t.:, t*:, t!:, t.*:, t.!:, t*!:, u:, u.:, u*:, u!:, u.*:, u.!:, u*!:, v:, v.:, v*:, v!:, v.*:, v.!:, v*!:, w:, w.:, w*:, w!:, w.*:, w.!:, w*!:, x:, x.:, x*:, x!:, x.*:, x.!:, x*!:, y:, y.:, y*:, y!:, y.*:, y.!:, y*!:, z:, z.:, z*:, z!:, z.*:, z.!:, z*!:,%
A:, A.:, A*:, A!:, A.*:, A.!:, A*!:, B:, B.:, B*:, B!:, B.*:, B.!:, B*!:, C:, C.:, C*:, C!:, C.*:, C.!:, C*!:, D:, D.:, D*:, D!:, D.*:, D.!:, D*!:, E:, E.:, E*:, E!:, E.*:, E.!:, E*!:, F:, F.:, F*:, F!:, F.*:, F.!:, F*!:, G:, G.:, G*:, G!:, G.*:, G.!:, G*!:, H:, H.:, H*:, H!:, H.*:, H.!:, H*!:, I:, I.:, I*:, I!:, I.*:, I.!:, I*!:, J:, J.:, J*:, J!:, J.*:, J.!:, J*!:, K:, K.:, K*:, K!:, K.*:, K.!:, K*!:, L:, L.:, L*:, L!:, L.*:, L.!:, L*!:, M:, M.:, M*:, M!:, M.*:, M.!:, M*!:, N:, N.:, N*:, N!:, N.*:, N.!:, N*!:, O:, O.:, O*:, O!:, O.*:, O.!:, O*!:, P:, P.:, P*:, P!:, P.*:, P.!:, P*!:, Q:, Q.:, Q*:, Q!:, Q.*:, Q.!:, Q*!:, R:, R.:, R*:, R!:, R.*:, R.!:, R*!:, S:, S.:, S*:, S!:, S.*:, S.!:, S*!:, T:, T.:, T*:, T!:, T.*:, T.!:, T*!:, U:, U.:, U*:, U!:, U.*:, U.!:, U*!:, V:, V.:, V*:, V!:, V.*:, V.!:, V*!:, W:, W.:, W*:, W!:, W.*:, W.!:, W*!:, X:, X.:, X*:, X!:, X.*:, X.!:, X*!:, Y:, Y.:, Y*:, Y!:, Y.*:, Y.!:, Y*!:, Z:, Z.:, Z*:, Z!:, Z.*:, Z.!:, Z*!:%
},
  keywordstyle=[1]\color{green},
  commentstyle=\color{lightgray},
  morestring=[b]",
stringstyle=\color{myorange},
moredelim=[s][\color{myorange}]{\{}{\}},
}


% define custom terminal for TEA language snippets

\tcbset{
  teaterminalstyle/.style={
    enhanced,
    colback=myteal,
    coltext=white,
    fontupper=\ttfamily,
    boxrule=0pt,
    arc=0pt,
    outer arc=0pt,
    left=2mm, right=2mm, top=1mm, bottom=1mm,
    sharp corners,
    listing only,
    listing options={
      language=TEA,
     basicstyle=\ttfamily,
%keywordstyle=\color{cyan}\bfseries,
%commentstyle=\color{green}\itshape,
%stringstyle=\color{yellow}
    }
  }
}



% from google-gemini for verbatim listings:
\tcbuselibrary{listings,breakable}

% Define a language with no syntax highlighting
\lstdefinelanguage{none}{}

% Define a new tcblisting environment for verbatim content
\newtcblisting{tcbverbatim}[1][]{
  % Pass any user options to the new environment
  #1,
  breakable, % Allow long lines to wrap
  listing only, % The box contains only a listing
  listing options={
    language=none, % The 'none' language disables highlighting
    basicstyle=\ttfamily, % Use the typewriter font
    columns=flexible, % Allow flexible column widths for wrapping
    breaklines=true, % Enable line breaking
  },
}



%%---------------CONFIG JAVASCRIPT

%\usepackage{listings}
%\usepackage{xcolor}

% Define custom colors
\definecolor{jskeyword}{RGB}{0,0,180}
\definecolor{jsstring}{RGB}{163,21,21}
\definecolor{jscomment}{RGB}{0,128,0}
\definecolor{jsnumber}{RGB}{128,0,128}
\definecolor{jsbackground}{RGB}{245,245,245}

% JavaScript style for listings
\lstdefinelanguage{JavaScript}{
  keywords={break, case, catch, continue, debugger, default, delete, do, else, 
    finally, for, function, if, in, instanceof, new, return, switch, this, throw, 
    try, typeof, var, void, while, with, const, let, class, extends, super, import, 
    export, yield, async, await},
  keywordstyle=\color{jskeyword}\bfseries,
  ndkeywords={boolean, number, string, null, undefined, true, false, Array, Date, 
    eval, function, Math, Object, RegExp},
  ndkeywordstyle=\color{jsnumber}\bfseries,
  identifierstyle=\color{black},
  sensitive=true,
  comment=[l]{//},
  morecomment=[s]{/*}{*/},
  commentstyle=\color{jscomment}\ttfamily,
  stringstyle=\color{jsstring}\ttfamily,
  morestring=[b]',
  morestring=[b]"
}


% General listings setup
%\lstset{
%  language=JavaScript,
%  backgroundcolor=\color{jsbackground},
%  basicstyle=\ttfamily\small,
%  numbers=left,
%  numberstyle=\tiny\color{gray},
%  stepnumber=1,
%  numbersep=8pt,
%  showstringspaces=false,
%  tabsize=2,
%  breaklines=true,
%  frame=single,
%  rulecolor=\color{gray},
%  captionpos=b
%}



%%---------------END CONFIG JAVASCRIPT



% for maths
\usepackage{amsmath}
% for number sets symbols
\usepackage{amssymb}
%\usepackage{ntheorem}
\usepackage{amsthm}


% for writing our theorems and defs...
\newtheorem{comp}{Computation}
\newtheorem{theo}{Theorem}
\newtheorem{defn}{Definition}
\newtheorem{lem}{Lemma}
\newtheorem{prop}{Proposition}
\newtheorem{axiom}{Axiom}
\newtheorem{post}{Postulate}
\newtheorem{trans}{Transformation}
\newtheorem{transf}{Transformer}
\newtheorem{law}{Law}
\newtheorem{prob}{Problem}
\newtheorem{soln}{Solution}
\newtheorem{alg}{Algorithm}

\title{\textbf{NOVUS MODERNUS GRIMOIRE AETERNUS MAGIA LUMTAUTO} --- A Modern Grimoire of Eternal Magickal Languages\thanks{Proceed with Caution. This is \textbf{A Call to Practice} Occult Mysteries. There is no guarantee that tears wont be involved\cite{crowley1948magick}.}}

\usepackage{amsmath, amssymb, graphicx}

% to include pdf pages
\usepackage{pdfpages}

% Define \invpi to flip the pi symbol and use it as a function
\newcommand{\invpi}[1]{\mathop{\rotatebox[origin=c]{180}{$\pi$}}#1}
\newcommand{\invdel}[1]{\mathop{\rotatebox[origin=c]{180}{$\Delta$}}#1}

\author{\textbf{M*A*P} Adept Psymaz\thanks{\textbf{Most Ancient Priest}, also known as Fut. Prof. J. Willrich Lutalo C.M.R.W; Curator, PI and President at Nuchwezi Research, GARUGA, Uganda. \textbf{ORCID:} \url{https://orcid.org/0000-0002-0002-4657}}\\Nuchwezi Research\\\href{mailto:joewillrich@gmail.com}{joewillrich@gmail.com}, \href{mailto:jwl@nuchwezi.com}{jwl@nuchwezi.com}}

%\date \today
\date {EDITION-DRAFT: \textbf{16}$^{th}$ \textbf{NOV}, \texttt{2025}}


\begin{document}

% insert [front] cover --- could just be a PNG or PDF
\includepdf[pages=1]{resources/front_cover.pdf}

\maketitle

\begin{abstract}
%\Large
In this manuscript, intentionally designed like a grimoire, we are to present for the first time, a proper distillation of research and applications in the use of esoteric languages for the purpose of performing occult operations as explored by initiates at Nuchwezi Esoteric School (NES) for the past 1 decade. This is an original contribution to the universal esoteric tradition and is meant to help curate, promulgate and advance a sane, thoughtful appreciation of the mysteries --- a line of work that goes back through the ages, to the first psy-ops and occult workings attempted by primitive man, through generations of diverse and varying explorations by mystics and initiates from all kinds of schools, cultures and traditions, all the way to modern approaches best known to the true initiates of illuminism. In particular though, this manuscript shall focus on 4 different ORIGINAL RESULTS of hard-work, study and yes, applying occult philosophy at Nuchwezi;
\begin{enumerate}
\item \textbf{LUMTAUTO} --- an application of computational mysticism in the form of an algorithmic cipher that can transform ordinary English or any language into a form suitable for occult operations and conjurations.
\item \textbf{The Grand Myrrh Transform} --- a related, but otherwise later and shorter, more occult TEA\cite{cli_tttt} algorithm for turning ordinary phrases into sacred words of power reminiscent of sacred languages such as Hebrew and Aramaic. 
\item  The \textbf{Ozin Cipher} --- a visual code first presented in an earlier work\cite{lutalo_2025_trans_genetics}, useful in expressing secret or special messages in an occult and psychologically charged hand that was first developed by the Illuminates of Nuchwezi Angelic (IoNA) via esoteric workings reminiscent of the methods of medieval angel-working wizards such as John Dee and Edward Kelley.
\item \textbf{Crypt of Medina} --- a special occult cipher also first developed at NES, and which, unlike most ciphers ancient or modern, allows for the visual encoding of occult messages in such a way that \textbf{one must explicitly use the method of reading between the lines} in order to understand its messages.
\end{enumerate} 
We shall [briefly] look at the underlying philosophies and supporting literature; shall look at many guidelines concerning how to practically apply the presented ideas; shall treat of the matter of mixing modern computer technology in applying these ideas, and shall share lots of visuals, links to supporting videos, community and online tools to help practitioners further and deepen their appreciation of these modern mysteries. This is a work for the Illuminati.
 \newline\newline
     \textbf{Keywords}: Foundations, Psy-Ops, Modern Magick, Computational Mysticism, Grimoire
\end{abstract}

\begin{figure}[H]
  \begin{center}
   \includegraphics[scale=0.8]{resources/emblem_ion.pdf}\\
   \caption{Illuminates of Nuchwezi}
  \label{FIG1}
  \end{center}
\end{figure}


\section{An Introduction to Magickal Languages}
\label{SECINTRO}



In the preface of his treatise on applying the new mathematics of Transformatics\cite{Lutalo2025_transformatics_thesis} to Genetics\cite{lutalo_2025_trans_genetics}, the author appeals to the sacred scriptures, particularly to the first book of the Torah (also known as the ``Pentateuch") --- which, not just for Kabbalists and Jewish mystics might be considered the foundational book of essential laws and instruction, but which also serves as a foundational text for many judaic-associated faiths and traditions --- the likes of Christians, Moslems, Rastafarians but also, and not very surprising, Theistic Satanists\cite{wikipedia_theistic_satanism}\footnote{Some critics might argue that actually, modern satanism only borrows the concept of Satan (as \textit{ha-satan}) from the Hebrew bible but doesn't appeal to Judaic roots or laws and that it instead is founded on twisting and elevating the Christian idea of Satan as a fallen angel\cite{copilot_assistant}\cite{wikipedia_theistic_satanism}, however, and logically so, it does make sense to pin them down, and assert their undeniable roots in ancient Judaism and its traditions for that reason alone. Also, note that we talk of \textbf{theistic satanism} here and not \textit{atheistic satanism} (also \textbf{LaVeyan Satanism}); Satan as a symbol not as a deity, and also not \textit{acosmic satanism} (more popular with edge-lords and ecclectic modern LHP philosophers); satan as the principle that opposes order --- perhaps \textit{chaos}?)}!. Lutalo calls us to consider the strong directive given man by God, when he is called to \textbf{``be fruitful, multiply, fill the earth and subdue it"} (Genesis 1:28).

In another work by Lutalo --- \textbf{3 Core Ideas in Computational Mysticism}\cite{Lutalo2024_3c}, a September 2024 paper that laid down the foundations of computational mysticism, he clearly lays down the fundamental significance of \textbf{language} not only in its use for bringing about transformations and the manifestation of will via computers, but also for general human life and affairs, when he says this in the abstract of that mini-paper:


\noindent
\begin{minipage}{1\textwidth}
\vspace{1em}
\begin{quotation}
{\ttfamily

Programming languages create a medium via which one can define and execute orders with certain effects at will, and certainly so. Basing on how language underlies the ability for humans to formulate and share thoughts with each other, we also see how the use of certain special languages underlies man's ability to command and control reality since ancient times.

}
\end{quotation}
\vspace{1em}
\end{minipage}


As we shall see and come to appreciate in this grimoire, a careful and willful use of ``special" languages --- especially, and from the perspective of the theme of this work, languages both inspired by or based on natural, but also artificial or \textit{synthetic} languages, can readily help yield results that we shall soon come to appreciate to be what ``magic"\cite{butler1952magic} or rather ``magick"\cite{crowley1929magick} is all about. Essentially, Lutalo tells us, in \cite{Lutalo2024_3c} that:


\noindent
\begin{minipage}{1\textwidth}
\vspace{1em}
\begin{quotation}
{\ttfamily

Basically, we see how it is indeed language, or rather, its use via communication --- a willful application
of language, that makes possible the creation and transmission of [any] thought.

}
\end{quotation}
\vspace{1em}
\end{minipage}


Thus, we argue that, \textbf{it is essentially via the willful application of language that man can come to subdue reality.} Moreover, and as we shall soon see when we consider the acceptable definitions of magick, any such willful acts, whether they leverage language in the form of expressions in the mind (``thoughtforms"), expressions in sound (``incantations", ``spells", ``mantras", ``commands", etc.), expressions in body/body-language (``gestures", ``mudras", ``assanas", ``signs", etc.) or as visual expressions (``sigils", ``mandalas", ``glyphs", ``signs" or ``symbols", etc.) are what make magick possible.


And as for the traditional concept of \textit{language}, still, \cite{Lutalo2024_3c}  offers a compelling and reliable working definition. But what of the idea of a \textbf{magickal language}? First, we shall return to the [modern] classics, \textbf{Aleister Crowley}\footnote{Apart from being a renown initiate into the ancient mysteries --- initiated circa 1898 into the \textbf{Hermetic Order of the Golden Dawn}\cite{cassiel1990encyclopedia}, he's also a famous and prolific writer and researcher on all matters occult and esoteric during the early 20$^{th}$ century, and was also well-known to have not only founded the tradition of THELEMA\cite{crowley1929magick} that has inspired many modern magical traditions such as the \textit{Wiccans} and several ``new-age" spirituality paths, but that he also considered himself not just a ``magus", but as also the ``BEAST"!} being one very undeniably reliable authority on the subject, and shall start by considering the ideas he presents in \textbf{Magick in Theory and Practice}\cite{crowley1929magick}:




\noindent
\begin{minipage}{1\textwidth}
\vspace{1em}
\begin{quotation}
{\ttfamily

Magick is the Science and Art of causing Change to occur in conformity with Will.

}
\end{quotation}
\vspace{1em}
\end{minipage}



To the best of our knowledge, it is not until when Crowley penned this definition, that magic as a stream of ``esotericism" or ``occult philosophy" and not as ``stage magic" or the practice of ``illusionism" passed from the realm of mere mysticism into a kind of formal science --- arguably, a great contribution to the galvanizing of so-called \textit{Scientific Illuminism}. Earlier authorities such as \textbf{Cornelius Agrippa}, despite having penned incredible tomes and compendiums on western esotericism and occult philosophy\cite{agrippa2014occult}, never actually, or rather, explicitly offered nor extended any such formal definitions that we know of.

That said, note that Crowley's definition isn't the only authoritative, nor usable \textbf{working definition} of magick, and as for that matter, though we shall not attempt to enumerate all of them here, there is a rare compilation of such definitions that was prepared by the author of this grimoire as far back as 2014\cite{lutalo_2025_definitions}, and that it not only lists Crowley's definition among a whooping total of \textbf{34 distinct definitions of magic[k]}, but that is also clearly offers their associated sources (authors and books/articles/websites/traditions, etc.). Among these, let us just recall but only 3:



\noindent
\begin{minipage}{1\textwidth}
\vspace{1em}
\begin{quotation}
\noindent {\ttfamily

The science and art of causing change (in consciousness) to occur in conformity with will, using means not currently understood by traditional Western science

}
\hspace*{\fill} --- \textbf{Modern Magick}, \textit{2010}, Donald Michael Kraig\cite{kraig2010modern}
\end{quotation}
\vspace{1em}
\end{minipage}


That one, especially resurfaced here, because DMK has really helped modern magick practitioners working outside of traditional initiatory systems and who might not be able to access proper initiators into hermeticism and practical western esotericism to actually get busy and attain results. His book\cite{kraig2010modern} on magick is very resourceful --- especially for solo practitioners, and we shall come back to it several times in later parts of this grimoire.



\noindent
\begin{minipage}{1\textwidth}
\vspace{1em}
\begin{quotation}
\noindent {\ttfamily

The Highest, most Absolute, and most Divine Knowledge of Natural Philosophy, advanced in its works and wonderful operations by a right understanding of the inward and occult virtue of things; so that true Agents being applied to proper Patients, strange and admirable effects will thereby be produced.

}
\hspace*{\fill} --- \textbf{The Goetia of the Lemegeton of King Solomon}, \textit{1904}, S. L. MacGregor Mathers and Aleister Crowley\cite{mathers1904goetia}
\end{quotation}
\vspace{1em}
\end{minipage}


That second one, not only because it is one of few that appeals to ``ancients" such as \textbf{Cornelius Agrippa}\cite{agrippa2014occult} or \textbf{Eliphas Levi}\footnote{This is the pen name of Alphonse Louis Constant (1810–1875), a highly influential French occult author and ceremonial magician whose writings significantly contributed to the revival of magic and esoteric thought in the 19th century\cite{wordweb_assistant}.} that especially championed concepts such as ``High Magic" or rather highly-eclectic Ceremonial [and priestly] Magick, but also because it might align well with classical kinds of magick such as the \textit{Alchemy} that scientists such as \textbf{Sir Isaac Newton} dabbled in occasionally, the \textit{Hermetic Medicine} that medieval esotericists such as \textbf{Paracelsus}\footnote{Also known as \textbf{Philippus Aureolus Theophrastus Bombastus von Hohenheim}\cite{FasanoSequeira2017}} --- also ``Father of Toxicology", championed, as well as mystics practicing arts such as \textit{Theurgy} --- the likes of \textbf{Emanuel Swedenborg}\footnote{Swedish scientist, theologian and mystic (1688-1772)\cite{wordweb_assistant}}. Moreover, and also a contemporary of Crowley, \textbf{Mathers} does deserve a special place in the hearts of modern occultists for his many contributions to formalizing and promulgating ancient mysteries in the 20$^{th}$ century.


Finally, we shall also re-surface this definition:


\noindent
\begin{minipage}{1\textwidth}
\vspace{1em}
\begin{quotation}
\noindent {\ttfamily

The enhancement of the probabilities/likelihood of a desired outcome/result.

}
\hspace*{\fill} --- \textbf{PsyberMagick: Advanced Ideas in Chaos Magick}, \textit{1995}, Peter Carroll\cite{Carroll1995}
\end{quotation}
\vspace{1em}
\end{minipage}


That final one, especially because the founder of the \textit{Chaos Magick} meta-paradigm modern magick tradition has also greatly helped inspire and liberate many practitioners from ideological slavery to ancient and medieval dogmas, but also that, as one might find when studying many of \textbf{Peter Carroll}'s works, he, like the associated magical communities he inspired, such as the \textbf{Illuminates of Thanateros}\footnote{See IoT German Section: \url{https://iot-d.de/} or IoT BIS: \url{https://iotbritishisles.com/}}, but also our own \textbf{Illuminates of Nuchwezi}\footnote{Refer to IoNA home page: \url{https://iona.nuchwezi.com/}} and individual modern chaos magicians such as \textbf{Joshua Madara}\footnote{Originally \url{http://hyperritual.com/}, now \url{https://eldri.tech/}} find much utility in associating modern Tech, Maths and Science sensibilities with Esotericism and Magick\footnote{See for example, our take on Probabilistic Metaphysics\cite{Lutalo2023_metaphysics}, which strongly reflects or aligns with Carroll's definition and ideas of Magick.}.


And so, having looked at all the past authorities, we shall consider just one more definition --- essentially, the one we consider to be our authoritative working definition, as laid out below:




\fbox{\begin{minipage}{\textwidth}
\large

Given the working definition of a \textbf{Certain Manifestor}\cite{Lutalo2025transpsy}\cite{Lutalo2023_metaphysics}:\\


\begin{transf}[The \textbf{Certain Manifestor}]
\label{TRANSFCM}
In a reality space $\Psi:N \times \psi_{k}$, \\
a \textbf{certain manifestor}, $\mathbb{k} : \Psi(\mathbb{k}): \psi_{\tau} : \tau \implies \mathbb{k} \quad \land \quad \invpi(\psi_\tau \in \Psi) = 1 \quad \\
\forall \mathbb{k} \in [1,|\Psi|]$ is the following operator:

\begin{trans}
 $\langle \tau \rangle \xrightarrow{O_{\lambda}(\cdot)}  \psi_\tau $\\
 \end{trans}
\end{transf}

we know that the operator $O_{\lambda}(\cdot)$, also defined as the \textbf{Certain Manifestor}, is a \textbf{Magician}, if we define magick as such:\\

\begin{defn}[\textbf{MAGICK}]
\label{DEFMAGICK}

Given some potential distinct event $e$ from the space of all possible events $\langle e* \rangle$, it can be manifested when requested for, willed or wanted --- by the application of strong faith, belief or conscious effort on the part of the operator relative to that event, $\lambda (\langle e* \rangle)(e)$, in an act that is essentially one of applying psychology, but otherwise which we shall call \textbf{MAGICK}. Essentially, then, \textbf{Magick is any act that can make the following transformation happen:}\\

{\Large

\begin{trans}
\label{TRANSMAGICK}
 $\langle e \rangle \xrightarrow{O_{\lambda}(\psi_\infty)(e)} \overset{>}{\psi_e} \approx  \psi_e $
 \end{trans}
 
 }


\end{defn} 

\end{minipage}}
\\

The critical term $ \overset{>}{\psi_e} $ in \textbf{\hyperref[TRANSMAGICK]{Transformation \ref{TRANSMAGICK}}} is meant to symbolize the ``desired result" --- a \textit{willed outcome}, that essentially is encoded in the language of transformatics\cite{Lutalo2025_transformatics_thesis} here as a kind of [identifying particular modal] sequence that captures the gist of that event or phenomena\footnote{Note that, as we saw in the important fundamental law --- \textbf{IGS}: \textbf{Identity Genome Sequence Law} first laid down in \cite{lutalo_2025_trans_genetics}, any potential event or phenomena might be somehow reducible to or expressible as some distinct sequence, and as per the sensibilities of transformatics, any such sequence might be related to or reducible to some symbol set.}, whereas $\langle e \rangle$ could be just a symbol, a word, thought or label depicting that which is desired or wanted.


A \textbf{magickal language} then, is a concept we might formally define as such:\\

\begin{defn}[A \textbf{Magickal Language}: $\mathbb{L}: \mathbb{N} \times \psi_\infty$]
\label{DEFMAGICKLANG}

Any system of encoding will, and particularly, such as would allow one to encode some desire or need such as in the input of \textbf{\hyperref[TRANSMAGICK]{Transformation \ref{TRANSMAGICK}}}, so that, by presenting it to a certain manifestor --- a special kind of operator --- essentially a magician, the corresponding desired event or result can then readily be manifested, is a \textbf{Magical Language}, $\mathbb{L}: \mathbb{N} \times \psi_\infty$, that \textit{might} span the infinite set of symbols or events $\psi_\infty$.

\end{defn} 



At this juncture, and given what we have just clarified, it might help to set some things clearer concerning how modern magick needs be approached, but also how it might be appreciated. First, note that, magick, despite being an art, is also a science\cite{lutalo_2025_definitions}. In fact, even long before modern authorities on magick such as the Computer Scientist Peter Carroll gave us a somewhat mathematical treatment\cite{Carroll1995}\cite{carroll2010octavo} of an originally mostly speculative and mystical craft, and yet, in their work on the theory and practice of magick, Aleister Crowley did also greatly help cast magick as an enterprise one might come to properly appreciate or evaluate through the lens of a mathematician, when he laid down the following theorem and remarks:


\noindent
\begin{minipage}{1\textwidth}
\vspace{1em}
\begin{quotation}
\noindent {\ttfamily

THEOREMS.
1) Every intentional act is a Magickal act.

By ``intentional" I mean ``willed". But even unintentional acts so seeming are not truly so. Thus, breathing is an act of the Will to Live.

}
\hspace*{\fill} --- \textbf{Magick in Theory and Practice}, \textit{1929}, Aleister Crowley\cite{crowley1929magick}
\end{quotation}
\vspace{1em}
\end{minipage}

And so that, for our case, we can rest assured, that the following theorem likewise shall drive the point home concerning why developing, knowing and applying a magical language might be more useful than not:\\

\begin{theo}[\textbf{Effectiveness of a Magickal Language}]
\label{THEOMAGLANG}

Given some desire, $\tau$, and a magician that can operate on it, $\lambda(\cdot)$, we can assert that: presenting the suitable encoding of $\tau$ in some magical language that the operator can process effectively, shall yield better results/increase the likelihood of manifesting the desired outcome than not.

\begin{proof}
The proofs are several, but we shall call out just two:
\begin{enumerate}
\item Follows from \textbf{\hyperref[DEFMAGICK]{Definition \ref{DEFMAGICK}}} and \textbf{\hyperref[DEFMAGICKLANG]{Definition \ref{DEFMAGICKLANG}}}.
\item Experience and generations of successful practitioners confirm this; the most effective magical operations require or assume the need to express desire using some encoding method such as sigilization, mantras, visualization, specific asanas, specific programs in specific languages, etc. so as to readily or quickly and most efficiently bring the [desired] result into manifestation.
\end{enumerate}
$\qed$
\end{proof}

\end{theo}


By now, \textbf{\hyperref[THEOMAGLANG]{Theorem \ref{THEOMAGLANG}}} should be obvious to anyone reading this grimoire --- like, for example, considering the otherwise commonplace case of \textit{wanting to eat food presented on a plate}: no matter what one might do, say or think, unless they actually go ahead and employ the ``right" language of ``eating" --- which entails presenting something to the mouth, chewing and/or swallowing it so as to actually ``eat" it; any other actions the operator might apply --- such as smiling at the food, looking at a picture of the food, or perhaps merely smelling it, might only leave the food turning cold, and perhaps only satiate the mind (that ``I have food"), but otherwise leave the person not nourished and possibly still hungry (the food hasn't been ingested, and neither has it been digested). A related, and totally absurd case might be that of \textit{wanting to bare a [normal human] child} without having sex nor having ones sperm presented to someone's ready ova. It just wouldn't work!\footnote{Well, one might will a child into existence [without sex], say, by having some donor offer them a baby, or by adopting an abandoned kid, or adopting an already impregnated girl etc. However, the original intent of bearing one's own literal child shall not be met --- preternatural cases such as the immaculate conception of the Blessed Virgin Mary [dogma], and subsequently her \textbf{virgin birth} of Jesus Christ[Matthew (1:18–25), Luke (1:26–38)] might of course deviate from this, but also then, the birth of Jesus without Joseph having to copulate Mary perhaps didn't originate from an intent of Joseph's desire to bear \textit{his own child} --- as careful analysis of scripture\cite{newjerusalem1985} does indeed confirm.} 


In the rest of this manuscript, we are thus going to spend time exploring, in theory, but also in practice, several original and multipurpose (and arguably ``general", but not necessarily infallible) magickal languages that have been developed or discovered by the author while exploring and applying Magick at NES, and yes, in sane, and for sane reasons\footnote{There is nothing as insane as assuming that a limited operator [\textit{an automaton and not a psymaton, a psymaton and not a man, a man and not an angel, an angel and not a god, a god and not God}] might be able to process an infinite sequence of [magickal] languages when presented with them or an infinite sequence of desires encoded using them, however, we shall still proceed to present these [limited] languages, which, even though they might seem finite and constrained for one or some operator, could otherwise find [infinite] purpose in the hands of yet another [well-prepared] operator, and thus, it might then not be insane to consider that all of magick might be somehow reducible to just the mastery of [one of] these four languages we are about to unravel.}.




\fbox{\begin{minipage}{\textwidth}
\large

\textbf{Concerning Relevance of [Practicing] Magick}:\\

Especially for the uninitiated, much of what is being talked about in this volume might seem or come off as strange gibberish, total mysticism or scientific absurdity! However, like the scripture warns the wise to \textit{not throw pearls at swine};


\begin{quotation}
\noindent {\ttfamily

Do not give dogs what is holy; and do not throw your pearls in front of pigs, or they may trample them and then turn on you and tear you to pieces.

}
\hspace*{\fill} --- \textbf{Sermon on the Mount}, \textit{Mathew 7:6}, Jesus Christ\cite{newjerusalem1985}
\end{quotation}
\vspace{1em}
 

We [the initiated, genuine students and explorers of the sacred and mystical], ought be, first, unbothered that the mundane don't see any value in dedicated study and careful practice of Occult Science. But also that, we ought not concern ourselves with trying to appeal to plebeians and the unworthy, for the true rewards of this science and art are innumerable, ancient and eternal --- as one authority, \textbf{Walter Ernest Butler}, shared in his foundational treatise on the subject:

\begin{quotation}
\noindent {\ttfamily

...the old Hermetic axiom \textit{Solve et coagula}, which may be rendered as ``Dissolve and re-form," and so he uses the rites of the High Magic to effect both that dissolution and that reformation. But what is dissolved, and what is reformed? ...it is the personal self which he has for so long regarded as his only real self, this personality which he has so tenaciously clung to and defended, has pampered and indulged --- it this \textit{persona} this mask of the real man which must be dissolved and reformed. But how shall that which is itself imperfect produce perfection? ``Nature unaided, fails," said the old alchemists, and in the Scriptures we read ``Except the Lord build the House, the workman worketh in vain." So the magician in all humility seeks the Knowledge and Conversation of his Holy Guardian Angel --- that True Self of which his earthly personality is but the mask. This is the supreme aim of the magician. All else, spells and charms, rituals and circles, swords, wands and fumigations, all are but means by which he may accomplish that end. Then, being united with that True Self --- if only for a brief time --- he is instructed by that Inner Ruler in that Higher Magic which will one day bring up his manhood into his Godhood and will achieve that which the True Mysteries have ever declared to be the true end of man --- Deification.


}
\hspace*{\fill} --- \textbf{Magic: Its Ritual, Power and Purpose}, \textit{1952}, W.E. BUTLER\cite{butler1952magic}
\end{quotation}

\end{minipage}}
\\





\begin{figure}[H]
  \begin{center}
   \includegraphics[scale=0.8]{resources/iona.pdf}\\
   \caption{Illuminates of Nuchwezi Angelic}
  \label{FIG2}
  \end{center}
\end{figure}


\begin{figure}[H]
  \begin{center}
   \includegraphics[width=\textwidth]{resources/lumtauto_slogan.pdf}\\
  \end{center}
\end{figure}

\section{Language $\rightarrow$ LUMTAUTO}
\label{SECLUMTAUTO}

\begin{transf}[The \textbf{Magical Language \texttt{Lumtauto}}]
\label{TRANSFLUMTAUTO}
If $\Theta^n$ is a sequence of $n > 0$ symbols (the original message) spanning the \textbf{Latin Alphabet} or the symbol set $\psi_{az}$, such that:

\begin{multline}
\label{EQLATINALPHABET}
\psi_{az} = \langle a, b, c, d, e, f, g, h, i, j, k, l, m, n, o, p, q, r, s, t, u, v, w, x, y, z \rangle: \invpi(\psi_{az}) = 26 \\ \quad \land \quad \Theta^n:\mathbb{N} \times \psi_{az}
\end{multline}

then the following transformation:\\

\begin{trans}
\label{TRANSLUMTAUTO}
$\Theta^n \xrightarrow{O_{lauto(\cdot)}} \Theta^* = \Omega^n;$\\
$\invpi(\Theta^n) = \invpi(\Theta^*) = \invpi(\Omega^n)$\\
$\land \quad \forall \theta_{i \in [1,n]} \in \Theta^n \quad \exists \omega_{j \in [1,n]} \in \Omega^n \quad \land \quad \invpi(\theta_i \in \Theta^n) = \invpi(\omega_i \in \Omega^n) = 1$\\
$\land \quad \forall \alpha \in \psi(\Theta^n): \invpi(\alpha \in \psi(\Theta^n)) = f_\alpha \implies \alpha \in \psi(\Omega^n): \invpi(\alpha \in \psi(\Omega^n)) = f_\alpha$\\
$\land \quad \overset{>}{\psi(\Theta^n)} = \overset{>}{\psi(\Omega^n)} \lor \overset{>}{\psi(\Theta^n)} \neq \overset{>}{\psi(\Omega^n)}$\\
$\land \quad \tilde{A}(\Theta^n \rightarrow \Omega^n) > 1 \qed$
\end{trans}

is guaranteed to always produce/generate a derivative message --- $\Theta^*$ that has the following properties:\\


\begin{multline}
\label{EQLUMTAUTO}
\forall \alpha \in \Theta^n \implies \beta \in \Omega^n \implies \begin{cases}
a \rightarrow u, & \\ \text{what happened: }\\ a \in \Theta^n \text{ became } u \in \Omega\\ \land \quad I(a,\Theta^n) = I(u,\Omega^n) = i \quad iff \quad \theta_{i=I(u,\Omega^n)} = a\\
b \rightarrow y, & \\ \text{what happened: }\\b \in \Theta^n \text{ became } y \in \Omega\\ \land \quad I(b,\Theta^n) = I(u,\Omega^n) = i \quad iff \quad \theta_{i=I(y,\Omega^n)} = b\\
c \rightarrow x,& \\
d \rightarrow w,& \\
e \rightarrow o,& \\
f \rightarrow f,& \\
g \rightarrow t,& \\
h \rightarrow s,& \\
i \rightarrow i,& \\
j \rightarrow q,& \\
k \rightarrow p,& \\
l \rightarrow l,& \\
m \rightarrow n,& \\
n \rightarrow m,& \\
o \rightarrow e,& \\
p \rightarrow k,& \\
q \rightarrow j,& \\
r \rightarrow r,& \\
s \rightarrow h,& \\
t \rightarrow g,& \\
u \rightarrow a,& \\
v \rightarrow v,& \\
w \rightarrow d,& \\
x \rightarrow c,& \\
y \rightarrow b,& \\
z \rightarrow z& \\ \text{what happened: }\\z \in \Theta^n \text{ became } z \in \Omega\\ \land \quad I(z,\Theta^n) = I(z,\Omega^n) = i \quad iff \quad \theta_{i=I(z,\Omega^n)} = z\\
\end{cases}
\end{multline}
$\qed$

And so that, the resultant [transformed] message, $\Theta^* = \Omega^n$, despite being the same exact length as the original message, is not exactly equivalent to it, and is an instance of text in the language \textbf{LUMTAUTO}.

\end{transf}


\textbf{\hyperref[EQLUMTAUTO]{Equation \ref{EQLUMTAUTO}}} explicitly specifies the magickal language \textbf{LUMTAUTO}, in which, if one had a starting message such as ``language", it is then transformed into an equivalently correct \textbf{magical message} ``lumtauto" via the transformation that a transformer such as \textbf{\hyperref[TRANSFLUMTAUTO]{Transformer \ref{TRANSFLUMTAUTO}}}, also encoded as the operator $O_{lauto}(\cdot) = O_{lauto}(m:\mathbb{N} \times \psi_{az})$ in \textbf{\hyperref[TRANSLUMTAUTO]{Transformation \ref{TRANSLUMTAUTO}}}. This transformation is certain, and always guaranteed to occur, if one applies that lumtauto transformer to any message.

\subsubsection{Relevance and MAGICKAL IMPACT of the magickal language LUMTAUTO}
\label{SECRELLUMTAUTO}

Of course, merely knowing that we have a mechanism by which any letter in the latin alphabet can be mapped to another letter of the same set in a certain way and not any other, might not immediately strike some people as either \textbf{odd, relevant or potent} --- especially for those who have never attempted to actually practice magick literally or rather practically (and not just in theory or just wishfully --- as most plebeians and the uninitiated would or might). But, for a good starter at how powerful this magickal language is, consider the impact the following transformations might have:

\subsubsection{The Four Special Properties of LUMTAUTO}
\label{SECPROPERTIESLUMTAUTO}

\begin{enumerate}
\item First, on the actor or agent --- the magician, attempting to process; perform or utter, read or evoke, chant or vibrate, charge or apply or even merely cast or visualize these messages.
\item The immediate environment of whomever processes the resultant (transformed) messages.
\item The semantics or meaning of the resultant message vis-a-vis the original and its intent.
\item The structure of the resultant message vis-a-vis that of the original --- which, and provably so, has the \textbf{interesting properties} such as; all consonants in $\Theta^n$ are mapped to \textit{different}\footnote{\textit{Different} for consonants, because, we note that the \textbf{special quartet} $\{f, l, r, z \}$ is conserved under \textbf{\hyperref[TRANSLUMTAUTO]{Transformation \ref{TRANSLUMTAUTO}}}, while all other consonants in $\psi_{az}$ are not --- they are guaranteed to change under the lumtauto transformation.} consonants in $\Omega^n$, while all vowels in $\Theta^n$ are mapped to  \textbf{different} vowels in $\Omega^n$ \textbf{except} $i$.
\end{enumerate}


\subsubsection{FOUR Examples of Applying LUMTAUTO}
\label{SECEXAMPLESLUMTAUTO}

The four properties depicted in \textbf{\hyperref[SECPROPERTIESLUMTAUTO]{Section \ref{SECPROPERTIESLUMTAUTO}}}, shall be illustrated via the following four illustrative and relevant examples:


\begin{enumerate}
\item{\textbf{The Word ``language"}.
It becomes ``lumtauto" as we have already seen.
}

\item{\textbf{The \textbf{Magical Cogito Ergo Sum} Motto  ``I think, therefore I am"}. First attributed to classical philosopher \textbf{Rene Descartes} and discussed in \cite{Lutalo2025transpsy}, becomes the weird magical motto\footnote{For direct usage, prefer the formulation: {\LARGE \texttt{IGA SIMPE, GASOROFERO IUNA!}}} {\Large ``I gsimp, gsorofero I un"} under the lumtauto transformer (\textbf{\hyperref[TRANSFLUMTAUTO]{Transformer \ref{TRANSFLUMTAUTO}}}). 
}

\item{The \textbf{Magical Lord's Prayer} --- the \textit{Paternoster}, first attributed to god-man \textbf{Jesus Christ} of Nazareth and presented in \textbf{Mathew 6:9-13}\cite{newjerusalem1985}, here quoted verbatim as in the bible version \cite{newjerusalem1985}:\\


{\ttfamily

Our Father in heaven,
may your name be held holy,
your kingdom come,
your will be done,
on earth as in heaven.
Give us today our daily bread.
And forgive us our debts,
as we have forgiven those who are in debt to us.
And do not put us to the test,
but save us from the Evil One.

}

\vspace{2em}

becomes the [verbatim] and equivalent \textit{mystical} magical prayer\\

{\ttfamily

Ear Fugsor im souvom,
nub bear muno yo solw selb,
bear pimtwen xeno,
bear dill yo wemo,
em ourgs uh im souvom.
Tivo ah gewub ear wuilb yrouw.
Umw fertivo ah ear woygh,
uh do suvo fertivom gseho dse uro im woyg ge ah.
Umw we meg kag ah ge gso gohg,
yag huvo ah fren gso Ovil Emo.

}

\vspace{2em}

\item{The \textbf{Opening Sentence of the Bible}\footnote{For those familiar with the TEA programming language, note that this sentence has earlier on also shown up among the standard TEA program examples (check the TEA WEB IDE), when we notice the interesting result of applying the \textbf{modal sequence statistic} command: \texttt{u:} to it --- the result `` enthadgirIbGocvs." potentially telling that in transforming simple messages, one might many times uncover some very peculiar things!} --- also the first sentence in the \textbf{Pentateuch} and presented in \textbf{Genesis 1:1}\cite{newjerusalem1985}, here quoted verbatim as in the bible version \cite{newjerusalem1985}:\\


{\ttfamily

In the beginning God created heaven and earth.

}

\vspace{2em}

becomes the [verbatim] and corresponding magical utterance\\

{\ttfamily

Im gso yotimmimt Tew xrougow souvom umw ourgs.

}

\vspace{2em}

This particular example warrants some little more discussion concerning working with transformations of text for magical purposes. For example, \textbf{one of the major utilities of a magickal language like lumtauto would be to generate or construct arcane, charged, enchanting and perhaps obscure magical formulas, mantras and incantations} --- the stuff of serious, barbarous, ancient and seasoned magicians, sorcerers, wizards, witches, high-priests, spiritualists\footnote{Often needing strange utterances to compel or subdue spirits [using other spirits] or to command and operate spiritually using the strange \textit{tongues} of the Holy Spirit} and perhaps exorcists to name but a few. 

Concerning this then, it would be more useful --- especially for the case of rendering the outputs of such a transformer as \textbf{\hyperref[TRANSFLUMTAUTO]{Transformer \ref{TRANSFLUMTAUTO}}}, to be not only different from the original text, but that they are also rendered \sout{legible} readable and readily utterable so as to make them more useful in especially the sharing or reuse of standard and or, eternal spells, prayers and commands --- in a manner as how, using a normal natural language such as English, one can have the guarantee that the words and phrases thus encoded in a prayer, scripture or grimoire, shall stay readable and usable as originally intended across many generations of readers, users, students and practitioners. And thus, despite many phrases or words in English never requiring any modifications to their structure so as to make them readily and correctly pronounceable, and yet, for the outputs of the LUMTAUTO transformation thus presented, we find (especially out of experience, taste and necessity), that despite the succinctness and elegance of the transformer as presented either mathematically, or as a computer algorithm --- refer to ..., we might want to sometimes \textit{further massage} the lumtauto transformer output so as to make them better \textbf{for actual use}. 

Thus, for example, the above transformed opening sentence of the scriptures\footnote{And it should not come as a surprise to those who are learned or who know why certain books have survived and been passed on from generation to generation for millennia.. such as the Bible and other Sacred books --- well, not just for their explicit wisdom and stories, but the fact that they contain many [sometimes] hidden nuggets of magical wisdom and wizardry if one knows what to look out for, or how to \textit{creatively use} the written word.} might become:\\

{\ttfamily
\LARGE

Imi geso yotimemimet Tewa xarougow souvom umwa ouregus.

}

\vspace{2em}

And so that, people practicing magick --- especially \textbf{ritual or ceremonial magick}, where operators must speak, act, dance, shout, vibrate and do many expressive things during the operations or workings (so as to make them worthwhile and successful) --- so-called \textit{psycho-dramas}\cite{LaVey1969}\footnote{A term very popular with modern satanists such as \textbf{Anton Szandor LaVey}, the founder of the \textbf{Church of Satan} and author of \textbf{The Satanic Bible}\cite{LaVey1969}.}, can do so effectively, systematically and repeatably in a standardized way. 

\vspace{2em}

And so, with sufficient practice and experimentation transforming words and phrases or texts into magical versions using LUMTAUTO, you shall learn how to be creative, and formulate very potent and interesting spells for almost any purpose!

}



}
\end{enumerate}


\subsection{The LUMTAUTO Algorithm}

\begin{table}[H]
  \begin{tabular}{|p{0.95\textwidth}} % Left border only
    \hline
    \begin{figure}[H]
      \centering
      \includegraphics[width=0.9\textwidth]{resources/lumtauto.jpg}\\
    \end{figure} \\
    \cline{1-1} % Bottom border only
  \end{tabular}
\end{table}

Using a formalism familiar to computer scientists, we shall here formally specify the LUMTAUTO algorithm using a method that could readily be translated into a compartible computer program so that an interest magician or researcher can then use the magickal language with any computer or programming language available to or familiar to them.

\vspace{2em}


\begin{alg}[The \textbf{LUMTAUTO Algorithm}: \texttt{lauto(msg\_in)}]
\label{ALGLUMTAUTO}
$ $\\
\begin{enumerate}
\item \textbf{GIVEN} source sequence (a plain-text message), \texttt{msg\_in} of length $n$.
\item \textbf{GIVEN} latin alphabet (as a list of letter symbols), \texttt{list\_a\_z} of 26 elements from $\psi_{az}$.
\item \textbf{GIVEN} list of vowels, \texttt{list\_vowels} = $\langle a,e,i,o,u \rangle$.

\item \textbf{COMPUTE} list of mirrored vowels, \texttt{list\_vowels\_mirror} = \texttt{reversed(list\_vowels)}.

\item{ \textbf{COMPUTE} first version of target alphabet, \texttt{list\_a\_z\_intermediate} thus: 

\begin{enumerate}
\item \textbf{INITIALIZE} \texttt{list\_a\_z\_intermediate} by cloning \texttt{list\_a\_z}.
\item \textbf{UPDATE} \texttt{list\_a\_z\_intermediate} by replacing each of its members, $v_i$ at index $i$ in that list, with another character, $v_j$ from \texttt{list\_vowels\_mirror}, if the element $v_i$ is located at position $j$ in \texttt{list\_vowels}.
\end{enumerate}

}

\item{ \textbf{COMPUTE} final version of target alphabet, \texttt{list\_a\_z\_lumtauto} thus: 

\begin{enumerate}
\item \textbf{INITIALIZE} \texttt{list\_a\_z\_lumtauto} by cloning \texttt{list\_a\_z\_intermediate}.
\item{ \textbf{UPDATE} \texttt{list\_a\_z\_lumtauto} thus:\\

\textbf{FOR} each element $v_i$ in \texttt{list\_a\_z\_lumtauto} between positions 1 to $\frac{1}{2} \times$ \texttt{length(list\_a\_z\_lumtauto)}:
\begin{enumerate}
\item \textbf{SET} element at $i$ as $v_i = $ \texttt{list\_a\_z\_lumtauto[i]}.
\item \textbf{SET} element $vm_i = $ \texttt{list\_a\_z\_lumtauto[n - i]}.
\item{ \textbf{IF} $v_i$ \textbf{NOT IN} \texttt{list\_vowels} \textbf{AND} $vm_i$ \textbf{NOT IN} \texttt{list\_vowels}: 

\begin{enumerate}
\item \textbf{SWAP} $v_i$ at position $i$ in \texttt{list\_a\_z\_lumtauto} with $vm_i$.
\item \textbf{SWAP} $vm_i$ at position $n - i$ in \texttt{list\_a\_z\_lumtauto} with $v_i$.
\end{enumerate}
}

\end{enumerate}

}
\end{enumerate}

}

\item{ \textbf{COMPUTE} resultant, transformed/translated message, \texttt{msg\_lumtauto} thus:

\begin{enumerate}
\item \textbf{INITIALIZE} \texttt{msg\_lumtauto} = \texttt{msg\_in}.
\item{ \textbf{FOR} each letter $c_i$ in \texttt{msg\_lumtauto} at position $i$:

\begin{enumerate}
\item \textbf{COMPUTE} the corresponding mirror letter position $k = $ \texttt{index($c_i$ in list\_a\_z\_lumtauto)}.
\item \textbf{REPLACE} $c_i$ in \texttt{msg\_lumtauto} via the \textbf{OVERWRITE} operation: \texttt{msg\_lumtauto[i] = list\_a\_z\_lumtauto[k]}
\end{enumerate}
}
\end{enumerate}

}

\item \textbf{RETURN} resultant sequence, \texttt{msg\_lumtauto}.
\end{enumerate}
$\qed$
\end{alg}


And as for how to go about testing or implementing this, note that an example implementation in the popular \textbf{PYTHON} computer programming language is shown in \textbf{\hyperref[SECLUMTAUTO_PY]{Section \ref{SECLUMTAUTO_PY}}}, and is based on \textbf{\hyperref[ALGLUMTAUTO]{Algorithm \ref{ALGLUMTAUTO}}}, itself based on \textbf{\hyperref[TRANSFLUMTAUTO]{Transformer \ref{TRANSFLUMTAUTO}}}, while a two-way, encode-decode version in the language this algorithm was first implemented is shown in \textbf{\hyperref[SECLUMTAUTO_JS]{Section \ref{SECLUMTAUTO_JS}}}.



\subsubsection{The LUMTAUTO Algorithm in TEA}
\label{SECTEALUMTAUTO}

For purposes of helping you to immediately be able to study, apply and share this LUMTAUTO language and the associated text transformer, note that the following TEA\cite{cli_tttt}\cite{Lutalo2024TEATAZ} program is a first, and reliable, robust implementation of the transformer specified in \textbf{\hyperref[TRANSFLUMTAUTO]{Transformer \ref{TRANSFLUMTAUTO}}}.


 %\small
  \begin{tcolorbox}[teaterminalstyle, title=TEA Program: The LUMTAUTO Transformer, breakable]
  %\begin{lstlisting}[language=TEA, caption={TP C7}, label={LSTC7}, numbers=left]
  \begin{lstlisting}[language=TEA,breaklines=true]
i:{language} # given some message
v:vMESSAGE #store the original message

#COMPLETE LANGuage -> LUMTauto TRANSFORM
#lumtauto-TRANSFORM [lower-case]
#start transforming via the lumtauto cipher algorithm
r!:a:_%_ #U
r!:b:_%%_ #Y
r!:c:_%%%_ #X
r!:d:_%%%%_ #W
r!:e:_%%%%%_ #O
r!:f:_%%%%%%_ #F
r!:g:_%%%%%%%_ #T
r!:h:_%%%%%%%%_ #S
r!:i:_%%%%%%%%%_ #I
r!:j:_%%%%%%%%%%_ #Q
r!:k:_%%%%%%%%%%%_ #P
r!:l:_%%%%%%%%%%%%_ #L
r!:m:_%%%%%%%%%%%%%_ #N
r!:n:m
r!:o:e
r!:p:k
r!:q:j
r!:r:r
r!:s:h
r!:t:g
r!:u:a
r!:v:v
r!:w:d
r!:x:c
r!:y:b
r!:z:z
#complete the transform
r!:_%_:u
r!:_%%_:y
r!:_%%%_:x
r!:_%%%%_:w
r!:_%%%%%_:o
r!:_%%%%%%_:f
r!:_%%%%%%%_:t
r!:_%%%%%%%%_:s
r!:_%%%%%%%%%_:i
r!:_%%%%%%%%%%_:q
r!:_%%%%%%%%%%%_:p
r!:_%%%%%%%%%%%%_:l
r!:_%%%%%%%%%%%%%_:n
#FINISHED: for lower-case

#j:lFINISHED

#COMPLETE LUMTAUTO TRANSFORM [for uppercase]
#LUMTAUTO-TRANSFORM [upper-case]
#start transforming via the LUMTAUTO cipher algorithm
r!:A:_%_ #U
r!:B:_%%_ #Y
r!:C:_%%%_ #X
r!:D:_%%%%_ #W
r!:E:_%%%%%_ #O
r!:F:_%%%%%%_ #F
r!:G:_%%%%%%%_ #T
r!:H:_%%%%%%%%_ #S
r!:I:_%%%%%%%%%_ #I
r!:J:_%%%%%%%%%%_ #Q
r!:K:_%%%%%%%%%%%_ #P
r!:L:_%%%%%%%%%%%%_ #L
r!:M:_%%%%%%%%%%%%%_ #N
r!:N:M
r!:O:E
r!:P:K
r!:Q:J
r!:R:R
r!:S:H
r!:T:G
r!:U:A
r!:V:V
r!:W:D
r!:X:C
r!:Y:B
r!:Z:Z
#complete the transform
r!:_%_:U
r!:_%%_:Y
r!:_%%%_:X
r!:_%%%%_:W
r!:_%%%%%_:O
r!:_%%%%%%_:F
r!:_%%%%%%%_:T
r!:_%%%%%%%%_:S
r!:_%%%%%%%%%_:I
r!:_%%%%%%%%%%_:Q
r!:_%%%%%%%%%%%_:P
r!:_%%%%%%%%%%%%_:L
r!:_%%%%%%%%%%%%%_:N
#FINISHED: for upper-case

#Complete Original Message NOW Transformed

#then store transformed message :)
v:vTRANSFORMED_MESSAGE #such as "lumtauto"
   \end{lstlisting}
  \end{tcolorbox}
    \captionof{figure}{TEA Program: The LANGuage to LUMTauto transformer}
  \label{FIGLUMTAUTOTEACODE}

\vspace{2em}

\textbf{\hyperref[FIGLUMTAUTOTEACODE]{Figure \ref{FIGLUMTAUTOTEACODE}}} is the \textbf{source-code} of the non-interactive TEA program implementing this algorithm, and which, when actually cleaned of comments and MINIFIED, would be the program depicted in  \textbf{\hyperref[FIGLUMTAUTOTEACODE_CLEAN]{Figure \ref{FIGLUMTAUTOTEACODE_CLEAN}}}. However, and in case one wishes to look at the code, modify or run \textbf{an interactive version} of it like on the Linux, Unix, Windows or MAC OS command-line or on the WEB, the most recent version should be what you might find or run directly and live via:
  
  
\vspace{2em}

 \url{https://tea.nuchwezi.com/?i=put+your+message+here&fc=https://gist.githubusercontent.com/mcnemesis/ae9d6226d49f5a8601a84241a08f07c8/raw/lumtauto_language_transformer.tea}

\vspace{1em}


\textbf{ALTERNATIVELY} just use the short-link: \url{https://bit.ly/lumtauto}

\vspace{2em}


 %\small
  \begin{tcolorbox}[teaterminalstyle, title=CLEAN TEA Program: The LUMTAUTO Transformer, breakable]
  %\begin{lstlisting}[language=TEA, caption={TP C7}, label={LSTC7}, numbers=left]
  \begin{lstlisting}[language=TEA,breaklines=true]
f!:^$:lDONTPROMPT|i!:{Enter Message to be Encoded:}|i*:|i:{language}|l:lDONTPROMPT|v:vMESSAGE|r!:a:_%_|r!:b:_%%_|r!:c:_%%%_|r!:d:_%%%%_|r!:e:_%%%%%_|r!:f:_%%%%%%_|r!:g:_%%%%%%%_|r!:h:_%%%%%%%%_|r!:i:_%%%%%%%%%_|r!:j:_%%%%%%%%%%_|r!:k:_%%%%%%%%%%%_|r!:l:_%%%%%%%%%%%%_|r!:m:_%%%%%%%%%%%%%_|r!:n:m|r!:o:e|r!:p:k|r!:q:j|r!:r:r|r!:s:h|r!:t:g|r!:u:a|r!:v:v|r!:w:d|r!:x:c|r!:y:b|r!:z:z|r!:_%_:u|r!:_%%_:y|r!:_%%%_:x|r!:_%%%%_:w|r!:_%%%%%_:o|r!:_%%%%%%_:f|r!:_%%%%%%%_:t|r!:_%%%%%%%%_:s|r!:_%%%%%%%%%_:i|r!:_%%%%%%%%%%_:q|r!:_%%%%%%%%%%%_:p|r!:_%%%%%%%%%%%%_:l|r!:_%%%%%%%%%%%%%_:n|r!:A:_%_|r!:B:_%%_|r!:C:_%%%_|r!:D:_%%%%_|r!:E:_%%%%%_|r!:F:_%%%%%%_|r!:G:_%%%%%%%_|r!:H:_%%%%%%%%_|r!:I:_%%%%%%%%%_|r!:J:_%%%%%%%%%%_|r!:K:_%%%%%%%%%%%_|r!:L:_%%%%%%%%%%%%_|r!:M:_%%%%%%%%%%%%%_|r!:N:M|r!:O:E|r!:P:K|r!:Q:J|r!:R:R|r!:S:H|r!:T:G|r!:U:A|r!:V:V|r!:W:D|r!:X:C|r!:Y:B|r!:Z:Z|r!:_%_:U|r!:_%%_:Y|r!:_%%%_:X|r!:_%%%%_:W|r!:_%%%%%_:O|r!:_%%%%%%_:F|r!:_%%%%%%%_:T|r!:_%%%%%%%%_:S|r!:_%%%%%%%%%_:I|r!:_%%%%%%%%%%_:Q|r!:_%%%%%%%%%%%_:P|r!:_%%%%%%%%%%%%_:L|r!:_%%%%%%%%%%%%%_:N|l:lFINISHED|v:vTRANSFORMED_MESSAGE|i!:{In LUMTAUTO
}|x*!: vMESSAGE|x!: {
-- becomes --
}|x*!: vTRANSFORMED_MESSAGE|i*:|y:vTRANSFORMED_MESSAGE
   \end{lstlisting}
  \end{tcolorbox}
    \captionof{figure}{MINIFIED TEA Program: The basic LUMTAUTO transformer program source-code}
  \label{FIGLUMTAUTOTEACODE_CLEAN}
  
  
 \subsection{The Background of LUMTAUTO}


This language and the associated algorithm was first developed at Nuchwezi Esoteric School around February 2015 (almost \textbf{15 years ago!}) as per the official repository of the associated original project --- \textbf{Font Crypto}\cite{nuchweziCrypto} --- an online/web app project with the source-code plus several other occult ciphers, text-transformation programs and magickal languages (including text to \textbf{Hieroglyphics, text to Greek, Cuneiform and Masonic cipher} among others) we explored back then\footnote{Refer to \url{https://github.com/NuChwezi/font-crypto}}\cite{nuchweziCrypto}.

\vspace{1em}

Back then, Nuchwezi as a technology startup and research community was just about a year old (since its founding in July 2014). And so, given what we have seen of the latest way that this language is approached, utilized and formalized, such as in \textbf{\hyperref[TRANSFLUMTAUTO]{Transformer \ref{TRANSFLUMTAUTO}}} and actual modern code for the associated transformer program as in \textbf{\hyperref[SECTEALUMTAUTO]{Section \ref{SECTEALUMTAUTO}}}, surely, this language has now reached some appreciable maturity and can be properly applied in actual workings and research.

\vspace{1em}

Talking of its applications though, note that the first formal mention and heavy use of LUMTAUTO was in the literary fiction work --- the novel, \textbf{Shrines of The Free Men}\cite{shrinesjwl}, first made public around 2018. In that book, a stealth link to the above mentioned online tool is shared, as part of a snippet of a text-chat session between various students and netizens in a school alumni social media community as depicted on page 24 of \cite{shrinesjwl}. It was shared back then as the link:


\vspace{1em}

 \url{http://tiny.cc/cry_dept#scrt}

\vspace{1em}

However, that link does not seem to be working anymore, and instead, one would better access the \textbf{original JavaScript implementation} of LUMTAUTO, together with the other ciphers and magical languages developed back then, via the up-to-date link:



\vspace{1em}

 \url{http://crypto.nuchwezi.com}

\vspace{1em}

Also, and more concerning how it was used in that novel, note that there are several parts in that book, where the author deliberately chose to present certain text --- especially specially formatted spells and incantations, via projections from the original plain versions into corresponding versions in LUMTAUTO. We shall call out just a few examples, showing the lumtauto that was depicted in the book Vs the plain text version one might decipher from them using a two-way encoder/transformer of LUMTAUTO such as the original JavaScript transformer program could.






\begin{table}[H]
  \begin{tabular}{|p{0.95\textwidth}} % Left border only
    \hline
    \begin{figure}[H]
      \centering
      \includegraphics[width=0.9\textwidth]{resources/illustration_trudy_olga_spirit_fight_2.jpg}\\
  \caption{A Spiritual Battle with a Malevolent Spirit}
      \label{FIGOLGA}
    \end{figure} \\
    \cline{1-1} % Bottom border only
  \end{tabular}
\end{table}



The first example use case we are to look into occurs on on page 166 of \cite{shrinesjwl} --- it is the case of a ``powerful mantra" --- in the context of the story, being employed by Trudy inside a lucid dream, to exorcise a feminine spirit that had been impersonating her friend Olga, and which had suddenly turned malevolent:\\



%\begin{figure}[H]
  \begin{tcbverbatim}[title=A Mantra from the novel ``Shrines of The Free Men"]
I yimw bea im gso muno ef Rasumtu umw ull gso hkirigh ef mugaro umw
nb umxohgerh soro krohomg.
I yimw bea roturwlohh ef dsogsor bea uro soro im hkirig, nimw er yewb.
I yimw bea umw xag fren bea ull hearxoh ef kedor gsug uro Uminahgimt
bea ritsg med, im gso muno ef gso Nehg Kedorfal Rasumtu!
I sarl bea imge um ogormul krihem, I xag bea ge hsrowh, I hannem
gsamwor umw firo gsug ough ug gso yewb, nimw umw hkirig ge hgripo
umw wokrivo bea ef ull bear hgromtgs umw soulgs.
I hannem gso samtriohg ef soll'h hkirigh ge fouhg em bea.
I yimw umw wohgreb bea im Tew'h nehg selb, ampmedm munoh!
Ritsg Soro, Ritsg Med!
Yo Temo, Yo Temo fren gsih kluxo umw gino Ritsg Med!
Im gso muno ef gso Imfimigo, Uyhelago umw Nbhgorieah Tew!
  \end{tcbverbatim}
%\end{figure}

\vspace{2em}

Which, if we run it backwards --- basically, reverse/decipher from LUMTAUTO into plain text (without attempting to make any manual adjustments and assuming the presented text is proper LUMTAUTO text) --- something we might do well via the online \textbf{Crypto} tool shared above, we would then get:

\vspace{2em}


{\ttfamily

I bind you in the name of Ruhanga and all the spirits of nature and
my ancestors here present.
I bind you regardless of whether you are here in spirit, mind or body.
I bind you and cut from you all sources of power that are Animusting
you right now, in the name of the Most Powerful Ruhanga!
I hurl you into an eternal prison, I cut you to shreds, I summon
thunder and fire that eats at the body, mind and spirit to strike
and deprive you of all your strength and health.
I summon the hungriest of hell's spirits to feast on you.
I bind and destroy you in God's most holy, unknown names!
Right Here, Right Now!
Be Gone, Be Gone from this place and time Right Now!
In the name of the Infinite, Absolute and Mysterious God!

}

\vspace{2em}

Which perhaps needs no explanations, apart from the word ``Animusting" potentially having been modified in the original LUMTAUTO version, so that it might possibly have been ``Animating", which would have been \textit{Uminugimt} in Lumtauto\footnote{Further support for this analysis is by the fact that in that book, we later come across creative use of two spirit names --- ``Uminu" and ``Uminah", corresponding to the \textit{anima} and the other the \textit{animus}\cite{jung1964symbols} of the protagonist \textbf{Ignatius Irumba} --- also refer to page 353 of \cite{shrinesjwl}.}.


\vspace{2em}


And so, with what we know now, a better, more practical rendition of this mantra\footnote{\textbf{mantra}(n): (Sanskrit) literally a `sacred utterance' in Vedism; one of a collection of orally transmitted poetic hymns\cite{wordweb_assistant}} and \textbf{exorcising formula} would be the following universally useful formulation:\\


  \begin{tcbverbatim}[title=A Universal EXORCISM Mantra against Malevolence]
Ayimwe bea imi geso muno, efa, RU USU OSOIOS BOSOHASAU!

Umwa ulli, geso hakirigaha, efe mugaro umwa!

Nuba umixo, higerah soro karohomiga.

Iyimwe bea, roturwa-loheha efa disogesora
 
Bea uro soro ima hakiriga, nimwe era yewabu.

Iyimwe bea umwa xagi, frena bea ulla hearaxohu,
 
Efa kedora, Efa kedora!

Gisuga uro uminugimate, bea ritasagi medu!
 
Ima geso, muno efe gaso!

Nehagi Kedora-falu, RU USU OSOIOS BOSOHSAU!

Isarale bea, image uma ogoremule kirihema,
Xagi bea; ge hasoro-wahe, hannemi
Gasamwora Gasamwora!

Umwe firo gasuga oua gahe!
 
Uga, Uga, Uga! Geso yewabu,
Nimwe umwa hikiriga.
 
Ge hagripo, Ge umwa wokarivo bea efa ulla beara

Hagiro mitugis umwa! Soulu gase ula ro uwebu!
 
Ihanneme geso samite riohaga efe!
Solulah Solulah Soluleh!
 
Hakiri gehe, Hakiri geha!
Gege fou Gege hagi, Gege eme bea.

Iyimwa umwa, woha gureba
Belebele bea ima Tewha!
 
Neha neha gise laaba! 
Amapenduma munoh munoh!

Rita sagu, Soro, Ritasage Medu!
Yo Tema, Yo Temu, fareni gisiha kalu exa!
 
Umwe gano-gine ARARITA! Sagu MEDUSA!

Ime gaso muno efe gaso Imi-fifi-migo..
 
Uya, hela, HERA! Go! Umwa Nabahigi!
Rie AHA Lerwa EFE Ulla Lerwah!
He Yo Igu! Unoma Unoma RU!
  \end{tcbverbatim}

\vspace{2em}

\begin{table}[H]
  \begin{tabular}{|p{0.95\textwidth}} % Left border only
    \hline
    \begin{figure}[H]
      \centering
      \includegraphics[width=0.9\textwidth]{resources/illustration_trudy_olga_spirit_fight.jpg}\\
  \caption{Warding Off Undesirable Spirits}
      \label{FIGFIGHT2}
    \end{figure} \\
    \cline{1-1} % Bottom border only
  \end{tabular}
\end{table}


\subsubsection{The ORIGINAL LUMTAUTO Algorithm in JAVASCRIPT}
\label{SECLUMTAUTO_JS}

Note that, for especially background reasons, and for those who wish to replicate the original implementation, note that the following JavaScript program would suffice to reflect the active and two-way (encode/decode) imeplemtation of  \textbf{\hyperref[ALGLUMTAUTO]{Algorithm \ref{ALGLUMTAUTO}}} as currently found in \cite{nuchweziCrypto}:\\


 %\small
  \begin{tcolorbox}[ title=JavaScript Program: DECODE-ENCODE LUMTAUTO transformer function, breakable]
  \begin{lstlisting}[language=JavaScript,breaklines=true]
function transform_chaos(src,target,reverse){
    var v_alphabet = 'aeiou'.split('');
    var v_al_mirror = 'uoiea'.split('');
    var vu_alphabet = 'AEIOU'.split('');
    var vu_al_mirror = 'UOIEA'.split('');
    var l_alphabet = []; var l_al_mirror = [];
    var u_alphabet = []; var u_al_mirror = [];

    for(var l='a'.charCodeAt(0); l <= 'z'.charCodeAt(0); l++){
        var s = String.fromCharCode(l);
        var S = s.toUpperCase();
        if(v_alphabet.indexOf(s) >= 0) {// a vowel
            l_alphabet.push(s);
            u_alphabet.push(S);
            var _s = v_al_mirror[v_alphabet.indexOf(s)];
            var _S = _s.toUpperCase();
            l_al_mirror.push(_s);
            u_al_mirror.push(_S);
        }else {
            l_alphabet.push(s);
            l_al_mirror.push(s);
            u_alphabet.push(S);
            u_al_mirror.push(S);
        }
    }

    for(var i=0; i < l_al_mirror.length * 0.5; i++){
        var t = l_al_mirror[i];
        var _t = l_al_mirror[l_al_mirror.length-1-i];
        if((v_alphabet.indexOf(t) < 0 ) && (v_alphabet.indexOf(_t) < 0 )) {// skip vowel
            l_al_mirror[i] = _t;
            l_al_mirror[l_al_mirror.length-1-i] = t;

            var T = u_al_mirror[i];
            u_al_mirror[i] = u_al_mirror[u_al_mirror.length-1-i];
            u_al_mirror[u_al_mirror.length-1-i] = T;
        }
    }

    var _in = $(src).val();
    var _out = _in.split("").map(function(c){ 
        if(reverse){
            var _c = u_al_mirror.indexOf(c) >= 0 ? u_al_mirror[u_alphabet.indexOf(c)] : (l_al_mirror.indexOf(c) >= 0 ? l_al_mirror[l_alphabet.indexOf(c)] : c); 
            return _c;
        }else{
            var _c = l_al_mirror.indexOf(c) >= 0 ? l_al_mirror[l_alphabet.indexOf(c)] : (u_al_mirror.indexOf(c) >= 0 ? u_al_mirror[u_alphabet.indexOf(c)] : c); 
            return _c;
        }
    }).join("");
    
    $(target).val(_out);
}
   \end{lstlisting}
  \end{tcolorbox}
    \captionof{figure}{JavaScript Program: DECODE-ENCODE LUMTAUTO transformer function}
  \label{FIGLUMTAUTOJSCODE}

\vspace{2em}


\subsubsection{The LUMTAUTO Algorithm in PYTHON}
\label{SECLUMTAUTO_PY}


The up-to-date (one-way/encode) implementation of LUMTAUTO in Python is as follows:

 %\small
  \begin{tcolorbox}[ title=PYTHON Program: ENCODE LUMTAUTO transformer program, breakable]
  \begin{lstlisting}[language=Python,breaklines=true]
#!/usr/bin/env python3
def lauto(msg_in):
    a_z = ['a','b','c','d','e','f','g','h','i','j','k','l','m','n','o','p','q','r','s','t','u','v','w','x','y','z']
    vowels = ['a','e','i','o','u']

    #first, compute mirror of vowels
    vowels_mirror = list(reversed(vowels))

    #compute a new alphabet with all vowels replaced by their mirrors
    a_z_vm = [c if not c in vowels else vowels_mirror[vowels.index(c)] for c in a_z]

    #next compute alphabet mirror, by replacing all non-vowels in a_z_vm with their mirror element from the intermediate alphabet
    a_z_mirror = a_z_vm #first, make a copy

    #thus, we update the intermediate alphabet thus:
    for i in range(int(len(a_z_mirror) * 0.5)):
        t = a_z_mirror[i]
        _t = a_z_mirror[len(a_z_mirror) - 1 - i]
        # ensure neither t nor _t are vowels:
        if not(t in vowels) and not(_t in vowels):
            #swap the two opposite letters
            a_z_mirror[i] = _t
            a_z_mirror[len(a_z_mirror) - 1 - i] = t


    msg_lumtauto = msg_in.lower() # only work on lowercase messages for now
    msg_lumtauto = "".join([a_z_mirror[a_z.index(l)] for l in msg_lumtauto])
    return msg_lumtauto

print(lauto("LANGuage")) #prints 'lumtauto'
   \end{lstlisting}
  \end{tcolorbox}
    \captionof{figure}{PYTHON Program: ENCODE LUMTAUTO transformer program}
  \label{FIGLUMTAUTOPYCODE}




\subsection{More LUMTAUTO Spells and Mantras from ``Shrines of The Free Men"\cite{shrinesjwl}}

Before we continue, note that, considering the kind of spells, mantras, conjurations and such that we are dealing with here and in most of this grimoire, the fact that we somewhat design or re-format our formulas to make them palatable to not just the tongue, but also the mind (and especially the sub-conscious), can be best understood when we reflect on the advise given by \textbf{Janet and Stewart Farrar} in their essential book on spells:

\noindent
\begin{minipage}{1\textwidth}
\vspace{1em}
\begin{quotation}
\noindent {\ttfamily

A traditional spell is often rather like poetry, in that the power can manifest even though the reasons for its working remain a mystery. It is best to keep a cautiously open mind. 

}
\hspace*{\fill} --- \textbf{Spells and How They Work}, \textit{1990}, Janet and Stewart Farrar\cite{farrar1990spells}
\end{quotation}
\vspace{1em}
\end{minipage}

In fact, that book does remind and compel us to not readily discard or take for granted, spells we find well preserved in spell books, grimoires, prayer-books and the kind often passed down from generation to generation via oral tradition alone. This, especially because we might not always understand why the spell works, and yet it actually does work! See that they tell us:


\noindent
\begin{minipage}{1\textwidth}
\vspace{1em}
\begin{quotation}
\noindent {\ttfamily

The formulae, actions and objects or substances they involve many have many different purposes. Sometimes they may be, frankly, just to impress the uninitiated. Sometimes they may be a distorted memory of something which originally had a physically, psychologically or psychically practical reason. And even oftener they may be calling upon a symbolism, whether conscious or unconscious, which aids the inter-level communication... What one should not do is to dismiss them out of hand just because the reason for their form is not immediately obvious to rational analysis. You may find that they work though you don't know why --- that the power of their symbology is buried too deeply in your personal Unconscious (or even in the Collective Unconscious) for your awareness to find it.

}
\hspace*{\fill} --- \textbf{Spells and How They Work}, \textit{1990}, Janet and Stewart Farrar\cite{farrar1990spells}
\end{quotation}
\vspace{1em}
\end{minipage}



The next spell we are going to consider, is that regarding using magick to get visions concerning events or people not immediately within physical reach --- kind of like \textbf{remote viewing} or a kind of \textbf{clairvoyance} --- powers normally exercised by spirit mediums, seers, scryers and people with their \textbf{third eye} open and working actively.

\begin{table}[H]
  \begin{tabular}{|p{0.95\textwidth}} % Left border only
    \hline
    \begin{figure}[H]
      \centering
      \includegraphics[width=0.9\textwidth]{resources/conjurer.jpg}\\
  \caption{Conjuring Visions}
      \label{FIGCONJ1}
    \end{figure} \\
    \cline{1-1} % Bottom border only
  \end{tabular}
\end{table}


It is demonstrated on page 171 of \cite{shrinesjwl} and the equivalent literal wording would be:


%\begin{figure}[H]
  \begin{tcbverbatim}[title=A Conjuration for Visions from the novel ``Shrines of The Free Men"]
Mwimuke inywe abalibata omumuro
Mwimuke inywe abarora omukizima
Mwimuke inywe abarora omumbeho
Mwimuke inywe abarora emizimu
Mwimuke inywe abarubata omumwanya
Mwijje munyoleke omwana wange Nyamwezi.
  \end{tcbverbatim}
%\end{figure}

\vspace{2em}

However, that particular spell was for conjuring visions and images to help one old woman --- a witch and grandmother of a girl child that had gone missing in the deep of night while she slept. So, for purposes of rendering it generally usable, and applicable to any matter concerning wanting to solicit for visions from the spiritual realm, we instead have the following conjuration spell:


{\LARGE

 \begin{tcbverbatim}[title=A Universal Vision CONJURATION Spell Leveraging Familiar Spirits]
Nana dinape imabado uyuliyugu enanare
Nana dinapo imabado uyureru enapizinu
Nana dinape imabado uyureru enanyose
Nana dinapo imabado uyureru onizina
Nana dinape dinapo inapo imabado uyura-yugu, enanu-dumba
Nana dinapo imabado uyure-lara suru!
Nedoqqo, namebele ape pima padomawa padogoti iraze!

________ napi mabelo papa bobo sugi suma, enyahi, huma munura apu!
  \end{tcbverbatim}
  }
 

\vspace{2em}

\begin{table}[H]
  \begin{tabular}{|p{0.95\textwidth}} % Left border only
    \hline
    \begin{figure}[H]
      \centering
      \includegraphics[width=0.9\textwidth]{resources/conjurer_4.jpg}\\
  \caption{Conjuring Visions With an Assistant Medium or Seer}
      \label{FIGCONJ1}
    \end{figure} \\
    \cline{1-1} % Bottom border only
  \end{tabular}
\end{table}


In that spell, it shall be useful to utter some details about the actual vision being sought, in the space left blank (with blank line) in the conjuration formula. Also, in keeping with the aesthetics and method of the original source of that spell\cite{shrinesjwl}, one might want to have the following as part of their operation and operating space:

\begin{enumerate}
\item An altered state of consciousness --- performing the spell itself could cause a change in consciousness on the part of the magician, but it might be better sharped and enhanced if other methods of inducing trance or gnosis were employed as earlier steps --- for example, meditation, vigorous or shamanic dance, intoxication with alcohol or a psycho-active substance, etc.
\item A source of creative power --- depending on taste and tradition, but some good alternatives might be proximity to a water source or focus on some reflective substance --- a mirror, water in a dark bowl, gazing at a lake, clouds, TV static/white-noise, etc.
\item A transmutation power source --- especially via the element of fire; could be a candle, a fireplace, a red cloth or blood.
\item An anchorage point for the spiritual entities inducing the visions --- could be a skull of a dead but familiar or benevolent person such as an ancestor or partner, could be just the bones of a dead animal (especially those of ones familiars or totem animal), could be belongings of such an entity or person or a symbolic expression of any such spiritual powers --- sigilized names of angels or some god, a sacred symbol of a Godform, an identifying sigil for a legion, etc.
\item A means to copy, capture or record any such visions once they are obtained --- might be as simple as having a blank piece of paper and a pen or pencil, might be a camera if the visions are to be projected or sourced external to the operator, might be a blank canvas, paints and a brush if the visions are to be obtained as via an invocation or artistic mediumship, etc.
\end{enumerate}



\subsubsection{Inducing Trance While Invoking Solar Deities}


Often times, there are moments during a ritual, when the operator needs the support or energy of his working to be boosted further, leveraging the powers of a legion or an entire army! A well coordinated chant, especially by the onlookers or subordinate attendants at a group ritual or communal working are what can readily make this proceed fruitfully --- essentially, while the main operator does their thing (perhaps performing a spell, casting or channeling some powers), the others would not merely stand or sit watching, but would join in, and sing or vibrate a particular power-evoking mantra, and if they do it in sync and with sufficient vigor, the entire space, room or the magic circle, would suddenly be filled with sufficient energy to help the magician effectively execute their operation.

\vspace{2em}

We see something of the sort happening, on page 365 of \cite{shrinesjwl}, when four spiritual beings working inside a pristine occult temple deep in the astral, chant continuously, the following LUMTAUTO mantra for raising power and invoking a solar deity...

{\LARGE

 \begin{tcbverbatim}[title=RA Invocation Mantra]
Epatiziydo Piguru! Piguru!
Epatiziydo Piguru! Piguru!
Epatiziydo PiguruI Piguru!
Epatiziydo Piguru! Piguru!
Epatiziydo Piguru! Piguru!
Epatiziydo Piguru! Piguru!
  \end{tcbverbatim}
  
}
  
  
So, while they repetitively vibrate that formula... essentially recanting the phrase ``Epatizido Piguru! Piguru!" again and again... things start happening... and the operator, but also the members participating, shall experience something of a solar deity... most likely \textbf{AMUN RA} or his attributes\footnote{It shall help some readers to realize that the preferred name ``Amun Ra" might vary [significantly] from what some authorities and especially Egyptologists might prefer --- for example, \textbf{Paul Johnson} prefers the phrasing ``Amun-Re"\cite{johnson2000civilization}. However, in practice, such as during invocations and such, one might default to which version resonates best with them.}, manifesting in the air all about them. It can be a great companion mantra to casting out evil spirits, performing healing, dispatching intents or charging tools.

\vspace{2em}

It is left as an exercise to the reader, to find out what exactly those words means. The necessary tools have already been shared.




\begin{figure}[H]
  \begin{center}
   \includegraphics[scale=0.8]{resources/maiu_sigil.pdf}\\
   \caption{The Mysteries}
  \label{FIG2}
  \end{center}
\end{figure}


\begin{figure}[H]
  \begin{center}
   \includegraphics[width=\textwidth]{resources/myrrh_slogan.pdf}\\
  \end{center}
\end{figure}

\section{The Grand Myrrh: \textbf{MYRRH LANGUAGE}}
\label{SECMYRRH}



Without wasting time, first, consider the following TEA program:\\


 %\small
  \begin{tcolorbox}[teaterminalstyle, title=TEA Program: The Grand Myrrh Transform, breakable]
  %\begin{lstlisting}[language=TEA, caption={TP C7}, label={LSTC7}, numbers=left]
  \begin{lstlisting}[language=TEA,breaklines=true]
m!:
r!:y:yua
r!:ht:th
r!:dn:dun
r!:tn:tan
r!:rp:rupa
r!:sy:s y
r!:tc:tauch
r!: :a 
r!:[gG]:su
r!:[dD]:v
z:
   \end{lstlisting}
  \end{tcolorbox}
    \captionof{figure}{The Grand Myrrh Transform}
  \label{FIGTEAMYRRH}


\vspace{2em}


That simple TEA program\footnote{Much more succinct and yet no less powerful than what we saw with LUMTAUTO in its simplest form --- refer to \textbf{\hyperref[FIGLUMTAUTOTEACODE_CLEAN]{Figure \ref{FIGLUMTAUTOTEACODE_CLEAN}}}} has \textbf{a very very powerful application} in Occult Science and Esotericism. Before we move on, note that it is all we need to specify \textbf{the magickal language MYRRH}\footnote{Pronounced ``mira", and also sometimes to be referred to as ``Grand Myrrh", because of the original TEA program it was based on --- early in the days (circa 2021 or so) when TEA was still just a small and niche android app (TTTT) with a very limited instruction set\cite{lutalo2024tea}.}. 

\vspace{2em}

The \textbf{MYRRH Language} can take any ordinary word, phrase or text, and apply some subtle transforms to it that automagically turn it into something of \textbf{the sacred words, words of power or magical spells!}


\vspace{2em}


Take for example, a basic, very common phrase --- actually, perhaps one of the most \textit{religious test-cases} in Computer Science and Software Engineering ever\footnote{If for no other reason, because all students and initiates of computer programming must at some point have to test their teeth at a new or unfamiliar language, and most likely, their teacher or initiator shall task them to write the so-called ``Hello World Program". But also, it is one of the best test-cases for evaluating or comparing [computer] languages\cite{lutalo2020dnap}.} --- such as: 

\vspace{2em}


{\ttfamily

Hello World

}


\vspace{2em}


Transformed using the MYRRH Transform depicted in \textbf{\hyperref[FIGTEAMYRRH]{Figure \ref{FIGTEAMYRRH}}}, it becomes as an occult conjuration...

\vspace{2em}

{\ttfamily
\LARGE

avalaraoawa aoalalaeaha

}

\vspace{2em}



Clearly then, the Grand Myrrh language has immediate use as an effective means of preparing payloads for magical incantations in say a ritual, as part of occult mantras or spell casting. In fact, two more observations might help drive the point home...\\

\begin{enumerate}
\item Of all the nifty little spell-preparation programs we have ever attempted to develop at NES, none has ever been so confounding as the Myrrh Transform! First, because, the way it was discovered... somewhat trial and error, but also creative and perhaps inspired hacks while exploring and studying several traditional and classic spells and grimoires. The original version merely had the name ``myrrh transform", mostly because of the opening TEA instruction required to make the entire transform work --- the \texttt{m:} TEA primitive command also formally known as \textbf{MIRROR} command\cite{Lutalo2024TEATAZ}. However, like the mind-altering, sacred and rare magical incense, myrrh\footnote{Gum myrrh was one of the gifts brought by the Magi in the biblical nativity story\cite{wordweb_assistant}.}, that was offered Jesus and his family at his birth in Bethlehem (\textbf{Mathew 2:12}\cite{newjerusalem1985}), how the discovery of this precious transformer program occurred is but a miracle! We might as well consider it a somewhat divine gift to the illuminates of Nuchwezi... perhaps by angels we don't clearly know yet!

\item{Especially thanks to TEA, it is short, concise and just works! Note that, as we observed and recommended in \textbf{\hyperref[SECEXAMPLESLUMTAUTO]{Section \ref{SECEXAMPLESLUMTAUTO}}} concerning payloads developed using the LUMTAUTO transformer, one might often find that they need to somewhat massage or further edit the resultant payload before it can be adequate for \textit{smooth use} in say a magickal operation. Take for example, the case of ``Hello World" as we have just encountered; if we apply \textbf{\hyperref[TRANSFLUMTAUTO]{Transformer \ref{TRANSFLUMTAUTO}}} to it, it shall result in the phrase:\\

{\centering
\ttfamily

Solle Derlw

}

\vspace{2em}
Which, and clearly, is no match to what we can accomplish with the Grand Myrrh Transformer:

{\centering
\ttfamily

avalaraoawa aoalalaeaha

}

\vspace{2em}

 Of course, experimentation and experience shall prove that each language has its place and purpose, depending on context or problem at hand, but in terms of usability, and with the examples we are going to consider, one shall come to surely love the MYRRH language!

}
\end{enumerate}

\vspace{2em}

Students and practitioners of magick and occult philosophy shall truly come to love such nifty, sacred transformer programs such as this one. Especially by realizing that.. it is ancient wisdom, that to readily bring about a transformation in real life, often, it starts with bringing about a corresponding transformation in the mind... (particularly, via proper manipulation of the subconscious mind) and the careful use of language (especially via magical transformations or projections of will.. such as this TEA program can make possible) underlie many such fundamental practices in (especially western) magick; 

\vspace{2em}

One no longer needs to cram or memorize arcane spells or mysterious incantations from ancient or closely guarded grimoires (\textbf{Ars Goetia, Emerald Tablets of Thoth, Bhagavad Gita,}...etc), and neither might they need to learn ``magical phrases or words of power" from some occult language (such as when a Roman Catholic resorts to use of \textbf{Latin, Aramaic or Hebrew} for the very occult parts of their [mass] rituals.. or when modern Occultists resort to incantations in supposedly ancient ``magical languages" such as \textbf{Egyptian, Sumerian, Aztec, Yiddish, Hindu, Arabic,}... etc for their more arcane/exacting rituals).

\vspace{2em} 

Instead, \textbf{one merely needs have a robust and reliable text-transformation formula or algorithm fit for magical purposes}; such as a [text-processing] programming utility like TEA (the Transforming Executable Alphabet), and knowledge of how to implement such magical transformations for themselves as say via a tool like the TEA WEB IDE\footnote{Checkout \url{https://tea.nuchwezi.com}} or that failing, perhaps having access to a specially crafted, and ready-to-use phrase transformer program such as we see in the above example, and then, voila! They can take any desire, prayer or spell in ordinary natural language (doesn't have to be English), and the transformer program shall automatically generate for them a corresponding, magically (and psychologically potent) version ready to adapt or apply as is.

\vspace{2em}

So, with that introduction out of the way, let us first clearly understand what the underlying algorithm is --- mathematically or rather, formally so..

\vspace{2em}


\begin{transf}[The \textbf{Magical Language \texttt{MYRRH}}]
\label{TRANSFMYRRH}
If $\Theta^n$ is a sequence of $n > 0$ symbols (the original message) spanning the \textbf{Latin Alphabet} or the symbol set $\psi_{az}$ --- such as in \textbf{\hyperref[EQLATINALPHABET]{Equation \ref{EQLATINALPHABET}}}, then the following transformation:\\

\begin{trans}
\label{TRANSMYRRH}
$\Theta^n \xrightarrow{O_{gmyrrh(\cdot)}} \Theta^* = \Omega^k;$\\
$\invpi(\Theta^n) < \invpi(\Theta^*) = \invpi(\Omega^k) : n < k: k \geq n + 2 \quad \forall n,k \in \mathbb{N}$\\
$\land \quad \Omega^k \supset \Theta^n \quad \land \quad \Omega^k \approx \lnot \Theta^n $\\
$\land \quad \xi(\Theta^n \rightarrow \Omega^k) > 1 \quad \land \quad \Psi(\Theta^n \rightarrow \Omega^k) \geq 0$\\
$\qed$
\end{trans}

is guaranteed to always produce/generate a derivative message --- $\Theta^*$ that has the following \textbf{extra} properties:\\


\begin{multline}
\label{EQMYRRH}
\forall \alpha \in \Theta^n \implies \beta \in \Omega^k \implies \begin{cases}
y \rightarrow yua, & \\
ht \rightarrow th,& \\
dn \rightarrow dun,& \\
tn \rightarrow tan,& \\
rp \rightarrow rupa,& \\
sy \rightarrow s y,& \\
tc \rightarrow tauch,& \\
g|G \rightarrow su,& \\
d|D \rightarrow v,& \\
. \rightarrow a.a,& \\ \text{what happened: every original character is padded by `a'}\\
. \in \psi_{AZ} \rightarrow . \in \psi_{az}& \\ \text{what happened: resultant is entirely all lowercase}\\
\end{cases}
\end{multline}
$\qed$

And so that, the resultant [transformed] message, $\Theta^* = \Omega^k$, despite being \textbf{somewhat} composed of the same exact symbols as in the original message, is not exactly equivalent to it, is also more like an \textbf{a-protracted} lateral mirror inversion\cite{transformatics} of the original message, and is an instance of text in the language \textbf{MYRRH}.

\end{transf}



\vspace{2em}

So, now that we have formally defined this language, it is time to explore some of its interesting applications and ``magical properties"!


\subsection{The Four Special Properties of MYRRH}


\begin{enumerate}
\item First of all, note that all phrases in this language a naturally pronounceable or utterable --- makes it a very useful language for practical use in all kinds of magick operations for practitioners of all levels of skill and experience.
\item The ordinary, but also magical phrases transformed into this MYRRH language are somewhat longer than their original versions, but not for the wrong reasons --- naturally, spells meant to be applied via sound/voice are either cast using exacting commands or enchanting conjurations and mantras. So, for the later two cases, longer phrases are naturally fine --- one would prefer to vibrate ``AARAAARIIITAAA" [as it is literally expressed here in a MYRRH-like phrasing] than trying to guess at how to \textit{correctly} vibrate ``ARARITA"\cite{kraig2010modern}.
\item It is arguably ideal for creating \textbf{reversal spells} --- the kind that are fit for effecting healing, undoing harm or hexes, returning lost or stolen things, bringing the dead back to life, resurfacing lost of forgotten memories, etc. Essentially, because it is founded on a mirror operation, and because we know\cite{farrar1990spells}, that mirror operations are a great basic technique in most practical magick, for reversing ill, deflecting curses and bad intent as well as helping reverse black magick.
\item It is ideal, if not one of the best methods of creating \textbf{little spells} or ``basis words of power" --- this, best reflected by the fact that, any single letter, when transformed using this method, shall become a word! Take the example ``R" $\rightarrow$ ``ARA", ``B" $\rightarrow$ ``ABA"\footnote{Perhaps reminiscent of the main incantation or words that Jesus spoke in Gethsemane in his most trying hour (\textbf{Mark 14:36})! The fact that it relates to ``Abba", aramaic for ``Father", and that, its use is closely associated with invoking divine assistance or that of the \textbf{Holy Spirit} (\textbf{Romans 8:15}, \textbf{Galatians 4:6}) would surely make this little facility of the MYRRH transform very powerful!}. But also the protraction of basic phrases such as ``AE" $\rightarrow$ ``AEAAA"\footnote{``Who sweeps" in Runyakitara!}, ``RA" $\rightarrow$ ``AAARA"\footnote{Funnily, but perhaps relevant --- ``Aaara" would translate as ``Who cries" in RUNYORO-RUNYAKITARA, the ancient language of the CWEZI, but also, this little find is interesting magically... it somewhat then synchronizes with ``Who Scries" --- essentially, who has magical sight or intuition,... and this, all just from ``RA", an ancient [solar] God-name also known to be associated with magick!}, etc.  
\end{enumerate}



\subsection{THREE Examples of Applying MYRRH}
\label{SECEXAMPLESMYRRH}


 %\small
  \begin{tcolorbox}[teaterminalstyle, title=TEA Program: The FINAL Grand Myrrh Transform, breakable]
  %\begin{lstlisting}[language=TEA, caption={TP C7}, label={LSTC7}, numbers=left]
  \begin{lstlisting}[language=TEA,breaklines=true]
m!:|r!:y:yua|r!:ht:th|r!:dn:dun|r!:tn:tan|r!:rp:rupa|r!:sy:s y|r!:tc:tauch|r!: :a|r!:[gG]:su|r!:[dD]:v|z:|z*:
   \end{lstlisting}
  \end{tcolorbox}
    \captionof{figure}{The Final Grand Myrrh Transform}
  \label{FIGTEAMYRRHFIN}


\vspace{2em}


\textbf{\hyperref[FIGTEAMYRRH]{Figure \ref{FIGTEAMYRRH}}} is the \textbf{source-code} of the non-interactive TEA program implementing this algorithm, and which, when enhanced with just one more step --- transforming the output of the original MYRRH transformer so that, instead of all lowercase output, and since the use-case is to generate spells or words of power, applies the useful Title-Case transform to the final output, and so that, the final, work-friendly version of that program (also minified) is as depicted in \textbf{\hyperref[FIGTEAMYRRHFIN]{Figure \ref{FIGTEAMYRRHFIN}}}. However, and in case one wishes to look at the code, modify/improve or run \textbf{a live version} of it like on the Linux, Unix, Windows or MAC OS command-line or the WEB, the most recent version should be what you might find or run directly and live via:
  
  
\vspace{1em}

 \url{https://tea.nuchwezi.com/?i=hello+world&fc=https://gist.githubusercontent.com/mcnemesis/89a8030ba026977574930f87273fedd0/raw/grand_myrrh.tea}

\vspace{1em}


\textbf{ALTERNATIVELY} just use the short-link: \url{https://bit.ly/gmyrrh}


\subsubsection{MIRROR Reversal SPELLS}


In exploring how one might go about applying the \textbf{GRAND MYRRH} magickal language and their associated text-transformer (\textbf{\hyperref[TRANSFMYRRH]{Transformer \ref{TRANSFMYRRH}}}), we shall start by revisiting one interesting (and important) case of how mirrors are vital in \textbf{Psychic Self-Defence} --- a very important addition to any magicians arsenal of weapons and utilities\footnote{Because, like soldiers or any army, it is important to keep in mind that where there are means for people or entities to cause harm, induce attacks or even merely make things happen, there is also need to be able to do the reverse; defend, prevent or oppose some actions from happening or perhaps to mirror-back any such undesirable actions [by others?]}. The example is first going to be picked verbatim from Farrar's book on spells\cite{farrar1990spells}:




\noindent
\begin{minipage}{1\textwidth}
\vspace{1em}
\begin{quotation}
\noindent {\ttfamily

Susa Morgan Black sends us another way of ensuring the Boomerang Effect and deflecting bad energy back to the sender...

`Take a special hand-held mirror,' she says, `and simply turn completely around with the mirror reflecting outward and state an affirmation like -

\vspace{1em}

{\centering

`Circle of Reflection,
Circle of Protection,
May the sender of all harm
Feel the power of this charm.'

}

\vspace{1em}

We have used mirrors for this purpose ourselves when we have known who was working against us, and in what direction he or she lived. We would put a mirror in a suitable window facing outwards in that direction, willing it to send the malevolence back to its source.

}
\hspace*{\fill} --- \textbf{Spells and How They Work}, \textit{1990}, Janet and Stewart Farrar\cite{farrar1990spells}
\end{quotation}
\vspace{1em}
\end{minipage}


Thus, for a compelling first example application of our MYRRH language, let us merely enhance the evil-reversal spell cited above, [hopefully] without undoing its original power!


{\LARGE

%\begin{figure}[H]
  \begin{tcbverbatim}[title=A Reversal Spell Against Black Magick]
Aamaraaahaca Asaiataha Afaoa Araeawaoapa Aeataha Alaeaeafa
Amaraaaha Alalaaa Afaoa Araeavauanaeasa Aeataha Ayauaaaaama
Aanaoaiataaauacahaeataoarapa Afaoa Aealacaraiaca
Aanaoaiataaauacahaealafaeara Afaoa Aealacaraiaca.
  \end{tcbverbatim}
%\end{figure}
}

\vspace{2em}


\subsubsection{NAMES of POWER and DIVINE NAMES}


Away from spells, let us first return to the stuff that many ceremonial magicians enjoy --- the stuff of \textbf{Assuming Godforms}\cite{kraig2010modern}! Essentially, there are certain parts of arcane and HIGHER MAGICK, that call for the operator to being something \textbf{greater than} their normal (basal) selves; something along the lines of \textbf{becoming ones Higher Self} --- this, so that then, the magician thus transformed and empowered, can execute tasks and manifest things that a normal human wouldn't be able to, or which they otherwise would only be able to do wishfully but not actually.

\vspace{2em}

One of the well-known methods for assuming a god-form is depicted in the modern magicians training and workbook manuscript attributed to \textbf{Donald Michael Kraig}, and which book, among many things, is a great guide to self-initiation into hermetics and especially the modern \textbf{Golden Dawn} tradition. And so, we see, on page 37 of DMK's magical treatise, the following step that is an essential part of practicing and performing the \textbf{Lesser Banishing Ritual of The Pentagram} (LBRP):


 

\noindent
\begin{minipage}{1\textwidth}
\vspace{1em}
\begin{quotation}
\noindent {\ttfamily

STEP THREE. Step forward with the left foot. At the same time thrust your hands forward so that they point at the exact middle of the glowing blue pentagram in front of you (this position is known as a ``GodForm" and is the God Form known as ``The Enterer"). As you do this you should exhale and feel the energy come back up your body, out your arms and hands, through the pentagram and to the ends of the universe. You should use the entire exhalation to vibrate the God Name: \textit{Yud-Heh-Vahv-Heh}.


}
\hspace*{\fill} --- \textbf{Modern Magick: Twelve Lessons in the High Magickal Arts}, \textit{2010}, Donald Michael Kraig\cite{kraig2010modern}
\end{quotation}
\vspace{2em}
\end{minipage}


And talking of god-forms, it shall help the reader to realize that the above procedure entails knowing a whole lot of theory and practical knowledge in order to pull it off well or effectively. Basically, it is not enough to just know or vibrate the \textbf{sacred names} --- such as the \textbf{TETRAGRAMMATON} --- YHWH --- sometimes expressed as DMK shows in the above instruction, but other times [and though it is generally known that it is not only a \textit{fearsome} name to utter, but is also without explicit pronunciation] as ``Yahweh", ``Jehovah" and other variations. One needs to known and employ their \textbf{entire self} when performing these acts of assuming a godform --- essentially, the meat of what invocations are all about; thus, details such as:

\begin{enumerate}
\item  What is expressed \textbf{physically}; such as the bits about which posture/asana to assume or express, what clothing/vestments/robes to wear, etc.
\item  \textbf{Psychologically}; usually, while in a relaxed but commanding and detached state of mind.
\item \textbf{Psychically}; such as when surrounded by symbols and implements of power --- pentagrams or hexagrams such as in the above example, but also [and based on school or tradition] what decorum the operator employs, etc.
\item \textbf{Astrally}; for example, in the DMK recipe for the LBRP, one comes to this step after having already performed the Kabbalistic Cross ritual, by which time, the magician's astral body is most likely way more pronounced and more protracted than their physical body, etc.
\end{enumerate}


\vspace{2em}



\begin{table}[H]
  \begin{tabular}{|p{0.95\textwidth}} % Left border only
    \hline
    \begin{figure}[H]
      \centering
      \includegraphics[width=0.9\textwidth]{resources/a_modern_chaos_magick_operating_circle_by_nemesisfixx_digtctm.jpg}\\
  \caption{A Magic Circle prepared for an operation at Nuchwezi Esoteric School}
      \label{FIGOLGA}
    \end{figure} \\
    \cline{1-1} % Bottom border only
  \end{tabular}
\end{table}


However, and back to our core purpose here, one needs to perform such ritual magick when they are well employed with the RIGHT WORDS or POWER FORMULAE. Sometimes it helps to know why a certain ritual is supposed to be performed how it is taught --- for example, why it is that DMK instructs the neophyte to vibrate that particular \textbf{word of power} and not any other --- even though it is the fact that there are other sacred names\cite{kraig2010modern} by which a Godform such as the one appealed to in such kaballistic rites might respond to:

\vspace{2em}
\begin{itemize}
\item YUD-HEH-VAHV-HE --- YHVH
\item AH-DOH-NYE --- ADONAI
\item EH-LOH-HEEM --- ELOHIM
\item EL
\item ME-AH-RAHB --- MEARAB
\item EH-HEH-YEH --- EHIEH
\item AH-GLAH --- AGLA
\end{itemize}
\vspace{2em}

But, and it is important to note this; \textbf{it is not ethical to knowingly corrupt or modify the sacred names} --- especially when they are meant to be used in sacred rituals or as part of a communal tradition or rite --- for example, it is well known that the name ``JESUS" isn't exactly what the original literal name of the Son of God was, based on the original language of the original scriptures:

\vspace{2em}
\begin{quotation}
{\ttfamily

The original name of Jesus of Nazareth was not ``Jesus." In his own time, he would have been called Yeshua in Hebrew/Aramaic, a shortened form of Yehoshua, meaning ``Yahweh is salvation." The name ``Jesus" is the English form that developed through Greek (Iēsous) and Latin (Iesus) translations.

}\cite{copilot_assistant}. 
\end{quotation}
\vspace{2em}


BUT, it is also important to realize that, as practicing magicians, sometimes one finds themselves in a tricky situation --- they know or have some form of a sacred name or word of power, such as \texttt{YHVH}, but they are not sure how best to go about pronouncing or uttering it while performing their ritual. And so, \textbf{it is acceptable that one modify or adapt a sacred word or name to a version easier or readily utterable.} And ADDITIONALLY and also out of experience, one finds that \textbf{it might also help to not work within the context of a ritual while using literal names of deities or entities --- principals and powers}. This might be for several reasons;

\begin{enumerate}
\item To conceal a secret from the uninitiated --- like when one must perform a ritual in the presence or within hearing distance of non-members or possibly adversaries.
\item To enhance the potency of a ritual --- just like one might prefer to perform a conjuration or cast a spell in some alien or barbarous language\footnote{Talking of barbarous languages used by actual contemporary magicians, one might for example appreciate the use of [artificial?] magical languages such as \textbf{Ouranian Barbaric}\cite{madara2019robotheosis} --- see \url{http://www.chaosmatrix.org/library/ob.php} --- employed by \textbf{The Illuminati} of IoT, but also several affiliate chaos magicians, and known to have been invented by \textbf{Peter J. Carroll}\cite{copilot_assistant}.} such as our own case of LUMTAUTO already well introduced in \textbf{\hyperref[SECLUMTAUTO]{Section \ref{SECLUMTAUTO}}}.
\item To help readily \textbf{short-circuit} programming of the unconscious mind --- as we have already encountered in the note from the Farrars\cite{farrar1990spells}, but also as eminent scholars like Jung taught concerning use of clever symbolic programming for effecting psycho-social change\cite{jung1964symbols} --- a topic so well covered by the present author in \cite{lutalo2025concerning_trans}.
\end{enumerate}

But also just for creativity's sake. Thus, we might for example prefer to recite mass, while referring to JESUS as ``YEHESHUA" --- a name that one shall find in our virtual shrines at \url{https://iona.nuchwezi.com}, but also, one might instead decide to take ordinary names and words of power, and use their transformed versions --- such as we show in the following table, applying the MYRRH transformer to traditional \textbf{Divine Names}:



\begin{table}[H]
  %\centering
{

\footnotesize  
  
	\begin{tabular}[t]{|l|l|l|}
\rowcolor{lightgray}\bfseries	\textbf{NAME} & \textbf{MYRRH'ed VERSION} & \textbf{RELEVANCE\cite{GoogleAI2025}\cite{copilot_assistant}}\\
	\hline
 \hline
 	\textbf{RA} & AAARA & Ancient Solar Deity in Khem/Egypt\\
	\hline
	\textbf{AMUN RA} & ARA ANAUAMAAA & Another form of RA\\
	\hline
	\textbf{KA} & AAAKA & The human soul/spirit (Egyptian)\\
	\hline
	\textbf{TETU} & AUATAEATA & \makecell[l]{Ancient Deity (Egyptian) also ``THOTH", ``TAHUTI",\\ ``HERMES" (Greek); of Magick and Writing}\\
	\hline
	\textbf{HECATE} & AEATACAEAHA & \makecell[l]{Deity (Greek) also ``HEKATE";\\ of Magick, the Moon, \textit{Liminal Spaces}, and VooDoo!}\\
	
	\hline
	\hline
	\textbf{YHVH} & AHAVAHAYA &\makecell[l]{The Tetragrammaton/God of Israel\\Creator of The Universe}\\
	\hline
	\textbf{ADONAI} & AIAAANAOAVAAA &{Deity Honor (Kabalah), ``My Lords"}\\
	\hline
	\textbf{EHIEH} & AHAEAIAHAEA &\makecell[l]{Deity Title (Kabalah) of Mysterious God of Moses}\\
	\hline
	\textbf{AGLA} & AAALASUAAA &{Deity Honor for ``ADONAI" (Kabalah)}\\
	\hline
	\textbf{EL} & ALAEA &\makecell[l]{Deity (Kabalah, Canaanite tradition)}\\
	\hline
	\textbf{IAO} & AOAAAIA &\makecell[l]{Greek form of ``YHVH", Invokable God}\\
	\hline
	\textbf{MAMMON} & ANAOAMAMAAAMA & \makecell[l]{An Arch-Daemon of Wealth and Money\\also ``NAMON" (Illuminati)}\\
	
	\hline
	\hline
	\textbf{YAHWEH} & AHAEAWAHAAAYA & \makecell[l]{A popular Christianized Deity\\A non-material Omnipotent Spirit}\\
	\hline
	\textbf{JESUS} & ASAUASAEAJA &\makecell[l]{God Incarnate (Christian Mysticism)\\a Benevolent, Merciful Lord; the Saviour}\\
	\hline
	\textbf{YEHESHUA} & AUAHASAEAHAEAYA & \makecell[l]{Also ``JESUS", who understands Humans and Mystics}\\
	\hline
	\textbf{ALLAH} & AHAAALALAAA &\makecell[l]{Deity (Arabic), also ``Al-Ilāh"; the ``God" in Judaism}\\
	
	\hline
	\hline
	\textbf{SATAN} & ANAAATAAASA & \makecell[l]{An Arch-Angel, the ``Devil", ``Adversary",\\ also ``HUGUM" (Illuminati); a Pagan Godform}\\
	\hline
	\textbf{BAPHOMET} & ATAEAMAOAHAPABA & \makecell[l]{Fictitious Deity (Knights Templars)\\often confused with ``MAHOMET" (Islam)}\\
	\hline
	\textbf{ASTAROTH} & ATAHAOARAAATASA & \makecell[l]{Demonised-Deity, Legion Commander\\in sync with Astarte, Lucifer and Beelzebub}\\
	
	\hline
	\hline
	\textbf{ISHTAR} & ARAAATAHASAIA & Deity (Sumerian) also ``INANNA"; of war, love, sex.\\
	\hline
	\textbf{ANU} & AUANAAA & Ancient Deity (Sumerian), ``Father of Gods"\\
	\hline
	\textbf{MARDUK} & AKAUAVARAAAMA & Deity (Babylon) of Creation\\
	
	\hline
	\hline
	\textbf{RAMA} & AMAAARA & \makecell[l]{Deity (Hindu) also ``RAMACHANDRA", ``VISHNU"; of Ideals}\\
	\hline
	\textbf{SHIVA} & AVAIAHASA & \makecell[l]{Deity (Hindu) also ``NATARAJA";\\ of Transformation and Drums}\\
	\hline
	\textbf{KAMA} & AMAAAKA & \makecell[l]{Deity (Hindu) also ``KAMADEVA"; of Love, Eros,\\ Desires, and Wishes}\\
	\hline
	\textbf{MAHE} & AEAHAAAMA & Deity (Hindu) also ``SHIVA", ``MAHENYU" (Cwezi); of Armies\\

	\hline	
	\hline
	\textbf{BANGA} & ASUANAAABA &\makecell[l]{Sovereign Deity (Ganda), also ``RUHANGA" (Cwezi);\\of Aliens, Invisibility and Space.\\In the form ``TONDA", as Creator}\\
	\hline
	\textbf{KIBUKA} & AKAUABAIAKA &\makecell[l]{Deity (Ganda), also ``OMUMBALE", sovereign over ``Air",\\Aerial Travel, Birds and Aerial Spirits}\\
	\hline
	\textbf{KINTU} & AUATAAANAIAKA &\makecell[l]{Deity Attribute (Ganda) of Matter, sovereign over ``Earth";\\ a kind of ``Adam"}\\
	\hline
	\textbf{MANYA} & AYANAAAMA &\makecell[l]{Deity Honor (Ganda) sovereign over Names and Knowledge}\\

	\hline	
	\hline
	\textbf{AHA} & AAAHAAA & \makecell[l]{Deity Honor (Cwezi) of Divine Providence, Wealth and Money}\\
	\hline
	\textbf{WEIRA} & ARAIAEAWA & \makecell[l]{Deity Honor (Cwezi) Mysterious ``First Cause"}\\
	\hline
	\textbf{IRAKA} & AKAAARAIA & \makecell[l]{Deity Attribute (Cwezi) of Communication and Languages}\\
	\hline
	\textbf{MANI} & AIANAAAMA & \makecell[l]{Deity Attribute (Cwezi) Mysterious ``Powers", an ``Almighty"}\\
	
	\hline	
	\hline
	\textbf{HAZORAHIN} & ANAIAHARAOAZAHA & \makecell[l]{Deity Honor (Illuminati) Mysterious 1\\of Psychic Powers}\\
	\hline
	\textbf{HANUZAMEK} & AKAEAMAZAUANAHA & \makecell[l]{Deity Honor (Illuminati) Mysterious 1\\also ``ZHANAMUKE", ``AMENUZAHK", ``AHEZANKUM",\\``KAHENAMUZ", and ``NKUZAMAHE"; of Fun and Drama!}\\
	\hline
	              
\end{tabular}
}
\caption{\texttt{The PANTHEONICS:} \textbf{Divine Names} of POWER: in MYRRH language, Explained}
  \label{TABMYRRHNAMES}
\end{table}


\textbf{NOTE:} Because of space constraints, some of the MYRRH-form of the names in \textbf{\hyperref[TABMYRRHNAMES]{Table \ref{TABMYRRHNAMES}}} have been simplified by reducing extraneous `A's where necessary. However, those interested in securing the proper MYRRH-form of any of the listed divine names, can readily do so using the programs and algorithms already shared.



\subsubsection{WORDS of POWER, FORMULAE for BLACK and WHITE MAGICK}

\begin{figure}[H]
  \begin{center}
   \includegraphics[width=\textwidth]{resources/om_mani.jpg}\\
  \end{center}
  \caption{The \texttt{OM MANI PADME HUM} mantra}
  \label{FIGOMMANI}
\end{figure}


Having considered \textbf{Names of POWER} in \textbf{\hyperref[TABMYRRHNAMES]{Table \ref{TABMYRRHNAMES}}}, and which names might be well applied during either \textbf{Invocations} (for God-forms) or in \textbf{Evocations} (for daemons, thoughtforms or angels), or which might be incorporated into other formulations and arbitrary rituals as one sees fit, let us now turn our attention to \textbf{Words of POWER} --- in fact, \textbf{MANTRAS of POWER} and some \textbf{Magical Formulae}\footnote{\textbf{NOTE}: It might help some readers unfamiliar with how things typically work, to realize that unlike ``Words or Names of Power" that require \textit{other stuff} in order to properly use or apply them, \texttt{MANTRAS} and the kind of magical formulae being discussed in this section can usually be used as-is --- i.e. one can just pick up the mantra and start uttering or vibrating it however way they see fit, and it shall just work! But also, these mantras and formulae might be incorporated into other, longer or more sophisticated spells or ritual formulations by experienced students and magicians who know what they want and know what they are doing.}



\begin{table}[H]
{

\footnotesize  

\begin{tabular}[t]{|l|l|l|}
 \hline
 
	\rowcolor{lightgray}\bfseries \textbf{PHRASE/FORMULA} & \makecell[l]{\textbf{MYRRH'ed UTTERANCE}\\or \textbf{MANTRA}} & \textbf{RELEVANCE/NOTES}\cite{GoogleAI2025}\cite{copilot_assistant}\\
	\hline
	\hline
	
	
	\makecell[l]{\textbf{OM AH HUM}\\\textbf{VAJRA GURU}\\ \textbf{PADMA SIDDHI HUM}} & \makecell[l]{AMAUAHA\\AIAHAVAVAIASA\\AMAVAAAPA\\AUARAUASUA\\ARAJAAAVA\\AMAUAHA\\AHAAA\\AMAOA} &\makecell[l]{\textbf{For Defense and Healing:}\\\\Also as: \texttt{OM AH HUNG BENZA GURU}\\ \texttt{PEMA SIDDHI HUNG} --- for Tibetan Buddhists,\\Attributed to \textit{Padmasambhava} --- founder of\\Tibetan Budhism\cite{Rinpoche1992}. It is said to be:\\\\``the mantra of all the budhas, masters, and\\ realized beings, and so uniquely powerful for\\ peace, for healing, for transformation and for\\protection in this violent, chaotic age."\cite{Rinpoche1992}\\}\\
	\hline
	\hline
	
	
	\makecell[l]{\textbf{OM MANI}\\\textbf{PADME HUM}} & \makecell[l]{AMAUAHA\\AEAMAVAAAPA\\AIANAAAMA\\AMAOA} &\makecell[l]{\textbf{For Compassion and Mercy:}\\\\Also as: \texttt{OM MANI PEME HUNG}, it is attributed\\to Tibetans, and this mantra \\(of \textit{Avalokiteshvara} --- the Budha\\ of Compassion), apart from invoking blessings,\\shall help a deceased family member\\ or friend peacefully pass on into the after-life\\ --- especially when performed while in the\\ presence of the dead person's body or grave\cite{Rinpoche1992}.\\}\\
	\hline
	\hline
	

		\makecell[l]{\textbf{LEAD ME}\\\textbf{OUT OF PRISON}\\\textbf{THAT I MAY PRAISE}\\\textbf{YOUR NAME O LORD.}} & \makecell[l]{AVARAOALA AOA\\AEAMAANA ARAUAOAYA\\AEASAIAARAPA AYAAMA\\ AIA ATAAHATA\\ANAOASAIARAPA AFAOA\\ATAUAOA AEAMA\\AVAAEALA} & \makecell[l]{\textbf{A Powerful Liberation Psalm:}\\attributed to King David (\textbf{Psalms 142:7})\cite{newjerusalem1985}.\\As with most Psalms, its worth is not\\just in its poetic value, but also the spiritual \\value;\\\\ ``The spiritual riches of the Psalter need no\\ commendation... prayers of the Old Testament\\in which God inspired the feelings that his\\children ought to have towards him and\\the words they ought to use when speaking\\to him. They were recited by Jesus himself, \\by the Virgin Mary, the apostles and the\\early martyrs."\cite{newjerusalem1985}}\\
	\hline
	\hline
	
	\makecell[l]{\textbf{MAY WORDS}\\\textbf{SPOKEN BECOME}} & \makecell[l]{AEAMAOACAEABA\\ANAEAKAOAPASA\\ASAVARAOAWA\\AYAAMA} & \makecell[l]{\textbf{For Charging and Manifesting Intents:}\\attributed to JOHN The Apostle (\textbf{John 1:14})\cite{newjerusalem1985}, \\this magical formula is derived from the mystical\\statement:\\\\ \texttt{And the Word became flesh}\\\texttt{and dwelt among us.}\\\\Which, when applied to magick, especially \\ritual magick, is leveraged such that,\\the desire or will/intent(s) of \\the operator or gathered magicians \\(say at a ceremony such as mass or a seance),\\and which typically are expressed as \\either intentions spoken, written or expressed\\in some creative way, are thus charged\\and so willed into being via the mantra,\\and which is founded on this Occult Philosophy\\that John appeals to. Thus then,\\this formula is BEST APPLIED at the very end\\of the ceremony or [perhaps] before [and after]\\ the expression of intents\\ --- such as after the 5$^{th}$ bead of the\\Transformative Rosary Rite\cite{transformation_rosary_rite}.}\\
	\hline
	\hline

	              
\end{tabular}
}
\caption{\texttt{The BENEVOLENT MANTRAS}: \textbf{WORDS of Power} Transformed by MYRRH}
  \label{TABMYRRHWORDS}
\end{table}








\begin{table}[H]
{

\footnotesize  

\begin{tabular}[t]{|l|l|l|}
 \hline
 
	\rowcolor{lightgray}\bfseries \textbf{PHRASE/FORMULA} & \makecell[l]{\textbf{MYRRH'ed UTTERANCE}\\or \textbf{MANTRA}} & \textbf{RELEVANCE/NOTES}\cite{GoogleAI2025}\cite{copilot_assistant}\\
	\hline
	\hline
	
	\textbf{I AM THAT I AM} & \makecell[l]{AMAAA\\AIA\\ATAAAHATA\\AMAAA\\AIA} & \makecell[l]{\textbf{For Commanding Powers:}\\\\Of MASONIC relevance, also formula\\ of God's name given to MOSES at the\\burning bush (\textbf{Exodus 3:14})\cite{newjerusalem1985}\cite{butler1952magic}.\\Great for affirming power, divinity, and command.\\}\\
	\hline
	\hline
	
	
	\makecell[l]{\textbf{PROCUL PROCUL}\\\textbf{ESTE PROFANI}} & \makecell[l]{AIANAAFAOARAPA\\AEATASAEA\\ALAUACAOARAPA\\ALAUACAOARAPA} & \makecell[l]{\textbf{Banishing/Suspending Profane Reality:}\\\\Originally Latin for ``Far hence, far hence,\\ye profane!" and attributed to Book VI\\of \textbf{Virgil}'s epic poem, the \textit{Aeneid}\cite{virgil_aeneid_book6},\\this phrase is a somewhat \textit{underground}\\opening utterance for arcane\\rituals in several secret societies only for\\the initiated, and like how formulas\\such as ``The License to Depart" might\\chase away demons after an evocation, this \\formula is useful in ``chasing away" normal\\consciousness so the participants\\can operate while in a [sufficient]\\altered state of consciousness.}\\
	\hline
	\hline
	
	
	\makecell[l]{\textbf{MORTEM MIHI}\\\textbf{INDUITE NOVUM}\\\textbf{HOMINEM}} & \makecell[l]{AMAEANAIAMAOAHA\\AMAUAVAOANA\\AEATAIAUAVANAIA\\AIAHAIAMA\\AMAEATARAOAMA} & \makecell[l]{\textbf{Induce Metamorphosis or Transfiguration:}\\\\In English:``Darkness, clothe me with\\ a different body."\\\\It is attributed to Lutalo's novel;\\ \textit{Shrines of The Free Men}\cite{shrinesjwl}, and this formula\\is a kind of Death Magick practiced by Black \\Magicians exploring \textbf{Vampire Magick},\\ OBEs or Astral Travel. The basic idea is to enter\\a state of deep sleep-like trance or perhaps\\ a Near-Death Experience (NDE) after\\ performing the mantra, and that, upon waking\\or regaining awareness (possibly in the dream\\ or altered-state), the magician shall find that they\\are someone else or are inside someone else's body!}\\
	\hline
	\hline
	              
\end{tabular}
}
\caption{\texttt{The ARCANE MANTRAS}: \textbf{WORDS of Power} Transformed by MYRRH}
  \label{TABMYRRHWORDS2}
\end{table}


And thus we come to the conclusion of this treatment of the \textbf{MYRRH Language}. As a final note, realize that, even though we have split the above two sets of mantras or words of power collections into ``Benevolent" and ``Arcane", and yet, in reality, either can be used or applied in contexts relating to White or Black Magick --- for example, many people might regard the idea of performing magick that raises the dead or which causes a dead person to ``prematurely" return to life as a kind of necromancy or black magick, and yet, we see such acts having been performed well by the [well-intending and benevolent] Jesus Christ as depicted in the scriptures!


\begin{figure}[H]
  \begin{center}
   \includegraphics[scale=0.9]{resources/ozin_code.pdf}\\
   \caption{CODE: \textbf{Church of Dance Eternal}}
  \label{FIGCODE}
  \end{center}
\end{figure}


\begin{figure}[H]
  \begin{center}
   \includegraphics[width=\textwidth]{resources/ozin_slogan.pdf}\\
  \end{center}
\end{figure}


\section{The OZIN Cipher and Magickal Language}
\label{SECOZIN}

\begin{figure}[H]
  \begin{center}
   \includegraphics[height=0.9\textheight]{resources/COMPLETE_OZINLANGUAGE.pdf}\\
   \caption{The COMPLETE OZIN ALHABET Symbol Set}
  \label{FIGOZINALPHABET}
  \end{center}
\end{figure}


Next, we turn our attention to a language unlike any that we have covered yet in this grimoire; \textbf{a visual magickal language!} Yes, much as the two languages, LUMTAUTO (\textbf{\hyperref[TRANSFLUMTAUTO]{Transformer \ref{TRANSFLUMTAUTO}}}) and MYRRH (\textbf{\hyperref[TRANSFMYRRH]{Transformer \ref{TRANSFMYRRH}}}) are each powerful and have special applications as we have seen in their respective chapters, and yet, they can not do certain things that only a language such as \textbf{OZIN} can do. But what exactly is this OZIN language all about? Why is it special? 

\vspace{2em}


\subsubsection{Some History on Use of Visual Languages in Magick}


First, before we dive into exploring our third magical language system --- the \textbf{OZIN Magickal Language}\cite{lutalo_2025_ozin} --- refer to \textbf{\hyperref[FIGOZINALPHABET]{Figure \ref{FIGOZINALPHABET}}} , let us first jump into a time machine, and travel back to the most ancient days... let us first trace our steps back to when writing and use of written language was still a thing exclusively only for high priests and the most dignified occultists in the land... back in [and before!] the days of \textbf{Hieroglyphics} and \textbf{the earliest forms of writing and written expressions}!

\vspace{2em}


In the [now-rare] encyclopedic compendium about the history and progress of human languages and communication --- \textbf{Communication and Language: Networks of Thought and Action} edited by \textbf{Sir Gerald Barry}, Dr. J. Bronowski, James Fisher and \textbf{Sir Julian Huxley}, we see, in the introduction, a telling of how the magical faculties of formal communication that humans exhibit first came to be...


\begin{figure}[H]
  \begin{center}
   \includegraphics[width=1\textwidth]{resources/kids_in_ancient_khem_studying_rubics_games_LEARNING.jpg}\\
  \end{center}
\end{figure}



\noindent
\begin{minipage}{1\textwidth}
\vspace{1em}
\begin{quotation}
\noindent {\ttfamily

Most animals communicate with their kind in one degree or another, some by methods unknown to or imperfectly understood by men. But man alone has acquired the faculty of communication by \textit{speech}. This unique achievement has been the biggest single factor in the success of \textit{Homo sapiens} in developing complex societies. Only now are we beginning to realize how deeply relevant communication is to the story of human progress. Communication theory, as it is called, is now one of the basic areas of research into human intercourse and understanding.

Speech was the first great leap forward in the development of human communication. The second was the invention of \textit{writing}. By this means, what men thought to themselves or said to one another could be recorded, read by others, and stored for the benefit of future generations. There now existed a communal ``memory."

}
\hspace*{\fill} --- \textbf{Communication and Language: Networks of Thought and Action}, \textit{1965}, Gerald Barry et al.\cite{Barry1965}
\end{quotation}
\vspace{1em}
\end{minipage}



Thus, we start to realize that, apart from [formal] languages underlying our distinctiveness as humans when compared to beasts, they also serve the purpose of setting us apart --- particularly, knowledge of and possession of, ability to read, speak or wield a certain formal language can be all \textbf{the difference between those with a power and privilege and those without!} We clearly see this, when Lutalo, in his treatment of special symbols, words and languages\cite{lutalo2025concerning_trans}, says, as part of his analysis of \textbf{Professor Pierre Bourdieu}'s classic text ``LANGUAGE and SYMBOLIC POWER":




\noindent
\begin{minipage}{1\textwidth}
\vspace{1em}
\begin{quotation}
\noindent {\ttfamily

Also, we see that in many instances of control of access to power, authority and in the exercising of order in society --- especially where many classes and diversities are concerned, that certain ``special" knowledge such as that of ``special" languages, endows or lifts certain members of the community/society to a status and privilege that isn't or can't be readily shared by everyone.


}
\hspace*{\fill} --- \textbf{Concerning a Transformative Power in Certain Symbols, Letters and Words}, \textit{2025}, Joseph Willrich Lutalo\cite{lutalo2025concerning_trans}
\end{quotation}
\vspace{1em}
\end{minipage}


And he goes on to cite Pierre Bourdieu on this one, and which is also relevant to our present explorations, when he informs us that:\\


 \begin{quotation}
%\small
{\ttfamily

The members of these local bourgeoisie of priests, doctors or teachers, who owed their position to their mastery of the instruments of expression, had everything to gain from the Revolutionary policy of linguistic unification... defacto monopoly of politics, and more generally of communication with the central government and its representatives

}
\end{quotation}


\vspace{2em}


Not to deviate significantly from our main discourse, note that in that last quote, the context was about use of and access to ``special languages" --- or particularly, \textit{special dialects} --- by a few privileged people and social-classes during the French Revolution. However, these scenarios and cases aren't few across human history, and neither are they absent in contemporary life and reality. Moreover, and with regards to \textit{the kind} of communication that is most leveraged in magick --- essentially, and usually, languages and expression forms that mostly appeal to the unconscious mind\footnote{For a great and up-to-date model of the mind, treating of the conscious, subconscious and overall unconscious and their attributes, also refer \textbf{a model of the mind} depicted in Lutalo's treatment of modern Psychology\cite{Lutalo2025transpsy}.} --- tools and methods with which not only individuals, but entire communities, generations and all of the human race might be conditioned and programmed, the special use of symbolic forms of expression is one of the cornerstones of many power-wielding elites in all circles ofBa human life since antiquity. 


Concerning use of and particularly, the significance of using symbols in communication, note what Barry et al. (1965) say:



\noindent
\begin{minipage}{1\textwidth}
\vspace{1em}
\begin{quotation}
\noindent {\ttfamily

So far we have looked mainly at symbols that merely convey information. But there is another class of symbols that both carry information and also stand for a body of thought or belief that is not easily expressed in any other way. Just as a simple word can in the course of time gather around it a mass of associations and a variety of meanings that are not easily explained by using other words, so, too, a symbol can stand for much more than its basic meaning in the original code. In other words, symbols can become ``charged" with meaning and with the power to evoke emotions.

}
\hspace*{\fill} --- \textbf{Communication and Language: Networks of Thought and Action}, \textit{1965}, Gerald Barry et al.\cite{Barry1965}
\end{quotation}
\vspace{1em}
\end{minipage}



In that quote's opening sentence, most likely, the reference is to \textit{signs} --- a matter perhaps best treated of in philosophies and studies such as \textbf{Semiotics}. However, the core of that quote and its relevance to us, concerns use of special symbols and writing systems whose value or impact has less to do with the explicit messages or information they carry, and more to do with what they evoke when one encounters them or attempts to process them. In that book\cite{Barry1965}, Barry et al. go on to show us some examples such as the use of the ``fish" symbol to depict or relate ideas to Jesus and/or Christianity especially among the early Christians, but also shows us some other common symbols such as ``Tau" cross and others as depicted in \textbf{\hyperref[FIGCHRISTSYMBOLS]{Figure \ref{FIGCHRISTSYMBOLS}}}\footnote{Note that this figure attempts to express the symbols as close as possible to what Barry et al. (1965) show in their book, but doesn't exactly use their drawing style verbatim.}




\begin{figure}[H]
  \begin{center}
   \includegraphics[width=\textwidth]{resources/earlychristiansymbols.pdf}\\
   \caption{\textbf{Symbolism from Early Christianity:} The Fish Sign (``Pisces" also 12$^{th}$ sign of the Zodiac) (left). Then first two letters of Christos superimposed --- the \textbf{Christogram} (center) and then the ``Tau" Cross with Greek letters from the beginning and end of the alphabet --- ``I am the Alpha and the Omega, the beginning and the end."}
         \label{FIGCHRISTSYMBOLS}
  \end{center}
\end{figure}



Please note that sometimes, correct or proper rendition of ancient symbols can be tricky or troublesome. For example, concerning the last symbol in \textbf{\hyperref[FIGCHRISTSYMBOLS]{Figure \ref{FIGCHRISTSYMBOLS}}} --- the special \textbf{Tau Cross}, Barry et al. use the sometimes confusing rendition that mixes an upper-case ``alpha" with a lower-case ``omega" --- as shown in figures 



\begin{figure}[H]
  \begin{center}
   \includegraphics[scale=0.8]{resources/tau_cross_earlychristiansymbols.pdf}\\
   \caption{The decorated Tau Cross as depicted in Gerald Barry et al. (1965)}
         \label{FIGCHRISTSYMBOLSTAU}
  \end{center}
\end{figure}


\begin{figure}[H]
  \begin{center}
   \includegraphics[scale=0.3]{resources/1091px-Chrisme_Colosseum_Rome_Italy.jpg}\\
   \caption{The decorated Christogram sourced from Wikipedia}
         \label{FIGCHRISTSYMBOLCHRISTOGRAM}
  \end{center}
\end{figure}


Moreover, that same confusion might arise with also the decorated Christogram\footnote{Source: \url{https://en.wikipedia.org/wiki/File:Chrisme_Colosseum_Rome_Italy.jpg}} as depicted in \textbf{\hyperref[FIGCHRISTSYMBOLCHRISTOGRAM]{Figure \ref{FIGCHRISTSYMBOLCHRISTOGRAM}}}, and so that, both precaution and care need be taken when trying to reproduce symbols or symbolic messages from antiquity. However, given that sometimes the necessary fonts or typefaces are missing, and yet one might wish to write or draw such symbols electronically, it then also helps to invest some effort into understanding where they originated from, how they are meant to be used and/or what the symbols might mean --- so as to properly reproduce or perhaps adapt them for their original purposes or in new cases and contexts.


And so, back to our tracing of the history of \textbf{visual magical languages}. Again, we find in Barry et al. an enlightening discussion under the title ``Pictures and Magic", that shall help offer proper context to our use of and development of the language \textbf{OZIN}\cite{lutalo_2025_trans_genetics}. We learn that\cite{Barry1965}:




\noindent
\begin{minipage}{1\textwidth}
\vspace{1em}
\begin{quotation}
\noindent {\ttfamily

Long before man had learned to write, and perhaps even before he had developed a proper spoken language, he was painting pictures.

}
\end{quotation}
\vspace{1em}
\end{minipage}


Moreover, we further learn that, not only because of the style in which these painting were made, but also where they were made, that it is very likely that these primitive people had a truly special purpose for making these drawings\cite{Barry1965}:


\noindent
\begin{minipage}{1\textwidth}
\vspace{1em}
\begin{quotation}
\noindent {\ttfamily

...we have quite recently discovered animal paintings that are at least 20,000 years old. They have survived only because they are far from the cave entrances and therefore unaffected by weather conditions. Their remoteness also makes it certain that these parts of the caves were not lived in, but [were] only visited for special purposes.

}
\end{quotation}
\vspace{1em}
\end{minipage}


Also, it is worth noting that, it is not just about ``how" or ``where" these drawings were made, but also about the ``why", that researchers and students of these early forms of visual communication might be concerned. \cite{Barry1965} puts it clearly enough for our purpose:



\noindent
\begin{minipage}{1\textwidth}
\vspace{1em}
\begin{quotation}
\noindent {\ttfamily

...they were painted under extremely difficult conditions. What was their purpose? The question is important because these paintings are the earliest known attempts by man to represent the world in which he lived. They are the beginnings of art --- a major form of visual communication. In questioning their purpose, we are asking: ``With whom was the artist communicating?"

...The skills that have gradually made us masters of the earth...

Primitive man is a weakling in the world of animals, and yet he has to gain mastery over them or die. One way in which he can build up his courage against animals bigger and stronger than himself is to make pictures of them.

}
\end{quotation}
\vspace{1em}
\end{minipage}


In that final statement, we see the traces of a special logic that underlies much of what actually goes on in the modern practice of magick --- actually, not just in magick, but also, and based on our experience and knowledge, also heavily utilized in hard-science fields such as in Architecture (the drawing of plans ahead of actual construction of structures), in Computer Science (the drawing of system design and flow-charts, state-diagrams, etc. ahead of actual writing or implementation of computer programs), etc. --- whereby, the skilled and learned magician, takes their intent, desire or objective --- typically referred to as a \textbf{Statement of Intent} --- and then, after either writing or thinking about it as clearly as possible, goes ahead to transform it into a visual form; by modern standards, as a \textbf{SIGIL} --- a special, magically relevant kind of picture, and which they then proceed to operate on or with, so as to resolve or \textit{subdue} their main problem. Thus, to the uninitiated, such procedures and rigor might seem useless or ``over-complicating things", and yet, as Barry et al. argue, this is pretty basic and ancient magical wisdom\cite{Barry1965}:


\noindent
\begin{minipage}{1\textwidth}
\vspace{1em}
\begin{quotation}
\noindent {\ttfamily

By making these pictures, Neolithic man was talking to himself, thinking about his coming combats. In effect he was saying: ``This is the animal that I have to kill. He is part of me and the world I live in; by painting this picture I am that much nearer achieving my object." He believed that the act of picturing the animal magically ensured that it would be killed.

}
\end{quotation}
\vspace{1em}
\end{minipage} 


\subsubsection{The Development of Writing Systems and The Priesthood of Thoth}


\begin{figure}[H]
  \begin{center}
   \includegraphics[width=1\textwidth]{resources/hieroglyphics-ancient-egypt-time-machine-images-examples---ANCIENT-scribes-home.jpg}\\
  \end{center}
\end{figure}


We shall not attempt to exhaust our explorations of, analysis of or interpretation of ancient art nor the use of visual or rather, pictographic writing and communication in early and primitive man societies, however, and of interest to us as the Illuminates of Nuchwezi, is the ancient visual communication systems that were employed by the elite and especially the various priesthood of ancient Africa. We especially wish to bring attention to the fact that the \textbf{Priesthood of Tehuti} not only deified Thoth as a god of writing and magick, but also, and jealously so, practiced and preserved the use of writing and especially visual expressions (in the form of \textit{hieroglyphics}) in both scientific and religious, but also for political ends. We for example learn from \textbf{Paul Johnson}'s reference book on many things concerning ancient Egypt:




\noindent
\begin{minipage}{1\textwidth}
\vspace{1em}
\begin{quotation}
\noindent {\ttfamily

The cult of the supergod and of theological imperialism, in which gods left their localities and crossed frontiers without loosing their power, necessarily presupposed the possibility of monotheism, of a universal, all-powerful and solitary deity... A papyrus in the Cairo Museum contains a hymn to Amun-Re with the following passage:\\

Thou art the sole one, who made all there is.\\
The solitary one, who made what exists,\\
From whose eyes mankind came forth,\\
And upon whose mouth the gods came into being...\\
Hail to thee who did this!\\
Solitary sole one, with many hands.\\\\

...Sometimes, the idea was reformulated in a characteristic Egyptian way to embrace the three leading supergods in a trinitarian concept of three natures in the same god. Thus a papyrus now in Leyden dating from the Nineteenth Dynasty gives this formulation: ``All gods are three, Amun, Re and Ptah, and there is no second to them. `Hidden' is his name Amun. He is Re in face and his body is Ptah. Their cities are on earth for ever: Thebes, Heliopolis and Mephis to eternity."\\\\

If there was one supreme god, not confined to a particular place, what was his relationship to non-Egyptians? With the growth of empire, and the parallel tendency for foreigners to settle in Egypt, there is evidence that the latter were granted the protection of the Egyptian pantheon, though not exactly of equal status. The tomb document known as the \textit{Book of the Gates} credits Horus with creating all the main races of mankind; Horus or Sekhmet `protects the souls' of Asians, Negroes and Libyans. The illustrations in the \textit{Book of the Dead} certainly imply that foreigners and interpreters will be present in eternity. Thoth, the god of wisdom and scribes, was credited with inventing all languages, thus giving a kind of legitimacy to foreign customs and culture.

}
\hspace*{\fill} --- \textbf{The Civilization of Ancient Egypt}, \textit{2000}, Paul Johnson\cite{johnson2000civilization}
\end{quotation}
\vspace{1em}
\end{minipage}


We see here, the indications that despite not having originally been a part of the core pantheon of deities in Egypt, and yet, the need to write, or rather, the need to work formal magick, possibly introduced to Egypt by foreigners or perhaps attributed to an originally `alien deity', eventually took its rightful place among the most revered gods, and unto this day, we might not be able to easily dismiss the role that deities such as Thoth and his priesthood played in making modern writing and much of civilization possible. Perhaps Johnson puts it best, when he informs us that:



\noindent
\begin{minipage}{1\textwidth}
\vspace{1em}
\begin{quotation}
\noindent {\ttfamily

And, to judge by tomb biographies, most men who made their way to the top of the State served at some time in roles which demanded a degree of literacy. Literacy was clearly an enormous advantage in the highly centralized, and therefore highly bureaucratic, Egyptian theocracy.\\

But if most important men in Egypt were partly literate, the complexity of the written language, and the multiplicity of the scripts, created a need for professional writers or scribes. Each important religious establishment had a `House of Life', which was a scriptorium and library. Not all priests were professional scribes: scribal priests and lector-priests formed specialized branches of the clergy. Each department of government had its own special scribes: army scribes, navy scribes, treasury scribes, and so forth, who tended to develop specialized scripts of their own. There were business scribes and accountant scribes too. Scribes had their own god, Thoth, the baboon, sometimes portrayed with an ibis-head.

}
\hspace*{\fill} --- \textbf{The Civilization of Ancient Egypt}, \textit{2000}, Paul Johnson\cite{johnson2000civilization}
\end{quotation}
\vspace{1em}
\end{minipage}


And so we start to appreciate the central role of writing and language systems in all of government, religion and even domestic culture! But that's not all.. Johnson does add some few details here, that might further illuminate the practice and style with which these ancient scribes approached their craft:




\noindent
\begin{minipage}{1\textwidth}
\vspace{1em}
\begin{quotation}
\noindent {\ttfamily

They wrote sitting with their legs crossed before them using palettes of wood, with recesses for black and red inks (sometimes other colors); they kept their brushes --- pens, after the Greeks introduced them in the third century BC --- in recesses in the middle, sometimes with a sliding wooden cover, so that these boxes, which they carried in satchels, were the same in all essentials as the pencil boxes in use in Western schools until a few years ago.

}
\end{quotation}
\vspace{1em}
\end{minipage}



 Thus, we can surely appreciate that to be a well-endowed modern magician, one surely ought also have a good taste for not only writing, but also for practicing their craft with the care and attention that our ancient predecessors gave to their craft and profession. Moreover, concerning the importance of picking up, polishing and creatively using a [writing] system or language, he goes on to inform us thus:
 
 
 
 \noindent
\begin{minipage}{1\textwidth}
\vspace{1em}
\begin{quotation}
\noindent {\ttfamily

As among the Christian clergy later, scribes were recruited from even the lowest classes --- especially orphans --- and it was one of the ways in which a poor man could set his children's feet on the bottom ladder of social advancement.


}
\end{quotation}
\vspace{1em}
\end{minipage}



But it wasn't so easy, and not everyone could pick up a language and master it...



 \noindent
\begin{minipage}{1\textwidth}
\vspace{1em}
\begin{quotation}
\noindent {\ttfamily

Scribal exercises, on papyrus, board and other materials, form one of the largest categories of surviving writings from ancient Egypt, and it is plain from them that acquiring professional status as a scribe was a very long and arduous business --- devoted chiefly to the copying of classical texts and didactic exercises --- whose rigours the scribes and their teachers justified to themselves by dwelling on the dignity and security of their profession. 


}
\end{quotation}
\vspace{1em}
\end{minipage}


Which reminds us that, to become a scribe, much as it was not easy, and that it was jealously guarded, brought with it not just the comfort of being in a class different from the ordinary, but, and as we see in the next excerpt, also had its certain rewards:



 \noindent
\begin{minipage}{1\textwidth}
\vspace{1em}
\begin{quotation}
\noindent {\ttfamily

Scribes, they boasted, were exempt from the \textit{corv\'ee}, from active military service, and from land taxes. Thus an Egyptian learning proverb states: ``What you gain in one day at school is for eternity. The work done there lasts as long as mountains." Or again: ``Do you not carry a palette? That is what constitutes the difference between you and a man who has to pull an oar." Or yet again: ``Put writing in they heart [i.e. learn to write], so that thou mayest protect thine own person from any kind of manual labour, and be a respected official."


}
\end{quotation}
\vspace{1em}
\end{minipage}


Finally, and back to the point we raised earlier, that the development of writing systems in Africa underlies much of later human progress and civilization, Johson has this to say:


  \noindent
\begin{minipage}{1\textwidth}
\vspace{1em}
\begin{quotation}
\noindent {\ttfamily

The Neoplatonists, like Plato himself, were actually looking for metaphysical short-cuts to the tedious business of acquiring knowledge by exact observation and logical proof --- what we call science. They thought they could fathom the secrets of the universe in their own minds provided they had the right key and methodology. Plotinus, the most influential of the Neoplatonists, believed and taught that the Egyptians had discovered (or perhaps had been given by `Theuth' or Hermes) this methodology, and put it into hieroglyphs. Thus they wrote with separate pictures of individual objects --- not only letters expressing sounds and syllables --- and these pictures did not directly portray the objects they apparently represented but penetrated to the very essence of things and encapsulated knowledge possessed only by the gods. The Egyptians had thus discovered `pure' philosophy, uncomplicated and unclouded by the barrier of an alphabetic and phonetic language. This material was embodied in a collection called \textit{Corpus Hermeticum}, reputedly by `Hermes Trismegistus' compiled in the late antiquity but supposed, even then, to go back to prehistoric times.\\


}
\end{quotation}
\vspace{1em}
\end{minipage}


and did this knowledge and ancient wisdom continue to allure and draw hordes of modern elites and especially magicians or perhaps the Illuminati...

 \noindent
\begin{minipage}{1\textwidth}
\vspace{1em}
\begin{quotation}
\noindent {\ttfamily

The medieval Christians in the West knew nothing of these texts at first hand... In 1414, fragments of Ammianus, the last great historian of antiquity, turned up in German monastery. He referred to the hieroglyphs on the obelisks brought to imperial Rome and summarized the view that throughout ancient times scholars had drawn on the ancient, god-given knowledge of the Egyptian seers and priests.\\\


...Egyptian knowledge, or pseudo-knowledge, operated at a number of levels simultaneously. Until the seventeenth century, when an unabridged fissure opened between empirical science, on the one hand, and metaphysics, astrology and esoteric knowledge on the other, men saw natural philosophy as a \textit{continuum}. Even as late a figure as Isaac Newton thought the fissure could be crossed. Hence in the sixteenth century, a great mathematician and physicist like John Dee studied the Hermetic texts with fervent hope... imperial Prague was both the leading scientific center in Europe and the place where Egyptian Hieroglyphics and the secrets they were believed to contain were most eagerly scrutinized.


}
\end{quotation}
\vspace{1em}
\end{minipage}


So, both mainstream occultists, but also empirical scientists were passionate about and engrossed substantially in the study of these ancient pictograms...


 \noindent
\begin{minipage}{1\textwidth}
\vspace{1em}
\begin{quotation}
\noindent {\ttfamily

At the level of allegoric art, compendia of hieroglyphic signs, real and invented, such as Colonna's \textit{Hypnerotomachia Polifil} (2499) and Valeriano's \textit{Hieroglyphica}, were used by court artists throughout the sixteenth and seventeenth centuries. Masons and other secret societies such as the Rosicrucians used these signs and symbols also, especially in relation to the Roman obelisks; they were among the first to relate the esoteric knowledge of ancient Egypt to its physical memorials in stone, and to the mythology of Egyptian religion


}
\end{quotation}
\vspace{1em}
\end{minipage}


\begin{figure}[H]
  \begin{center}
   \includegraphics[width=1\textwidth]{resources/kids_in_ancient_khem_studying_rubics_games_LEARNING_B.jpg}\\
  \end{center}
\end{figure}


{%\LARGE
%\hl{TODO: ADD SOME HIEROGLYPHICS PHOTOGRAPHS from WIKIPEDIA HERE}

}


\subsubsection{Modern Use of Visual Languages in Magick}


Away from primitive man, note that visual languages are heavily utilized in modern magical systems and practices --- both privately and in public, and for purposes and levels of applications spanning all the way from basic and near mundane (such as the use of a ``heart" symbol in a modern social-media account profile name, tag-line or inline a text-message so as to attract or influence peers and onlookers towards ``loving" the one thus communicating) to sophisticated and complex political, commercial and/or religious mind-programming of the citizenry (partisan symbolism and national identity for example), consumers (subliminal messaging in advertising and branding) and/or believers or followers (faith-specific symbols of God, use of color to depict or distinguish spiritual powers, etc.) respectively.



\begin{figure}[H]
  \begin{center}
   \includegraphics[width=1\textwidth]{resources/hieroglyphics-ancient-egypt-time-machine-images-examples---ANCIENT-SEX-MAGiCK-temple.jpg}\\
  \end{center}
\end{figure}



\vspace{2em}


For the practicing magician, one interesting case of where use of visual writing or visual communication --- especially, and \textbf{specifically pictorial communication} and not sounds or mere verbal writing as with ordinary alphabets such as the commonplace Latin Alphabet, can be seen in a ritual recipe that 20$^{th}$ century magician and author, \textbf{Donald Michael Kraig} gives in his treatment of use of \textbf{Sigilization in Sex Magick}. We shall not attempt to reproduce the entire ritual formula here, nor attempt to provide all the essential theory and context necessary to make these methods sensible, but shall focus on just the essential bits given our present discussion:



\noindent
\begin{minipage}{1\textwidth}
\vspace{1em}
\begin{quotation}
\noindent {\ttfamily

Let both participants be aware that a magickal act, \textbf{a spiritual act} is about to be performed, not just a common act of sexual intercourse.

STEP ONE. Let both participants know the purpose of this act. A divination should be done, with both persons present, to insure the ``karmic correctness" of the magickal act.

STEP TWO. Let a suitable sigil, representing the purpose of the magickal act, be designed. Although a sigil taken from a grimoire will do, designing an original sigil is a good idea. The system of A. O. Spare is quite good for this.

STEP THREE. Let large versions of the above sigil be made and placed around the room. This must include the ceiling so that no matter which way you look you will see the magickal sigil.

}
\hspace*{\fill} --- \textbf{Modern Magick}, \textit{2010}, Donald Michael Kraig\cite{kraig2010modern}
\end{quotation}
\vspace{1em}
\end{minipage}


Thus, we see that, it is not a by-the-way, or even a last-step procedure, but is actually at the CORE of how modern magick might be approached or performed. DMK does help us here, by bringing up the name of another 20$^{th}$ century magician and authority in both the creative and magickal fields --- \textbf{Austin Osman Spare}!



\subsection{The OZIN Language System}

\begin{defn}[The \textbf{Extended Latin Alphabet}, $\psi_{09az}$]
\label{DEFXLATINALPHABET}
The 26 symbols of the standard Latin Alphabet, $\psi_{az}$ as expressed in \textbf{\hyperref[EQLATINALPHABET]{Equation \ref{EQLATINALPHABET}}}, prefixed by the 10 symbols of the standard base-10 symbol set, $\psi_{10} = \psi_{09}$, fully specify the symbol set we are calling \textbf{The Extended Latin Alphabet}. Equivalently:

Given

\begin{equation}
\label{EQDECIMALSYMBOLSET}
\psi_{09} = \langle 0, 1, 2, 3, 4, 5, 6, 7, 8, 9 \rangle
\end{equation}

Then

\begin{equation}
\label{EALPHANUMSYMBOLSET}
\psi_{09az} = \psi_{09} \cdot \psi_{az} = \langle 0, 1, 2, 3, 4, 5, 6, 7, 8, 9, a, b, c, d, e, f, g, h, i, j, k, l, m, n, o, p, q, r, s, t, u, v, w, x, y, z \rangle
\end{equation}

The special symbol set $\psi_{09az}$, which, based on \textbf{Theorem 2} in \cite{base36paper} we might also refer to as the \textbf{Base-36} symbol set, $\psi_{36}$, is also equivalently, the \textbf{Extended Latin Alphabet} for our purposes in this manuscript and future purposes.
\end{defn}

\vspace{2em}

And with that definition out of the way, we can then formally define the ozin language as such:

\vspace{2em}


\begin{transf}[The \textbf{Magical Language \texttt{OZIN}}]
\label{TRANSFLUMTAUTO}
If $\Theta^n$ is a sequence of $n > 0$ symbols (the original message) spanning the \textbf{Extended Latin Alphabet} or the symbol set $\psi_{36}$, 

then the following transformation:\\

\begin{trans}
\label{TRANSOZIN}
$\Theta^n \xrightarrow{O_{lozin(\cdot)}} \Theta^* = \Omega^n : \mathbb{N} \times (\psi_{36} \rightarrow \psi_{ozin}) : \mathbb{N} \times \psi_{ozin};$\\
$\invpi(\Theta^n) = \invpi(\Theta^*) = \invpi(\Omega^n)$\\
$\land \quad \forall \theta_{i \in [1,n]} \in \Theta^n \quad \exists \omega_{j \in [1,n]} \in \Omega^n \quad \land \quad \invpi(\theta_i \in \Theta^n) = \invpi(\omega_i \in \Omega^n) = 1$\\
$\land \quad \forall \alpha \in \psi(\Theta^n): \invpi(\alpha \in \psi(\Theta^n)) = f_\alpha \implies \beta \in \psi(\Omega^n): \invpi(\beta \in \psi(\Omega^n)) = f_\alpha  \quad \text{ iff } \quad \alpha \in \psi_{36} \implies \beta \in \psi_{ozin}$\\
$\land \quad \overset{>}{\psi(\Theta^n)} \implies \overset{>}{\psi(\Omega^n)}$\\
$\land \quad \tilde{A}(\Theta^n \rightarrow \Omega^n)_{\psi_{36}} = 0 \qed$
\end{trans}

Which is one way of saying, the transformation of a message from the Extended Latin Alphabet into the OZIN language, is guaranteed to always produce/generate a derivative message --- $\Theta^*$ that has the following properties:\\


\begin{multline}
\label{EQOZIN}
\forall \alpha \in \Theta^n \implies \beta \in \Omega^n: \mathbb{N} \times \psi_{ozin}
\end{multline}

Where $\psi_{ozin}$ is an extension of the symbol set $\psi_{oz}$ first introduced in \textbf{Figure 10.2} of \cite{lutalo_2025_trans_genetics}\footnote{For those that had not yet become familiar with on-going research at Nuchwezi concerning development and formalization of the new mathematics of Transformatics\cite{transformatics} \cite{Lutalo2025_transformatics_thesis}, the OZIN language we are treating of here, was actually first well introduced in a mini-treatise on applying Transformatics in GENETICS\cite{lutalo_2025_trans_genetics}, and which work, focus was mostly placed on using just the number-subset of $\psi_{36}$, so that then, in a hypothetical genome expression system --- the \textbf{Lu Genome Expression System}(LGES), it would be possible to take a DNA or na-Sequence describing some species or organism or entity, and based on some sound logic relating to how living things manifest based on their genetic code, render or approximate/predict a visual-spatial expression that best describes the simplest form of that resulting or associated entity. The key assumption then, was that, unlike the typical DNA code that spans the symbols A-T-G-C, that these codes would need to be mapped to a sequence of just digits. There might be special occult science applications for that model too, however, in this grimoire, we shall not treat of that application of OZIN in GENETICS, and shall instead focus on its use as  normal magickal language.} , and which is now expressed in full as depicted in \cite{lutalo_2025_ozin} and \textbf{\hyperref[FIGOZINALPHABET]{Figure \ref{FIGOZINALPHABET}}} as a \textbf{one-to-one} mapping from \textbf{Base-36 symbol set}, $\psi_{36} \rightarrow \psi_{ozin}$.
$\qed$

\vspace{2em}

And so that, the resultant [transformed] message, $\Theta^* = \Omega^n \approx \Theta^n$, is not only semantically equivalent to the original message, but also its length \textbf{is exactly equivalent to that of the original message}, with the only basic different being that its message is expressed using the alternative symbols or visual glyphs different from the original symbol set, but otherwise corresponding to it item-wise. Any such instance of text thus transformed is then an expression in the language \textbf{OZIN}.

\end{transf}


\vspace{2em} For clearer appreciation and study of the $\psi_{ozin}$ alphabet, let us consider its elements when split up into two figures --- \textbf{\hyperref[FIGOZINALPHABETNUMBERS]{Figure \ref{FIGOZINALPHABETNUMBERS}}} for just the numbers: mappings from $\psi_{10} = \psi_{09} \rightarrow \psi_{ozin}$, while \textbf{\hyperref[FIGOZINALPHABETLETTERS]{Figure \ref{FIGOZINALPHABETLETTERS}}} for the letters only: mappings from $\psi_{az} \rightarrow \psi_{ozin}$.



\begin{figure}[H]
  \begin{center}
   \includegraphics[width=\textwidth]{resources/NUMBERS_OZINLANGUAGE.pdf}\\
   \caption{The NUMBER OZIN ALHABET: $\psi_{10} = \psi_{09} \rightarrow \psi_{ozin}$}
  \label{FIGOZINALPHABETNUMBERS}
  \end{center}
\end{figure}



\begin{figure}[H]
  \begin{center}
   \includegraphics[height=1\textheight]{resources/LETTERS_OZINLANGUAGE.pdf}\\
   \caption{The LETTERS OZIN ALHABET: $\psi_{az} \rightarrow \psi_{ozin}$}
  \label{FIGOZINALPHABETLETTERS}
  \end{center}
\end{figure}


Next, let us consider some of the interesting magical applications of the OZIN language.



\subsection{5 Examples of Applying OZIN}
\label{SECEXAMPLEOZIN}



\subsubsection{OZIN MAGICK}



\begin{figure}[H]
  \begin{center}
   \includegraphics[width=1\textwidth]{resources/OZIN_MAGICK.pdf}\\
   \caption{The message ``Ozin Magick", expressed using the OZIN language}
  \label{FIGOZINEX1}
  \end{center}
\end{figure}


\begin{figure}[H]
  \begin{center}
   \includegraphics[scale=1]{resources/philosopher.pdf}\\
     \label{FIGPHILOS}
   \caption{The Philosopher}
  \end{center}
\end{figure}


\begin{figure}[H]
  \begin{center}
   \includegraphics[width=\textwidth]{resources/medina_slogan.pdf}\\
  \end{center}
\end{figure}

\section{The Crypt of MEDINA}
\label{SECMEDINA}




blah blah


\section{Finale}
\label{SECFIN}

blah blah...

\bibliographystyle{unsrt}
\bibliography{references}


\vspace{5cm}
\fbox{
\begin{minipage}{0.9\textwidth}
\textbf{TO CITE:}\\

Lutalo, Joseph Willrich (2025). \textbf{TRANSFORMATICS 101 - explained.} figshare. Thesis. \url{https://doi.org/10.6084/m9.figshare.30305056}

\end{minipage}}
\\
%}


% insert [front] cover --- could just be a PNG or PDF
\includepdf[pages=1]{resources/back_cover.pdf}

\end{document}

% try to explore how to fit the entire paper on 1 page. Especially using A4 size paper.